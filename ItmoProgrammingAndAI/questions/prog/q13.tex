\subsection{Параллельное программирование. Консенсусное число. Стек Трайбера. Очередь Майкла-Скотта.}

\href{https://www.babichev.org/tpmtp/Lecture09.pdf}{Про консенсус}.
\D{
	Задача консенсуса: есть $N$ потоков, нужно чтоб они все пришли к согласию по поводу какого-то одного значения.
}

\D{
	Любой последовательный объект можно реализовать без ожидания (wait-free) для N потоков используя консенсусный протокол для N потоков.
	
	Такое построение называется универсальная конструкция
}	

\href{https://neerc.ifmo.ru/wiki/index.php?title=%D0%A1%D1%82%D0%B5%D0%BA_%D0%A2%D1%80%D0%B0%D0%B9%D0%B1%D0%B5%D1%80%D0%B0}{Стек Трайбера}

\href{https://neerc.ifmo.ru/wiki/index.php?title=%D0%9E%D1%87%D0%B5%D1%80%D0%B5%D0%B4%D1%8C_%D0%9C%D0%B0%D0%B9%D0%BA%D0%BB%D0%B0_%D0%B8_%D0%A1%D0%BA%D0%BE%D1%82%D1%82%D0%B0}{Очередь Майкла-Скотта}.