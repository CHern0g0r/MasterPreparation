\subsection{Java. Java Reflections. Работа с метаинформацией классов в процессе исполнения программ}

Java Reflection API~---~это программный интерфейс в языке Java, который позволяет приложениям анализировать свои компоненты и программное окружение, изменять собственное поведение и структуру. Позволяет исследовать информацию о полях, методах и конструкторах классов.

С помощью механизма рефлексии можно обрабатывать типы, которые отсутствовали при компиляции, но появились во время выполнения программы. Рефлексия и наличие логически целостной модели выдачи информации об ошибках позволяют создавать корректный динамический код. 

Помимо самомодификации, API способен проводить самопроверку и самоклонирование. Чаще всего рефлексию Java используют:
\begin{enumerate}
	\item для получения информации о классах, интерфейсах, функциях, конструкторах, методах и модулях;
	\item изменения имен функций и классов во время выполнения программы;
	\item создания новых экземпляров классов;
	\item анализа и исполнения кода, поступающего из программного окружения;
	\item преобразования классов из одного типа в другой;
	\item создания массивов данных и манипуляций с ними;
	\item установления значений полей объектов по именам;
	\item получения доступа к переменным и методам, включая приватные, и к внешним классам;
	\item вызова методов объектов по именам.
\end{enumerate}
\href{https://blog.skillfactory.ru/glossary/java-reflection-api/#:~:text=Java%20Reflection%20API%20%E2%80%94%20%D1%8D%D1%82%D0%BE%20%D0%BF%D1%80%D0%BE%D0%B3%D1%80%D0%B0%D0%BC%D0%BC%D0%BD%D1%8B%D0%B9,%D0%BF%D0%BE%D0%BB%D1%8F%D1%85%2C%20%D0%BC%D0%B5%D1%82%D0%BE%D0%B4%D0%B0%D1%85%20%D0%B8%20%D0%BA%D0%BE%D0%BD%D1%81%D1%82%D1%80%D1%83%D0%BA%D1%82%D0%BE%D1%80%D0%B0%D1%85%20%D0%BA%D0%BB%D0%B0%D1%81%D1%81%D0%BE%D0%B2.}{Подробнее + пример}