\subsection{Java. Устройство сборщика мусора в JVM.}

\href{https://medium.com/nuances-of-programming/%D1%81%D0%B1%D0%BE%D1%80%D0%BA%D0%B0-%D0%BC%D1%83%D1%81%D0%BE%D1%80%D0%B0-%D0%B2-java-%D1%87%D1%82%D0%BE-%D1%8D%D1%82%D0%BE-%D1%82%D0%B0%D0%BA%D0%BE%D0%B5-%D0%B8-%D0%BA%D0%B0%D0%BA-%D1%80%D0%B0%D0%B1%D0%BE%D1%82%D0%B0%D0%B5%D1%82-%D0%B2-jvm-25bb2570b44c}{Подробнее тут}
Сборка мусора в Java~---~это процесс, с помощью которого программы Java автоматически управляют памятью. Java-программы компилируются в байт-код, который запускается на виртуальной машине Java (JVM).

Когда Java-программы выполняются на JVM, объекты создаются в куче, которая представляет собой часть памяти, выделенную для них.

Пока Java-приложение работает, в нем создаются и запускаются новые объекты. В конце концов некоторые объекты перестают быть нужны. Можно сказать, что в любой момент времени память кучи состоит из двух типов объектов:
\begin{enumerate}
	\item Живые~---~эти объекты используются, на них ссылаются откуда-то еще.
	\item Мертвые~---~эти объекты больше нигде не используются, ссылок на них нет.
\end{enumerate}

Сборщик мусора находит эти неиспользуемые объекты и удаляет их, чтобы освободить память.

\textbf{Этапы сборки мусора в Java:}
\begin{enumerate}
	\item Пометка объектов как живых
	\item Зачистка мертвых объектов
	\item Компактное расположение оставшихся объектов в памяти
\end{enumerate} 

\textbf{Сбор мусора по поколениям.}

Сборщики мусора в Java реализуют стратегию сбора мусора поколений, которая классифицирует объекты по возрасту.

Область памяти кучи в JVM разделена на три секции:
\begin{enumerate}
	\item \emph{Молодое поколение.} 
	Вновь созданные объекты начинаются в молодом поколении. Молодое поколение далее подразделяется на две категории.
	\begin{enumerate}
		\item Пространство Эдема~---~все новые объекты начинают здесь, и им выделяется начальная память.
		\item Пространства выживших (FromSpace и ToSpace)~---~объекты перемещаются сюда из Эдема после того, как пережили один цикл сборки мусора.
	\end{enumerate}
	
	\item \emph{Старшее поколение.}
	Объекты-долгожители в конечном итоге переходят из молодого поколения в старшее. Оно также известно как штатное поколение и содержит объекты, которые долгое время оставались в пространствах выживших.
	\item \emph{Постоянное поколение} (\emph{Мета-пространство}, начиная с Java 8).
	Метаданные, такие как классы и методы, хранятся в постоянном поколении. JVM заполняет его во время выполнения на основе классов, используемых приложением. Классы, которые больше не используются, могут переходить из постоянного поколения в мусор.
	
	Начиная с Java 8, на смену пространству постоянного поколения (PermGen) приходит пространство памяти MetaSpace. Реализация отличается от PermGen — это пространство кучи теперь изменяется автоматически.
\end{enumerate}