\subsection{Архитектура ЭВМ. Кэш-память. Многоуровневая организация кэш-памяти. Протоколы когерентности кэш-памяти}

\subsubsection{Архитектура ЭВМ}
\textbf{Архитектура ЭВМ}~---~
это модель, устанавливающая принципы организации вычислительной системы, состав, 
порядок и взаимодействие основных частей ЭВМ, функциональные возможности, 
удобство эксплуатации, стоимость, надежность.

\subsubsection{Кэш-память. Многоуровневая организация кэш-памяти. Протоколы когерентности кэш-памяти}
\textbf{Кэширование}~---~это использование дополнительной быстродействующей памяти (кэш-памяти) для хранения копий блоков информации из основной (оперативной) памяти, вероятность обращения к которым в ближайшее время велика.

\textbf{Аспекты кэшей}:
\begin{enumerate}
	\item Кэш-линии~---~вся память выровнена и разбита на непересекающиеся отрезки по 64 байта (последние 6 бит не используются как тег в ассоциативном кэше). 
	\item Кэши делятся на уровни (ближе к процессору $\Rightarrow$ больше скорость, меньше размер). 
	\begin{enumerate}
		\item L1 (32k, делится на данные и команды, Associativity = 4)
		\item L2 (256k, Associativity = 8)
		\item L2 (8Mb, Associativity = 16)
	\end{enumerate}
	\item Ассоциативность. Полная асс. - это когда мы просто пишем в кэш и для поиска нужного адреса нужно бежать по всем линиям. Асс =1 - это когда мы просто мапим по хвосту адреса (тегу) в таблицу. (типа хеш-таблица). Тогда быстро искать, но часто будем промахиваться. Если Acc = k - то мапим в корзины по k и получаем компромисс.
	\item  Эксклюзивность/инклюзивность - данные хранятся только в одном кэш (- эффективность из-за поиска, + размер) или они дублируются в уровнях (в памяти) ниже. (+ скорость, - размер)
	\item Когда у нас много процессоров, то возникает необходимость использовать протоколы когерентности. (MSI, MESI, MESIF, MOESI). Это необходимо, чтобы один процессор мог знать о том, что данные изменились только в кеше одно из его соседей. Для этого вводятся разные состояния владения памятью (invalid, shared, modified + owned/forward, exclusive), чтобы как можно более тоньше развести их и поменьше сбрасывать кэши.
\end{enumerate}

\subsubsection{Многоуровневая организация кэш-памяти (подробнее)}

\emph{ВОДА:}

Современные технологии позволяют разместить КЭШ-память и ЦП на общем кристалле. Такая внутренняя КЭШ-память строится по технологии статического ОЗУ и является наиболее быстродействующей. 

Емкость ее обычно не превышает 64 Кбайт. Попытки увеличения емкости обычно приводят к снижению быстродействия, главным образом, из-за усложнения схем управления и дешифрации адреса. 

Общую емкость КЭШ-памяти ЭВМ увеличивают за счет второй (внешней) КЭШ-памяти, расположенной между внутренней КЭШ- памятью и ОЗУ. Такая система известна под названием двухуровневой, где внутренней КЭШ-памяти отводится роль первого уровня (L1), а внешней — второго уровня (L2). Емкость L2 может быть значительной (до 1 МБ). 

При доступе к памяти ЦП сначала обращается к КЭШ-памяти первого уровня. В случае промаха производится обращение к КЭШ-памяти второго уровня. Если информация отсутствует и в L2, выполняется обращение к ОЗУ и соответствующий блок заносится сначала в L2, а затем и в L1. Благодаря такой процедуре часто запрашиваемая информация может быть быстро восстановлена из КЭШ-памяти второго уровня. Для ускорения обмена информацией между ЦП и L2 между ними часто вводят специальную шину, так называемую шину заднего плана, в отличие от шины переднего плана, связывающую ЦП с основной памятью. 

Количество уровней КЭШ-памяти не ограничивается двумя. В некоторых ЭВМ можно встретить КЭШ-память третьего уровня (L3). Ведутся активные дискуссии о введении также и КЭШ-памяти четвертого уровня (L4). Характер взаимодействия очередного уровня с предшествующим аналогичен описанному для L1 и L2. Таким образом, можно говорить об иерархии КЭШ-памяти. Каждый последующий уровень характеризуется большей емкостью, меньшей стоимостью, но и меньшим быстродействием, хотя оно все же выше, чем у ЗУ основной памяти.