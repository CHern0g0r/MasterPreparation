\subsection{Операционные системы. Виртуальная память: MMU, TLB, таблицы страниц, аллокаторы и менеджеры виртуальной памяти.}

\subsubsection{Виртуальная память: MMU, TLB.}

\href{https://habr.com/ru/post/211150/}{Подробнее тут}

Блок управления памятью или устройство управления памятью memory management unit, MMU)~---~компонент аппаратного обеспечения компьютера, отвечающий за управление доступом к памяти, запрашиваемым центральным процессором.

Его функции заключаются в трансляции адресов виртуальной памяти в адреса физической памяти (то есть управление виртуальной памятью), защите памяти, управлении кэш-памятью, арбитражем шины и, в более простых компьютерных архитектурах (особенно 8-битных), переключением блоков памяти. 

Принцип работы современных MMU основан на разделении виртуального адресного пространства (одномерного массива адресов, используемых центральным процессором) на участки одинакового, как правило, несколько килобайт, хотя, возможно, и существенно большего, размера, равного степени 2, называемые страницами. Младшие n бит адреса (смещение внутри страницы) остаются неизменными. Старшие биты адреса представляют собой номер (виртуальной) страницы. MMU обычно преобразует номера виртуальных страниц в номера физических страниц, используя буфер ассоциативной трансляции (Translation Lookaside Buffer, TLB).

Если преобразование при помощи TLB невозможно, включается более медленный механизм преобразования, основанный на специфическом аппаратном обеспечении или на программных системных структурах. Данные в этих структурах, как правило, называются элементами таблицы страниц (page table entries (PTE)), а сами структуры — таблицами страниц (англ. page table (PT)). Конкатенация номера физической страницы со смещением внутри страницы даёт физический адрес.

Элементы PTE или TLB могут также содержать дополнительную информацию: бит признака записи в страницу ( dirty bit), время последнего доступа к странице (accessed bit), какие процессы (пользовательские (user mode) или системные (supervisor mode)) могут читать или записывать данные в страницу, необходимо ли кэшировать страницу.

\subsubsection{Таблицы страниц.}

\href{https://ru.wikipedia.org/wiki/%D0%A2%D0%B0%D0%B1%D0%BB%D0%B8%D1%86%D0%B0_%D1%81%D1%82%D1%80%D0%B0%D0%BD%D0%B8%D1%86}{Википедия}

Таблица страниц~---~это структура данных, используемая системой виртуальной памяти в операционной системе компьютера для хранения сопоставления между виртуальным адресом и физическим адресом. Виртуальные адреса используются выполняющимся процессом, в то время как физические адреса используются аппаратным обеспечением, или, более конкретно, подсистемой ОЗУ. Таблица страниц является ключевым компонентом преобразования виртуальных адресов, который необходим для доступа к данным в памяти.

\subsubsection{Аллокаторы.}

\href{https://habr.com/ru/post/505632/}{Подробнее тут}

Аллокатор или распределитель памяти в языке программирования C++ ~---~ специализированный класс, реализующий и инкапсулирующий малозначимые (с прикладной точки зрения) детали распределения и освобождения ресурсов компьютерной памяти.

Концептуально выделяется пять основных операции, которые можно осуществить над аллокатором:
\begin{enumerate}
	\item \emph{create}~---~создает аллокатор и отдает ему в распоряжение некоторый объем памяти;
	\item \emph{allocate}~---~выделяет блок определенного размера из области памяти, которым распоряжается аллокатор;
	\item \emph{deallocate}~---~освобождает определенный блок;
	\item \emph{free}~---~освобождает все выделенные блоки из памяти аллокатора (память, выделенная аллокатору, не освобождается);
	\item \emph{destroy}~---~уничтожает аллокатор с последующим освобождением памяти, выделенной аллокатору.
\end{enumerate}

\subsubsection{Менеджеры виртуальной памяти.}

Менеджер виртуальной памяти (далее просто «менеджер памяти») ~---~ часть операционной системы, благодаря которой можно адресовать память большую, чем объем физической памяти (ОЗУ).

Благодаря виртуальной памяти можно запускать множество ресурсоёмких приложений, требующих большого объёма ОЗУ. Максимальный объём виртуальной памяти, который можно получить, используя 24-битную адресацию, — 16 мегабайт. С помощью 32-битной адресации можно адресовать до 4 ГБ виртуальной памяти. А 64-битная адресация позволяет работать уже с 16 эксабайтами памяти.

Применение механизма виртуальной памяти позволяет:

\begin{enumerate}
	\item упростить адресацию памяти клиентским программным обеспечением;
	\item рационально управлять оперативной памятью компьютера (хранить в ней только активно используемые области памяти);
	\item изолировать процессы друг от друга (процесс полагает, что монопольно владеет всей памятью).
\end{enumerate}