\subsection{Операционные системы. Файловые системы (UNIX): файлы и директории, inode, контроль доступа. Файловые системы (реализация в ядре): VFS, блочные устройства, планировщик IO}

\subsubsection{Файловые системы (UNIX): файлы и директории, inode, контроль доступа.}

Все файлы, с которыми могут манипулировать пользователи, располагаются в файловой системе, представляющей собой дерево, промежуточные вершины которого соответствуют каталогам, и листья - файлам и пустым каталогам. 

Каждый каталог и файл файловой системы имеет уникальное полное имя. Каталог, являющийся корнем файловой системы, в любой файловой системе имеет предопределенное имя "/" (слэш). Полное имя файла, например, /bin/sh означает, что в корневом каталоге должно содержаться имя каталога bin, а в каталоге bin должно содержаться имя файла sh. 
Коротким или относительным именем файла (relative pathname) называется имя (возможно, составное), задающее путь к файлу от текущего рабочего каталога. 
В каждом каталоге содержатся два специальных имени, имя ".", именующее сам этот каталог, и имя "..", именующее "родительский" каталог данного каталога, т.е. каталог, непосредственно предшествующий данному в иерархии каталогов.

\textbf{Incode.} \href{https://freehost.com.ua/faq/articles/inode-v-linux--chto-eto-takoe/}{Подробнее туту}

Для ОС Linux есть такое понятие, как Inode или индексный дескриптор. Индексные дескрипторы в файловых системах (таких как ext4) предназначены для хранения метаданных о файлах, каталогах и др. объектах.

Представим иерархическую структуру файловой системы Линукс в упрощенном виде:
\begin{enumerate}
	\item верхушка иерархии — это сама файловая система;
	\item уровнем ниже идут имена файлов (папок);
	\item имена файлов ссылаются на inode;
	\item inode ссылаются на физические данные.
\end{enumerate}

Таким образом, файловая система Linux содержит блоки для хранения данных и inodes. По умолчанию, в ext4, 4092 байта — это размер одного блока. Любой файл в каталоге ОС Linux имеет имя файла и номер inode. Пользователь может узнать метаданные этого файла, указав его номер inode.

Как правило, каждый Inode хранит следующие атрибуты:
\begin{enumerate}
	\item размер;
	\item владелец;
	\item дата/время;
	\item разрешения и контроль доступа;
	\item расположение на диске;
	\item тип файла;
	\item количество ссылок;
	\item дополнительные метаданные о файле.
\end{enumerate}

Таблица с Inode размещена в начале раздела диска, после нее уже идут блоки с данными. Директории в ОС Линукс рассматриваются как Inode типа «директория», в них содержатся списки имен файлов и номера их inode.

Для ОС Линукс также важно понятие о ссылках (символические и жесткие ссылки).

Символическая ссылка — это по своей сути «ярлык», она содержит адрес файла.

Если вы попытаетесь открыть такую ссылку, то откроется соответствующий файл (папка). Если удалить данный файл (папку), символическая ссылка не удалится, но при попытке открыть ее — она приведет «в никуда». Номер Inоdе «символической ссылки» отличается от номера inоde того файла, на который она ссылается.

Если же вы используете «жесткие ссылки», то ваш конкретный файл находится только в определенном месте жесткого диска, а уже именно на это место и ведут сразу несколько ссылок. Каждая «жесткая ссылка» представлена в виде отдельного файла, однако все такого вида ссылки указывают на один и тот же участок диска (даже если мы перемещаем этот файл между разными каталогами). Жесткая ссылка в системе идет под таким же номером Inode, как и фaйл, на который она ссылается.


\subsubsection{Файловые системы (реализация в ядре): VFS, блочные устройства, планировщик IO}

\href{https://docstore.mik.ua/unix2/glava_13.htm}{Подробнее тут (про блоки)}

Файловая система обычно размещается на дисках или других устройствах внешней памяти, имеющих блочную структуру. Кроме блоков, сохраняющих каталоги и файлы, во внешней памяти поддерживается еще несколько служебных областей. 

В мире UNIX существует несколько разных видов файловых систем со своей структурой внешней памяти. Наиболее известны традиционная файловая система UNIX System V (s5) и файловая система семейства UNIX BSD (ufs). Файловая система s5 состоит из четырех секций (рисунок 2.2,a). В файловой системе ufs на логическом диске (разделе реального диска) находится последовательность секций файловой системы 

Кратко опишем суть и назначение каждой области диска:
\begin{enumerate}
	\item Boot-блок содержит программу раскрутки, которая служит для первоначального запуска ОС UNIX. В файловых системах s5 реально используется boot-блок только корневой файловой системы. В дополнительных файловых системах эта область присутствует, но не используется.
	\item Суперблок - это наиболее ответственная область файловой системы, содержащая информацию, которая необходима для работы с файловой системой в целом. Суперблок содержит список свободных блоков и свободные i-узлы (information nodes - информационные узлы). В файловых системах ufs для повышения устойчивости поддерживается несколько копий суперблока. Каждая копия суперблока имеет размер 8196 байт, и только одна копия суперблока используется при монтировании файловой системы (см. ниже). Однако, если при монтировании устанавливается, что первичная копия суперблока повреждена или не удовлетворяет критериям целостности информации, используется резервная копия. 
	\item Блок группы цилиндров содержит число i-узлов, специфицированных в списке i-узлов для данной группы цилиндров, и число блоков данных, которые связаны с этими i-узлами. Размер блока группы цилиндров зависит от размера файловой системы. Для повышения эффективности файловая система ufs старается размещать i-узлы и блоки данных в одной и той же группе цилиндров. 
	\item Список i-узлов (ilist) содержит список i-узлов, соответствующих файлам данной файловой системы. Максимальное число файлов, которые могут быть созданы в файловой системе, определяется числом доступных i-узлов. В i-узле хранится информация, описывающая файл: режимы доступа к файлу, время создания и последней модификации, идентификатор пользователя и идентификатор группы создателя файла, описание блочной структуры файла и т.д.
	\item Блоки данных - в этой части файловой системы хранятся реальные данные файлов. В случае файловой системы ufs все блоки данных одного файла пытаются разместить в одной группе цилиндров. Размер блока данных определяется при форматировании файловой системы командой mkfs и может быть установлен в 512, 1024, 2048, 4096 или 8192 байтов. 
\end{enumerate}

\textbf{Виртуальная файловая система (VFS)}~---~уровень абстракции поверх конкретной реализации файловой системы. Целью VFS является обеспечение единообразного доступа клиентских приложений к различным типам файловых систем. VFS может быть использована для доступа к локальным устройствам и файлам (fat32, ext4, ntfs), сетевым устройствам и файлам на них (nfs), а также к устройствам, не предназначенным для хранения данных (procfs[1]). VFS декларирует программный интерфейс между ядром и конкретной файловой системой, таким образом, для добавления поддержки новой файловой системы не требуется вносить изменений в ядро операционной системы.

\href{https://russianblogs.com/article/825789153/}{Тут про VFS}

\textbf{Планировщик IO.}

\href{https://xakep.ru/2014/05/11/input-out-linux-planning/}{Подробнее тут}

Планировщик IO отвечает за распределение дисковых операций по процессам.  В ранних ядрах Linux (как минимум в ядре 2.4) существовал только один планировщик — Linus Elevator. Он был слишком примитивным, и поэтому в ядре 2.6 появились еще три планировщика, часть из которых ныне уже ушла в небытие. Таким образом, сейчас в ядре существует три планировщика, а в ближайшее время, возможно, прибавится еще и четвертый:
\begin{enumerate}
	\item \emph{NOOP}~---~наиболее простой планировщик. Он банально помещает все запросы в очередь FIFO и исполняет их вне зависимости от того, пытаются ли приложения читать или писать. Планировщик этот, тем не менее, пытается объединять однотипные запросы для сокращения операций ввода/вывода.
	\item \emph{CFQ} был разработан в 2003 году. Заключается его алгоритм в следующем. Каждому процессу назначается своя очередь запросов ввода/вывода. Каждой очереди затем присваивается квант времени. Планировщик же циклически обходит все процессы и обслуживает каждый из них, пока не закончится очередь либо не истечет заданный квант времени. Если очередь закончилась раньше, чем истек выделенный для нее квант времени, планировщик подождет (по умолчанию 10 мс) и, в случае напрасного ожидания, перейдет к следующей очереди. Отмечу, что в рамках каждой очереди чтение имеет приоритет над записью.
	\item \emph{Deadline} в настоящее время является стандартным планировщиком, был разработан в 2002 году. В основе его работы, как это ясно из названия, лежит предельный срок выполнения — то есть планировщик пытается выполнить запрос в указанное время. В дополнение к обычной отсортированной очереди, которая появилась еще в Linus Elevator, в нем есть еще две очереди — на чтение и на запись. Чтение опять же более приоритетно, чем запись. Кроме того, запросы объединяются в пакеты. Пакетом называется последовательность запросов на чтение либо на запись, которая идет в сторону больших секторов («алгоритм лифта»). После его обработки планировщик смотрит, есть ли запросы на запись, которые не обслуживались длительное время, и в зависимости от этого решает, создавать ли пакет на чтение либо же на запись.
	\item \emph{BFQ (Budget Fair Queueing)}~---~относительно новый планировщик. Базируется на CFQ. Если не вдаваться в технические подробности, каждой очереди (которая, как и в CFQ, назначается попроцессно) выделяется свой «бюджет», и, если процесс интенсивно работает с диском, данный «бюджет» увеличивается.
\end{enumerate}