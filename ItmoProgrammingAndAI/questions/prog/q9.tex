\subsection{Операционные системы. Процессы: вытесняющая и кооперативная многозадачность, планировщики, многопроцессорные машины.}

\textbf{Кооперативная многозадачность}.

Тип многозадачности, при котором фоновые задачи выполняются только во время простоя основного процесса и только в том случае, если на это получено разрешение основного процесса.

Кооперативную многозадачность можно назвать многозадачностью “второй ступени” поскольку она использует более передовые методы, чем простое переключение задач, реализованное многими известными программами (например, МS-DOS shell из МS-DOS 5.0 при простом переключении активная программа получает все процессорное время, а фоновые приложения полностью замораживаются). При кооперативной многозадачности приложение может захватить фактически столько процессорного времени, сколько оно считает нужным. Все приложения делят процессорное время, периодически передавая управление следующей задаче.

\textbf{ Вытесняющая многозадачность}.

Вид многозадачности, в котором операционная система сама передает управление от одной выполняемой программы другой. Распределение процессорного времени осуществляется планировщиком процессов. Этот вид многозадачности обеспечивает более быстрый отклик на действия пользователя.

Вытесняющая многозадачность~---~это вид многозадачности при котором планирование процессов основывается на абсолютных приоритетах. Процесс с меньшим приоритетом (например пользовательская программа) может быть вытеснен при его выполнении более приоритетным процессом (например системной или диагностической программой). Иногда этот вид многозадачности называют приоритетным.

Каждая работающая программа имеет свое защищенное адресное пространство. Многопоточное выполнение отдельных задач позволяет при задержке в выполнении одного потока не останавливать задачу полностью, а работать со следующим потоком.

\textbf{Планировщик.}

\href{https://habr.com/ru/post/154609/}{Подробнее тут}

Планировщик~---~часть операционной системы, которая отвечает за (псевдо)параллельное выполнения задач, потоков, процессов. Планировщик выделяет потокам процессорное время, память, стек и прочие ресурсы. Планировщик может принудительно забирать управление у потока (например по таймеру или при появлении потока с большим приоритетом), либо просто ожидать пока поток сам явно(вызовом некой системной процедуры) или неявно(по завершении) отдаст управление планировщику.
Первый вариант работы планировщика называется реальным или вытесняющим(preemptive), второй, соответственно, не вытесняющим (non-preemptive).

\textbf{Многопроцессорные машины.}

\href{https://docstore.mik.ua/skbd/glava_10.htm}{Подробнее тут}

\href{https://ru.wikipedia.org/wiki/%D0%9C%D0%BD%D0%BE%D0%B3%D0%BE%D0%BF%D1%80%D0%BE%D1%86%D0%B5%D1%81%D1%81%D0%BE%D1%80%D0%BD%D0%BE%D1%81%D1%82%D1%8C}{В Википедии тоже хорошо написано}

Любая вычислительная система (будь то супер-ЭВМ или персональный компьютер) достигает своей наивысшей производительности благодаря использованию высокоскоростных элементов и параллельному выполнению большого числа операций. Именно возможность параллельной работы различных устройств системы (работы с перекрытием) является основой ускорения основных операций. 

Параллельные ЭВМ часто подразделяются по классификации Флинна на машины типа SIMD (Single Instruction Multiple Data - с одним потоком команд при множественном потоке данных) и MIMD (Multiple Instruction Multiple Data - с множественным потоком команд при множественном потоке данных). Как и любая другая, приведенная выше классификация несовершенна: существуют машины прямо в нее не попадающие, имеются также важные признаки, которые в этой классификации не учтены. В частности, к машинам типа SIMD часто относят векторные процессоры, хотя их высокая производительность зависит от другой формы параллелизма - конвейерной организации машины. Многопроцессорные векторные системы, типа Cray Y-MP, состоят из нескольких векторных процессоров и поэтому могут быть названы MSIMD (Multiple SIMD).

