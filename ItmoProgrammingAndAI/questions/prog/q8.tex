\subsection{Функциональное программирование. Чистые объекты. Функторы. Аппликативы. Монада. Взаимодействие с внешним миром.}

\subsubsection{Функциональное программирование. Чистые объекты.}
\textbf{Функциональное программирование}~---~парадигма программирования, в которой процесс вычисления трактуется как вычисление значений функций в математическом понимании последних (в отличие от функций как подпрограмм в процедурном программировании).

\textbf{Чистыми} называют функции, которые не имеют побочных эффектов ввода-вывода и памяти (они зависят только от своих параметров и возвращают только свой результат). Чистые функции обладают несколькими полезными свойствами, многие из которых можно использовать для оптимизации кода:
\begin{enumerate}
	\item если результат чистой функции не используется, её вызов может быть удалён без вреда для других выражений;
	\item результат вызова чистой функции может быть мемоизирован, то есть сохранён в таблице значений вместе с аргументами вызова;
	\item если нет никакой зависимости по данным между двумя чистыми функциями, то порядок их вычисления можно поменять или распараллелить (говоря иначе, вычисление чистых функций удовлетворяет принципам потокобезопасности);
	\item если весь язык не допускает побочных эффектов, то можно использовать любую политику вычисления. Это предоставляет свободу компилятору комбинировать и реорганизовывать вычисление выражений в программе (например, исключить древовидные структуры).
\end{enumerate}

\subsubsection{Функторы}

\href{https://habr.com/ru/post/183150/}{Объяснение на пальцах}

\href{http://cmc-msu-ai.github.io/haskell-course/lecture/2013/09/07/functors.html}{Подробнее про функторы}

\textbf{Функтором} называется класс типов, который декларирует единственный метод «fmap». Интуитивно, «fmap» применяет функцию a -> b к значению типа f a, чтобы получить значение типа f b. С другой стороны, можно рассматривать «fmap» как функцию высшего порядка, преобразующую «простую» функцию a -> b в «составную» функцию f a -> f b. Важно отметить, что структура значения типа f после применения «fmap» должна оставаться неизменной.

\begin{lstlisting}[language=Haskell]
	class Functor f where
	fmap :: (a -> b) -> f a -> f b
\end{lstlisting}

\subsubsection{Аппликативы.}
\href{http://cmc-msu-ai.github.io/haskell-course/lecture/2013/09/08/applicative-and-monad.html}{Подробнее тут}

Естественным продолжением класса Functor является класс \textbf{Applicative} (аппликативный функтор), определенный в модуле Control.Applicative:

\begin{lstlisting}[language=Haskell]
	class Functor f => Applicative f where
	pure  :: a -> f a
	(<*>) :: f (a -> b) -> f a -> f b
\end{lstlisting}

\textbf{Законы:}
\begin{enumerate}
	\item Помещение тождественной функции в «чистый» контекст и применение к аргументу в контексте не меняет ни значение, ни контекст.
	\begin{lstlisting}[language=Haskell]
		pure id <*> x == x
	\end{lstlisting}
	\item Применение чистой функции к чистому аргументу в контексте «по умолчанию» должно быть эквивалентно применению функции, а затем помещению результата в контекст.
	\begin{lstlisting}[language=Haskell]
		pure f <*> pure x == pure (f x)
	\end{lstlisting}
	\item При применении функции u с побочными эффектами к чистому аргументу y порядок вычисления функции и аргумента неважен.
	\begin{lstlisting}[language=Haskell]
		u <*> pure y == pure ($ y) <*> u
	\end{lstlisting}
	\item Некоторый аналог композиции для аппликативных функторов.
	\begin{lstlisting}[language=Haskell]
		u <*> (v <*> w) == pure (.) <*> u <*> v <*> w
	\end{lstlisting}
\end{enumerate} 

\subsubsection{Монады}
\href{http://cmc-msu-ai.github.io/haskell-course/lecture/2013/09/08/applicative-and-monad.html}{Подробнее тут}
\begin{lstlisting}[language=Haskell]
	class Monad m where
	return :: a -> m a
	(>>=) :: m a -> (a -> m b) -> mb
	(>>) :: m a -> m b -> m b
	m >> n = m >>= \_ -> n
	
	fail :: String -> m a
\end{lstlisting}

Функция return по типу очень напоминает функцию pure из класса Applicative. И, в действительности, return и есть pure, хоть и с не самым удачным названием (return в Haskell совсем не то же, что return в обычных императивных языках вроде C или Java). С математической точки зрения, любая монада является аппликативным функтором (но не наоборот). Но по историческим причинам,в описании класса это не указано.

Как следует из определения, операция ($>>$) является частным случаем операции ($>>=$)

Функция fail осталась в классе по историческим причинам, хотя никакого отношения к монадам реально не имеет.

\textbf{Законы:}
\begin{lstlisting}[language=Haskell]
	return a >>= k = k a
	m >>= return = m
	m >>= (\x -> k x >>= h) = (m >>= k) >>= h
	fmap f xs = xs >>= return . f = liftM f xs
\end{lstlisting}

\subsubsection{Взаимодействие с внешним миром.}
Взаимодействие с "внешним миром" (побочные эффекты вычислений) можно реализовать с помощью специальной монады IO.

Грубое приближение монады IO — это взятие пары с контекстом: "(a, RealWorld)". При операциях с монадой состояние RealWorld может меняться.

Есть операция return, погружающая объект в окружение IO («выпускающая во внешний мир»), а вот обратного преобразования нет.

\begin{lstlisting}[language=Haskell]
	putChar :: Char -> IO () 
\end{lstlisting}~---~берёт символ и возвращает новый «мир», в котором растворился (был напечатан в консоли) этот символ.

\begin{lstlisting}[language=Haskell]
	getChar :: IO Char
\end{lstlisting} --- мы можем получить символ из внешнего мира, но только внутри монады.

Достать его из монады мы не можем, но можем работать сним внутри IO с помощью $>>=$.

В процессе выполнения программы, содержащей IO, объекты типов IO a остаются временно невычисленными, как задумки.

Например, если мы где-то напишем putChar ' a' , то символ не будет тут же напечатан.

Вместо этого нужно дождаться, пока соберётся «главный» объект типа IO (), и уже при его вычислении все операции с внешним миром будут выполнены, причём в правильном порядке.