\subsection{Машинное обучение. Многослойная нейронная сеть и алгоритм обратного распространения ошибок. Методы инициализации и методы оптимизации в нейронных сетях. Архитектуры сетей, их применение}

\href{https://ru.wikipedia.org/wiki/%D0%9C%D0%B5%D1%82%D0%BE%D0%B4_%D0%BE%D0%B1%D1%80%D0%B0%D1%82%D0%BD%D0%BE%D0%B3%D0%BE_%D1%80%D0%B0%D1%81%D0%BF%D1%80%D0%BE%D1%81%D1%82%D1%80%D0%B0%D0%BD%D0%B5%D0%BD%D0%B8%D1%8F_%D0%BE%D1%88%D0%B8%D0%B1%D0%BA%D0%B8}{Метод обратного распространения ошибки}

\href{https://machinelearningmastery.ru/initialization-techniques-for-neural-networks-f4ce8e64effc/}{Методы инициализации для нейронных сетей}

\href{https://habr.com/ru/company/skillfactory/blog/552394/}{Принцип действия оптимизаторов для нейронных сетей}

\href{https://habr.com/ru/post/318970/}{Еще про оптимизацию}.

Как я понял, оптимизация для нейронных сетей - это всякие навесы на градиентный спуск.

\href{https://habr.com/ru/company/oleg-bunin/blog/340184/}{Введение в архитектуры нейронных сетей}