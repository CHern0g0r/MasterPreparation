\subsection{С++. Процесс компиляции и линковки. .cpp, .h, .i, .o файлы.}
Понятно и подробно описано: 
\href{https://habr.com/ru/post/478124/}{https://habr.com/ru/post/478124/}

Кратко и структурировано: \href{https://server.179.ru/tasks/cpp/total/105.html}{https://server.179.ru/tasks/cpp/total/105.html}

\textbf{Компиляция}~---~трансляция программы, составленной на исходном языке высокого уровня, в эквивалентную программу на низкоуровневом языке, близком машинному коду (абсолютный код, объектный модуль, иногда на язык ассемблера). Входной информацией для компилятора (исходный код) является описание алгоритма или программа на объектно-ориентированном языке, а на выходе компилятора—эквивалентное описание алгоритма на машинно-ориентированном языке (объектный код).

\subsubsection{Заголовочные файлы (.h)}

%Целью заголовочных файлов является удобное хранение набора объявлений объектов для их последующего использования в других программах. 
В языках программирования Си и C++ заголовочные файлы~---~основной способ подключить к программе типы данных, структуры, прототипы функций, перечисляемые типы и макросы, используемые в другом модуле. По умолчанию используется расширение .h; иногда для заголовочных файлов языка C++ используют расширение .hpp.

Чтобы избежать повторного включения одного и того же кода, используются директивы \#ifndef, \#define, \#endif.

Заголовочный файл в общем случае может содержать любые конструкции языка программирования, но на практике исполняемый код (за исключением inline-функций в C++) в заголовочные файлы не помещают.