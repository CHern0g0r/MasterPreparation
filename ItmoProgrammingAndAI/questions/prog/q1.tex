\subsection{Архитектура ЭВМ. Архитектура фон Неймана и гарвардская архитектура. Основные принципы и их альтернативы. Архитектура набора команд (ISA), CISC и RISC архитектуры.}

\subsubsection{Архитектура ЭВМ}
\textbf{Архитектура ЭВМ}~---~
это модель, устанавливающая принципы организации вычислительной системы, состав, 
порядок и взаимодействие основных частей ЭВМ, функциональные возможности, 
удобство эксплуатации, стоимость, надежность.

\subsubsection{Архитектура фон Неймана и гарвардская архитектура. Основные принципы и их альтернативы.}


\href{https://www.currentschoolnews.com/ru/%D0%BD%D0%BE%D0%B2%D0%BE%D1%81%D1%82%D0%B8-%D0%BE%D0%B1%D1%80%D0%B0%D0%B7%D0%BE%D0%B2%D0%B0%D0%BD%D0%B8%D1%8F/%D0%A0%D0%B0%D0%B7%D0%BD%D0%B8%D1%86%D0%B0-%D0%BC%D0%B5%D0%B6%D0%B4%D1%83-%D1%84%D0%BE%D0%BD-%D0%9D%D0%B5%D0%B9%D0%BC%D0%B0%D0%BD%D0%B0-%D0%B8-%D0%93%D0%B0%D1%80%D0%B2%D0%B0%D1%80%D0%B4%D1%81%D0%BA%D0%BE%D0%B9-%D0%B0%D1%80%D1%85%D0%B8%D1%82%D0%B5%D0%BA%D1%82%D1%83%D1%80%D1%8B/}{Подробнее тут}
\\

\textbf{Особенности архитектуры фон Неймана:}
\begin{enumerate}
	\item Архитектура фон Неймана - это теоретический проект, основанный на концепции компьютера с хранимой программой.
	\item Архитектура фон Неймана имеет только одну шину, которая используется как для извлечения инструкций, так и для передачи данных. Что еще более важно, операции должны быть запланированы, потому что они не могут быть выполнены одновременно.
	\item В архитектуре фон Неймана процессору потребовалось бы два тактовых цикла для выполнения инструкции.
	\item Архитектура фон Неймана обычно используется буквально на всех машинах, от настольных компьютеров, ноутбуков, высокопроизводительных компьютеров до рабочих станций.
\end{enumerate}

\textbf{Особенности Гарвардской aрхитектуры:}
\begin{enumerate}
	\item Гарвардская архитектура - это современная компьютерная архитектура, основанная на компьютерной модели ретранслятора Harvard Mark I.
	\item Гарвардская архитектура имеет отдельное пространство памяти для инструкций и данных, которое физически разделяет сигналы и код хранения и память данных, что, в свою очередь, позволяет получить доступ к каждой из систем памяти одновременно.
	\item В гарвардской архитектуре процессор может выполнить инструкцию за один цикл, если были установлены соответствующие планы конвейерной обработки.
	\item Гарвардская архитектура - это новая концепция, используемая специально в микроконтроллерах и цифровой обработке сигналов (DSP).
	\item Гарвардская архитектура - сложный вид архитектуры, поскольку в ней используются две шины для команд и данных, что делает разработку блока управления сравнительно более дорогой.
\end{enumerate}

\subsubsection{Архитектура набора команд (ISA), CISC и RISC архитектуры.}

\textbf{ISA}~---~архитектура набора команд, которая включает в себя систему и режимы адресации, спецификацию команд процессора, ригистры и типы данных, систему прерываний (для обработки ошибок во время вычисления)

Процессоры можно разделить по сложности набора команд:
\emph{CISC} (Complex Inst. Set Computer) vs \emph{RISC} (Reduced Inst. Set Computer).

\textbf{RISC}~---~Главная идея в поддержании небольшого набора простых и быстрых команд, под которые соптимизирован процессор. Предполагалось, что  сложные вызываются редко. Для этой архитектуры характерен фиксированная длина кодов команд, так как их проще декодировать и она не большая.

\textbf{CISC}~---~Тут у нас много команд для всяких сложных операций которые реализованы непосредственно на плате, что позволяет их ускорить, но это усложняет архитектуру и может мешать эффективной реализации простых команд, поэтому по количество операций в секунду такие процессоры проигрывают RICS. Так как количество команды большое, то характерна переменная длина кода (с Хаффман кодированием).

В современном мире по факту используется компромисс между этими подходами. Процессоры - CISC по спецификации, но внутри скорее RICS и конвертируют сложные в более простые + 0-level cache.