\subsection{Задача Коши для системы обыкновенных дифференциальных уравнений. Существование и единственность решения. Устойчивость.}

\href{https://ru.wikipedia.org/wiki/%D0%97%D0%B0%D0%B4%D0%B0%D1%87%D0%B0_%D0%9A%D0%BE%D1%88%D0%B8}{Вики}

\D{
	Обыкновенное дифференциальное уравнение (ОДУ) - дифференциальное уравнение для функции от одной переменной типа $F(x, y(x), y'(x), y''(x), \ldots, y^{(n)}(x)) = 0$. Нужно решить относительно функции $y$, т.е. найти такую функцию (или класс функций), что равенство верно в любой точке.
	
	Порядок ОДУ - максимальная производная. В примере выше - порядок $n$.
	
	Переменная при $y$ часто опускается.
}

В задаче Коши на неизвестную функцию накладывается \textbf{начальное условие}, т.е. значение в некоторой $x_0$.

Пример задачи Коши для ОДУ первого порядка:
\begin{align*}
	\begin{cases}
		y' = f(x, y)\\
		y(x_0) = y_0 
	\end{cases}
\end{align*}

Пример задачи Коши для многих функций ($y_1, y_2, \ldots, y_n$) только с первыми производными, иначе говоря \textit{Система $n$ ОДУ первого порядка}
\begin{align*}
	\begin{cases}
		y'_1 = f_1(x, y_1, y_2, \ldots, y_n)\\
		\vdots\\
		y'_n = f_n(x, y_1, y_2, \ldots, y_n)\\
		y_1(x_0) = y_{01}\\
		\vdots\\
		y_n(x_0) = y_{0n}
	\end{cases}
\end{align*}

Заметим, что здесь в правой части нет производных, т.е. эта система разрешена относительно производных (это называется нормальная форма). Еще, наверное, важно, что начальная точка одна для всех функций.

Конечно, можно сделать ОДУ порядка > 1, но будем честны, кого ебет.

\subsubsection{Существование и единственность решения}
Сформулируем для простейшей задачи Коши вида:
\begin{align*}
	\begin{cases}
		y' = f(x, y)\\
		y(x_0) = y_0 
	\end{cases}
\end{align*}

\href{http://twt.mpei.ac.ru/math/ode/odef/ODEf_04070000.html}{Формулировка.}

Основная мораль - что при определенных условиях на $f$ задача Коши имеет единственное решение в окрестности точки $x_0$. 

\subsubsection{Устойчивость}
Идея: если мы изменим начальное условие, изменится решение. Решение $\varphi(x)$ для начального условия $(x_0, y_{01})$ называется устойчивым по Ляпунову, если для любого другого решения $y$ с начальным условием $(x_0, y_{02})$ верно $\forall \varepsilon \exists \delta:\ \forall\ (x > x_0)\  (|y_{01} - y_{02}| < \delta \rightarrow |\varphi(x) - y(x)| < \varepsilon)$

\href{http://twt.mpei.ac.ru/math/ode/odef/ODEf_04100000.html}{Формулировка.} 

