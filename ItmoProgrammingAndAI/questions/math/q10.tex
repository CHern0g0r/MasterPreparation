\subsection{Вероятностные неравенства Йенсена, Маркова и Чебышёва. Правило трёх сигм. Закон больших чисел}

\subsubsection{Неравенство Йенсена}

\D {Пусть $(\Omega, \mathcal{F}, \mathbb{P})$ -- вероятностное пространство, и $X: \Omega \rightarrow \mathbb{R}$ -- определенная на нем случайная величина. Пусть также $\phi: \mathbb{R} \rightarrow \mathbb{R}$ -- выпуклая (вниз) борелевская функция. Тогда если $X, \phi(X) \in L^1 (\Omega, \mathcal{F}, \mathbb{P})$, то
	$$\phi(EX) \le E(\phi(X))$$. Также можно добавить условность}

\subsubsection{Неравенство Маркова}

\D {Пусть неотрицательная случайная величина $X: \Omega \rightarrow \mathbb{R}^{+}$ определена на вероятностном пространстве, и ее матожидание конечно. Тогда 
	
	$\mathbb{P}(X \ge a) \le \frac{EX}{a}$}

{\bf Док-во:}

Пусть неотрицательная лучайная величина $X$ имеет плотность распределения $p(x)$, тогда для $a > 0$

$EX = \int\limits_{0}^{\infty} xp(x)dx \ge \int\limits_{a}^{\infty}xp(x)dx \ge \int\limits_{a}^{\infty}ap(x)dx = a \mathbb{P}(X \ge a)$


\subsubsection{Неравенство Чебышева}

\D {Пусть случайная величина $X$ определена на вероятностном пространстве, а ее матожидание $\mu$ и дисперсия $\sigma^2$ конечны. Тогда
	
	$P(|X - \mu| \ge a) \le \frac{\sigma^2}{a^2}, a > 0$
	
	Если $a = k\sigma$, где $\sigma$ -- стандартное отклонение, $k > 0$, то получаем
	
	$P(|X - \mu| >\ge k\sigma) \le \frac{1}{k^2}$
}


\subsubsection{Правило трех сигм}

Если случайная величина распределена нормально, то абсолютная величина ее отклонения от матожидания не превосходит утроенного среднего квадратического отклонения

$P(|X - \mu| \ge 3 \sigma) \le \frac{1}{9}$


\subsubsection{Закон больших чисел}

\D {Рассмотрим последовательность независимых в совокупности случайных величин $X_1, X_2, ...$ интегрируемых по Лебегу, которые имеют одинаковые распределения, следовательно, и одинаковые матожидания. Обозначим через $\overline{X}_n$ среднее арифметическое рассматриваемых случайных величин. Оно сходится к матожиданию}

{\bf Слабый закон}

Сходится по вероятности к матожиданию 

$\overline{X}_n \rightarrow \mu$ при $n \rightarrow 0$