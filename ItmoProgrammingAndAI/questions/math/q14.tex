\subsection{Математическая логика. Понятие доказательства. Правила вывода. Теоремы Гёделя.}

\D{
	Формальная дедуктивная теория состоит из
	\begin{itemize}
		\item Алфавита
		\item Правила образований формул (чисто синтаксически)
		\item Аксиом (подмножества формул)
		\item Правил вывода (способов получать из одних формул другие)
	\end{itemize}
}

Пример - \href{https://ru.wikipedia.org/wiki/%D0%9B%D0%BE%D0%B3%D0%B8%D0%BA%D0%B0_%D0%B2%D1%8B%D1%81%D0%BA%D0%B0%D0%B7%D1%8B%D0%B2%D0%B0%D0%BD%D0%B8%D0%B9#%D0%90%D0%BA%D1%81%D0%B8%D0%BE%D0%BC%D1%8B_%D0%B8_%D0%BF%D1%80%D0%B0%D0%B2%D0%B8%D0%BB%D0%B0_%D0%B2%D1%8B%D0%B2%D0%BE%D0%B4%D0%B0_%D1%84%D0%BE%D1%80%D0%BC%D0%B0%D0%BB%D1%8C%D0%BD%D0%BE%D0%B9_%D1%81%D0%B8%D1%81%D1%82%D0%B5%D0%BC%D1%8B_%D0%BB%D0%BE%D0%B3%D0%B8%D0%BA%D0%B8_%D0%B2%D1%8B%D1%81%D0%BA%D0%B0%D0%B7%D1%8B%D0%B2%D0%B0%D0%BD%D0%B8%D0%B9}{логика исчисления высказываний}.

Самым распространенным (а в случае с логикой исчисления высказываний единственным) правилом вывода является Modus ponens:
\begin{align*}
	\dfrac{A\ A \rightarrow B}{B}
\end{align*}

\D{
Формула $F$ называется выводимой из множества формул $\Gamma$, если ее можно получить из $\Gamma$ при помощи правил вывода.
}

\D{
Формула $F$ называется доказуемой, если ее можно вывести из аксиом. 
}

Более подробно про вывод \href{http://fkn.univer.omsk.su/kursi/disc/prlogic.htm#2.9}{тут}. (или можете у меня (Антона) спросить, я вроде чета понимаю про это).

У логик есть два важных свойства:
\begin{itemize}
	\item \textbf{Непротиворечивость.} Теория, в которой множество теорем покрывает всё множество формул (все формулы являются теоремами, «истинными высказываниями»), называется противоречивой. В противном случае теория называется непротиворечивой.
	\item \textbf{Полнота.} Теория называется полной, если в ней для любого предложения $F$ выводимо либо само $F$, либо его отрицание $\neg F$. В противном случае, теория содержит недоказуемые утверждения (утверждения, которые нельзя ни доказать, ни опровергнуть средствами самой теории), и называется неполной. 
\end{itemize}

\D{
	Арифметика - система аксиом Пеано для натуральных чисел.
}

Вики про \href{https://ru.wikipedia.org/wiki/%D0%90%D0%BA%D1%81%D0%B8%D0%BE%D0%BC%D1%8B_%D0%9F%D0%B5%D0%B0%D0%BD%D0%BE}{акисомы Пеано}.

\T[Первая теорема Гёделя (О неполноте)]{
	Если арифметика непротиворечива, то она не полна. 
	
	Иначе говоря, если с помощью аксиом Пеано не выводятся вообще все утверждения о натуральных числах, то существует утверждение, которое нельзя ни доказать, ни опровергнуть. 
}
\begin{proof}
	Если вкратце, то в арифметике можно закодировать высказывание $A$ "это высказывание не выводимо в арифметике". Понятно, что, как и с парадоксом лжеца, это высказывание нельзя ни доказать (предъявить вывод), ни опровергнуть (предъявить вывод отрицания).
	
	На \href{https://ru.wikipedia.org/wiki/%D0%A2%D0%B5%D0%BE%D1%80%D0%B5%D0%BC%D0%B0_%D0%93%D1%91%D0%B4%D0%B5%D0%BB%D1%8F_%D0%BE_%D0%BD%D0%B5%D0%BF%D0%BE%D0%BB%D0%BD%D0%BE%D1%82%D0%B5}{вики} есть набросок доказательства.
\end{proof}

\T[Вторая теорема Гёделя]{
	В формальной арифметике также можно закодировать формулу $G$, говорящую "формальная арифметика непротиворечива". И если арифметика непротиворечива, то эту формулу нельзя вывести.
	
	Другими словами, изнутри теории нельзя доказать ее собственную непротиворечивость. 
	
	Эта формула является конкретным примером к первой теореме.
}
\begin{proof}
	Назовем формулу, говорящую "арифметика непротиворечива" $Con$ (эту формулу вполне можно выписать на языке арифметики). Вспомним формулу $A$ из первой теоремы, говорящую "$A$ не выводима в арифметике".
	
	Тогда первая теорема Гёделя выглядит так:
	\begin{align*}
		Con \rightarrow A
	\end{align*}
	
	Мы знаем, что это высказывание верно (и доказательство может быть превращено в формальный арифметический вывод, т.е. оно выводимо в арифметике). 
	
	Теперь, если $Con$ выводима, то мы сможем ее вывести, а затем, применив Modus ponens к ней и первой теореме Гёделя, выведем $A$, но $A$ не выводима, если арифметика непротиворечива. Таким образом, $Con$ не выводима.
\end{proof}