\subsection{Кратные, поверхностные и криволинейные интегралы. Формулы Грина, Стокса и Остроградского}


\subsubsection{Интеграль4ики}
\D{Пусть дана $f(x)$ -- функция действительной переменной. {\bf Неопределенным интегралом} функции $f(x)$, или ее первообразной, называется такая функция $F(x)$, производная которой равна $f(x)$, т.е. $F^{'}(x) = f(x)$. Обозначается $F(x) = \int f(x)dx$ }


\D{Кратным интегралом называют множество интегралов, взятых от $d > 1$, например  $$\underbrace{\int...\int f(x_1,...,x_d)dx_{1}...dx_{d}}_{d}$$}

Замечание -- кратный интеграл -- определенный интеграл, при его вычислении всегда получается число

\D{Криволинейный интеграл -- интеграл вычисляемый вдоль какой-либо прямой.
	
	Пусть $l$ -- пгладкая, без особых точек и пересечений кривая (может быть замкнутой), заданая параметрически $l: r(t)$, где $r$ -- радиус вектор, конец которого описывает кривую, а параметр $t$ направлен от начального значения $a$ к конечному значению $b$. Для интеграла второго рода направление, в котором движется параметр, определяет направление кривой $l$.
	
	Также есть скалярная или векторная функция , которая рассматривается вдоль кривой $l: f(r)$
	
	Еще есть разбиение отрезка параметризации, и разбиение кривой. Они соответствуют друг-другу (параметризация от параметра по факту сопостовляет точке из отрезка параметризации $[a, b]$ точку на прямой, и по разбиению параметризации разбивается кривая по соответствующим точкам, подробнее можно почитать на вики ссылку вставить не получилось:( )
	
	Интегральная сумма для интеграла {\bf первого рода} -- сумма вида $\sum\limits_{k=1}^{n} f(r(\xi_i))\cdot|l_k|$ где $|l_k|$ -- длина соответствующего отрезка, $\xi_i$ -- точка на соответствующем отрезке
	
	Интегральная сумма для интеграла {\bf первого рода} -- сумма вида $\sum\limits_{k=1}^{n} f(r(\xi_i))\cdot(r(t_k) = r(t_{k - 1})))$
	
	Собственно, криволинейный интеграл это интегральная сумма с $n$ устремленным в бесконечность
}

Похоже на обычный определенный интеграл, только тут мы вместо оси выравниваемся на кривую какую-то, и по факту считаем площидь криволинейного цилиндра между кривой в пространстве и ее проекцией (вроде бы, но это не точно)


\D{Пусть $\Phi$ -- гладкая, ограниченная полная поверхность. Пусть далее на $\Phi$ задана функция $f(M) = f(x, y, z)$. Рассмотрим разбиение $T$ этой поверхности на часть $\Phi_i (i=1, ..., n)$ кусочно-гладкими кривыми и на каждой такой части выберем произвольную точку $M_i(x_i, y_i, z_i)$. Вычислив значение функции в этой точке $f(M_i) = f(x_i, y_i, z_i)$ и, приняв за $\sigma_i$ площадь поверхность $\Phi_i$, рассмотрим сумму $$I\{\Phi_i, M_i\} = \sum_i f(M_i)\sigma_i$$. Тогда число I называется пределом сумм $i\{\Phi_i, M_i\}$ если $$\forall \epsilon > 0 \exists \delta > 0 \forall T: d(T) < \delta \forall \{M_i\} |I\{\Phi_i, M_i\} - I| < \epsilon$$
	Предел $I$ сумм $I\{\Phi_i, M_i\}$ при $d(T) \rightarrow 0$
	называется {\bf поверхностным интегралом первого рода} от функции $f(M)$ по поверхности $\Phi$ и обозначается $$I = \iint\limits_{\Phi} f(M)d\sigma$$} 

По сути -- берем поверхность в пространстве, а дальше как в криволинейном -- вместо отрезков оже куски пространства и т.д. получается магия какая-то.

Тут может быть полезно вспомнить про \href{https://mph.phys.spbu.ru/~budylin/meth/node15.html}{дифференциальные формы}

\subsubsection{Формула Грина}

\D{Пусть $C$ -- положительно ориентированная кусочно-гладкая замкнутая кривая на плоскости, а $D$ -- область, ограниченная кривой $C$. Если фунеции $P = P(x, y)$, $Q = Q(x, y)$ определены в области $D$ и имеют неприрывные частные производные $\frac{\partial P}{\partial y}, \frac{\partial Q}{\partial x}$, то $$\oint Pdx + Qdy = \iint\limits_{D}(\frac{\partial Q}{\partial x} - \frac{\partial P}{\partial y})dxdy$$}
{\bf Док-во и еще:} \href{https://ru.wikipedia.org/wiki/%D0%A2%D0%B5%D0%BE%D1%80%D0%B5%D0%BC%D0%B0_%D0%93%D1%80%D0%B8%D0%BD%D0%B0}{here}

\subsubsection{Формула Стокса}

\D{Пусть на ориентируемом многообразии $M$ размерности $n$ заданы положительно ориентированное ограниченное $p-$мерное подмногообразие $\sigma (1 \le p \le n)$ и дифференциальная форма $\omega$ степени $p - 1$ класса $C^1$. Тогда если граница подмногообразия $\partial \sigma$ положительно ориентированаб то $$\int\limits_{\sigma}d\omega = \int\limits_{\partial \sigma \omega}$$}

Грубо говоря взяли поверхность, и с помощью дифференциалов перешли к интегралу по границе поверхности, как-то так, но надо глубже разбираться потому что очень много определений которые надо помнить

\subsubsection{Формула Остроградского (Гаусс сосать)}

\D{Пусть теперь $\partial V$ -- кусочно-гладкая гипперповерхность $(p = n - 1)$, ограничивающая некоторую область $V$ в $n-$мерном пространстве. Тогда интеграл дивергенции (это оператор который отображает векторное поле на скалярное -- $div F = \lim\limits_{V \rightarrow 0} \frac{\Phi_F}{V}$, где $\Phi_F$ -- поток векторного поля $F$ через сферическую поверхность площадью $S$ ограничивающую объем $V$, хуита какая-то хочу объяснение на пальцах) поля по области равен потоку поля через границу области $\partial V$: $$\int\limits_{V} div F dV = \int\limits_{\partial V} F d \Sigma$$.
	
	В трехмерном пространстве $(n = 3)$ с координатами $\{x, y, z\}$ эквивалентнно $$\int\limits_{\partial V} F d \Sigma = \int\limits_{V}(\frac{\partial P}{\partial x} + \frac{\partial Q}{\partial y} + \frac{\partial R}{\partial z}) dV$$, или $$\iiint\limits_{\partial V} Pdydz + Qdzdx + Rdxdy = \iint\limits_{V} (\frac{\partial P}{\partial x} + \frac{\partial Q}{\partial y} + \frac{\partial R}{\partial z})dxdydz$$ }


Понятно что тут взяли и применили стокса на какой-то случай, но чет пиздец ребята)))