\subsection{Определители и их свойства. Системы линейных алгебраических уравнений и их исследование. Методы решения систем линейных алгебраических уравнений.}

\subsubsection{Определитель}

\D {Определитель -- скалярная величина, которая характеризует ориентированное "растяжение"\ или "сжатие" \  многомерного евклидова пространства после преобразования матрицей. Имеет смысл только для квадратных матриц. Стандартные обозначения -- $\det(A), |A|, \Delta (A)$ }

{\bf Определение через перестановки}

Для квадратной матрицы $A = (a_{ij})$ размера $n \times n$ ее определитель вычисляется по формуле $$det A = \sum\limits_{\alpha_1, \alpha_2, ..., \alpha_n}(-1)^{N(\alpha_1, \alpha_2, ..., \alpha_n)} \cdot a_{1 \alpha_1} a_{2 \alpha_2} ... a_{n \alpha_n}$$

Где суммирование проводится по всем перестановкам $\alpha_1, \alpha_2, ..., \alpha_n$ чисел $1, 2, ..., n$, а $N(\alpha_1, \alpha_2, ..., \alpha_n)$ обозначает число инверсий в перестановке $\alpha_1, ..., \alpha_n$

Таким образом в определитель входит $n!$ слагаемых.


{\bf Аксиоматическое построение}

Понятие определителя может быть введено на основе его свойств. А именно, определителем вещественной матрицы называется функция $det: \mathbb{R}^{n \times n} \rightarrow \mathbb{R}$, обладающая следующими тремя свойствами

\begin{itemize}
	\item $\det(A)$ -- кососимметрическая функция строк(столбцов) матрицы $A$, т.е. не меняется при четных перестановках аргументов
	\item $\det(A)$ -- полилинейная функция строк (столбцов) матрицы $A$
	\item $\det(E) = 1$, где $E$ -- единичная $n \times n$ матрица.
\end{itemize}

Еще свойства

\begin{enumerate}
	\item $\det E = 1$
	\item $\det cA = c^n \det A$
	\item $\det A^T = \det A$
	\item $\det(AB) = \det A \cdot \det B$
	\item $\det A^{-1} = (\det A)^{-1}$, причем матрица обратима тогда и только тогда, когда обратим ее определитель
	\item Существует ненулевое решение уравнения $AX = 0$ тогда и только тогда, когда $\det A = 0$ 
\end{enumerate}

\subsubsection{Системы линейных уравнений}

В классическом варианте коэффициенты при переменных, свободные члены и неизвестные считаются вещественными числами

Общий вид системы линейных алгебраических уравнений:

$$
{\begin{cases}a_{11}x_{1}+a_{12}x_{2}+\dots +a_{1n}x_{n}=b_{1}\\a_{21}x_{1}+a_{22}x_{2}+\dots +a_{2n}x_{n}=b_{2}\\\dots \\a_{m1}x_{1}+a_{m2}x_{2}+\dots +a_{mn}x_{n}=b_{m}\\\end{cases}}$$

где $m$ -- количество уравнений, а $n$ -- количество переменных.

Система называется {\bf однородной}, если все ее свободные члены ($b_i$) равны нулю, иначе -- {\bf неоднородной}

Система называется {\bf совместной}, если она имеет хотя бы одно решение, иначе несовместной. Решения считаются различными, если хотя бы одно из значений переменных не совпадает. Если решение одно, то система {\bf определенная}

Также есть запись в матричной форме

$
\begin{pmatrix}
	a_{11} & a_{12} & \cdots & a_{1n} \\
	a_{21} & a_{22} & \cdots & a_{2n} \\
	\vdots & \vdots & \ddots & \vdots \\
	a_{m1} & a_{m2} & \cdots & a_{mn} 
\end{pmatrix}
\begin{pmatrix}
	x_1 \\
	x_2 \\
	\vdots \\
	x_n
\end{pmatrix} 
=
\begin{pmatrix}
	b_1 \\
	b_2 \\
	\vdots \\
	b_m
\end{pmatrix}$

или $Ax=b$. Если к матрицу $A$ приписать справа столбец свободных членов, матрица будет называться расширенной.

Системы называются {\bf эквивалентными}, если множество их решений совпадает, т.е. если решение одной системы является решением другой.

Можно менять уравнения домножением на константу кроме 0, на сумму с другим уравнением, на линейную комбинацию с учетом этой. Будут получаться эквивалентные системы.

\subsubsection{Методы решения систем уравнений}

\begin{itemize}
	\item {\it Метод Гаусса}
	
	Приводим матрицу к ступенчатому виду, остались какие-то переменные. Назовем главными те, которые на диагонали, остальные -- свободные. Теперь переносим свободные через =, и присваивая им все возможные значения легко получить решения для главных, а значит для всей системы.
	
	Подробнее \href{https://ru.wikipedia.org/wiki/%D0%9C%D0%B5%D1%82%D0%BE%D0%B4_%D0%93%D0%B0%D1%83%D1%81%D1%81%D0%B0}{здесь}
	
	\item {\it Метод Гаусса-Жордана}
	
	\begin{enumerate}
		\item Выбирают первый слева столбец матрицы, в котором есть хоть одно отличное от нуля значение.
		\item Если самое верхнее число в этом столбце ноль, то меняют всю первую строку матрицы с другой строкой матрицы, где в этой колонке нет нуля.
		\item Все элементы первой строки делят на верхний элемент выбранного столбца.
		\item Из оставшихся строк вычитают первую строку, умноженную на первый элемент соответствующей строки, с целью получить первым элементом каждой строки (кроме первой) ноль.
		\item Далее проводят такую же процедуру с матрицей, получающейся из исходной матрицы после вычёркивания первой строки и первого столбца.
		\item После повторения этой процедуры $(n-1)$ раз получают верхнюю треугольную матрицу
		\item Вычитают из предпоследней строки последнюю строку, умноженную на соответствующий коэффициент, с тем, чтобы в предпоследней строке осталась только 1 на главной диагонали.
		\item Повторяют предыдущий шаг для последующих строк. В итоге получают единичную матрицу и решение на месте свободного вектора (с ним необходимо проводить все те же преобразования).
		
	\end{enumerate}
	
	Короче приводим к единичной, что осталось у свободных членов и есть решение
	
	\item {\it Метод Крамера}
	
	Для системы $n$ линейных уравнений с $n$ неизвестными (над произвольным полем)
	
	$\begin{cases}a_{11}x_{1}+a_{12}x_{2}+\ldots +a_{1n}x_{n}=b_{1}\\a_{21}x_{1}+a_{22}x_{2}+\ldots +a_{2n}x_{n}=b_{2}\\\cdots \cdots \cdots \cdots \cdots \cdots \cdots \cdots \cdots \cdots \\a_{n1}x_{1}+a_{n2}x_{2}+\ldots +a_{nn}x_{n}=b_{n}\\
	\end{cases}$
	
	с определителем матрицы системы $\Delta$ , отличным от нуля, решение записывается в виде
	
	$ x_{i}={\frac {1}{\Delta }}{\begin{vmatrix}a_{11}&\ldots &a_{1,i-1}&b_{1}&a_{1,i+1}&\ldots &a_{1n}\\a_{21}&\ldots &a_{2,i-1}&b_{2}&a_{2,i+1}&\ldots &a_{2n}\\\ldots &\ldots &\ldots &\ldots &\ldots &\ldots &\ldots \\a_{n-1,1}&\ldots &a_{n-1,i-1}&b_{n-1}&a_{n-1,i+1}&\ldots &a_{n-1,n}\\a_{n1}&\ldots &a_{n,i-1}&b_{n}&a_{n,i+1}&\ldots &a_{nn}\\\end{vmatrix}}$
	(i-ый столбец матрицы системы заменяется столбцом свободных членов).
	
	Подставляем вместо соответствующего столбца свободные члены, считаем определитель, делим на определитель всей матрицы, и получаем $x_i$ соответствующий данному столбцу.
	
	\item {\it Матричный метод}
	
	Есть система вида $AX=B$, тогда решением будет $X=A^{-1}B$
	
	Чтобы работало, нужно чтобы матрица $A$ была невырождена, т.е. чтобы определитель был не равен 0
	
\end{itemize}

Еще есть какие-то итерационные и другие методы, но они звучат и выглядят не очень полезными, но можно посмотреть \href{https://ru.wikipedia.org/wiki/%D0%A1%D0%B8%D1%81%D1%82%D0%B5%D0%BC%D0%B0_%D0%BB%D0%B8%D0%BD%D0%B5%D0%B9%D0%BD%D1%8B%D1%85_%D0%B0%D0%BB%D0%B3%D0%B5%D0%B1%D1%80%D0%B0%D0%B8%D1%87%D0%B5%D1%81%D0%BA%D0%B8%D1%85_%D1%83%D1%80%D0%B0%D0%B2%D0%BD%D0%B5%D0%BD%D0%B8%D0%B9#:~:text=%D0%A1%D0%B8%D1%81%D1%82%D0%B5%D0%BC%D0%B0%20%D0%BB%D0%B8%D0%BD%D0%B5%D0%B9%D0%BD%D1%8B%D1%85%20%D0%B0%D0%BB%D0%B3%D0%B5%D0%B1%D1%80%D0%B0%D0%B8%D1%87%D0%B5%D1%81%D0%BA%D0%B8%D1%85%20%D1%83%D1%80%D0%B0%D0%B2%D0%BD%D0%B5%D0%BD%D0%B8%D0%B9%20(%D0%BB%D0%B8%D0%BD%D0%B5%D0%B9%D0%BD%D0%B0%D1%8F,%D0%BB%D0%B8%D0%BD%D0%B5%D0%B9%D0%BD%D1%8B%D0%BC%20%E2%80%94%20%D0%B0%D0%BB%D0%B3%D0%B5%D0%B1%D1%80%D0%B0%D0%B8%D1%87%D0%B5%D1%81%D0%BA%D0%B8%D0%BC%20%D1%83%D1%80%D0%B0%D0%B2%D0%BD%D0%B5%D0%BD%D0%B8%D0%B5%D0%BC%20%D0%BF%D0%B5%D1%80%D0%B2%D0%BE%D0%B9%20%D1%81%D1%82%D0%B5%D0%BF%D0%B5%D0%BD%D0%B8}{тут}
