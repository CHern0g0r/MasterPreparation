\subsection{Марковские цепи, Эргодические цепи, Регулярные цепи. Алгоритм Витерби.}

\href{https://neerc.ifmo.ru/wiki/index.php?title=%D0%9C%D0%B0%D1%80%D0%BA%D0%BE%D0%B2%D1%81%D0%BA%D0%B0%D1%8F_%D1%86%D0%B5%D0%BF%D1%8C}{Марковская цепь}

На состояниях Марковской цепи можно построить ориентированный граф.
\D{
	Эргодическая цепь Маркова - цепь, имеющая сильно связный граф. 
}
\href{https://neerc.ifmo.ru/wiki/index.php?title=%D0%AD%D1%80%D0%B3%D0%BE%D0%B4%D0%B8%D1%87%D0%B5%D1%81%D0%BA%D0%B0%D1%8F_%D0%BC%D0%B0%D1%80%D0%BA%D0%BE%D0%B2%D1%81%D0%BA%D0%B0%D1%8F_%D1%86%D0%B5%D0%BF%D1%8C}{Эргодическая цепь}

\D{
	Цепь регулярна тогда и только тогда, когда существует такое $n$, что в матрице $P^n$ все элементы ненулевые, то есть из любого состояния можно перейти в любое за $n$ переходов.
}
\href{https://neerc.ifmo.ru/wiki/index.php?title=%D0%A0%D0%B5%D0%B3%D1%83%D0%BB%D1%8F%D1%80%D0%BD%D0%B0%D1%8F_%D0%BC%D0%B0%D1%80%D0%BA%D0%BE%D0%B2%D1%81%D0%BA%D0%B0%D1%8F_%D1%86%D0%B5%D0%BF%D1%8C}{Регулярная цепь}.

\subsubsection{Алгоритм Витерби}
\href{https://neerc.ifmo.ru/wiki/index.php?title=%D0%A1%D0%BA%D1%80%D1%8B%D1%82%D1%8B%D0%B5_%D0%9C%D0%B0%D1%80%D0%BA%D0%BE%D0%B2%D1%81%D0%BA%D0%B8%D0%B5_%D0%BC%D0%BE%D0%B4%D0%B5%D0%BB%D0%B8}{Скрытая Марковская цепь}

\href{https://neerc.ifmo.ru/wiki/index.php?title=%D0%90%D0%BB%D0%B3%D0%BE%D1%80%D0%B8%D1%82%D0%BC_%D0%92%D0%B8%D1%82%D0%B5%D1%80%D0%B1%D0%B8}{Алгоритм Витерби}