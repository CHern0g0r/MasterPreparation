\subsection{Непрерывные случайные величины и их функции распределения. Математическое ожидание и дисперсия. Стандартные непрерывные распределения (равномерное, показательное, нормальное).}

\subsubsection{Непрерывные случайные величины}

\D { Непрерывной случайной величиной называют случайную величину, которая в результате испытания принимает все значения из некоторого числового промежутка. Число возможных значений непрерывной случайной величины бесконечно.}

$F_X(x) = P(X \le x)$


В непрерывном случае $\P(X = x) = 0 \forall x \in \mathbb{R}$
и $F_X(x - 0) = F_X(x), \forall x \in \mathbb{R}$

А следовательно формулы имеют вид $P(X \in |a, b|) = F_X(b) - F_X(a)$



Матожидание для непрерывных величин считается по формуле $EX = \int\limits_{-\infty}^{+\infty}xf(x)dx$, где $f(x)$ -- функция плотности распределения (штука от которой интеграл равен 1, говорит с какой вероятностью выпадет то или иное значение (или отношение для разных значений в непрерывном случае))


Дисперсия же вычисляется по формуле $DX = \int\limits_{-\infty}^{+\infty} x^2 f(x) dx - (EX)^2$

\subsubsection{Стандартные равномерные распределения}

{\bf Равномерное распределение}

Распределение, которое принимает значения из некоторого промежутка конечной длины, при этом плотность вероятности на этом промежутке почти всюду постоянна.

Т.е. с равной вероятностью может выпасть любое значение из промежутка.

Плотность имеет вид $f_X(x) = \begin{cases}
	\frac{1}{b - a}, x \in [a, b]\\
	0, x \not\in [a, b]
\end{cases}$

{\bf Показательное (экспоненциальное) распределение}

Случайная величина имеет экспоненциальное распределение с параметром $\lambda > 0$, если ее плотность вероятности имеет вид

$f_X(x) = \begin{cases}
	\lambda e^{-\lambda x}, x \ge 0\\
	0, x < 0
\end{cases}$

{\bf Нормальное распределение}


Плотность вероятности $f(x) = \frac{1}{\sigma \sqrt{2 \pi}}e^{-\frac{1}{2}(\frac{x - \mu}{\sigma})^2}$

где $\mu$ -- матожидание

$\sigma$ -- среднеквадратическоее отклонение

Данное распределение моделирует ситуацию когда есть какое-то среднее и какая-то дисперсия, Наибольшая вероятность получить среднее, в зависимости от дисперсии получаем значения дальше или ближе к среднему.