\subsection{Числовые ряды. Абсолютная и условная сходимость. Признаки сходимости числовых рядов.}

\subsubsection{Числовые ряды}

$\sum\limits_{k=1}^{\infty} a_{k} = a_{1} + a_{2} + a_{3} + \cdots$ -- числовой ряд

Сходимость ряда означает существование конечной суммы, т.е. $\sum\limits_{k=1}^{\infty} a_{k} = S$ где $S$ -- конечное число, иначе ряд считается расходящимся.

\subsubsection{Абсолютная и условная сходимость}

Ряд $\sum\limits_{k=1}^{\infty} a_{k}$ называется {\bf абсолютно} сходящимся, если сходится ряд из модулей $\sum\limits_{k=1}^{\infty} |a_{k}|$, иначе ряд называется {\bf условно} сходящимся

\subsubsection{Признаки сходимости числовых рядов} 

{\bf Знакоположительные ряды} (ряды с положительными членами):

Критерий сходимости знакоположительных рядов-- знакоположительный ряд $\sum\limits_{k=1}^{\infty} a_{k}$ сходится тогда и только тогда, когда последовательность его частичных сумм $S(n) = \sum\limits_{k=1}^{k=n}a_{k}$ ограничена сверху

{\bf Док-во:}

=>: ряд сходится, значит последовательность частичных сумм $\S(n) =\sum\limits_{k=1}^{n} a_{k}$ имеет предел равный $\sum\limits_{k=1}^{\infty} a_{k} = S$

<=: Пусть дан положительный ряд и последовательность частичных сумм ограничена сверху, заметим что последовательность частичных сумм неубывающая:
$$S_{n + 1} - S_{n} = a_{n + 1} \ge 0$$. Используя свойство из теоремы о монотонной последовательности получаем, что т.к. последовательность частичных сумм монотонно не убывает и ограничена сверху, значит она сходится и потому ряд сходится по определению.

{\bf Признак сравнения с мажорантой}

Пусть даны два положительных ряда $\sum\limits_{k=1}^{\infty} a_{k}$ и $\sum\limits_{k=1}^{\infty} b_{k}$. Если начиная с некоторого номера $n > N$ выполняется неравенство $0 \le a_n \le b_n$, то:

\begin{itemize}
	\item из сходимости рядя $\sum\limits_{k=1}^{\infty} b_{k}$ следует сходимость ряда $\sum\limits_{k=1}^{\infty} a_{k}$
	\item из расходимости ряда $\sum\limits_{k=1}^{\infty} a_{k}$ следует расходимость $\sum\limits_{k=1}^{\infty} b_{k}$
\end{itemize}

{\bf Док-во:}

Из неравенств на члены следует неравенство на частичные суммы $0 \le S_n \le \sigma_n$, дальше очев.


{\bf Признак Раабе}

Если для ряда $\sum\limits_{k=1}^{\infty} a_{k}$ существует предел $$R = \lim\limits_{n \rightarrow \infty} n (\frac{a_n}{a_{n+1}} - 1)$$, то при $R > 1$ ряд сходится, а при $R < 1$ -- расходится. Если $R = 1$, то жанный признак не говорит ничего.

{\bf Признак Гаусса}

Пусть для знакоположительного ряда $\sum\limits_{n=1}^{\infty} a_{n}$ отношение $\frac{a_n}{a_{n + 1}}$ может быть представлено в виде $$\frac{a_n}{a_{n + 1}} = \lambda + \frac{\mu}{n} + \frac{\theta_n}{n^2}$$, где $\lambda, \mu$ -- постоянные, а последовательность $\theta_n$ ограничена. Тогда 
\begin{itemize}
	\item ряд расходится если либо $\lambda > 1$, либо $\lambda = 1, \mu > 1$
	\item ряд расходится, если либо $\lambda < 1$, либо $\lambda = 1, \mu \le 1$
\end{itemize}


{\bf Знакопеременные ряды}

\D{Знакопеременными называются ряды, члены которых могут (стоять) быть как положительными, так и отрицательными.}


{\bf Признак Даламбера}

Слабее признака Коши, но зато проще

Если существует $\lim\limits_{n \rightarrow \infty}|\frac{a_{n + 1}}{a_n}| = r$, то 

\begin{itemize}
	\item если $r < 1$, то ряд абсолютно сходится
	\item если $r > 1$, то ряд расходится
	\item если $r = 1$, то данный признак ничего не говорит (сука)
\end{itemize}

{\bf Док-во:}

1. Пусть начиная с некоторого номера N верно неравенство $|\frac{a_{n+1}}{a_n}| \le q, 0 < q < 1$. Тогда перемножив члены начиная с N будем иметь что $\frac{a_{N+n}}{a_N} \le q^n$ откуда $|a_{N+n}| \le |a_{N}q^n|$, значит ряд $|a_{N+1}| + |a_{N+2}| + ...$ меньше бесконечной суммы убывающей геометрической прогрессии, поэтому он сходится

2. $|\frac{a_{n + 1}}{a_n}| \ge 1$ (с некоторого N), тогда можно записать $|a_{n+1}| \ge |a_n|$ значит модуль членов $a$ не стремится к 0 на бесконечности, значит последовательность не стремится к 0 а значит ряд не сходится.

3. Если просто меньше 1 до там хуйня какая-то мне впадлу
\\

{\bf Радикальный признак Коши} (ебаная оппозиция)

Если существует $\lim\lim\limits_{n \rightarrow \infty} \sqrt[n]{|a_n|} = r$, то

\begin{itemize}
	\item если $r < 1$ то ряд сходится абсолютно
	\item если $r > 1$ то ряд расходится
	\item если $r = 1$ то хз (опять??)
\end{itemize}

{\bf Док-во:} \href{https://ru.wikipedia.org/wiki/%D0%A0%D0%B0%D0%B4%D0%B8%D0%BA%D0%B0%D0%BB%D1%8C%D0%BD%D1%8B%D0%B9_%D0%BF%D1%80%D0%B8%D0%B7%D0%BD%D0%B0%D0%BA_%D0%9A%D0%BE%D1%88%D0%B8}{тут}
\\

{\bf Признак Лейбница}

Пусть для знакочередующегося ряда $$S = \sum\limits_{n=1}^{\infty}(-1)^{n-1}a_n, a_n \ge 0$$
выполняются следующие условия

\begin{itemize}
	\item С некоторого $N$ последовательность $a$ монотонно убывает, т.е. $a_{n+1} \le a_n$
	\item $\lim\limits_{n \rightarrow \infty}a_n = 0$
\end{itemize}

Тогда такой ряд сходится

{\bf Док-во:} \href{https://ru.wikipedia.org/wiki/%D0%A2%D0%B5%D0%BE%D1%80%D0%B5%D0%BC%D0%B0_%D0%9B%D0%B5%D0%B9%D0%B1%D0%BD%D0%B8%D1%86%D0%B0_%D0%BE_%D1%81%D1%85%D0%BE%D0%B4%D0%B8%D0%BC%D0%BE%D1%81%D1%82%D0%B8_%D0%B7%D0%BD%D0%B0%D0%BA%D0%BE%D1%87%D0%B5%D1%80%D0%B5%D0%B4%D1%83%D1%8E%D1%89%D0%B8%D1%85%D1%81%D1%8F_%D1%80%D1%8F%D0%B4%D0%BE%D0%B2}{здесь}\\

{\bf Признак Абеля}

\T {Числовой ряд $\sum\limits_{n=1}^{\infty}a_nb_n$ сходится, если выполнены следующие условия
	
	\begin{itemize}
		\item Последовательность \{$a_n$\} монотонна и ограничена
		\item Ряд $\sum\limits_{n=1}^{\infty}b_n$ сходится
	\end{itemize}
}
{\bf Proof:} \href{https://ib.mazurok.com/2015/06/16/%D0%BF%D1%80%D0%B8%D0%B7%D0%BD%D0%B0%D0%BA%D0%B8-%D0%B0%D0%B1%D0%B5%D0%BB%D1%8F-%D0%B8-%D0%B4%D0%B8%D1%80%D0%B8%D1%85%D0%BB%D0%B5/}{вот}\\

{\bf Признак Дирихле}

\T{Пусть выполнены условия:
	\begin{itemize}
		\item последовательность частичных сумм $B_n = \sum\limits_{k=1}^{n}$ ограничена
		\item последовательность $a_n$, начиная с некоторого номера, монотонно убывает $a_n \ge a_{n+1}$
		\item $\lim\limits_{n\rightarrow\infty}a_n = 0$
	\end{itemize}
	Тогда ряд $\sum\limits_{n=1}^{\infty}a_nb_b$ сходится
}

{\bf Proof:} \href{https://ib.mazurok.com/2015/06/16/%D0%BF%D1%80%D0%B8%D0%B7%D0%BD%D0%B0%D0%BA%D0%B8-%D0%B0%D0%B1%D0%B5%D0%BB%D1%8F-%D0%B8-%D0%B4%D0%B8%D1%80%D0%B8%D1%85%D0%BB%D0%B5/}{вот}\\