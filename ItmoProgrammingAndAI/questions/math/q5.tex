\subsection{Линейные операторы в конечномерном пространстве и их матричное представление. Характеристический многочлен, собственные числа и собственные вектора линейного оператора. Сопряженные и самосопряженные операторы.}

\subsubsection{ Линейные операторы}

\D {Пусть $X$ и $Y$ -- линейные пространства над полем $F$. Отображение $\mathcal{A}: X \rightarrow Y$ называется линейным оператором, если $\forall x_1, x_2 \in X, \forall \lambda \in F$
	\begin{itemize}
		\item $\mathcal{A}(x_1 + x_2) = \mathcal{A}(x_1) + \mathcal{A}(x_2)$
		\item $\mathcal{A}(\lambda \cdot x_1) = \lambda \cdot \mathcal{A}(x_1)$
	\end{itemize}
	
	Линейный оператор $\mathcal{A}: X \rightarrow X$ называется автоморфизмом (или гомоморфизмом)
	
}

Операторы равны, если переводят элементы первого пространства в одинаковые элементы второго пространства.

\D{Пусть $\mathcal{A}: X \rightarrow Y$
	
	Пусть п.п. $X \leftrightarrow \{e_k\}_{k=1}^{n},\ \dim X = n$
	
	Пусть п.п. $Y \leftrightarrow \{h_k\}_{k=1}^{n},\ \dim Y = m$
	
	$\underset{{1\le k \le n}}{\mathcal{A}e_k} = \sum\limits_{i = 1}^{m} \alpha_k^i \cdot h_i \Rightarrow A = ||\alpha_k^i||,$ где $1 \le i \le m, 1 \le k \le m$
	
	$A = \begin{pmatrix}
		\alpha_1^1 & ... & \alpha_n^1\\
		\alpha_1^2 & ... & \alpha_n^2\\
		... & ... & ...\\
		\alpha_1^n & ... & \alpha_n^n\\
		
	\end{pmatrix}$
	
}

\subsubsection{Характеристический многочлен}

\D {Для данной матрицы $A$, $\chi(\lambda) = \det (A - \lambda E)$, где $E$ -- единичная матрица, является многочленом от $\lambda$, который называется {\bf характеристическим многочленом} матрицы $A$ (видимо можно отождествить матрицу с линейным оператором, тогда будет многочлен для оператора)}

Ценность характеристического многочлена в том, то собственные значения матрицы являются его корнями. Действительно, если уравнение $Av = \lambda v$ имеет ненулевое решение, то $(A - \lambda E)v = 0$, значит матрица $A - \lambda E$ вырождена и ее определитель $\det (A - \lambda E) = \chi(\lambda)$ равен 0

{\bf Свойства}

\begin{itemize}
	\item Для матрицы $n \times n$ характеристический многочлен имеет степень $n$
	\item Все корни характеристического многочлена матрицы являются ее собственными значениями
	\item Теорема Гамильтона-Кэли -- если $\chi(\lambda)$ -- характеристический многочлен матрицы $A$, то $\chi(A) = 0$
	\item Характеристические многочлены подобных матриц совпадают
	\item Характеристический многочлен обратной матрицы $\chi_{A^{-1}}(\lambda) = \frac{(-\lambda)^n}{\det A}\chi_A(1/\lambda)$
	\item Если $A$ и $B$ две матрицы $n \times n$, то $\chi_{AB} = \chi_{BA}$. В частности $tr(AB) = tr(BA), \det(AB) = \det (BA)$
	\item В более общем виде, если $A$ --матрица $m \times n$, а $B$ -- матрица $n \times m$, причем $m < n$, так что $AB$ и $BA$ --квадратные матрицы размеров $m$ и $n$ соответственно, то $\chi_{BA}(\lambda) = \lambda ^{n - m}\chi_{AB}(\lambda)$ 
\end{itemize}


\D{Пусть $L$ -- линейное пространство над полем $K$,
	
	$\mathcal{A}:L \rightarrow L$ -- линейный оператор
	
	{\bf Собственным вектором} линейного оператора $\mathcal{A}$ называется такой ненулевой вектор $x \in L$, что для некоторого $\lambda \in K: \mathcal{A}x = \lambda x$
	
	При этом $\lambda$ называют {\bf собственным числом} оператора $\mathcal{A}$
	
}

{\bf Свойства}

\begin{itemize}
	\item Собственные векторы, отвечающие различным собственным значениям, образуют ЛНЗ набор
	\item Еще какие-то леммы есть, подробнее см на \href{https://neerc.ifmo.ru/wiki/index.php?title=%D0%A1%D0%BE%D0%B1%D1%81%D1%82%D0%B2%D0%B5%D0%BD%D0%BD%D1%8B%D0%B5_%D0%B2%D0%B5%D0%BA%D1%82%D0%BE%D1%80%D1%8B_%D0%B8_%D1%81%D0%BE%D0%B1%D1%81%D1%82%D0%B2%D0%B5%D0%BD%D0%BD%D1%8B%D0%B5_%D0%B7%D0%BD%D0%B0%D1%87%D0%B5%D0%BD%D0%B8%D1%8F}{говне}
\end{itemize}

\subsubsection{Сопряженные и самосопряженные операторы}

\D{Пусть $E, L$ -- линейные пространства, а $E^*, L^*$ -- сопряженные линейные пространства (пространства линейных функционалов, определенных на $E$ и $L$). Тогда для любого линейного оператора $\mathcal{A}: E \rightarrow L$ и любого линейного функционала $g \in L^*$ определен линейный функционал $F \in E^*$ -- суперпозиция $g$ и $A: f(x) = g(A(x))$. Отображение $g \rightarrow f$ называется сопряженным линейным оператором и обозначается $\mathcal{A^*}: L^* \rightarrow E^*$. Если кратко, то $(\mathcal{A^*}g, x) = (g, \mathcal{A}x)$
	
	Если же $\mathcal{A^*} = \mathcal{A}$, то такой оператор называется самосопряженным, для него $(\mathcal{A}x, y) = (x, \mathcal{A}y)$
	
}