\subsection{Градиентные методы. Метод сопряжения градиентов. Минимизация квадратичных функций. Метод Ньютона.}

\href{https://ru.wikipedia.org/wiki/%D0%93%D1%80%D0%B0%D0%B4%D0%B8%D0%B5%D0%BD%D1%82%D0%BD%D1%8B%D0%B5_%D0%BC%D0%B5%D1%82%D0%BE%D0%B4%D1%8B}{Градиентные методы}

\href{http://www.machinelearning.ru/wiki/index.php?title=%D0%9C%D0%B5%D1%82%D0%BE%D0%B4_%D1%81%D0%BE%D0%BF%D1%80%D1%8F%D0%B6%D1%91%D0%BD%D0%BD%D1%8B%D1%85_%D0%B3%D1%80%D0%B0%D0%B4%D0%B8%D0%B5%D0%BD%D1%82%D0%BE%D0%B2}{Метод сопряженных градиентов}

Вроде для минимизации квадратичных функций применяются все те же градиентные методы.

\D{
	Квадратичная функция - функция вида $F(x) = x^TAx + x^Tb + c$, где $A$ - симметричная матрица. (все это векторы, умножение скалярное, в итоге получается число) 
}

\href{https://ru.wikipedia.org/wiki/%D0%9C%D0%B5%D1%82%D0%BE%D0%B4_%D0%9D%D1%8C%D1%8E%D1%82%D0%BE%D0%BD%D0%B0}{Метод Ньютона}

\href{https://ru.algorithmica.org/cs/numerical/newton/}{Вот здесь} про Ньютона приятнее, чем на Вики

