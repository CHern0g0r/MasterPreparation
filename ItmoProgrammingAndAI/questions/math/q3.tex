\subsection{Функциональные ряды, свойства равномерно сходящихся функциональных рядов. Степенные ряды. Ряд Тейлора.}

\subsubsection{Функциональные ряды}

\D {Функциональный ряд -- ряд, каждым членом которого является функция $u_k(x)$
	
	Обозначается $\sum\limits_{k=1}^{\infty} u_k(x)$}

Функциональная последовательность $u_k(x)$ сходится {\bf поточечно} к функции $u(x)$, если $\forall x \in E \exists \lim\limits_{k \rightarrow \infty} u_k(x) = u(x)$

{\bf Равномерная сходимость} -- существует функция $u(x): E \rightarrow \mathbb{C}$ такая, что 

$sup |u_k(x) - u(x)| \xrightarrow {k \rightarrow \infty} 0, x \in E$

Функциональный ряд называется сходящимся {\bf поточечно}, если последовательность $S_n(x) = \sum\limits_{k=1}^{n} u_k(n)$ сходится поточечно. Аналогично для равномерной сходимости.

{\bf Необходимое условие равноменой сходимости ряда}

$u_k(x) \rightrightarrows 0$ при $k \rightarrow \infty$

Или, что эквивалентно $\forall \epsilon > 0 \exists n_0(\epsilon) \in \mathbb{N} : \forall x \in X, \forall n > n_0 |u_n(x)| < \epsilon$, где $X$ -- область сходимости

{\bf Свойства}

\begin{enumerate}
	\item {\bf Теоремы о непрерывности}
	
	Последовательность непрерывных в точке функций сходится к функции, непрерывной в этой точке.
	
	Последовательность $u_k(x) \rightrightarrows u(x)$
	
	$\forall k:$ функция $u_k(x)$ непрерывна в точке $x_0$
	
	Тогда и $u(x)$ непрерывна в $x_0$
	
	Ряд непрерывных в точке функций сходится к функции, непрерывной в этой точке.
	
	Ряд $\sum\limits_{k=0}^{\infty}u_k(x) \rightrightarrows S(x)$
	
	$\forall k$: функция непрерывна в точкке $x_0$
	
	Тогда $S(x)$ непрерывна в  $x_0$
	
	\item {\bf Теоремы об интегрировании}
	
	Рассматриваются действительнозначные функции на отрезке действительной оси
	
	{\it Теорема о переходе к пределу под знаком интеграла}
	
	$\forall k:$ функция $u_k(x)$ непрерывна на отрезке $[a, b]$
	
	$u_k(x) \rightrightarrows u(x)$ на $[a, b]$
	
	Тогда числовая последовательность $\{\int\limits_{a}^{b} u_k(x) dx\}$ сходится к конечному пределу $\int\limits_a^b u(x) dx$
	
	{\it Теорема о почленном интегрировании}
	
	$\forall k:$ функция $u_k(x)$ непрерывна на отрезке $[a, b]$
	
	$\sum\limits_{k=1}^{\infty}u_k(x) \rightrightarrows S(x)$ на $[a, b]$
	
	Тогда числовой ряд $\sum\limits_{k=1}^{\infty}\int\limits_{a}^{b} u_k(x) dx$ сходится и равен $\int\limits_a^b S(x) dx$
	
	\item {\bf Теоремы о дифференцировании}
	
	Рассматриваются действительнозначные функции на отрезке действительной оси
	
	{\it Теорема о дифференцировании под пределом}
	
	$\forall k:$ функция $u_k(x)$ дифференцируема (имеет непрерывную производную) на отрезке $[a, b]$
	
	$\exists c \in [a, b]: u_k(c)$ сходится к конечному пределу
	
	$u_k^{\prime}(x)  \rightrightarrows \omega(x)$ на отрезке $[a, b]$
	
	Тогда $\exists u(x): u_k(x) \rightrightarrows u(x),\ u(x)$ -- дифференцируема на $[a, b],\ u^{\prime}(x) = \omega(x)$ на $[a, b]$
	
	{\it Теорема о почленном дифференцировании}
	
	$\forall k:$ функция $u_k(x)$ -- дифференцируема на отрезке $[a, b]$
	
	$\exists c \in [a, b]: \sum\limits_{k=1}^{\infty} u_k(c)$ сходится
	
	$\sum\limits_{k=1}^{\infty}u_k^{\prime}(x)$ равномерно сходится на отрезке $[a, b]$
	
	Тогда $\exists S(x): \sum\limits_{k=1}^{\infty}u_k(x) \rightrightarrows S(x),\ S(x)$ -- дифференцируем на $[a, b], S^{\prime}(x) = \sum\limits_{k=1}^{\infty}u_k^{\prime}(x)$ на $[a, b]$
	
\end{enumerate}

\subsubsection{Степенные ряды}

\D {{\bf Степенной ряд с одной переменной} -- это формальное алгебраическое вырадение вида $$F(x) = \sum\limits_{n=0}^{\infty}a_nX^n$$ в котором коэффициенты $a_n$ берутся из некоторого кольца $R$, обычно вещественные или комплексные числа}

Для степенных рядов есть несколько теорем об их сходимости

\begin{itemize}
	\item Певая теорема Абеля
	
	Пусть ряд $\sum a_n x^n$ сходится в точке $x_0$. Тогда этот ряд сходится абсолютно в круге $|x| < |x_0|$ и равномерно по $x$ на любом компактном подмножестве этого круга.
	
	Отсюда можно сделать вывод что если ряд расходится при $x = x_0$, то он расходится при всех $|x| > |x_0|$
	
	Появляется понятие радиуса сходимости $R$, при котором при $|x| < R$ ряд сходится абсолютно, про $|x| > R$ расходится
	
	\item Формула Коши-Адамара (Коши-Амидамару)
	
	Значение радиуса сходимости степенного ряда может быть вычислено по формуле $\frac{1}{R} = \uplim\limits_{n \rightarrow +\infty}|a_n|^{1/n}$
	
	\item Признак Даламбера
	
	Если при $n > N$ и $\alpha > 1$ выполнено неравенство $|\frac{a_n}{a_{n+1}}| \ge R(1 + \frac{\alpha}{n})$ тогда степенной ряд $\sum a_n x^n$ сходится во всех точках окружности $|x| = R$ абсолютно и равномерно по $x$
	
	\item Признак Дирихле
	
	Если все коэффициенты степенного ряда $\sum a_n x^n$ положительны и последовательность $a_n$ монотонно сходится к 0б тогда этот ряд сходится во всех точках окружности $|x| = 1$, кроме, может быть, точки $x = 1$
\end{itemize}


\subsubsection{Ряд Тейлора}

\D {Ряд Тейлора -- разложение функции в бесконечную сумму степенных функций
	
	Многочленом Тейлора функции $f(x)$ вещественной переменной $x$, дифференцируемой $k$ раз в точке $a$ называется конечная сумма 
	$$f(x) = \sum\limits_{n=0}{k}\frac{f^{(n)}(a)}{n!} (x - a)^n = f(a) + f^{\prime}(a)(x - a) + \frac{f^{(2)}(a)}{2!}(x - a)^2 + ... + \frac{f^{(k)}(a)}{k!}(x - k)^k$$
	
	Рядом Тейлора в точке $a$ функции $f(x)$ , бесконечно диффиренцируемой в окрестности точки $a$, называется формальный степенной ряд
	$$f(x) = \sum\limits_{n=0}^{+\infty}\frac{f^{(n)}(a)}{n!}(x - a)^n$$
	
	Другими словами, рядом Тейлора функции $f(x)$ в точке $a$ называется ряд разложения функции по положительным степеням двучлена $(x - a)$
	
}

Еще есть формула Тейлора, это просто частичная сумма ряда вроде как.

В случае $a = 0$ это все безобразие -- {\bf ряд Маклорена}