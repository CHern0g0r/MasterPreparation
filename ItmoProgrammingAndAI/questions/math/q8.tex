\subsection{Дискретные случайные величины. Математическое ожидание и дисперсия. Стандартные дискретные распределения (Бернулли, биномиальное, геометрическое, Пуассона).}

\D { Случайная величина -- отображение из множества элементарных исходов в множество вещественных чисел}

\D { {\bf Дискретной случайной величиной} называется случайная величина, множество значений которой не более чем счетно, причем принятие ею каждого из значений есть случайное событие с определенной вероятностью}

\D {{\bf Функция распределения} случайной величины -- функция $F(x)$, определенная на $\mathbb{R}$ как $P(\xi \le x)$ т.е. выражаюшая вероятность того, что $\xi$ примет значение меньшее или равное $x$}

Матожидание дискретной величины -- $E\xi = \sum \xi_i p(\xi_i)$, где $\xi_i$ -- возможный исход, а $p_{\xi_i}$ -- его вероятность.

Дисперсия пересчитывается по формуле $DX = EX^2 - (EX)^2$

\subsubsection{ Стандартные дискретные распределения}

{\bf Распределение Бернулли}

Величина принимает всего 2 значения -- $1$ и $0$ с вероятностями $p$ и $q = 1 - p$ соответственно

{\bf Биноминальное распределение}

Это распределение количества "успехов" \ в последовательности из $n$ независимых случайных экспериментов, таких, что вероятность успеха в них одинакова и равна $p$.

Т.е. есть последовательность $X_1, ..., X_n$ -- независимых случайных величин имеющих распределение Бернулли с параметром $p$, тогда $Y = X_1 + ... + X_n$ имеет биноминальное распределени с параметрами $n$ и $p = Bin(n, p)$ 

В этом случае функция вероятности задается формулой $\mathbb{P}(Y = k) = {n \choose k} p^k q^{n - k}, k = 0, ..., n$

${n \choose k} = C_n^k = \frac{n!}{k!(n-k)!}$

{\bf Геометрическое распределение}

Одно из двух

\begin{enumerate}
	\item распределение вероятностей случайной величины $X$ равой номеру первого успеха в серии испытаний Бернулли и принимающей значение $n = 1, 2, 3,...$
	\item Распределение вероятностей случайной величины $Y = X - 1$ равной числу неудач до первого успеха, и принимающей значения $n = 0, 1, 2,...$
\end{enumerate}

В cлучае первого успеха $P(X = n) = (1 - p)^{n - 1}p$

В случае количества неудач -- $P(Y = n) = (1 - p)^n p$


{\bf Распределение Пуассона}

Распределение числа событий, произошедших за фиксированое время, при условии что они происходят с некоторой фиксированной средней интенсивностью и независимо друг от друга

Выберем фиксированное $\lambda > 0$ и определим дискретное распределение по функции вероятности:

$p(k) = P(Y = k) = \frac{\lambda^k}{k!}e^{-\lambda}$

где 

$k$ -- количество событий

$\lambda$ -- матожидание величины 