\subsection{Контекстно-свободные грамматики. Эффективные методы разбора: LL(k)-, LR(k)- и LALR-грамматики.}

\D{
	Формальная грамматика (англ. Formal grammar) — способ описания формального языка, представляющий собой четверку
	
	$\Gamma = \langle \Sigma, N, S \in N, P\rangle$
	
	, где:
	\begin{itemize}
		\item $\Sigma$ — алфавит, элементы которого называют терминалами (англ. terminals);
		\item $N$ — множество, элементы которого называют нетерминалами (англ. nonterminals);
		\item $S$ — начальный символ грамматики (англ. start symbol);
		\item $P$ — набор правил вывода (англ. production rules или productions) $\alpha \rightarrow \beta$.
	\end{itemize}
}

\D{
	Левосторонним выводом слова (англ. leftmost derivation) $\alpha$ называется такой вывод слова $\alpha$, в котором каждая последующая строка получена из предыдущей путем замены по одному из правил самого левого встречающегося в строке нетерминала.
}

Контекстно-свободная грамматика: \href{https://neerc.ifmo.ru/wiki/index.php?title=%D0%9A%D0%BE%D0%BD%D1%82%D0%B5%D0%BA%D1%81%D1%82%D0%BD%D0%BE-%D1%81%D0%B2%D0%BE%D0%B1%D0%BE%D0%B4%D0%BD%D1%8B%D0%B5_%D0%B3%D1%80%D0%B0%D0%BC%D0%BC%D0%B0%D1%82%D0%B8%D0%BA%D0%B8,_%D0%B2%D1%8B%D0%B2%D0%BE%D0%B4,_%D0%BB%D0%B5%D0%B2%D0%BE-_%D0%B8_%D0%BF%D1%80%D0%B0%D0%B2%D0%BE%D1%81%D1%82%D0%BE%D1%80%D0%BE%D0%BD%D0%BD%D0%B8%D0%B9_%D0%B2%D1%8B%D0%B2%D0%BE%D0%B4,_%D0%B4%D0%B5%D1%80%D0%B5%D0%B2%D0%BE_%D1%80%D0%B0%D0%B7%D0%B1%D0%BE%D1%80%D0%B0}{neerc}.

Наша задача - понять принадлежит ли входное слово языку. Для этого нам нужно предъявить вывод из стартового нетерминала. 
\begin{itemize}
	\item $LR(k)$-анализатор просматривает символы слева направо, складывает их в стек. Если в стеке накопилась правая часть какого-то правила, нужно принимать решение: заменять ли содержание стека на левую часть этого правила или ждать еще, чтоб накопилась правая часть еще какого-то правила (являющаяся надстрокой для текущей гипотезы). Чтобы принять это решение, анализатор смотрит еще $k$ символов.
	
	\href{https://neerc.ifmo.ru/wiki/index.php?title=LR(k)-%D0%B3%D1%80%D0%B0%D0%BC%D0%BC%D0%B0%D1%82%D0%B8%D0%BA%D0%B8}{neerc}
	
	\item $LL(k)$-анализатор
	
	\href{https://neerc.ifmo.ru/wiki/index.php?title=LL(k)-%D0%B3%D1%80%D0%B0%D0%BC%D0%BC%D0%B0%D1%82%D0%B8%D0%BA%D0%B8,_%D0%BC%D0%BD%D0%BE%D0%B6%D0%B5%D1%81%D1%82%D0%B2%D0%B0_FIRST_%D0%B8_FOLLOW}{neerc}
	
	\item \href{https://ru.wikipedia.org/wiki/SLR(1)}{$SLR(1)$ грамматика} - это грамматика, в которой усовершенствованно построение таблицы $LR(0)$ грамматики при помощи применения FIRST и FOLLOW (см. $LL$-анализатор).
	
	\href{https://ru.wikipedia.org/wiki/LALR(1)}{$LALR(1)$} - усовершенствование $SLR(1)$
\end{itemize}
