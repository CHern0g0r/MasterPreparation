\subsection{Линейные обыкновенные дифференциальные уравнения и системы. Фундаментальная система решений. Метод вариации постоянных для решения неоднородных уравнений.}

\D{
	Обыкновенное дифференциальное уравнение (ОДУ) называется линейным, если оно имеет вид
	\begin{align*}
		y^{(n)} + a_{n-1}(x)y^{(n-1)} + \ldots + a_1(x)y' + a_0(x)y = f(x)
	\end{align*}
	Нужно решить относительно функции $y$, т.е. найти такую функцию (или класс функций), что равенство верно в любой точке.
	
	Мы решаем эту задачу для $x \in [a, b]$, и полагаем все $a_i$ и $f$ непрерывными на $[a, b]$.
}

Введем обозначения:
\begin{align*}
	Y = \begin{pmatrix}
		y_1\\
		\vdots\\
		y_n
	\end{pmatrix},\ 
	Y' = \begin{pmatrix}
		y_1'\\
		\vdots\\
		y_n'
	\end{pmatrix},\ 
	A(x) = \begin{pmatrix}
		a_{11}(x) & a_{12}(x) & \ldots & a_{1n}(x)\\
		a_{21}(x) & a_{22}(x) & \ldots & a_{2n}(x)\\
		\ldots & \ldots & \ldots & \ldots\\
		a_{n1}(x) & a_{n2}(x) & \ldots & a_{nn}(x)\\
	\end{pmatrix},\ 
	b(x) = \begin{pmatrix}
		b_1(x)\\
		\vdots\\
		b_n(x)\\
	\end{pmatrix}
\end{align*}

Тогда система дифференциальных уравнений в векторной (матричной) форме записывается в виде 
\begin{align*}
	Y' = A(x)Y + b(x)
\end{align*}

\T[Существование и единственность решения задачи Коши для линейной системы дифференциальных уравнений]{
	Если $A(x)$ и $b(x)$ непрерывны на отрезке $[a, b]$, то какова бы ни была начальная точка $(x_0, Y_0)$, задача Коши $Y' = A(x)Y + b(x), Y(x_0) = Y_0$, имеет единственное решение на $[a, b]$.  
}

\href{http://twt.mpei.ac.ru/math/ode/ODEsys/ODEsysup_08050000.html}{Формулировка.}

Принцип решения состоит в том, чтоб выражая и подставляя $y_i'$ как и в случае с числовыми системами, получить равенство, в которое входит только $y_i, y_i', \ldots y^{(n)}$, после чего решить ОДУ $n$-ого порядка. 
\href{https://math-it.petrsu.ru/users/semenova/DIFF_UR/Lections/Diff_UR_8.pdf}{Подробнее.} 

\subsubsection{Фундаментальная система решений}

\D{
	Если $b(x) = 0$, система называется однородной. Иначе - неоднородной.
}

\D{
	Фундаментальная система решений (ФСР) системы дифференциальных линейных однородных уравнений — максимальный (то есть содержащий наибольшее возможное число элементов) набор линейно независимых на $[a, b]$ решений этой системы
}

\D{
	Фундаментальная матрица - матрица, столбцы которой образуют ФСР
}

\T{Линейная однородная система уравнений имеет фундаментальную матрицу}

\T{
	Рассмотрим однородное уравнение $Y' = A(x)Y$. Если $W(x)$ - его фундаментальная матрица, то любое его решение $Y(x)$ представимо в виде $Y(x) = W(x)C$, где $C$ - произвольный постоянный вектор столбец.
}

Далее смотри \href{https://math-it.petrsu.ru/users/semenova/DIFF_UR/Lections/Diff_UR_8.pdf}{пособие} начиная со страницы 6, теоремы 5.

В пособии другие обозначения: 
\begin{align*}
	Y \rightarrow X\\
	A \rightarrow A\\
	x \rightarrow t\\
	b \rightarrow F(t)
\end{align*}

\subsubsection{Метод вариации постоянных для решения неоднородных уравнений}
Смотри \href{https://math-it.petrsu.ru/users/semenova/DIFF_UR/Lections/Diff_UR_8.pdf}{пособие} со страницы 7.