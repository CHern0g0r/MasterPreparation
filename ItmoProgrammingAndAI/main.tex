%----------------------------------------------------------------------------------------
%	PACKAGES AND OTHER DOCUMENT CONFIGURATIONS
%----------------------------------------------------------------------------------------

\documentclass{article}
% \documentclass[14pt]{extarticle}
\usepackage{pdfpages}
\usepackage{float}

\input{structure.tex} % Include the file specifying the document structure and custom commands

%----------------------------------------------------------------------------------------
%	ASSIGNMENT INFORMATION
%----------------------------------------------------------------------------------------

\title{Programming and AI} % Title of the assignment

%----------------------------------------------------------------------------------------

\DeclareMathOperator*\uplim{\overline{lim}}

\usepackage[parfill]{parskip}
\usepackage{listings}

\begin{document}
	
\maketitle % Print the title
\tableofcontents
\section{Математика и Теоретическая информатика}

\subsection{Числовые ряды. Абсолютная и условная сходимость. Признаки сходимости числовых рядов.}

\subsubsection{Числовые ряды}

$\sum\limits_{k=1}^{\infty} a_{k} = a_{1} + a_{2} + a_{3} + \cdots$ -- числовой ряд

Сходимость ряда означает существование конечной суммы, т.е. $\sum\limits_{k=1}^{\infty} a_{k} = S$ где $S$ -- конечное число, иначе ряд считается расходящимся.

\subsubsection{Абсолютная и условная сходимость}

Ряд $\sum\limits_{k=1}^{\infty} a_{k}$ называется {\bf абсолютно} сходящимся, если сходится ряд из модулей $\sum\limits_{k=1}^{\infty} |a_{k}|$, иначе ряд называется {\bf условно} сходящимся

\subsubsection{Признаки сходимости числовых рядов} 

{\bf Знакоположительные ряды} (ряды с положительными членами):

Критерий сходимости знакоположительных рядов-- знакоположительный ряд $\sum\limits_{k=1}^{\infty} a_{k}$ сходится тогда и только тогда, когда последовательность его частичных сумм $S(n) = \sum\limits_{k=1}^{k=n}a_{k}$ ограничена сверху

{\bf Док-во:}

=>: ряд сходится, значит последовательность частичных сумм $\S(n) =\sum\limits_{k=1}^{n} a_{k}$ имеет предел равный $\sum\limits_{k=1}^{\infty} a_{k} = S$

<=: Пусть дан положительный ряд и последовательность частичных сумм ограничена сверху, заметим что последовательность частичных сумм неубывающая:
$$S_{n + 1} - S_{n} = a_{n + 1} \ge 0$$. Используя свойство из теоремы о монотонной последовательности получаем, что т.к. последовательность частичных сумм монотонно не убывает и ограничена сверху, значит она сходится и потому ряд сходится по определению.

{\bf Признак сравнения с мажорантой}

Пусть даны два положительных ряда $\sum\limits_{k=1}^{\infty} a_{k}$ и $\sum\limits_{k=1}^{\infty} b_{k}$. Если начиная с некоторого номера $n > N$ выполняется неравенство $0 \le a_n \le b_n$, то:

\begin{itemize}
	\item из сходимости рядя $\sum\limits_{k=1}^{\infty} b_{k}$ следует сходимость ряда $\sum\limits_{k=1}^{\infty} a_{k}$
	\item из расходимости ряда $\sum\limits_{k=1}^{\infty} a_{k}$ следует расходимость $\sum\limits_{k=1}^{\infty} b_{k}$
\end{itemize}

{\bf Док-во:}

Из неравенств на члены следует неравенство на частичные суммы $0 \le S_n \le \sigma_n$, дальше очев.


{\bf Признак Раабе}

Если для ряда $\sum\limits_{k=1}^{\infty} a_{k}$ существует предел $$R = \lim\limits_{n \rightarrow \infty} n (\frac{a_n}{a_{n+1}} - 1)$$, то при $R > 1$ ряд сходится, а при $R < 1$ -- расходится. Если $R = 1$, то жанный признак не говорит ничего.

{\bf Признак Гаусса}

Пусть для знакоположительного ряда $\sum\limits_{n=1}^{\infty} a_{n}$ отношение $\frac{a_n}{a_{n + 1}}$ может быть представлено в виде $$\frac{a_n}{a_{n + 1}} = \lambda + \frac{\mu}{n} + \frac{\theta_n}{n^2}$$, где $\lambda, \mu$ -- постоянные, а последовательность $\theta_n$ ограничена. Тогда 
\begin{itemize}
	\item ряд расходится если либо $\lambda > 1$, либо $\lambda = 1, \mu > 1$
	\item ряд расходится, если либо $\lambda < 1$, либо $\lambda = 1, \mu \le 1$
\end{itemize}


{\bf Знакопеременные ряды}

\D{Знакопеременными называются ряды, члены которых могут (стоять) быть как положительными, так и отрицательными.}


{\bf Признак Даламбера}

Слабее признака Коши, но зато проще

Если существует $\lim\limits_{n \rightarrow \infty}|\frac{a_{n + 1}}{a_n}| = r$, то 

\begin{itemize}
	\item если $r < 1$, то ряд абсолютно сходится
	\item если $r > 1$, то ряд расходится
	\item если $r = 1$, то данный признак ничего не говорит (сука)
\end{itemize}

{\bf Док-во:}

1. Пусть начиная с некоторого номера N верно неравенство $|\frac{a_{n+1}}{a_n}| \le q, 0 < q < 1$. Тогда перемножив члены начиная с N будем иметь что $\frac{a_{N+n}}{a_N} \le q^n$ откуда $|a_{N+n}| \le |a_{N}q^n|$, значит ряд $|a_{N+1}| + |a_{N+2}| + ...$ меньше бесконечной суммы убывающей геометрической прогрессии, поэтому он сходится

2. $|\frac{a_{n + 1}}{a_n}| \ge 1$ (с некоторого N), тогда можно записать $|a_{n+1}| \ge |a_n|$ значит модуль членов $a$ не стремится к 0 на бесконечности, значит последовательность не стремится к 0 а значит ряд не сходится.

3. Если просто меньше 1 до там хуйня какая-то мне впадлу
\\

{\bf Радикальный признак Коши} (ебаная оппозиция)

Если существует $\lim\lim\limits_{n \rightarrow \infty} \sqrt[n]{|a_n|} = r$, то

\begin{itemize}
	\item если $r < 1$ то ряд сходится абсолютно
	\item если $r > 1$ то ряд расходится
	\item если $r = 1$ то хз (опять??)
\end{itemize}

{\bf Док-во:} \href{https://ru.wikipedia.org/wiki/%D0%A0%D0%B0%D0%B4%D0%B8%D0%BA%D0%B0%D0%BB%D1%8C%D0%BD%D1%8B%D0%B9_%D0%BF%D1%80%D0%B8%D0%B7%D0%BD%D0%B0%D0%BA_%D0%9A%D0%BE%D1%88%D0%B8}{тут}
\\

{\bf Признак Лейбница}

Пусть для знакочередующегося ряда $$S = \sum\limits_{n=1}^{\infty}(-1)^{n-1}a_n, a_n \ge 0$$
выполняются следующие условия

\begin{itemize}
	\item С некоторого $N$ последовательность $a$ монотонно убывает, т.е. $a_{n+1} \le a_n$
	\item $\lim\limits_{n \rightarrow \infty}a_n = 0$
\end{itemize}

Тогда такой ряд сходится

{\bf Док-во:} \href{https://ru.wikipedia.org/wiki/%D0%A2%D0%B5%D0%BE%D1%80%D0%B5%D0%BC%D0%B0_%D0%9B%D0%B5%D0%B9%D0%B1%D0%BD%D0%B8%D1%86%D0%B0_%D0%BE_%D1%81%D1%85%D0%BE%D0%B4%D0%B8%D0%BC%D0%BE%D1%81%D1%82%D0%B8_%D0%B7%D0%BD%D0%B0%D0%BA%D0%BE%D1%87%D0%B5%D1%80%D0%B5%D0%B4%D1%83%D1%8E%D1%89%D0%B8%D1%85%D1%81%D1%8F_%D1%80%D1%8F%D0%B4%D0%BE%D0%B2}{здесь}\\

{\bf Признак Абеля}

\T {Числовой ряд $\sum\limits_{n=1}^{\infty}a_nb_n$ сходится, если выполнены следующие условия
	
	\begin{itemize}
		\item Последовательность \{$a_n$\} монотонна и ограничена
		\item Ряд $\sum\limits_{n=1}^{\infty}b_n$ сходится
	\end{itemize}
}
{\bf Proof:} \href{https://ib.mazurok.com/2015/06/16/%D0%BF%D1%80%D0%B8%D0%B7%D0%BD%D0%B0%D0%BA%D0%B8-%D0%B0%D0%B1%D0%B5%D0%BB%D1%8F-%D0%B8-%D0%B4%D0%B8%D1%80%D0%B8%D1%85%D0%BB%D0%B5/}{вот}\\

{\bf Признак Дирихле}

\T{Пусть выполнены условия:
	\begin{itemize}
		\item последовательность частичных сумм $B_n = \sum\limits_{k=1}^{n}$ ограничена
		\item последовательность $a_n$, начиная с некоторого номера, монотонно убывает $a_n \ge a_{n+1}$
		\item $\lim\limits_{n\rightarrow\infty}a_n = 0$
	\end{itemize}
	Тогда ряд $\sum\limits_{n=1}^{\infty}a_nb_b$ сходится
}

{\bf Proof:} \href{https://ib.mazurok.com/2015/06/16/%D0%BF%D1%80%D0%B8%D0%B7%D0%BD%D0%B0%D0%BA%D0%B8-%D0%B0%D0%B1%D0%B5%D0%BB%D1%8F-%D0%B8-%D0%B4%D0%B8%D1%80%D0%B8%D1%85%D0%BB%D0%B5/}{вот}\\

\subsection{Архитектура ЭВМ. Кэш-память. Многоуровневая организация кэш-памяти. Протоколы когерентности кэш-памяти}

\subsubsection{Архитектура ЭВМ}
\textbf{Архитектура ЭВМ}~---~
это модель, устанавливающая принципы организации вычислительной системы, состав, 
порядок и взаимодействие основных частей ЭВМ, функциональные возможности, 
удобство эксплуатации, стоимость, надежность.

\subsubsection{Кэш-память. Многоуровневая организация кэш-памяти. Протоколы когерентности кэш-памяти}
\textbf{Кэширование}~---~это использование дополнительной быстродействующей памяти (кэш-памяти) для хранения копий блоков информации из основной (оперативной) памяти, вероятность обращения к которым в ближайшее время велика.

\textbf{Аспекты кэшей}:
\begin{enumerate}
	\item Кэш-линии~---~вся память выровнена и разбита на непересекающиеся отрезки по 64 байта (последние 6 бит не используются как тег в ассоциативном кэше). 
	\item Кэши делятся на уровни (ближе к процессору $\Rightarrow$ больше скорость, меньше размер). 
	\begin{enumerate}
		\item L1 (32k, делится на данные и команды, Associativity = 4)
		\item L2 (256k, Associativity = 8)
		\item L2 (8Mb, Associativity = 16)
	\end{enumerate}
	\item Ассоциативность. Полная асс. - это когда мы просто пишем в кэш и для поиска нужного адреса нужно бежать по всем линиям. Асс =1 - это когда мы просто мапим по хвосту адреса (тегу) в таблицу. (типа хеш-таблица). Тогда быстро искать, но часто будем промахиваться. Если Acc = k - то мапим в корзины по k и получаем компромисс.
	\item  Эксклюзивность/инклюзивность - данные хранятся только в одном кэш (- эффективность из-за поиска, + размер) или они дублируются в уровнях (в памяти) ниже. (+ скорость, - размер)
	\item Когда у нас много процессоров, то возникает необходимость использовать протоколы когерентности. (MSI, MESI, MESIF, MOESI). Это необходимо, чтобы один процессор мог знать о том, что данные изменились только в кеше одно из его соседей. Для этого вводятся разные состояния владения памятью (invalid, shared, modified + owned/forward, exclusive), чтобы как можно более тоньше развести их и поменьше сбрасывать кэши.
\end{enumerate}

\subsubsection{Многоуровневая организация кэш-памяти (подробнее)}

\emph{ВОДА:}

Современные технологии позволяют разместить КЭШ-память и ЦП на общем кристалле. Такая внутренняя КЭШ-память строится по технологии статического ОЗУ и является наиболее быстродействующей. 

Емкость ее обычно не превышает 64 Кбайт. Попытки увеличения емкости обычно приводят к снижению быстродействия, главным образом, из-за усложнения схем управления и дешифрации адреса. 

Общую емкость КЭШ-памяти ЭВМ увеличивают за счет второй (внешней) КЭШ-памяти, расположенной между внутренней КЭШ- памятью и ОЗУ. Такая система известна под названием двухуровневой, где внутренней КЭШ-памяти отводится роль первого уровня (L1), а внешней — второго уровня (L2). Емкость L2 может быть значительной (до 1 МБ). 

При доступе к памяти ЦП сначала обращается к КЭШ-памяти первого уровня. В случае промаха производится обращение к КЭШ-памяти второго уровня. Если информация отсутствует и в L2, выполняется обращение к ОЗУ и соответствующий блок заносится сначала в L2, а затем и в L1. Благодаря такой процедуре часто запрашиваемая информация может быть быстро восстановлена из КЭШ-памяти второго уровня. Для ускорения обмена информацией между ЦП и L2 между ними часто вводят специальную шину, так называемую шину заднего плана, в отличие от шины переднего плана, связывающую ЦП с основной памятью. 

Количество уровней КЭШ-памяти не ограничивается двумя. В некоторых ЭВМ можно встретить КЭШ-память третьего уровня (L3). Ведутся активные дискуссии о введении также и КЭШ-памяти четвертого уровня (L4). Характер взаимодействия очередного уровня с предшествующим аналогичен описанному для L1 и L2. Таким образом, можно говорить об иерархии КЭШ-памяти. Каждый последующий уровень характеризуется большей емкостью, меньшей стоимостью, но и меньшим быстродействием, хотя оно все же выше, чем у ЗУ основной памяти.

\subsection{С++. Процесс компиляции и линковки. .cpp, .h, .i, .o файлы.}
Понятно и подробно описано: 
\href{https://habr.com/ru/post/478124/}{https://habr.com/ru/post/478124/}

Кратко и структурировано: \href{https://server.179.ru/tasks/cpp/total/105.html}{https://server.179.ru/tasks/cpp/total/105.html}

\textbf{Компиляция}~---~трансляция программы, составленной на исходном языке высокого уровня, в эквивалентную программу на низкоуровневом языке, близком машинному коду (абсолютный код, объектный модуль, иногда на язык ассемблера). Входной информацией для компилятора (исходный код) является описание алгоритма или программа на объектно-ориентированном языке, а на выходе компилятора—эквивалентное описание алгоритма на машинно-ориентированном языке (объектный код).

\subsubsection{Заголовочные файлы (.h)}

%Целью заголовочных файлов является удобное хранение набора объявлений объектов для их последующего использования в других программах. 
В языках программирования Си и C++ заголовочные файлы~---~основной способ подключить к программе типы данных, структуры, прототипы функций, перечисляемые типы и макросы, используемые в другом модуле. По умолчанию используется расширение .h; иногда для заголовочных файлов языка C++ используют расширение .hpp.

Чтобы избежать повторного включения одного и того же кода, используются директивы \#ifndef, \#define, \#endif.

Заголовочный файл в общем случае может содержать любые конструкции языка программирования, но на практике исполняемый код (за исключением inline-функций в C++) в заголовочные файлы не помещают.

\subsection{С++. Жизненный цикл объектов в С++. RAII.}
\href{https://docs.microsoft.com/en-us/cpp/cpp/object-lifetime-and-resource-management-modern-cpp?view=msvc-170}{Документация} на английском

\subsection{Java. Устройство сборщика мусора в JVM.}

\href{https://medium.com/nuances-of-programming/%D1%81%D0%B1%D0%BE%D1%80%D0%BA%D0%B0-%D0%BC%D1%83%D1%81%D0%BE%D1%80%D0%B0-%D0%B2-java-%D1%87%D1%82%D0%BE-%D1%8D%D1%82%D0%BE-%D1%82%D0%B0%D0%BA%D0%BE%D0%B5-%D0%B8-%D0%BA%D0%B0%D0%BA-%D1%80%D0%B0%D0%B1%D0%BE%D1%82%D0%B0%D0%B5%D1%82-%D0%B2-jvm-25bb2570b44c}{Подробнее тут}
Сборка мусора в Java~---~это процесс, с помощью которого программы Java автоматически управляют памятью. Java-программы компилируются в байт-код, который запускается на виртуальной машине Java (JVM).

Когда Java-программы выполняются на JVM, объекты создаются в куче, которая представляет собой часть памяти, выделенную для них.

Пока Java-приложение работает, в нем создаются и запускаются новые объекты. В конце концов некоторые объекты перестают быть нужны. Можно сказать, что в любой момент времени память кучи состоит из двух типов объектов:
\begin{enumerate}
	\item Живые~---~эти объекты используются, на них ссылаются откуда-то еще.
	\item Мертвые~---~эти объекты больше нигде не используются, ссылок на них нет.
\end{enumerate}

Сборщик мусора находит эти неиспользуемые объекты и удаляет их, чтобы освободить память.

\textbf{Этапы сборки мусора в Java:}
\begin{enumerate}
	\item Пометка объектов как живых
	\item Зачистка мертвых объектов
	\item Компактное расположение оставшихся объектов в памяти
\end{enumerate} 

\textbf{Сбор мусора по поколениям.}

Сборщики мусора в Java реализуют стратегию сбора мусора поколений, которая классифицирует объекты по возрасту.

Область памяти кучи в JVM разделена на три секции:
\begin{enumerate}
	\item \emph{Молодое поколение.} 
	Вновь созданные объекты начинаются в молодом поколении. Молодое поколение далее подразделяется на две категории.
	\begin{enumerate}
		\item Пространство Эдема~---~все новые объекты начинают здесь, и им выделяется начальная память.
		\item Пространства выживших (FromSpace и ToSpace)~---~объекты перемещаются сюда из Эдема после того, как пережили один цикл сборки мусора.
	\end{enumerate}
	
	\item \emph{Старшее поколение.}
	Объекты-долгожители в конечном итоге переходят из молодого поколения в старшее. Оно также известно как штатное поколение и содержит объекты, которые долгое время оставались в пространствах выживших.
	\item \emph{Постоянное поколение} (\emph{Мета-пространство}, начиная с Java 8).
	Метаданные, такие как классы и методы, хранятся в постоянном поколении. JVM заполняет его во время выполнения на основе классов, используемых приложением. Классы, которые больше не используются, могут переходить из постоянного поколения в мусор.
	
	Начиная с Java 8, на смену пространству постоянного поколения (PermGen) приходит пространство памяти MetaSpace. Реализация отличается от PermGen — это пространство кучи теперь изменяется автоматически.
\end{enumerate}

\subsection{Задача Коши для системы обыкновенных дифференциальных уравнений. Существование и единственность решения. Устойчивость.}

\subsection{Метапрограммирование. Шаблоны и Generics. Частичная специализация шаблонов.}

\subsubsection{Метапрограммирование}

\textbf{Метапрограммирование}~---~создание программ, которые создают другие программы как результат своей работы (либо — частный случай — изменяющие или дополняющие себя во время выполнения).

Метапрограммирование можно разделить на 2 направления: на стадии компиляции (генерация кода) и на стадии выполнения (самомодифицирующийся код).

Первое направление позволяет получить программу при меньших затратах времени и усилий, чем если бы программист писал её вручную. Второе — расширяет возможности программиста.

\textbf{Генерация кода} (это не обязательно)

При этом подходе код программы не пишется вручную, а создается автоматически программой-генератором на основе другой, более простой программы.
Такой подход приобретает смысл, если при программировании вырабатываются различные дополнительные правила (более высокоуровневые парадигмы, выполнение требований внешних библиотек, стереотипные методы реализации определенных функций и пр.). При этом часть кода теряет содержательный смысл и становится лишь механическим выполнением правил. Когда эта часть становится значительной, возникает мысль задавать вручную лишь содержательную часть, а остальное добавлять автоматически. Это и проделывает генератор. Реализуется 2 основными методами:

\begin{enumerate}
	\item  \emph{Шаблоны} (наиболее известные случаи применения — препроцессор C и шаблоны в C++). Решают задачу, если соблюдение «правил» сводится к вставке в программу повторяющихся (или почти повторяющихся) кусков кода. Помимо этого, обладают еще рядом достоинств: например, помогают повторному использованию.
	\item \emph{Внешнеязыковые средства} (пример: генераторы синтаксических и лексических анализаторов lex, yacc, bison). Применяются в случаях, если простых средств вроде шаблонов недостаточно. Язык генератора составляется так, чтобы автоматически или с минимальными усилиями со стороны программиста реализовывать правила парадигмы или необходимые специальные функции. Фактически, это — более высокоуровневый язык программирования, а генератор — не что иное, как транслятор.
	Генераторы пишутся, как правило, для создания специализированных программ, в которых очень значительная часть стереотипна, либо для реализации сложных парадигм (таких, как паттерны проектирования).
\end{enumerate}

\textbf{Самомодифицирующийся код} (это не обязательно)

Возможность изменять или дополнять себя во время выполнения превращает программу в виртуальную машину. Хотя такая возможность существовала уже давно на уровне машинных кодов (и активно использовалась, например, при создании полиморфных вирусов), с метапрограммированием обычно связывают перенос подобных технологий в высокоуровневые языки. Основные методы реализации:
\begin{enumerate}
	\item Интроспекция — представление внутренних структур языка в виде переменных встроенных типов с возможностью доступа к ним из программы. Позволяет во время выполнения смотреть, создавать и изменять определения типов, стек вызовов, обращаться к переменной по имени, получаемому динамически и пр.
	Например, Пространство имён System.Reflection и тип System.Type в .NET; классы Class, Method, Field в Java; представление пространств имен и определений типов через встроенные типы данных в Python
	\item  Интерпретация произвольного кода, представленного в виде строки.
	Существует естественным образом во множестве интерпретируемых языков, например eval() в PHP.
	Для C++ есть библиотека, позволяющая «на лету» компилировать и генерировать исполняемый код (используется урезанный компилятор gcc).
	Принципиальный недостаток технологий этого направления — неприменимость к компилируемым языкам. Можно ввести в такой язык интерпретатор, как в вышеуказанной библиотеке для С++, но это
	практически сведет на нет главное преимущество данных языков — производительность.
\end{enumerate}

\subsubsection{Шаблоны и Generics. Частичная специализация шаблонов.}

\textbf{Шаблоны:}

В языке C++ обобщённое программирование основывается на понятии «шаблон», обозначаемом ключевым словом template.

\begin{lstlisting}[language=C++]
	template < typename T> T max (T x , T y ) {
		if (x < y)
		return y;
		else
		return x;
	}
\end{lstlisting}

Интересное применение нашли шаблоны в языке C++. Оказалось, что шаблоны в этом языке являются тьюринг-полным функциональном языком. Другими словами на шаблонах С++ можно написать программу, реализующую произвольный алгоритм, и эта программа выполнится в момент компиляции.

К примеру, можно предпосчитать $50$-е число Фибоначчи. Тогда во время выполнения программы не придется тратить время на его вычисления. Одной интересной особенностью такого программирования на шаблонах, является встроенный механизм мемоизации (сохранения результата вычисления функции). Это значит, что рекурсивный алгоритм для вычисления k -го числа Фибоначчи работающий «в лоб» сделает порядка k операций (вместо ожидаемых 2k).

\begin{lstlisting}[language=C++]
	template <int i> struct fib { 
		static const int val = fib<i - 1>::val + fib<i - 2>::val;
	};
	template <> struct fib <1> { static const int val = 1; };
	template <> struct fib <2> { static const int val = 1; };
\end{lstlisting}

Но это не самое интересное: программирование на шаблонах С++ позволяет общаться с типом как с обычным объектом. К примеру, можно составить список типов, удалить из него все встроенные типы, а
из оставшегося списка создать объект, который будет унаследован от всех типов из данного списка. Для такого метапрограммирования была написана специальная библиотека MPL (MetaProgramming Library).

\textbf{Generics}

Язык Java предоставляет средства обобщённого программирования, синтаксически основанные на C++. В Java generics (параметризованные типы или родовые типы) имеют мнимое сходство с шаблонами C++ как
по синтаксису, так и по ожидаемому месту их применения (например, в качестве контейнерных классов).

Но это сходство только поверхностное — родовые типы в языке программирования Java почти полностью реализуются в компиляторе, который выполняет проверку типов и выявление типа (type inference)
и, затем, генерирует обычные не параметризованные байткоды. Такая техника реализации, называемая стиранием (когда компилятор использует информацию о родовом типе для контроля типов и удаляет ее перед генерированием байткода), имеет неожиданные, а иногда и непонятные последствия. В то время как родовые типы являются большим шагом на пути к безопасности Java-классов, изучение их использования почти наверняка будет вызывать некоторую озадаченность (а иногда и мучения).

\textbf{Частичная специализация шаблонов}

Если у шаблона класса есть несколько параметров, то можно специализировать его только для одного или нескольких аргументов, оставляя другие неспециализированными. Иными словами, допустимо написать шаблон, соответствующий общему во всем, кроме тех параметров, вместо которых подставлены фактические типы или значения. Такой механизм носит название частичной специализации шаблона класса. Она может понадобиться при определении реализации, более подходящей для конкретного набора аргументов. Например unique\_ptr имеет частичную специализацию для массивов (T[]).

\begin{lstlisting}[language=C++]
	template <typename T>
	class unique_ptr;
	
	template <typename T>
	class unique_ptr<T[]>;
\end{lstlisting}
Частичная специализация шаблона класса — это тоже шаблон, но список параметров здесь отличается от соответствующего списка параметров общего шаблона.

\subsection{Функциональное программирование. Чистые объекты. Функторы. Аппликативы. Монада. Взаимодействие с внешним миром.}

\subsubsection{Функциональное программирование. Чистые объекты.}
\textbf{Функциональное программирование}~---~парадигма программирования, в которой процесс вычисления трактуется как вычисление значений функций в математическом понимании последних (в отличие от функций как подпрограмм в процедурном программировании).

\textbf{Чистыми} называют функции, которые не имеют побочных эффектов ввода-вывода и памяти (они зависят только от своих параметров и возвращают только свой результат). Чистые функции обладают несколькими полезными свойствами, многие из которых можно использовать для оптимизации кода:
\begin{enumerate}
	\item если результат чистой функции не используется, её вызов может быть удалён без вреда для других выражений;
	\item результат вызова чистой функции может быть мемоизирован, то есть сохранён в таблице значений вместе с аргументами вызова;
	\item если нет никакой зависимости по данным между двумя чистыми функциями, то порядок их вычисления можно поменять или распараллелить (говоря иначе, вычисление чистых функций удовлетворяет принципам потокобезопасности);
	\item если весь язык не допускает побочных эффектов, то можно использовать любую политику вычисления. Это предоставляет свободу компилятору комбинировать и реорганизовывать вычисление выражений в программе (например, исключить древовидные структуры).
\end{enumerate}

\subsubsection{Функторы}

\href{https://habr.com/ru/post/183150/}{Объяснение на пальцах}

\href{http://cmc-msu-ai.github.io/haskell-course/lecture/2013/09/07/functors.html}{Подробнее про функторы}

\textbf{Функтором} называется класс типов, который декларирует единственный метод «fmap». Интуитивно, «fmap» применяет функцию a -> b к значению типа f a, чтобы получить значение типа f b. С другой стороны, можно рассматривать «fmap» как функцию высшего порядка, преобразующую «простую» функцию a -> b в «составную» функцию f a -> f b. Важно отметить, что структура значения типа f после применения «fmap» должна оставаться неизменной.

\begin{lstlisting}[language=Haskell]
	class Functor f where
	fmap :: (a -> b) -> f a -> f b
\end{lstlisting}

\subsubsection{Аппликативы.}
\href{http://cmc-msu-ai.github.io/haskell-course/lecture/2013/09/08/applicative-and-monad.html}{Подробнее тут}

Естественным продолжением класса Functor является класс \textbf{Applicative} (аппликативный функтор), определенный в модуле Control.Applicative:

\begin{lstlisting}[language=Haskell]
	class Functor f => Applicative f where
	pure  :: a -> f a
	(<*>) :: f (a -> b) -> f a -> f b
\end{lstlisting}

\textbf{Законы:}
\begin{enumerate}
	\item Помещение тождественной функции в «чистый» контекст и применение к аргументу в контексте не меняет ни значение, ни контекст.
	\begin{lstlisting}[language=Haskell]
		pure id <*> x == x
	\end{lstlisting}
	\item Применение чистой функции к чистому аргументу в контексте «по умолчанию» должно быть эквивалентно применению функции, а затем помещению результата в контекст.
	\begin{lstlisting}[language=Haskell]
		pure f <*> pure x == pure (f x)
	\end{lstlisting}
	\item При применении функции u с побочными эффектами к чистому аргументу y порядок вычисления функции и аргумента неважен.
	\begin{lstlisting}[language=Haskell]
		u <*> pure y == pure ($ y) <*> u
	\end{lstlisting}
	\item Некоторый аналог композиции для аппликативных функторов.
	\begin{lstlisting}[language=Haskell]
		u <*> (v <*> w) == pure (.) <*> u <*> v <*> w
	\end{lstlisting}
\end{enumerate} 

\subsubsection{Монады}
\href{http://cmc-msu-ai.github.io/haskell-course/lecture/2013/09/08/applicative-and-monad.html}{Подробнее тут}
\begin{lstlisting}[language=Haskell]
	class Monad m where
	return :: a -> m a
	(>>=) :: m a -> (a -> m b) -> mb
	(>>) :: m a -> m b -> m b
	m >> n = m >>= \_ -> n
	
	fail :: String -> m a
\end{lstlisting}

Функция return по типу очень напоминает функцию pure из класса Applicative. И, в действительности, return и есть pure, хоть и с не самым удачным названием (return в Haskell совсем не то же, что return в обычных императивных языках вроде C или Java). С математической точки зрения, любая монада является аппликативным функтором (но не наоборот). Но по историческим причинам,в описании класса это не указано.

Как следует из определения, операция ($>>$) является частным случаем операции ($>>=$)

Функция fail осталась в классе по историческим причинам, хотя никакого отношения к монадам реально не имеет.

\textbf{Законы:}
\begin{lstlisting}[language=Haskell]
	return a >>= k = k a
	m >>= return = m
	m >>= (\x -> k x >>= h) = (m >>= k) >>= h
	fmap f xs = xs >>= return . f = liftM f xs
\end{lstlisting}

\subsubsection{Взаимодействие с внешним миром.}
Взаимодействие с "внешним миром" (побочные эффекты вычислений) можно реализовать с помощью специальной монады IO.

Грубое приближение монады IO — это взятие пары с контекстом: "(a, RealWorld)". При операциях с монадой состояние RealWorld может меняться.

Есть операция return, погружающая объект в окружение IO («выпускающая во внешний мир»), а вот обратного преобразования нет.

\begin{lstlisting}[language=Haskell]
	putChar :: Char -> IO () 
\end{lstlisting}~---~берёт символ и возвращает новый «мир», в котором растворился (был напечатан в консоли) этот символ.

\begin{lstlisting}[language=Haskell]
	getChar :: IO Char
\end{lstlisting} --- мы можем получить символ из внешнего мира, но только внутри монады.

Достать его из монады мы не можем, но можем работать сним внутри IO с помощью $>>=$.

В процессе выполнения программы, содержащей IO, объекты типов IO a остаются временно невычисленными, как задумки.

Например, если мы где-то напишем putChar ' a' , то символ не будет тут же напечатан.

Вместо этого нужно дождаться, пока соберётся «главный» объект типа IO (), и уже при его вычислении все операции с внешним миром будут выполнены, причём в правильном порядке.

\subsection{Операционные системы. Процессы: вытесняющая и кооперативная многозадачность, планировщики, многопроцессорные машины.}

\textbf{Кооперативная многозадачность}.

Тип многозадачности, при котором фоновые задачи выполняются только во время простоя основного процесса и только в том случае, если на это получено разрешение основного процесса.

Кооперативную многозадачность можно назвать многозадачностью “второй ступени” поскольку она использует более передовые методы, чем простое переключение задач, реализованное многими известными программами (например, МS-DOS shell из МS-DOS 5.0 при простом переключении активная программа получает все процессорное время, а фоновые приложения полностью замораживаются). При кооперативной многозадачности приложение может захватить фактически столько процессорного времени, сколько оно считает нужным. Все приложения делят процессорное время, периодически передавая управление следующей задаче.

\textbf{ Вытесняющая многозадачность}.

Вид многозадачности, в котором операционная система сама передает управление от одной выполняемой программы другой. Распределение процессорного времени осуществляется планировщиком процессов. Этот вид многозадачности обеспечивает более быстрый отклик на действия пользователя.

Вытесняющая многозадачность~---~это вид многозадачности при котором планирование процессов основывается на абсолютных приоритетах. Процесс с меньшим приоритетом (например пользовательская программа) может быть вытеснен при его выполнении более приоритетным процессом (например системной или диагностической программой). Иногда этот вид многозадачности называют приоритетным.

Каждая работающая программа имеет свое защищенное адресное пространство. Многопоточное выполнение отдельных задач позволяет при задержке в выполнении одного потока не останавливать задачу полностью, а работать со следующим потоком.

\textbf{Планировщик.}

\href{https://habr.com/ru/post/154609/}{Подробнее тут}

Планировщик~---~часть операционной системы, которая отвечает за (псевдо)параллельное выполнения задач, потоков, процессов. Планировщик выделяет потокам процессорное время, память, стек и прочие ресурсы. Планировщик может принудительно забирать управление у потока (например по таймеру или при появлении потока с большим приоритетом), либо просто ожидать пока поток сам явно(вызовом некой системной процедуры) или неявно(по завершении) отдаст управление планировщику.
Первый вариант работы планировщика называется реальным или вытесняющим(preemptive), второй, соответственно, не вытесняющим (non-preemptive).

\textbf{Многопроцессорные машины.}

\href{https://docstore.mik.ua/skbd/glava_10.htm}{Подробнее тут}

\href{https://ru.wikipedia.org/wiki/%D0%9C%D0%BD%D0%BE%D0%B3%D0%BE%D0%BF%D1%80%D0%BE%D1%86%D0%B5%D1%81%D1%81%D0%BE%D1%80%D0%BD%D0%BE%D1%81%D1%82%D1%8C}{В Википедии тоже хорошо написано}

Любая вычислительная система (будь то супер-ЭВМ или персональный компьютер) достигает своей наивысшей производительности благодаря использованию высокоскоростных элементов и параллельному выполнению большого числа операций. Именно возможность параллельной работы различных устройств системы (работы с перекрытием) является основой ускорения основных операций. 

Параллельные ЭВМ часто подразделяются по классификации Флинна на машины типа SIMD (Single Instruction Multiple Data - с одним потоком команд при множественном потоке данных) и MIMD (Multiple Instruction Multiple Data - с множественным потоком команд при множественном потоке данных). Как и любая другая, приведенная выше классификация несовершенна: существуют машины прямо в нее не попадающие, имеются также важные признаки, которые в этой классификации не учтены. В частности, к машинам типа SIMD часто относят векторные процессоры, хотя их высокая производительность зависит от другой формы параллелизма - конвейерной организации машины. Многопроцессорные векторные системы, типа Cray Y-MP, состоят из нескольких векторных процессоров и поэтому могут быть названы MSIMD (Multiple SIMD).

\subsection{Операционные системы. Виртуальная память: MMU, TLB, таблицы страниц, аллокаторы и менеджеры виртуальной памяти.}

\subsubsection{Виртуальная память: MMU, TLB.}

\href{https://habr.com/ru/post/211150/}{Подробнее тут}

Блок управления памятью или устройство управления памятью memory management unit, MMU)~---~компонент аппаратного обеспечения компьютера, отвечающий за управление доступом к памяти, запрашиваемым центральным процессором.

Его функции заключаются в трансляции адресов виртуальной памяти в адреса физической памяти (то есть управление виртуальной памятью), защите памяти, управлении кэш-памятью, арбитражем шины и, в более простых компьютерных архитектурах (особенно 8-битных), переключением блоков памяти. 

Принцип работы современных MMU основан на разделении виртуального адресного пространства (одномерного массива адресов, используемых центральным процессором) на участки одинакового, как правило, несколько килобайт, хотя, возможно, и существенно большего, размера, равного степени 2, называемые страницами. Младшие n бит адреса (смещение внутри страницы) остаются неизменными. Старшие биты адреса представляют собой номер (виртуальной) страницы. MMU обычно преобразует номера виртуальных страниц в номера физических страниц, используя буфер ассоциативной трансляции (Translation Lookaside Buffer, TLB).

Если преобразование при помощи TLB невозможно, включается более медленный механизм преобразования, основанный на специфическом аппаратном обеспечении или на программных системных структурах. Данные в этих структурах, как правило, называются элементами таблицы страниц (page table entries (PTE)), а сами структуры — таблицами страниц (англ. page table (PT)). Конкатенация номера физической страницы со смещением внутри страницы даёт физический адрес.

Элементы PTE или TLB могут также содержать дополнительную информацию: бит признака записи в страницу ( dirty bit), время последнего доступа к странице (accessed bit), какие процессы (пользовательские (user mode) или системные (supervisor mode)) могут читать или записывать данные в страницу, необходимо ли кэшировать страницу.

\subsubsection{Таблицы страниц.}

\href{https://ru.wikipedia.org/wiki/%D0%A2%D0%B0%D0%B1%D0%BB%D0%B8%D1%86%D0%B0_%D1%81%D1%82%D1%80%D0%B0%D0%BD%D0%B8%D1%86}{Википедия}

Таблица страниц~---~это структура данных, используемая системой виртуальной памяти в операционной системе компьютера для хранения сопоставления между виртуальным адресом и физическим адресом. Виртуальные адреса используются выполняющимся процессом, в то время как физические адреса используются аппаратным обеспечением, или, более конкретно, подсистемой ОЗУ. Таблица страниц является ключевым компонентом преобразования виртуальных адресов, который необходим для доступа к данным в памяти.

\subsubsection{Аллокаторы.}

\href{https://habr.com/ru/post/505632/}{Подробнее тут}

Аллокатор или распределитель памяти в языке программирования C++ ~---~ специализированный класс, реализующий и инкапсулирующий малозначимые (с прикладной точки зрения) детали распределения и освобождения ресурсов компьютерной памяти.

Концептуально выделяется пять основных операции, которые можно осуществить над аллокатором:
\begin{enumerate}
	\item \emph{create}~---~создает аллокатор и отдает ему в распоряжение некоторый объем памяти;
	\item \emph{allocate}~---~выделяет блок определенного размера из области памяти, которым распоряжается аллокатор;
	\item \emph{deallocate}~---~освобождает определенный блок;
	\item \emph{free}~---~освобождает все выделенные блоки из памяти аллокатора (память, выделенная аллокатору, не освобождается);
	\item \emph{destroy}~---~уничтожает аллокатор с последующим освобождением памяти, выделенной аллокатору.
\end{enumerate}

\subsubsection{Менеджеры виртуальной памяти.}

Менеджер виртуальной памяти (далее просто «менеджер памяти») ~---~ часть операционной системы, благодаря которой можно адресовать память большую, чем объем физической памяти (ОЗУ).

Благодаря виртуальной памяти можно запускать множество ресурсоёмких приложений, требующих большого объёма ОЗУ. Максимальный объём виртуальной памяти, который можно получить, используя 24-битную адресацию, — 16 мегабайт. С помощью 32-битной адресации можно адресовать до 4 ГБ виртуальной памяти. А 64-битная адресация позволяет работать уже с 16 эксабайтами памяти.

Применение механизма виртуальной памяти позволяет:

\begin{enumerate}
	\item упростить адресацию памяти клиентским программным обеспечением;
	\item рационально управлять оперативной памятью компьютера (хранить в ней только активно используемые области памяти);
	\item изолировать процессы друг от друга (процесс полагает, что монопольно владеет всей памятью).
\end{enumerate}


\subsection{Операционные системы. Виртуальная память: MMU, TLB, таблицы страниц, аллокаторы и менеджеры виртуальной памяти.}

\subsubsection{Виртуальная память: MMU, TLB.}

\href{https://habr.com/ru/post/211150/}{Подробнее тут}

Блок управления памятью или устройство управления памятью memory management unit, MMU)~---~компонент аппаратного обеспечения компьютера, отвечающий за управление доступом к памяти, запрашиваемым центральным процессором.

Его функции заключаются в трансляции адресов виртуальной памяти в адреса физической памяти (то есть управление виртуальной памятью), защите памяти, управлении кэш-памятью, арбитражем шины и, в более простых компьютерных архитектурах (особенно 8-битных), переключением блоков памяти. 

Принцип работы современных MMU основан на разделении виртуального адресного пространства (одномерного массива адресов, используемых центральным процессором) на участки одинакового, как правило, несколько килобайт, хотя, возможно, и существенно большего, размера, равного степени 2, называемые страницами. Младшие n бит адреса (смещение внутри страницы) остаются неизменными. Старшие биты адреса представляют собой номер (виртуальной) страницы. MMU обычно преобразует номера виртуальных страниц в номера физических страниц, используя буфер ассоциативной трансляции (Translation Lookaside Buffer, TLB).

Если преобразование при помощи TLB невозможно, включается более медленный механизм преобразования, основанный на специфическом аппаратном обеспечении или на программных системных структурах. Данные в этих структурах, как правило, называются элементами таблицы страниц (page table entries (PTE)), а сами структуры — таблицами страниц (англ. page table (PT)). Конкатенация номера физической страницы со смещением внутри страницы даёт физический адрес.

Элементы PTE или TLB могут также содержать дополнительную информацию: бит признака записи в страницу ( dirty bit), время последнего доступа к странице (accessed bit), какие процессы (пользовательские (user mode) или системные (supervisor mode)) могут читать или записывать данные в страницу, необходимо ли кэшировать страницу.

\subsubsection{Таблицы страниц.}

\href{https://ru.wikipedia.org/wiki/%D0%A2%D0%B0%D0%B1%D0%BB%D0%B8%D1%86%D0%B0_%D1%81%D1%82%D1%80%D0%B0%D0%BD%D0%B8%D1%86}{Википедия}

Таблица страниц~---~это структура данных, используемая системой виртуальной памяти в операционной системе компьютера для хранения сопоставления между виртуальным адресом и физическим адресом. Виртуальные адреса используются выполняющимся процессом, в то время как физические адреса используются аппаратным обеспечением, или, более конкретно, подсистемой ОЗУ. Таблица страниц является ключевым компонентом преобразования виртуальных адресов, который необходим для доступа к данным в памяти.

\subsubsection{Аллокаторы.}

\href{https://habr.com/ru/post/505632/}{Подробнее тут}

Аллокатор или распределитель памяти в языке программирования C++ ~---~ специализированный класс, реализующий и инкапсулирующий малозначимые (с прикладной точки зрения) детали распределения и освобождения ресурсов компьютерной памяти.

Концептуально выделяется пять основных операции, которые можно осуществить над аллокатором:
\begin{enumerate}
	\item \emph{create}~---~создает аллокатор и отдает ему в распоряжение некоторый объем памяти;
	\item \emph{allocate}~---~выделяет блок определенного размера из области памяти, которым распоряжается аллокатор;
	\item \emph{deallocate}~---~освобождает определенный блок;
	\item \emph{free}~---~освобождает все выделенные блоки из памяти аллокатора (память, выделенная аллокатору, не освобождается);
	\item \emph{destroy}~---~уничтожает аллокатор с последующим освобождением памяти, выделенной аллокатору.
\end{enumerate}

\subsubsection{Менеджеры виртуальной памяти.}

Менеджер виртуальной памяти (далее просто «менеджер памяти») ~---~ часть операционной системы, благодаря которой можно адресовать память большую, чем объем физической памяти (ОЗУ).

Благодаря виртуальной памяти можно запускать множество ресурсоёмких приложений, требующих большого объёма ОЗУ. Максимальный объём виртуальной памяти, который можно получить, используя 24-битную адресацию, — 16 мегабайт. С помощью 32-битной адресации можно адресовать до 4 ГБ виртуальной памяти. А 64-битная адресация позволяет работать уже с 16 эксабайтами памяти.

Применение механизма виртуальной памяти позволяет:

\begin{enumerate}
	\item упростить адресацию памяти клиентским программным обеспечением;
	\item рационально управлять оперативной памятью компьютера (хранить в ней только активно используемые области памяти);
	\item изолировать процессы друг от друга (процесс полагает, что монопольно владеет всей памятью).
\end{enumerate}

\subsection{Операционные системы. Файловые системы (UNIX): файлы и директории, inode, контроль доступа. Файловые системы (реализация в ядре): VFS, блочные устройства, планировщик IO}

\subsubsection{Файловые системы (UNIX): файлы и директории, inode, контроль доступа.}

Все файлы, с которыми могут манипулировать пользователи, располагаются в файловой системе, представляющей собой дерево, промежуточные вершины которого соответствуют каталогам, и листья - файлам и пустым каталогам. 

Каждый каталог и файл файловой системы имеет уникальное полное имя. Каталог, являющийся корнем файловой системы, в любой файловой системе имеет предопределенное имя "/" (слэш). Полное имя файла, например, /bin/sh означает, что в корневом каталоге должно содержаться имя каталога bin, а в каталоге bin должно содержаться имя файла sh. 
Коротким или относительным именем файла (relative pathname) называется имя (возможно, составное), задающее путь к файлу от текущего рабочего каталога. 
В каждом каталоге содержатся два специальных имени, имя ".", именующее сам этот каталог, и имя "..", именующее "родительский" каталог данного каталога, т.е. каталог, непосредственно предшествующий данному в иерархии каталогов.

\textbf{Incode.} \href{https://freehost.com.ua/faq/articles/inode-v-linux--chto-eto-takoe/}{Подробнее туту}

Для ОС Linux есть такое понятие, как Inode или индексный дескриптор. Индексные дескрипторы в файловых системах (таких как ext4) предназначены для хранения метаданных о файлах, каталогах и др. объектах.

Представим иерархическую структуру файловой системы Линукс в упрощенном виде:
\begin{enumerate}
	\item верхушка иерархии — это сама файловая система;
	\item уровнем ниже идут имена файлов (папок);
	\item имена файлов ссылаются на inode;
	\item inode ссылаются на физические данные.
\end{enumerate}

Таким образом, файловая система Linux содержит блоки для хранения данных и inodes. По умолчанию, в ext4, 4092 байта — это размер одного блока. Любой файл в каталоге ОС Linux имеет имя файла и номер inode. Пользователь может узнать метаданные этого файла, указав его номер inode.

Как правило, каждый Inode хранит следующие атрибуты:
\begin{enumerate}
	\item размер;
	\item владелец;
	\item дата/время;
	\item разрешения и контроль доступа;
	\item расположение на диске;
	\item тип файла;
	\item количество ссылок;
	\item дополнительные метаданные о файле.
\end{enumerate}

Таблица с Inode размещена в начале раздела диска, после нее уже идут блоки с данными. Директории в ОС Линукс рассматриваются как Inode типа «директория», в них содержатся списки имен файлов и номера их inode.

Для ОС Линукс также важно понятие о ссылках (символические и жесткие ссылки).

Символическая ссылка — это по своей сути «ярлык», она содержит адрес файла.

Если вы попытаетесь открыть такую ссылку, то откроется соответствующий файл (папка). Если удалить данный файл (папку), символическая ссылка не удалится, но при попытке открыть ее — она приведет «в никуда». Номер Inоdе «символической ссылки» отличается от номера inоde того файла, на который она ссылается.

Если же вы используете «жесткие ссылки», то ваш конкретный файл находится только в определенном месте жесткого диска, а уже именно на это место и ведут сразу несколько ссылок. Каждая «жесткая ссылка» представлена в виде отдельного файла, однако все такого вида ссылки указывают на один и тот же участок диска (даже если мы перемещаем этот файл между разными каталогами). Жесткая ссылка в системе идет под таким же номером Inode, как и фaйл, на который она ссылается.


\subsubsection{Файловые системы (реализация в ядре): VFS, блочные устройства, планировщик IO}

\href{https://docstore.mik.ua/unix2/glava_13.htm}{Подробнее тут (про блоки)}

Файловая система обычно размещается на дисках или других устройствах внешней памяти, имеющих блочную структуру. Кроме блоков, сохраняющих каталоги и файлы, во внешней памяти поддерживается еще несколько служебных областей. 

В мире UNIX существует несколько разных видов файловых систем со своей структурой внешней памяти. Наиболее известны традиционная файловая система UNIX System V (s5) и файловая система семейства UNIX BSD (ufs). Файловая система s5 состоит из четырех секций (рисунок 2.2,a). В файловой системе ufs на логическом диске (разделе реального диска) находится последовательность секций файловой системы 

Кратко опишем суть и назначение каждой области диска:
\begin{enumerate}
	\item Boot-блок содержит программу раскрутки, которая служит для первоначального запуска ОС UNIX. В файловых системах s5 реально используется boot-блок только корневой файловой системы. В дополнительных файловых системах эта область присутствует, но не используется.
	\item Суперблок - это наиболее ответственная область файловой системы, содержащая информацию, которая необходима для работы с файловой системой в целом. Суперблок содержит список свободных блоков и свободные i-узлы (information nodes - информационные узлы). В файловых системах ufs для повышения устойчивости поддерживается несколько копий суперблока. Каждая копия суперблока имеет размер 8196 байт, и только одна копия суперблока используется при монтировании файловой системы (см. ниже). Однако, если при монтировании устанавливается, что первичная копия суперблока повреждена или не удовлетворяет критериям целостности информации, используется резервная копия. 
	\item Блок группы цилиндров содержит число i-узлов, специфицированных в списке i-узлов для данной группы цилиндров, и число блоков данных, которые связаны с этими i-узлами. Размер блока группы цилиндров зависит от размера файловой системы. Для повышения эффективности файловая система ufs старается размещать i-узлы и блоки данных в одной и той же группе цилиндров. 
	\item Список i-узлов (ilist) содержит список i-узлов, соответствующих файлам данной файловой системы. Максимальное число файлов, которые могут быть созданы в файловой системе, определяется числом доступных i-узлов. В i-узле хранится информация, описывающая файл: режимы доступа к файлу, время создания и последней модификации, идентификатор пользователя и идентификатор группы создателя файла, описание блочной структуры файла и т.д.
	\item Блоки данных - в этой части файловой системы хранятся реальные данные файлов. В случае файловой системы ufs все блоки данных одного файла пытаются разместить в одной группе цилиндров. Размер блока данных определяется при форматировании файловой системы командой mkfs и может быть установлен в 512, 1024, 2048, 4096 или 8192 байтов. 
\end{enumerate}

\textbf{Виртуальная файловая система (VFS)}~---~уровень абстракции поверх конкретной реализации файловой системы. Целью VFS является обеспечение единообразного доступа клиентских приложений к различным типам файловых систем. VFS может быть использована для доступа к локальным устройствам и файлам (fat32, ext4, ntfs), сетевым устройствам и файлам на них (nfs), а также к устройствам, не предназначенным для хранения данных (procfs[1]). VFS декларирует программный интерфейс между ядром и конкретной файловой системой, таким образом, для добавления поддержки новой файловой системы не требуется вносить изменений в ядро операционной системы.

\href{https://russianblogs.com/article/825789153/}{Тут про VFS}

\textbf{Планировщик IO.}

\href{https://xakep.ru/2014/05/11/input-out-linux-planning/}{Подробнее тут}

Планировщик IO отвечает за распределение дисковых операций по процессам.  В ранних ядрах Linux (как минимум в ядре 2.4) существовал только один планировщик — Linus Elevator. Он был слишком примитивным, и поэтому в ядре 2.6 появились еще три планировщика, часть из которых ныне уже ушла в небытие. Таким образом, сейчас в ядре существует три планировщика, а в ближайшее время, возможно, прибавится еще и четвертый:
\begin{enumerate}
	\item \emph{NOOP}~---~наиболее простой планировщик. Он банально помещает все запросы в очередь FIFO и исполняет их вне зависимости от того, пытаются ли приложения читать или писать. Планировщик этот, тем не менее, пытается объединять однотипные запросы для сокращения операций ввода/вывода.
	\item \emph{CFQ} был разработан в 2003 году. Заключается его алгоритм в следующем. Каждому процессу назначается своя очередь запросов ввода/вывода. Каждой очереди затем присваивается квант времени. Планировщик же циклически обходит все процессы и обслуживает каждый из них, пока не закончится очередь либо не истечет заданный квант времени. Если очередь закончилась раньше, чем истек выделенный для нее квант времени, планировщик подождет (по умолчанию 10 мс) и, в случае напрасного ожидания, перейдет к следующей очереди. Отмечу, что в рамках каждой очереди чтение имеет приоритет над записью.
	\item \emph{Deadline} в настоящее время является стандартным планировщиком, был разработан в 2002 году. В основе его работы, как это ясно из названия, лежит предельный срок выполнения — то есть планировщик пытается выполнить запрос в указанное время. В дополнение к обычной отсортированной очереди, которая появилась еще в Linus Elevator, в нем есть еще две очереди — на чтение и на запись. Чтение опять же более приоритетно, чем запись. Кроме того, запросы объединяются в пакеты. Пакетом называется последовательность запросов на чтение либо на запись, которая идет в сторону больших секторов («алгоритм лифта»). После его обработки планировщик смотрит, есть ли запросы на запись, которые не обслуживались длительное время, и в зависимости от этого решает, создавать ли пакет на чтение либо же на запись.
	\item \emph{BFQ (Budget Fair Queueing)}~---~относительно новый планировщик. Базируется на CFQ. Если не вдаваться в технические подробности, каждой очереди (которая, как и в CFQ, назначается попроцессно) выделяется свой «бюджет», и, если процесс интенсивно работает с диском, данный «бюджет» увеличивается.
\end{enumerate}

\subsection{Комбинаторные объекты. Коды Грея. Формула включения-исключения. Лемма Бернсайда и Теорема Пойа. Числа Стирлинга. Подсчёт деревьев. Метод производящих функций.}

\subsubsection{Коды Грея}
Код Грея - код для элементов упорядоченного множества (например, чисел от 1 до $n$), такой что расстояние Хэмминга между кодами соседних элементов = 1.
\begin{figure}[H]
	\centering
	\includegraphics[scale=0.4]{images/grey.png}
\end{figure}

Коды Грея легко получаются из двоичных чисел путём побитовой операции «Исключающее ИЛИ» с тем же числом, сдвинутым вправо на один бит и в котором старший разряд заполняется нулём. Следовательно, $i$-й бит кода Грея $G_i$ выражается через биты двоичного кода $B_i$ следующим образом:
\begin{align*}
G_i = B_i \oplus B_i + 1
\end{align*}

где $\oplus$ — операция «исключающее ИЛИ»; биты нумеруются справа налево, начиная с младшего. 

Декодинг происходит по формуле 
\begin{align*}
	B_i = B_{i + 1} \oplus G_i
\end{align*}

Код Грея назван «рефлексивным» (отражённым) из-за того, что первая половина значений при изменении порядка эквивалентна второй половине, за исключением старшего бита. Старший бит просто инвертируется. При делении каждой новой половины пополам это свойство сохраняется.

Код Грея используется в тех случаях, когда мы медленно считываем значения, а они меняются. Представим себе, что код (обычный двоичный) перескакивает $3\rightarrow4$, или $011_2 \rightarrow 100_2$. Если из-за несовершенства считывателя мы прочитаем первый бит от $011$, а остальные два — от $100$, мы получим $000_2=0$ — число, далёкое от реальных значений. В коде Грея никаких посторонних значений не будет: перескок будет в одном разряде, $010_G \rightarrow 110_G$, и мы считаем либо старое $010_G=3$, либо новое $110_G=4$. 

\subsubsection{Формула включения-исключения}

\T{
	Пусть $A = \bigcup\limits_{i = 1}^nA_i$, тогда по формуле включения-исключения:
	\begin{align*}
		|A| = \sum\limits_{I \in 2^n - 1}(-1)^{|I| + 1}|\bigcap_{j\in I}A_j|
	\end{align*}

	где $N = \{1, \ldots n\}$, $2^N - 1$ - множество всех непустых подмножеств $N$. 
}
\begin{proof}
	\href{https://neerc.ifmo.ru/wiki/index.php?title=%D0%A4%D0%BE%D1%80%D0%BC%D1%83%D0%BB%D0%B0_%D0%B2%D0%BA%D0%BB%D1%8E%D1%87%D0%B5%D0%BD%D0%B8%D1%8F-%D0%B8%D1%81%D0%BA%D0%BB%D1%8E%D1%87%D0%B5%D0%BD%D0%B8%D1%8F}{Доказательство}.
\end{proof}

Формулу включений исключений можно интерпретировать в вероятностном смысле, нужно лишь заменить множества на события, а мощности на вероятности. 

\subsubsection{Лемма Бернсайда}
Я чета в ахуе немного, это из теории групп. Про орбиты. 

\D{
	Говорят, что группа $G$ действует на множестве $M$ слева, если задано отображение $G \times M \rightarrow M$, такое что
	\begin{enumerate}
		\item $g(hm) = (gh)m$ для всех $g, h \in G, m \in M$
		\item $em = m$, где $e$ - нейтральный элемент $G$
	\end{enumerate}
}

\D{
	Подмножество
	\begin{align*}
		Gm = \{gm \mid g \in G\} \subset M
	\end{align*}

	Называется орбитой элемента $m$
}

\T[Лемма Бернсайда]{
	Пусть $G$ — конечная группа, действующая на множестве $X$. Тогда число орбит действия равно среднему количеству точек, фиксированных точек в $X$ элементами $G$.
	
	Точнее, для любого элемента $g \in G$ будем обозначать через $X^g$ множество элементов $X$, оставляемых на месте $g$, то есть
	\begin{align*}
		X^g = \{x \in X \mid gx = x\}
	\end{align*} 

	Тогда
	\begin{align*}
		|X/G| = \frac{1}{|G|}\sum\limits_{g \in G}|X^g|
	\end{align*}
	
	где $|X/G|$ - обозначает число орбит действия.
}
\begin{proof}
	\href{https://neerc.ifmo.ru/wiki/index.php?title=%D0%9B%D0%B5%D0%BC%D0%BC%D0%B0_%D0%91%D1%91%D1%80%D0%BD%D1%81%D0%B0%D0%B9%D0%B4%D0%B0_%D0%B8_%D0%A2%D0%B5%D0%BE%D1%80%D0%B5%D0%BC%D0%B0_%D0%9F%D0%BE%D0%B9%D0%B0#.D0.A2.D0.B5.D0.BE.D1.80.D0.B5.D0.BC.D0.B0_.D0.9F.D0.BE.D0.B9.D0.B0}{Доказательство}.
\end{proof}


\subsubsection{Теорема Пойа}
\href{https://e-maxx.ru/algo/burnside_polya}{Тут} максимально просто написано и про Пойа и про Бернсайда.

\subsubsection{Числа Стирлинга}
\begin{itemize}
	\item \textit{Первого рода.} Количество перестановок из $n$ элементов с $k$ циклами.
	
	Можно посчитать рекурсивно:
	\begin{align*}
		c(n, k) = c(n - 1, k - 1) + (n - 1)\cdot c(n - 1, k)
	\end{align*} 
	\item \textit{Второго рода.} Количество неупорядоченных разбиений $n$-элементного множества на $k$ непустых подмножеств.
	
	Можно посчитать рекурсивно:
	\begin{align*}
		S(n, k) = S(n - 1, k - 1) + k\cdot S(n - 1, k)
	\end{align*}
	
	Есть явная формула:
	\begin{align*}
		S(n, k) = \frac{1}{k!}\sum\limits(-1)^{k + j}{k \choose j}j^n
	\end{align*}
\end{itemize}

\subsubsection{Подсчет деревьев}
\D{
	$n$-ое число Каталана:
	\begin{align*}
		C_n = \frac{1}{n + 1}\binom{2n}{n}
	\end{align*}

	Числа Каталана - удовлетворяют рекуррентному соотношению
	\begin{align*}
		C_0 = 1\\
		C_n = \sum\limits_{i =0}^{n - 1}C_iC_{n - 1 - i}
	\end{align*}
	Например, это количество правильных скобочных последовательностей длины $2n$.
}

Более подробно на \href{https://ru.wikipedia.org/wiki/%D0%A7%D0%B8%D1%81%D0%BB%D0%B0_%D0%9A%D0%B0%D1%82%D0%B0%D0%BB%D0%B0%D0%BD%D0%B0}{Вики}.

\T{
	Количество неизоморфных упорядоченных бинарных деревьев с корнем и $n + 1$ листьями = $n$-ому числу Каталана. 
	
	Здесь \textit{упорядоченные} означает, что ребра, выходящие из каждой вершины, упорядочены.
}

\D{
	Помеченное дерево c $n$ вершинами - дерево c $n$ вершинами, вершинам которого взаимно однозначно соответствуют числа от 1 до $n$.
}

\T[Кэли]{
	Число помеченных деревьев с $n$ вершинами равняется $n^{n - 2}$.
}
\begin{proof}
	Доказательство и еще много всего интересного смотри \href{https://neerc.ifmo.ru/wiki/index.php?title=%D0%9F%D0%BE%D0%B4%D1%81%D1%87%D0%B5%D1%82_%D0%B4%D0%B5%D1%80%D0%B5%D0%B2%D1%8C%D0%B5%D0%B2}{тут}. 
\end{proof}

\subsubsection{Метод производящих функций}
Вот \href{https://neerc.ifmo.ru/wiki/index.php?title=%D0%9F%D1%80%D0%BE%D0%B8%D0%B7%D0%B2%D0%BE%D0%B4%D1%8F%D1%89%D0%B0%D1%8F_%D1%84%D1%83%D0%BD%D0%BA%D1%86%D0%B8%D1%8F}{отсюда} можно прочитать определение и пример для решения рекуррент


\subsection{Параллельное программирование. Консенсусное число. Стек Трайбера. Очередь Майкла-Скотта.}

\href{https://www.babichev.org/tpmtp/Lecture09.pdf}{Про консенсус}.
\D{
	Задача консенсуса: есть $N$ потоков, нужно чтоб они все пришли к согласию по поводу какого-то одного значения.
}

\D{
	Любой последовательный объект можно реализовать без ожидания (wait-free) для N потоков используя консенсусный протокол для N потоков.
	
	Такое построение называется универсальная конструкция
}	

\href{https://neerc.ifmo.ru/wiki/index.php?title=%D0%A1%D1%82%D0%B5%D0%BA_%D0%A2%D1%80%D0%B0%D0%B9%D0%B1%D0%B5%D1%80%D0%B0}{Стек Трайбера}

\href{https://neerc.ifmo.ru/wiki/index.php?title=%D0%9E%D1%87%D0%B5%D1%80%D0%B5%D0%B4%D1%8C_%D0%9C%D0%B0%D0%B9%D0%BA%D0%BB%D0%B0_%D0%B8_%D0%A1%D0%BA%D0%BE%D1%82%D1%82%D0%B0}{Очередь Майкла-Скотта}.

\subsection{Компьютерные сети. Модель OSI. Устройства коммутации и маршрутизации. Протоколы Ethernet, IP, TCP, UDP.}

\href{https://www.dropbox.com/sh/4st5b16mvdf8gkj/AAAI9sbKs_C3TFgRbqWGrAeca/Programming/16%20%D0%9A%D0%BE%D0%BC%D0%BF%D1%8C%D1%8E%D1%82%D0%B5%D1%80%D0%BD%D1%8B%D0%B5%20%D1%81%D0%B5%D1%82%D0%B8.pdf?dl=0}{dropbox}

\subsection{Контекстно-свободные грамматики. Эффективные методы разбора: LL(k)-, LR(k)- и LALR-грамматики.}

\subsection{Комбинаторная теория сложности. Временная и емкостная сложность. Сложностные классы P, NP, PS. Сведение, NP-полные задачи.}

\subsection{Марковские цепи, Эргодические цепи, Регулярные цепи. Алгоритм Витерби.}

\subsection{Линейные структуры данных. Амортизационный анализ. Поисковые структуры данных. Запросы на отрезках. Персистентные структуры данных.}

\subsection{Графы. Обход графов. Поиск кратчайших путей. Задача о паросочетании, максимальном потоке и максимальном потоке минимальной стоимости.}

\subsection{Строки. Поиск строки в подстроке. Бор, алгоритм Ахо-Корасика. Суффиксные массивы и деревья.}

\subsection{Постановка задачи линейного программирования. Двойственность задачи ЛП.}

\subsection{Градиентные методы. Метод сопряжения градиентов. Минимизация квадратичных функций. Метод Ньютона.}

\newpage

\section{Программирование и вычислительная техника}

\subsection{Числовые ряды. Абсолютная и условная сходимость. Признаки сходимости числовых рядов.}

\subsubsection{Числовые ряды}

$\sum\limits_{k=1}^{\infty} a_{k} = a_{1} + a_{2} + a_{3} + \cdots$ -- числовой ряд

Сходимость ряда означает существование конечной суммы, т.е. $\sum\limits_{k=1}^{\infty} a_{k} = S$ где $S$ -- конечное число, иначе ряд считается расходящимся.

\subsubsection{Абсолютная и условная сходимость}

Ряд $\sum\limits_{k=1}^{\infty} a_{k}$ называется {\bf абсолютно} сходящимся, если сходится ряд из модулей $\sum\limits_{k=1}^{\infty} |a_{k}|$, иначе ряд называется {\bf условно} сходящимся

\subsubsection{Признаки сходимости числовых рядов} 

{\bf Знакоположительные ряды} (ряды с положительными членами):

Критерий сходимости знакоположительных рядов-- знакоположительный ряд $\sum\limits_{k=1}^{\infty} a_{k}$ сходится тогда и только тогда, когда последовательность его частичных сумм $S(n) = \sum\limits_{k=1}^{k=n}a_{k}$ ограничена сверху

{\bf Док-во:}

=>: ряд сходится, значит последовательность частичных сумм $\S(n) =\sum\limits_{k=1}^{n} a_{k}$ имеет предел равный $\sum\limits_{k=1}^{\infty} a_{k} = S$

<=: Пусть дан положительный ряд и последовательность частичных сумм ограничена сверху, заметим что последовательность частичных сумм неубывающая:
$$S_{n + 1} - S_{n} = a_{n + 1} \ge 0$$. Используя свойство из теоремы о монотонной последовательности получаем, что т.к. последовательность частичных сумм монотонно не убывает и ограничена сверху, значит она сходится и потому ряд сходится по определению.

{\bf Признак сравнения с мажорантой}

Пусть даны два положительных ряда $\sum\limits_{k=1}^{\infty} a_{k}$ и $\sum\limits_{k=1}^{\infty} b_{k}$. Если начиная с некоторого номера $n > N$ выполняется неравенство $0 \le a_n \le b_n$, то:

\begin{itemize}
	\item из сходимости рядя $\sum\limits_{k=1}^{\infty} b_{k}$ следует сходимость ряда $\sum\limits_{k=1}^{\infty} a_{k}$
	\item из расходимости ряда $\sum\limits_{k=1}^{\infty} a_{k}$ следует расходимость $\sum\limits_{k=1}^{\infty} b_{k}$
\end{itemize}

{\bf Док-во:}

Из неравенств на члены следует неравенство на частичные суммы $0 \le S_n \le \sigma_n$, дальше очев.


{\bf Признак Раабе}

Если для ряда $\sum\limits_{k=1}^{\infty} a_{k}$ существует предел $$R = \lim\limits_{n \rightarrow \infty} n (\frac{a_n}{a_{n+1}} - 1)$$, то при $R > 1$ ряд сходится, а при $R < 1$ -- расходится. Если $R = 1$, то жанный признак не говорит ничего.

{\bf Признак Гаусса}

Пусть для знакоположительного ряда $\sum\limits_{n=1}^{\infty} a_{n}$ отношение $\frac{a_n}{a_{n + 1}}$ может быть представлено в виде $$\frac{a_n}{a_{n + 1}} = \lambda + \frac{\mu}{n} + \frac{\theta_n}{n^2}$$, где $\lambda, \mu$ -- постоянные, а последовательность $\theta_n$ ограничена. Тогда 
\begin{itemize}
	\item ряд расходится если либо $\lambda > 1$, либо $\lambda = 1, \mu > 1$
	\item ряд расходится, если либо $\lambda < 1$, либо $\lambda = 1, \mu \le 1$
\end{itemize}


{\bf Знакопеременные ряды}

\D{Знакопеременными называются ряды, члены которых могут (стоять) быть как положительными, так и отрицательными.}


{\bf Признак Даламбера}

Слабее признака Коши, но зато проще

Если существует $\lim\limits_{n \rightarrow \infty}|\frac{a_{n + 1}}{a_n}| = r$, то 

\begin{itemize}
	\item если $r < 1$, то ряд абсолютно сходится
	\item если $r > 1$, то ряд расходится
	\item если $r = 1$, то данный признак ничего не говорит (сука)
\end{itemize}

{\bf Док-во:}

1. Пусть начиная с некоторого номера N верно неравенство $|\frac{a_{n+1}}{a_n}| \le q, 0 < q < 1$. Тогда перемножив члены начиная с N будем иметь что $\frac{a_{N+n}}{a_N} \le q^n$ откуда $|a_{N+n}| \le |a_{N}q^n|$, значит ряд $|a_{N+1}| + |a_{N+2}| + ...$ меньше бесконечной суммы убывающей геометрической прогрессии, поэтому он сходится

2. $|\frac{a_{n + 1}}{a_n}| \ge 1$ (с некоторого N), тогда можно записать $|a_{n+1}| \ge |a_n|$ значит модуль членов $a$ не стремится к 0 на бесконечности, значит последовательность не стремится к 0 а значит ряд не сходится.

3. Если просто меньше 1 до там хуйня какая-то мне впадлу
\\

{\bf Радикальный признак Коши} (ебаная оппозиция)

Если существует $\lim\lim\limits_{n \rightarrow \infty} \sqrt[n]{|a_n|} = r$, то

\begin{itemize}
	\item если $r < 1$ то ряд сходится абсолютно
	\item если $r > 1$ то ряд расходится
	\item если $r = 1$ то хз (опять??)
\end{itemize}

{\bf Док-во:} \href{https://ru.wikipedia.org/wiki/%D0%A0%D0%B0%D0%B4%D0%B8%D0%BA%D0%B0%D0%BB%D1%8C%D0%BD%D1%8B%D0%B9_%D0%BF%D1%80%D0%B8%D0%B7%D0%BD%D0%B0%D0%BA_%D0%9A%D0%BE%D1%88%D0%B8}{тут}
\\

{\bf Признак Лейбница}

Пусть для знакочередующегося ряда $$S = \sum\limits_{n=1}^{\infty}(-1)^{n-1}a_n, a_n \ge 0$$
выполняются следующие условия

\begin{itemize}
	\item С некоторого $N$ последовательность $a$ монотонно убывает, т.е. $a_{n+1} \le a_n$
	\item $\lim\limits_{n \rightarrow \infty}a_n = 0$
\end{itemize}

Тогда такой ряд сходится

{\bf Док-во:} \href{https://ru.wikipedia.org/wiki/%D0%A2%D0%B5%D0%BE%D1%80%D0%B5%D0%BC%D0%B0_%D0%9B%D0%B5%D0%B9%D0%B1%D0%BD%D0%B8%D1%86%D0%B0_%D0%BE_%D1%81%D1%85%D0%BE%D0%B4%D0%B8%D0%BC%D0%BE%D1%81%D1%82%D0%B8_%D0%B7%D0%BD%D0%B0%D0%BA%D0%BE%D1%87%D0%B5%D1%80%D0%B5%D0%B4%D1%83%D1%8E%D1%89%D0%B8%D1%85%D1%81%D1%8F_%D1%80%D1%8F%D0%B4%D0%BE%D0%B2}{здесь}\\

{\bf Признак Абеля}

\T {Числовой ряд $\sum\limits_{n=1}^{\infty}a_nb_n$ сходится, если выполнены следующие условия
	
	\begin{itemize}
		\item Последовательность \{$a_n$\} монотонна и ограничена
		\item Ряд $\sum\limits_{n=1}^{\infty}b_n$ сходится
	\end{itemize}
}
{\bf Proof:} \href{https://ib.mazurok.com/2015/06/16/%D0%BF%D1%80%D0%B8%D0%B7%D0%BD%D0%B0%D0%BA%D0%B8-%D0%B0%D0%B1%D0%B5%D0%BB%D1%8F-%D0%B8-%D0%B4%D0%B8%D1%80%D0%B8%D1%85%D0%BB%D0%B5/}{вот}\\

{\bf Признак Дирихле}

\T{Пусть выполнены условия:
	\begin{itemize}
		\item последовательность частичных сумм $B_n = \sum\limits_{k=1}^{n}$ ограничена
		\item последовательность $a_n$, начиная с некоторого номера, монотонно убывает $a_n \ge a_{n+1}$
		\item $\lim\limits_{n\rightarrow\infty}a_n = 0$
	\end{itemize}
	Тогда ряд $\sum\limits_{n=1}^{\infty}a_nb_b$ сходится
}

{\bf Proof:} \href{https://ib.mazurok.com/2015/06/16/%D0%BF%D1%80%D0%B8%D0%B7%D0%BD%D0%B0%D0%BA%D0%B8-%D0%B0%D0%B1%D0%B5%D0%BB%D1%8F-%D0%B8-%D0%B4%D0%B8%D1%80%D0%B8%D1%85%D0%BB%D0%B5/}{вот}\\

\subsection{Архитектура ЭВМ. Кэш-память. Многоуровневая организация кэш-памяти. Протоколы когерентности кэш-памяти}

\subsubsection{Архитектура ЭВМ}
\textbf{Архитектура ЭВМ}~---~
это модель, устанавливающая принципы организации вычислительной системы, состав, 
порядок и взаимодействие основных частей ЭВМ, функциональные возможности, 
удобство эксплуатации, стоимость, надежность.

\subsubsection{Кэш-память. Многоуровневая организация кэш-памяти. Протоколы когерентности кэш-памяти}
\textbf{Кэширование}~---~это использование дополнительной быстродействующей памяти (кэш-памяти) для хранения копий блоков информации из основной (оперативной) памяти, вероятность обращения к которым в ближайшее время велика.

\textbf{Аспекты кэшей}:
\begin{enumerate}
	\item Кэш-линии~---~вся память выровнена и разбита на непересекающиеся отрезки по 64 байта (последние 6 бит не используются как тег в ассоциативном кэше). 
	\item Кэши делятся на уровни (ближе к процессору $\Rightarrow$ больше скорость, меньше размер). 
	\begin{enumerate}
		\item L1 (32k, делится на данные и команды, Associativity = 4)
		\item L2 (256k, Associativity = 8)
		\item L2 (8Mb, Associativity = 16)
	\end{enumerate}
	\item Ассоциативность. Полная асс. - это когда мы просто пишем в кэш и для поиска нужного адреса нужно бежать по всем линиям. Асс =1 - это когда мы просто мапим по хвосту адреса (тегу) в таблицу. (типа хеш-таблица). Тогда быстро искать, но часто будем промахиваться. Если Acc = k - то мапим в корзины по k и получаем компромисс.
	\item  Эксклюзивность/инклюзивность - данные хранятся только в одном кэш (- эффективность из-за поиска, + размер) или они дублируются в уровнях (в памяти) ниже. (+ скорость, - размер)
	\item Когда у нас много процессоров, то возникает необходимость использовать протоколы когерентности. (MSI, MESI, MESIF, MOESI). Это необходимо, чтобы один процессор мог знать о том, что данные изменились только в кеше одно из его соседей. Для этого вводятся разные состояния владения памятью (invalid, shared, modified + owned/forward, exclusive), чтобы как можно более тоньше развести их и поменьше сбрасывать кэши.
\end{enumerate}

\subsubsection{Многоуровневая организация кэш-памяти (подробнее)}

\emph{ВОДА:}

Современные технологии позволяют разместить КЭШ-память и ЦП на общем кристалле. Такая внутренняя КЭШ-память строится по технологии статического ОЗУ и является наиболее быстродействующей. 

Емкость ее обычно не превышает 64 Кбайт. Попытки увеличения емкости обычно приводят к снижению быстродействия, главным образом, из-за усложнения схем управления и дешифрации адреса. 

Общую емкость КЭШ-памяти ЭВМ увеличивают за счет второй (внешней) КЭШ-памяти, расположенной между внутренней КЭШ- памятью и ОЗУ. Такая система известна под названием двухуровневой, где внутренней КЭШ-памяти отводится роль первого уровня (L1), а внешней — второго уровня (L2). Емкость L2 может быть значительной (до 1 МБ). 

При доступе к памяти ЦП сначала обращается к КЭШ-памяти первого уровня. В случае промаха производится обращение к КЭШ-памяти второго уровня. Если информация отсутствует и в L2, выполняется обращение к ОЗУ и соответствующий блок заносится сначала в L2, а затем и в L1. Благодаря такой процедуре часто запрашиваемая информация может быть быстро восстановлена из КЭШ-памяти второго уровня. Для ускорения обмена информацией между ЦП и L2 между ними часто вводят специальную шину, так называемую шину заднего плана, в отличие от шины переднего плана, связывающую ЦП с основной памятью. 

Количество уровней КЭШ-памяти не ограничивается двумя. В некоторых ЭВМ можно встретить КЭШ-память третьего уровня (L3). Ведутся активные дискуссии о введении также и КЭШ-памяти четвертого уровня (L4). Характер взаимодействия очередного уровня с предшествующим аналогичен описанному для L1 и L2. Таким образом, можно говорить об иерархии КЭШ-памяти. Каждый последующий уровень характеризуется большей емкостью, меньшей стоимостью, но и меньшим быстродействием, хотя оно все же выше, чем у ЗУ основной памяти.

\subsection{С++. Процесс компиляции и линковки. .cpp, .h, .i, .o файлы.}
Понятно и подробно описано: 
\href{https://habr.com/ru/post/478124/}{https://habr.com/ru/post/478124/}

Кратко и структурировано: \href{https://server.179.ru/tasks/cpp/total/105.html}{https://server.179.ru/tasks/cpp/total/105.html}

\textbf{Компиляция}~---~трансляция программы, составленной на исходном языке высокого уровня, в эквивалентную программу на низкоуровневом языке, близком машинному коду (абсолютный код, объектный модуль, иногда на язык ассемблера). Входной информацией для компилятора (исходный код) является описание алгоритма или программа на объектно-ориентированном языке, а на выходе компилятора—эквивалентное описание алгоритма на машинно-ориентированном языке (объектный код).

\subsubsection{Заголовочные файлы (.h)}

%Целью заголовочных файлов является удобное хранение набора объявлений объектов для их последующего использования в других программах. 
В языках программирования Си и C++ заголовочные файлы~---~основной способ подключить к программе типы данных, структуры, прототипы функций, перечисляемые типы и макросы, используемые в другом модуле. По умолчанию используется расширение .h; иногда для заголовочных файлов языка C++ используют расширение .hpp.

Чтобы избежать повторного включения одного и того же кода, используются директивы \#ifndef, \#define, \#endif.

Заголовочный файл в общем случае может содержать любые конструкции языка программирования, но на практике исполняемый код (за исключением inline-функций в C++) в заголовочные файлы не помещают.

\subsection{С++. Жизненный цикл объектов в С++. RAII.}

\subsection{Java. Устройство сборщика мусора в JVM.}

\href{https://medium.com/nuances-of-programming/%D1%81%D0%B1%D0%BE%D1%80%D0%BA%D0%B0-%D0%BC%D1%83%D1%81%D0%BE%D1%80%D0%B0-%D0%B2-java-%D1%87%D1%82%D0%BE-%D1%8D%D1%82%D0%BE-%D1%82%D0%B0%D0%BA%D0%BE%D0%B5-%D0%B8-%D0%BA%D0%B0%D0%BA-%D1%80%D0%B0%D0%B1%D0%BE%D1%82%D0%B0%D0%B5%D1%82-%D0%B2-jvm-25bb2570b44c}{Подробнее тут}
Сборка мусора в Java~---~это процесс, с помощью которого программы Java автоматически управляют памятью. Java-программы компилируются в байт-код, который запускается на виртуальной машине Java (JVM).

Когда Java-программы выполняются на JVM, объекты создаются в куче, которая представляет собой часть памяти, выделенную для них.

Пока Java-приложение работает, в нем создаются и запускаются новые объекты. В конце концов некоторые объекты перестают быть нужны. Можно сказать, что в любой момент времени память кучи состоит из двух типов объектов:
\begin{enumerate}
	\item Живые~---~эти объекты используются, на них ссылаются откуда-то еще.
	\item Мертвые~---~эти объекты больше нигде не используются, ссылок на них нет.
\end{enumerate}

Сборщик мусора находит эти неиспользуемые объекты и удаляет их, чтобы освободить память.

\textbf{Этапы сборки мусора в Java:}
\begin{enumerate}
	\item Пометка объектов как живых
	\item Зачистка мертвых объектов
	\item Компактное расположение оставшихся объектов в памяти
\end{enumerate} 

\textbf{Сбор мусора по поколениям.}

Сборщики мусора в Java реализуют стратегию сбора мусора поколений, которая классифицирует объекты по возрасту.

Область памяти кучи в JVM разделена на три секции:
\begin{enumerate}
	\item \emph{Молодое поколение.} 
	Вновь созданные объекты начинаются в молодом поколении. Молодое поколение далее подразделяется на две категории.
	\begin{enumerate}
		\item Пространство Эдема~---~все новые объекты начинают здесь, и им выделяется начальная память.
		\item Пространства выживших (FromSpace и ToSpace)~---~объекты перемещаются сюда из Эдема после того, как пережили один цикл сборки мусора.
	\end{enumerate}
	
	\item \emph{Старшее поколение.}
	Объекты-долгожители в конечном итоге переходят из молодого поколения в старшее. Оно также известно как штатное поколение и содержит объекты, которые долгое время оставались в пространствах выживших.
	\item \emph{Постоянное поколение} (\emph{Мета-пространство}, начиная с Java 8).
	Метаданные, такие как классы и методы, хранятся в постоянном поколении. JVM заполняет его во время выполнения на основе классов, используемых приложением. Классы, которые больше не используются, могут переходить из постоянного поколения в мусор.
	
	Начиная с Java 8, на смену пространству постоянного поколения (PermGen) приходит пространство памяти MetaSpace. Реализация отличается от PermGen — это пространство кучи теперь изменяется автоматически.
\end{enumerate}

\subsection{Задача Коши для системы обыкновенных дифференциальных уравнений. Существование и единственность решения. Устойчивость.}

\subsection{Метапрограммирование. Шаблоны и Generics. Частичная специализация шаблонов.}

\subsubsection{Метапрограммирование}

\textbf{Метапрограммирование}~---~создание программ, которые создают другие программы как результат своей работы (либо — частный случай — изменяющие или дополняющие себя во время выполнения).

Метапрограммирование можно разделить на 2 направления: на стадии компиляции (генерация кода) и на стадии выполнения (самомодифицирующийся код).

Первое направление позволяет получить программу при меньших затратах времени и усилий, чем если бы программист писал её вручную. Второе — расширяет возможности программиста.

\textbf{Генерация кода} (это не обязательно)

При этом подходе код программы не пишется вручную, а создается автоматически программой-генератором на основе другой, более простой программы.
Такой подход приобретает смысл, если при программировании вырабатываются различные дополнительные правила (более высокоуровневые парадигмы, выполнение требований внешних библиотек, стереотипные методы реализации определенных функций и пр.). При этом часть кода теряет содержательный смысл и становится лишь механическим выполнением правил. Когда эта часть становится значительной, возникает мысль задавать вручную лишь содержательную часть, а остальное добавлять автоматически. Это и проделывает генератор. Реализуется 2 основными методами:

\begin{enumerate}
	\item  \emph{Шаблоны} (наиболее известные случаи применения — препроцессор C и шаблоны в C++). Решают задачу, если соблюдение «правил» сводится к вставке в программу повторяющихся (или почти повторяющихся) кусков кода. Помимо этого, обладают еще рядом достоинств: например, помогают повторному использованию.
	\item \emph{Внешнеязыковые средства} (пример: генераторы синтаксических и лексических анализаторов lex, yacc, bison). Применяются в случаях, если простых средств вроде шаблонов недостаточно. Язык генератора составляется так, чтобы автоматически или с минимальными усилиями со стороны программиста реализовывать правила парадигмы или необходимые специальные функции. Фактически, это — более высокоуровневый язык программирования, а генератор — не что иное, как транслятор.
	Генераторы пишутся, как правило, для создания специализированных программ, в которых очень значительная часть стереотипна, либо для реализации сложных парадигм (таких, как паттерны проектирования).
\end{enumerate}

\textbf{Самомодифицирующийся код} (это не обязательно)

Возможность изменять или дополнять себя во время выполнения превращает программу в виртуальную машину. Хотя такая возможность существовала уже давно на уровне машинных кодов (и активно использовалась, например, при создании полиморфных вирусов), с метапрограммированием обычно связывают перенос подобных технологий в высокоуровневые языки. Основные методы реализации:
\begin{enumerate}
	\item Интроспекция — представление внутренних структур языка в виде переменных встроенных типов с возможностью доступа к ним из программы. Позволяет во время выполнения смотреть, создавать и изменять определения типов, стек вызовов, обращаться к переменной по имени, получаемому динамически и пр.
	Например, Пространство имён System.Reflection и тип System.Type в .NET; классы Class, Method, Field в Java; представление пространств имен и определений типов через встроенные типы данных в Python
	\item  Интерпретация произвольного кода, представленного в виде строки.
	Существует естественным образом во множестве интерпретируемых языков, например eval() в PHP.
	Для C++ есть библиотека, позволяющая «на лету» компилировать и генерировать исполняемый код (используется урезанный компилятор gcc).
	Принципиальный недостаток технологий этого направления — неприменимость к компилируемым языкам. Можно ввести в такой язык интерпретатор, как в вышеуказанной библиотеке для С++, но это
	практически сведет на нет главное преимущество данных языков — производительность.
\end{enumerate}

\subsubsection{Шаблоны и Generics. Частичная специализация шаблонов.}

\textbf{Шаблоны:}

В языке C++ обобщённое программирование основывается на понятии «шаблон», обозначаемом ключевым словом template.

\begin{lstlisting}[language=C++]
	template < typename T> T max (T x , T y ) {
		if (x < y)
		return y;
		else
		return x;
	}
\end{lstlisting}

Интересное применение нашли шаблоны в языке C++. Оказалось, что шаблоны в этом языке являются тьюринг-полным функциональном языком. Другими словами на шаблонах С++ можно написать программу, реализующую произвольный алгоритм, и эта программа выполнится в момент компиляции.

К примеру, можно предпосчитать $50$-е число Фибоначчи. Тогда во время выполнения программы не придется тратить время на его вычисления. Одной интересной особенностью такого программирования на шаблонах, является встроенный механизм мемоизации (сохранения результата вычисления функции). Это значит, что рекурсивный алгоритм для вычисления k -го числа Фибоначчи работающий «в лоб» сделает порядка k операций (вместо ожидаемых 2k).

\begin{lstlisting}[language=C++]
	template <int i> struct fib { 
		static const int val = fib<i - 1>::val + fib<i - 2>::val;
	};
	template <> struct fib <1> { static const int val = 1; };
	template <> struct fib <2> { static const int val = 1; };
\end{lstlisting}

Но это не самое интересное: программирование на шаблонах С++ позволяет общаться с типом как с обычным объектом. К примеру, можно составить список типов, удалить из него все встроенные типы, а
из оставшегося списка создать объект, который будет унаследован от всех типов из данного списка. Для такого метапрограммирования была написана специальная библиотека MPL (MetaProgramming Library).

\textbf{Generics}

Язык Java предоставляет средства обобщённого программирования, синтаксически основанные на C++. В Java generics (параметризованные типы или родовые типы) имеют мнимое сходство с шаблонами C++ как
по синтаксису, так и по ожидаемому месту их применения (например, в качестве контейнерных классов).

Но это сходство только поверхностное — родовые типы в языке программирования Java почти полностью реализуются в компиляторе, который выполняет проверку типов и выявление типа (type inference)
и, затем, генерирует обычные не параметризованные байткоды. Такая техника реализации, называемая стиранием (когда компилятор использует информацию о родовом типе для контроля типов и удаляет ее перед генерированием байткода), имеет неожиданные, а иногда и непонятные последствия. В то время как родовые типы являются большим шагом на пути к безопасности Java-классов, изучение их использования почти наверняка будет вызывать некоторую озадаченность (а иногда и мучения).

\textbf{Частичная специализация шаблонов}

Если у шаблона класса есть несколько параметров, то можно специализировать его только для одного или нескольких аргументов, оставляя другие неспециализированными. Иными словами, допустимо написать шаблон, соответствующий общему во всем, кроме тех параметров, вместо которых подставлены фактические типы или значения. Такой механизм носит название частичной специализации шаблона класса. Она может понадобиться при определении реализации, более подходящей для конкретного набора аргументов. Например unique\_ptr имеет частичную специализацию для массивов (T[]).

\begin{lstlisting}[language=C++]
	template <typename T>
	class unique_ptr;
	
	template <typename T>
	class unique_ptr<T[]>;
\end{lstlisting}
Частичная специализация шаблона класса — это тоже шаблон, но список параметров здесь отличается от соответствующего списка параметров общего шаблона.

\subsection{Функциональное программирование. Чистые объекты. Функторы. Аппликативы. Монада. Взаимодействие с внешним миром.}

\subsubsection{Функциональное программирование. Чистые объекты.}
\textbf{Функциональное программирование}~---~парадигма программирования, в которой процесс вычисления трактуется как вычисление значений функций в математическом понимании последних (в отличие от функций как подпрограмм в процедурном программировании).

\textbf{Чистыми} называют функции, которые не имеют побочных эффектов ввода-вывода и памяти (они зависят только от своих параметров и возвращают только свой результат). Чистые функции обладают несколькими полезными свойствами, многие из которых можно использовать для оптимизации кода:
\begin{enumerate}
	\item если результат чистой функции не используется, её вызов может быть удалён без вреда для других выражений;
	\item результат вызова чистой функции может быть мемоизирован, то есть сохранён в таблице значений вместе с аргументами вызова;
	\item если нет никакой зависимости по данным между двумя чистыми функциями, то порядок их вычисления можно поменять или распараллелить (говоря иначе, вычисление чистых функций удовлетворяет принципам потокобезопасности);
	\item если весь язык не допускает побочных эффектов, то можно использовать любую политику вычисления. Это предоставляет свободу компилятору комбинировать и реорганизовывать вычисление выражений в программе (например, исключить древовидные структуры).
\end{enumerate}

\subsubsection{Функторы}

\href{https://habr.com/ru/post/183150/}{Объяснение на пальцах}

\href{http://cmc-msu-ai.github.io/haskell-course/lecture/2013/09/07/functors.html}{Подробнее про функторы}

\textbf{Функтором} называется класс типов, который декларирует единственный метод «fmap». Интуитивно, «fmap» применяет функцию a -> b к значению типа f a, чтобы получить значение типа f b. С другой стороны, можно рассматривать «fmap» как функцию высшего порядка, преобразующую «простую» функцию a -> b в «составную» функцию f a -> f b. Важно отметить, что структура значения типа f после применения «fmap» должна оставаться неизменной.

\begin{lstlisting}[language=Haskell]
	class Functor f where
	fmap :: (a -> b) -> f a -> f b
\end{lstlisting}

\subsubsection{Аппликативы.}
\href{http://cmc-msu-ai.github.io/haskell-course/lecture/2013/09/08/applicative-and-monad.html}{Подробнее тут}

Естественным продолжением класса Functor является класс \textbf{Applicative} (аппликативный функтор), определенный в модуле Control.Applicative:

\begin{lstlisting}[language=Haskell]
	class Functor f => Applicative f where
	pure  :: a -> f a
	(<*>) :: f (a -> b) -> f a -> f b
\end{lstlisting}

\textbf{Законы:}
\begin{enumerate}
	\item Помещение тождественной функции в «чистый» контекст и применение к аргументу в контексте не меняет ни значение, ни контекст.
	\begin{lstlisting}[language=Haskell]
		pure id <*> x == x
	\end{lstlisting}
	\item Применение чистой функции к чистому аргументу в контексте «по умолчанию» должно быть эквивалентно применению функции, а затем помещению результата в контекст.
	\begin{lstlisting}[language=Haskell]
		pure f <*> pure x == pure (f x)
	\end{lstlisting}
	\item При применении функции u с побочными эффектами к чистому аргументу y порядок вычисления функции и аргумента неважен.
	\begin{lstlisting}[language=Haskell]
		u <*> pure y == pure ($ y) <*> u
	\end{lstlisting}
	\item Некоторый аналог композиции для аппликативных функторов.
	\begin{lstlisting}[language=Haskell]
		u <*> (v <*> w) == pure (.) <*> u <*> v <*> w
	\end{lstlisting}
\end{enumerate} 

\subsubsection{Монады}
\href{http://cmc-msu-ai.github.io/haskell-course/lecture/2013/09/08/applicative-and-monad.html}{Подробнее тут}
\begin{lstlisting}[language=Haskell]
	class Monad m where
	return :: a -> m a
	(>>=) :: m a -> (a -> m b) -> mb
	(>>) :: m a -> m b -> m b
	m >> n = m >>= \_ -> n
	
	fail :: String -> m a
\end{lstlisting}

Функция return по типу очень напоминает функцию pure из класса Applicative. И, в действительности, return и есть pure, хоть и с не самым удачным названием (return в Haskell совсем не то же, что return в обычных императивных языках вроде C или Java). С математической точки зрения, любая монада является аппликативным функтором (но не наоборот). Но по историческим причинам,в описании класса это не указано.

Как следует из определения, операция ($>>$) является частным случаем операции ($>>=$)

Функция fail осталась в классе по историческим причинам, хотя никакого отношения к монадам реально не имеет.

\textbf{Законы:}
\begin{lstlisting}[language=Haskell]
	return a >>= k = k a
	m >>= return = m
	m >>= (\x -> k x >>= h) = (m >>= k) >>= h
	fmap f xs = xs >>= return . f = liftM f xs
\end{lstlisting}

\subsubsection{Взаимодействие с внешним миром.}
Взаимодействие с "внешним миром" (побочные эффекты вычислений) можно реализовать с помощью специальной монады IO.

Грубое приближение монады IO — это взятие пары с контекстом: "(a, RealWorld)". При операциях с монадой состояние RealWorld может меняться.

Есть операция return, погружающая объект в окружение IO («выпускающая во внешний мир»), а вот обратного преобразования нет.

\begin{lstlisting}[language=Haskell]
	putChar :: Char -> IO () 
\end{lstlisting}~---~берёт символ и возвращает новый «мир», в котором растворился (был напечатан в консоли) этот символ.

\begin{lstlisting}[language=Haskell]
	getChar :: IO Char
\end{lstlisting} --- мы можем получить символ из внешнего мира, но только внутри монады.

Достать его из монады мы не можем, но можем работать сним внутри IO с помощью $>>=$.

В процессе выполнения программы, содержащей IO, объекты типов IO a остаются временно невычисленными, как задумки.

Например, если мы где-то напишем putChar ' a' , то символ не будет тут же напечатан.

Вместо этого нужно дождаться, пока соберётся «главный» объект типа IO (), и уже при его вычислении все операции с внешним миром будут выполнены, причём в правильном порядке.

\subsection{Операционные системы. Процессы: вытесняющая и кооперативная многозадачность, планировщики, многопроцессорные машины.}

\textbf{Кооперативная многозадачность}.

Тип многозадачности, при котором фоновые задачи выполняются только во время простоя основного процесса и только в том случае, если на это получено разрешение основного процесса.

Кооперативную многозадачность можно назвать многозадачностью “второй ступени” поскольку она использует более передовые методы, чем простое переключение задач, реализованное многими известными программами (например, МS-DOS shell из МS-DOS 5.0 при простом переключении активная программа получает все процессорное время, а фоновые приложения полностью замораживаются). При кооперативной многозадачности приложение может захватить фактически столько процессорного времени, сколько оно считает нужным. Все приложения делят процессорное время, периодически передавая управление следующей задаче.

\textbf{ Вытесняющая многозадачность}.

Вид многозадачности, в котором операционная система сама передает управление от одной выполняемой программы другой. Распределение процессорного времени осуществляется планировщиком процессов. Этот вид многозадачности обеспечивает более быстрый отклик на действия пользователя.

Вытесняющая многозадачность~---~это вид многозадачности при котором планирование процессов основывается на абсолютных приоритетах. Процесс с меньшим приоритетом (например пользовательская программа) может быть вытеснен при его выполнении более приоритетным процессом (например системной или диагностической программой). Иногда этот вид многозадачности называют приоритетным.

Каждая работающая программа имеет свое защищенное адресное пространство. Многопоточное выполнение отдельных задач позволяет при задержке в выполнении одного потока не останавливать задачу полностью, а работать со следующим потоком.

\textbf{Планировщик.}

\href{https://habr.com/ru/post/154609/}{Подробнее тут}

Планировщик~---~часть операционной системы, которая отвечает за (псевдо)параллельное выполнения задач, потоков, процессов. Планировщик выделяет потокам процессорное время, память, стек и прочие ресурсы. Планировщик может принудительно забирать управление у потока (например по таймеру или при появлении потока с большим приоритетом), либо просто ожидать пока поток сам явно(вызовом некой системной процедуры) или неявно(по завершении) отдаст управление планировщику.
Первый вариант работы планировщика называется реальным или вытесняющим(preemptive), второй, соответственно, не вытесняющим (non-preemptive).

\textbf{Многопроцессорные машины.}

\href{https://docstore.mik.ua/skbd/glava_10.htm}{Подробнее тут}

\href{https://ru.wikipedia.org/wiki/%D0%9C%D0%BD%D0%BE%D0%B3%D0%BE%D0%BF%D1%80%D0%BE%D1%86%D0%B5%D1%81%D1%81%D0%BE%D1%80%D0%BD%D0%BE%D1%81%D1%82%D1%8C}{В Википедии тоже хорошо написано}

Любая вычислительная система (будь то супер-ЭВМ или персональный компьютер) достигает своей наивысшей производительности благодаря использованию высокоскоростных элементов и параллельному выполнению большого числа операций. Именно возможность параллельной работы различных устройств системы (работы с перекрытием) является основой ускорения основных операций. 

Параллельные ЭВМ часто подразделяются по классификации Флинна на машины типа SIMD (Single Instruction Multiple Data - с одним потоком команд при множественном потоке данных) и MIMD (Multiple Instruction Multiple Data - с множественным потоком команд при множественном потоке данных). Как и любая другая, приведенная выше классификация несовершенна: существуют машины прямо в нее не попадающие, имеются также важные признаки, которые в этой классификации не учтены. В частности, к машинам типа SIMD часто относят векторные процессоры, хотя их высокая производительность зависит от другой формы параллелизма - конвейерной организации машины. Многопроцессорные векторные системы, типа Cray Y-MP, состоят из нескольких векторных процессоров и поэтому могут быть названы MSIMD (Multiple SIMD).

\subsection{Операционные системы. Виртуальная память: MMU, TLB, таблицы страниц, аллокаторы и менеджеры виртуальной памяти.}

\subsubsection{Виртуальная память: MMU, TLB.}

\href{https://habr.com/ru/post/211150/}{Подробнее тут}

Блок управления памятью или устройство управления памятью memory management unit, MMU)~---~компонент аппаратного обеспечения компьютера, отвечающий за управление доступом к памяти, запрашиваемым центральным процессором.

Его функции заключаются в трансляции адресов виртуальной памяти в адреса физической памяти (то есть управление виртуальной памятью), защите памяти, управлении кэш-памятью, арбитражем шины и, в более простых компьютерных архитектурах (особенно 8-битных), переключением блоков памяти. 

Принцип работы современных MMU основан на разделении виртуального адресного пространства (одномерного массива адресов, используемых центральным процессором) на участки одинакового, как правило, несколько килобайт, хотя, возможно, и существенно большего, размера, равного степени 2, называемые страницами. Младшие n бит адреса (смещение внутри страницы) остаются неизменными. Старшие биты адреса представляют собой номер (виртуальной) страницы. MMU обычно преобразует номера виртуальных страниц в номера физических страниц, используя буфер ассоциативной трансляции (Translation Lookaside Buffer, TLB).

Если преобразование при помощи TLB невозможно, включается более медленный механизм преобразования, основанный на специфическом аппаратном обеспечении или на программных системных структурах. Данные в этих структурах, как правило, называются элементами таблицы страниц (page table entries (PTE)), а сами структуры — таблицами страниц (англ. page table (PT)). Конкатенация номера физической страницы со смещением внутри страницы даёт физический адрес.

Элементы PTE или TLB могут также содержать дополнительную информацию: бит признака записи в страницу ( dirty bit), время последнего доступа к странице (accessed bit), какие процессы (пользовательские (user mode) или системные (supervisor mode)) могут читать или записывать данные в страницу, необходимо ли кэшировать страницу.

\subsubsection{Таблицы страниц.}

\href{https://ru.wikipedia.org/wiki/%D0%A2%D0%B0%D0%B1%D0%BB%D0%B8%D1%86%D0%B0_%D1%81%D1%82%D1%80%D0%B0%D0%BD%D0%B8%D1%86}{Википедия}

Таблица страниц~---~это структура данных, используемая системой виртуальной памяти в операционной системе компьютера для хранения сопоставления между виртуальным адресом и физическим адресом. Виртуальные адреса используются выполняющимся процессом, в то время как физические адреса используются аппаратным обеспечением, или, более конкретно, подсистемой ОЗУ. Таблица страниц является ключевым компонентом преобразования виртуальных адресов, который необходим для доступа к данным в памяти.

\subsubsection{Аллокаторы.}

\href{https://habr.com/ru/post/505632/}{Подробнее тут}

Аллокатор или распределитель памяти в языке программирования C++ ~---~ специализированный класс, реализующий и инкапсулирующий малозначимые (с прикладной точки зрения) детали распределения и освобождения ресурсов компьютерной памяти.

Концептуально выделяется пять основных операции, которые можно осуществить над аллокатором:
\begin{enumerate}
	\item \emph{create}~---~создает аллокатор и отдает ему в распоряжение некоторый объем памяти;
	\item \emph{allocate}~---~выделяет блок определенного размера из области памяти, которым распоряжается аллокатор;
	\item \emph{deallocate}~---~освобождает определенный блок;
	\item \emph{free}~---~освобождает все выделенные блоки из памяти аллокатора (память, выделенная аллокатору, не освобождается);
	\item \emph{destroy}~---~уничтожает аллокатор с последующим освобождением памяти, выделенной аллокатору.
\end{enumerate}

\subsubsection{Менеджеры виртуальной памяти.}

Менеджер виртуальной памяти (далее просто «менеджер памяти») ~---~ часть операционной системы, благодаря которой можно адресовать память большую, чем объем физической памяти (ОЗУ).

Благодаря виртуальной памяти можно запускать множество ресурсоёмких приложений, требующих большого объёма ОЗУ. Максимальный объём виртуальной памяти, который можно получить, используя 24-битную адресацию, — 16 мегабайт. С помощью 32-битной адресации можно адресовать до 4 ГБ виртуальной памяти. А 64-битная адресация позволяет работать уже с 16 эксабайтами памяти.

Применение механизма виртуальной памяти позволяет:

\begin{enumerate}
	\item упростить адресацию памяти клиентским программным обеспечением;
	\item рационально управлять оперативной памятью компьютера (хранить в ней только активно используемые области памяти);
	\item изолировать процессы друг от друга (процесс полагает, что монопольно владеет всей памятью).
\end{enumerate}


\subsection{Операционные системы. Виртуальная память: MMU, TLB, таблицы страниц, аллокаторы и менеджеры виртуальной памяти.}

\subsubsection{Виртуальная память: MMU, TLB.}

\href{https://habr.com/ru/post/211150/}{Подробнее тут}

Блок управления памятью или устройство управления памятью memory management unit, MMU)~---~компонент аппаратного обеспечения компьютера, отвечающий за управление доступом к памяти, запрашиваемым центральным процессором.

Его функции заключаются в трансляции адресов виртуальной памяти в адреса физической памяти (то есть управление виртуальной памятью), защите памяти, управлении кэш-памятью, арбитражем шины и, в более простых компьютерных архитектурах (особенно 8-битных), переключением блоков памяти. 

Принцип работы современных MMU основан на разделении виртуального адресного пространства (одномерного массива адресов, используемых центральным процессором) на участки одинакового, как правило, несколько килобайт, хотя, возможно, и существенно большего, размера, равного степени 2, называемые страницами. Младшие n бит адреса (смещение внутри страницы) остаются неизменными. Старшие биты адреса представляют собой номер (виртуальной) страницы. MMU обычно преобразует номера виртуальных страниц в номера физических страниц, используя буфер ассоциативной трансляции (Translation Lookaside Buffer, TLB).

Если преобразование при помощи TLB невозможно, включается более медленный механизм преобразования, основанный на специфическом аппаратном обеспечении или на программных системных структурах. Данные в этих структурах, как правило, называются элементами таблицы страниц (page table entries (PTE)), а сами структуры — таблицами страниц (англ. page table (PT)). Конкатенация номера физической страницы со смещением внутри страницы даёт физический адрес.

Элементы PTE или TLB могут также содержать дополнительную информацию: бит признака записи в страницу ( dirty bit), время последнего доступа к странице (accessed bit), какие процессы (пользовательские (user mode) или системные (supervisor mode)) могут читать или записывать данные в страницу, необходимо ли кэшировать страницу.

\subsubsection{Таблицы страниц.}

\href{https://ru.wikipedia.org/wiki/%D0%A2%D0%B0%D0%B1%D0%BB%D0%B8%D1%86%D0%B0_%D1%81%D1%82%D1%80%D0%B0%D0%BD%D0%B8%D1%86}{Википедия}

Таблица страниц~---~это структура данных, используемая системой виртуальной памяти в операционной системе компьютера для хранения сопоставления между виртуальным адресом и физическим адресом. Виртуальные адреса используются выполняющимся процессом, в то время как физические адреса используются аппаратным обеспечением, или, более конкретно, подсистемой ОЗУ. Таблица страниц является ключевым компонентом преобразования виртуальных адресов, который необходим для доступа к данным в памяти.

\subsubsection{Аллокаторы.}

\href{https://habr.com/ru/post/505632/}{Подробнее тут}

Аллокатор или распределитель памяти в языке программирования C++ ~---~ специализированный класс, реализующий и инкапсулирующий малозначимые (с прикладной точки зрения) детали распределения и освобождения ресурсов компьютерной памяти.

Концептуально выделяется пять основных операции, которые можно осуществить над аллокатором:
\begin{enumerate}
	\item \emph{create}~---~создает аллокатор и отдает ему в распоряжение некоторый объем памяти;
	\item \emph{allocate}~---~выделяет блок определенного размера из области памяти, которым распоряжается аллокатор;
	\item \emph{deallocate}~---~освобождает определенный блок;
	\item \emph{free}~---~освобождает все выделенные блоки из памяти аллокатора (память, выделенная аллокатору, не освобождается);
	\item \emph{destroy}~---~уничтожает аллокатор с последующим освобождением памяти, выделенной аллокатору.
\end{enumerate}

\subsubsection{Менеджеры виртуальной памяти.}

Менеджер виртуальной памяти (далее просто «менеджер памяти») ~---~ часть операционной системы, благодаря которой можно адресовать память большую, чем объем физической памяти (ОЗУ).

Благодаря виртуальной памяти можно запускать множество ресурсоёмких приложений, требующих большого объёма ОЗУ. Максимальный объём виртуальной памяти, который можно получить, используя 24-битную адресацию, — 16 мегабайт. С помощью 32-битной адресации можно адресовать до 4 ГБ виртуальной памяти. А 64-битная адресация позволяет работать уже с 16 эксабайтами памяти.

Применение механизма виртуальной памяти позволяет:

\begin{enumerate}
	\item упростить адресацию памяти клиентским программным обеспечением;
	\item рационально управлять оперативной памятью компьютера (хранить в ней только активно используемые области памяти);
	\item изолировать процессы друг от друга (процесс полагает, что монопольно владеет всей памятью).
\end{enumerate}

\subsection{Операционные системы. Файловые системы (UNIX): файлы и директории, inode, контроль доступа. Файловые системы (реализация в ядре): VFS, блочные устройства, планировщик IO}

\subsubsection{Файловые системы (UNIX): файлы и директории, inode, контроль доступа.}

Все файлы, с которыми могут манипулировать пользователи, располагаются в файловой системе, представляющей собой дерево, промежуточные вершины которого соответствуют каталогам, и листья - файлам и пустым каталогам. 

Каждый каталог и файл файловой системы имеет уникальное полное имя. Каталог, являющийся корнем файловой системы, в любой файловой системе имеет предопределенное имя "/" (слэш). Полное имя файла, например, /bin/sh означает, что в корневом каталоге должно содержаться имя каталога bin, а в каталоге bin должно содержаться имя файла sh. 
Коротким или относительным именем файла (relative pathname) называется имя (возможно, составное), задающее путь к файлу от текущего рабочего каталога. 
В каждом каталоге содержатся два специальных имени, имя ".", именующее сам этот каталог, и имя "..", именующее "родительский" каталог данного каталога, т.е. каталог, непосредственно предшествующий данному в иерархии каталогов.

\textbf{Incode.} \href{https://freehost.com.ua/faq/articles/inode-v-linux--chto-eto-takoe/}{Подробнее туту}

Для ОС Linux есть такое понятие, как Inode или индексный дескриптор. Индексные дескрипторы в файловых системах (таких как ext4) предназначены для хранения метаданных о файлах, каталогах и др. объектах.

Представим иерархическую структуру файловой системы Линукс в упрощенном виде:
\begin{enumerate}
	\item верхушка иерархии — это сама файловая система;
	\item уровнем ниже идут имена файлов (папок);
	\item имена файлов ссылаются на inode;
	\item inode ссылаются на физические данные.
\end{enumerate}

Таким образом, файловая система Linux содержит блоки для хранения данных и inodes. По умолчанию, в ext4, 4092 байта — это размер одного блока. Любой файл в каталоге ОС Linux имеет имя файла и номер inode. Пользователь может узнать метаданные этого файла, указав его номер inode.

Как правило, каждый Inode хранит следующие атрибуты:
\begin{enumerate}
	\item размер;
	\item владелец;
	\item дата/время;
	\item разрешения и контроль доступа;
	\item расположение на диске;
	\item тип файла;
	\item количество ссылок;
	\item дополнительные метаданные о файле.
\end{enumerate}

Таблица с Inode размещена в начале раздела диска, после нее уже идут блоки с данными. Директории в ОС Линукс рассматриваются как Inode типа «директория», в них содержатся списки имен файлов и номера их inode.

Для ОС Линукс также важно понятие о ссылках (символические и жесткие ссылки).

Символическая ссылка — это по своей сути «ярлык», она содержит адрес файла.

Если вы попытаетесь открыть такую ссылку, то откроется соответствующий файл (папка). Если удалить данный файл (папку), символическая ссылка не удалится, но при попытке открыть ее — она приведет «в никуда». Номер Inоdе «символической ссылки» отличается от номера inоde того файла, на который она ссылается.

Если же вы используете «жесткие ссылки», то ваш конкретный файл находится только в определенном месте жесткого диска, а уже именно на это место и ведут сразу несколько ссылок. Каждая «жесткая ссылка» представлена в виде отдельного файла, однако все такого вида ссылки указывают на один и тот же участок диска (даже если мы перемещаем этот файл между разными каталогами). Жесткая ссылка в системе идет под таким же номером Inode, как и фaйл, на который она ссылается.


\subsubsection{Файловые системы (реализация в ядре): VFS, блочные устройства, планировщик IO}

\href{https://docstore.mik.ua/unix2/glava_13.htm}{Подробнее тут (про блоки)}

Файловая система обычно размещается на дисках или других устройствах внешней памяти, имеющих блочную структуру. Кроме блоков, сохраняющих каталоги и файлы, во внешней памяти поддерживается еще несколько служебных областей. 

В мире UNIX существует несколько разных видов файловых систем со своей структурой внешней памяти. Наиболее известны традиционная файловая система UNIX System V (s5) и файловая система семейства UNIX BSD (ufs). Файловая система s5 состоит из четырех секций (рисунок 2.2,a). В файловой системе ufs на логическом диске (разделе реального диска) находится последовательность секций файловой системы 

Кратко опишем суть и назначение каждой области диска:
\begin{enumerate}
	\item Boot-блок содержит программу раскрутки, которая служит для первоначального запуска ОС UNIX. В файловых системах s5 реально используется boot-блок только корневой файловой системы. В дополнительных файловых системах эта область присутствует, но не используется.
	\item Суперблок - это наиболее ответственная область файловой системы, содержащая информацию, которая необходима для работы с файловой системой в целом. Суперблок содержит список свободных блоков и свободные i-узлы (information nodes - информационные узлы). В файловых системах ufs для повышения устойчивости поддерживается несколько копий суперблока. Каждая копия суперблока имеет размер 8196 байт, и только одна копия суперблока используется при монтировании файловой системы (см. ниже). Однако, если при монтировании устанавливается, что первичная копия суперблока повреждена или не удовлетворяет критериям целостности информации, используется резервная копия. 
	\item Блок группы цилиндров содержит число i-узлов, специфицированных в списке i-узлов для данной группы цилиндров, и число блоков данных, которые связаны с этими i-узлами. Размер блока группы цилиндров зависит от размера файловой системы. Для повышения эффективности файловая система ufs старается размещать i-узлы и блоки данных в одной и той же группе цилиндров. 
	\item Список i-узлов (ilist) содержит список i-узлов, соответствующих файлам данной файловой системы. Максимальное число файлов, которые могут быть созданы в файловой системе, определяется числом доступных i-узлов. В i-узле хранится информация, описывающая файл: режимы доступа к файлу, время создания и последней модификации, идентификатор пользователя и идентификатор группы создателя файла, описание блочной структуры файла и т.д.
	\item Блоки данных - в этой части файловой системы хранятся реальные данные файлов. В случае файловой системы ufs все блоки данных одного файла пытаются разместить в одной группе цилиндров. Размер блока данных определяется при форматировании файловой системы командой mkfs и может быть установлен в 512, 1024, 2048, 4096 или 8192 байтов. 
\end{enumerate}

\textbf{Виртуальная файловая система (VFS)}~---~уровень абстракции поверх конкретной реализации файловой системы. Целью VFS является обеспечение единообразного доступа клиентских приложений к различным типам файловых систем. VFS может быть использована для доступа к локальным устройствам и файлам (fat32, ext4, ntfs), сетевым устройствам и файлам на них (nfs), а также к устройствам, не предназначенным для хранения данных (procfs[1]). VFS декларирует программный интерфейс между ядром и конкретной файловой системой, таким образом, для добавления поддержки новой файловой системы не требуется вносить изменений в ядро операционной системы.

\href{https://russianblogs.com/article/825789153/}{Тут про VFS}

\textbf{Планировщик IO.}

\href{https://xakep.ru/2014/05/11/input-out-linux-planning/}{Подробнее тут}

Планировщик IO отвечает за распределение дисковых операций по процессам.  В ранних ядрах Linux (как минимум в ядре 2.4) существовал только один планировщик — Linus Elevator. Он был слишком примитивным, и поэтому в ядре 2.6 появились еще три планировщика, часть из которых ныне уже ушла в небытие. Таким образом, сейчас в ядре существует три планировщика, а в ближайшее время, возможно, прибавится еще и четвертый:
\begin{enumerate}
	\item \emph{NOOP}~---~наиболее простой планировщик. Он банально помещает все запросы в очередь FIFO и исполняет их вне зависимости от того, пытаются ли приложения читать или писать. Планировщик этот, тем не менее, пытается объединять однотипные запросы для сокращения операций ввода/вывода.
	\item \emph{CFQ} был разработан в 2003 году. Заключается его алгоритм в следующем. Каждому процессу назначается своя очередь запросов ввода/вывода. Каждой очереди затем присваивается квант времени. Планировщик же циклически обходит все процессы и обслуживает каждый из них, пока не закончится очередь либо не истечет заданный квант времени. Если очередь закончилась раньше, чем истек выделенный для нее квант времени, планировщик подождет (по умолчанию 10 мс) и, в случае напрасного ожидания, перейдет к следующей очереди. Отмечу, что в рамках каждой очереди чтение имеет приоритет над записью.
	\item \emph{Deadline} в настоящее время является стандартным планировщиком, был разработан в 2002 году. В основе его работы, как это ясно из названия, лежит предельный срок выполнения — то есть планировщик пытается выполнить запрос в указанное время. В дополнение к обычной отсортированной очереди, которая появилась еще в Linus Elevator, в нем есть еще две очереди — на чтение и на запись. Чтение опять же более приоритетно, чем запись. Кроме того, запросы объединяются в пакеты. Пакетом называется последовательность запросов на чтение либо на запись, которая идет в сторону больших секторов («алгоритм лифта»). После его обработки планировщик смотрит, есть ли запросы на запись, которые не обслуживались длительное время, и в зависимости от этого решает, создавать ли пакет на чтение либо же на запись.
	\item \emph{BFQ (Budget Fair Queueing)}~---~относительно новый планировщик. Базируется на CFQ. Если не вдаваться в технические подробности, каждой очереди (которая, как и в CFQ, назначается попроцессно) выделяется свой «бюджет», и, если процесс интенсивно работает с диском, данный «бюджет» увеличивается.
\end{enumerate}

\subsection{Комбинаторные объекты. Коды Грея. Формула включения-исключения. Лемма Бернсайда и Теорема Пойа. Числа Стирлинга. Подсчёт деревьев. Метод производящих функций.}

\subsubsection{Коды Грея}
Код Грея - код для элементов упорядоченного множества (например, чисел от 1 до $n$), такой что расстояние Хэмминга между кодами соседних элементов = 1.
\begin{figure}[H]
	\centering
	\includegraphics[scale=0.4]{images/grey.png}
\end{figure}

Коды Грея легко получаются из двоичных чисел путём побитовой операции «Исключающее ИЛИ» с тем же числом, сдвинутым вправо на один бит и в котором старший разряд заполняется нулём. Следовательно, $i$-й бит кода Грея $G_i$ выражается через биты двоичного кода $B_i$ следующим образом:
\begin{align*}
G_i = B_i \oplus B_i + 1
\end{align*}

где $\oplus$ — операция «исключающее ИЛИ»; биты нумеруются справа налево, начиная с младшего. 

Декодинг происходит по формуле 
\begin{align*}
	B_i = B_{i + 1} \oplus G_i
\end{align*}

Код Грея назван «рефлексивным» (отражённым) из-за того, что первая половина значений при изменении порядка эквивалентна второй половине, за исключением старшего бита. Старший бит просто инвертируется. При делении каждой новой половины пополам это свойство сохраняется.

Код Грея используется в тех случаях, когда мы медленно считываем значения, а они меняются. Представим себе, что код (обычный двоичный) перескакивает $3\rightarrow4$, или $011_2 \rightarrow 100_2$. Если из-за несовершенства считывателя мы прочитаем первый бит от $011$, а остальные два — от $100$, мы получим $000_2=0$ — число, далёкое от реальных значений. В коде Грея никаких посторонних значений не будет: перескок будет в одном разряде, $010_G \rightarrow 110_G$, и мы считаем либо старое $010_G=3$, либо новое $110_G=4$. 

\subsubsection{Формула включения-исключения}

\T{
	Пусть $A = \bigcup\limits_{i = 1}^nA_i$, тогда по формуле включения-исключения:
	\begin{align*}
		|A| = \sum\limits_{I \in 2^n - 1}(-1)^{|I| + 1}|\bigcap_{j\in I}A_j|
	\end{align*}

	где $N = \{1, \ldots n\}$, $2^N - 1$ - множество всех непустых подмножеств $N$. 
}
\begin{proof}
	\href{https://neerc.ifmo.ru/wiki/index.php?title=%D0%A4%D0%BE%D1%80%D0%BC%D1%83%D0%BB%D0%B0_%D0%B2%D0%BA%D0%BB%D1%8E%D1%87%D0%B5%D0%BD%D0%B8%D1%8F-%D0%B8%D1%81%D0%BA%D0%BB%D1%8E%D1%87%D0%B5%D0%BD%D0%B8%D1%8F}{Доказательство}.
\end{proof}

Формулу включений исключений можно интерпретировать в вероятностном смысле, нужно лишь заменить множества на события, а мощности на вероятности. 

\subsubsection{Лемма Бернсайда}
Я чета в ахуе немного, это из теории групп. Про орбиты. 

\D{
	Говорят, что группа $G$ действует на множестве $M$ слева, если задано отображение $G \times M \rightarrow M$, такое что
	\begin{enumerate}
		\item $g(hm) = (gh)m$ для всех $g, h \in G, m \in M$
		\item $em = m$, где $e$ - нейтральный элемент $G$
	\end{enumerate}
}

\D{
	Подмножество
	\begin{align*}
		Gm = \{gm \mid g \in G\} \subset M
	\end{align*}

	Называется орбитой элемента $m$
}

\T[Лемма Бернсайда]{
	Пусть $G$ — конечная группа, действующая на множестве $X$. Тогда число орбит действия равно среднему количеству точек, фиксированных точек в $X$ элементами $G$.
	
	Точнее, для любого элемента $g \in G$ будем обозначать через $X^g$ множество элементов $X$, оставляемых на месте $g$, то есть
	\begin{align*}
		X^g = \{x \in X \mid gx = x\}
	\end{align*} 

	Тогда
	\begin{align*}
		|X/G| = \frac{1}{|G|}\sum\limits_{g \in G}|X^g|
	\end{align*}
	
	где $|X/G|$ - обозначает число орбит действия.
}
\begin{proof}
	\href{https://neerc.ifmo.ru/wiki/index.php?title=%D0%9B%D0%B5%D0%BC%D0%BC%D0%B0_%D0%91%D1%91%D1%80%D0%BD%D1%81%D0%B0%D0%B9%D0%B4%D0%B0_%D0%B8_%D0%A2%D0%B5%D0%BE%D1%80%D0%B5%D0%BC%D0%B0_%D0%9F%D0%BE%D0%B9%D0%B0#.D0.A2.D0.B5.D0.BE.D1.80.D0.B5.D0.BC.D0.B0_.D0.9F.D0.BE.D0.B9.D0.B0}{Доказательство}.
\end{proof}


\subsubsection{Теорема Пойа}
\href{https://e-maxx.ru/algo/burnside_polya}{Тут} максимально просто написано и про Пойа и про Бернсайда.

\subsubsection{Числа Стирлинга}
\begin{itemize}
	\item \textit{Первого рода.} Количество перестановок из $n$ элементов с $k$ циклами.
	
	Можно посчитать рекурсивно:
	\begin{align*}
		c(n, k) = c(n - 1, k - 1) + (n - 1)\cdot c(n - 1, k)
	\end{align*} 
	\item \textit{Второго рода.} Количество неупорядоченных разбиений $n$-элементного множества на $k$ непустых подмножеств.
	
	Можно посчитать рекурсивно:
	\begin{align*}
		S(n, k) = S(n - 1, k - 1) + k\cdot S(n - 1, k)
	\end{align*}
	
	Есть явная формула:
	\begin{align*}
		S(n, k) = \frac{1}{k!}\sum\limits(-1)^{k + j}{k \choose j}j^n
	\end{align*}
\end{itemize}

\subsubsection{Подсчет деревьев}
\D{
	$n$-ое число Каталана:
	\begin{align*}
		C_n = \frac{1}{n + 1}\binom{2n}{n}
	\end{align*}

	Числа Каталана - удовлетворяют рекуррентному соотношению
	\begin{align*}
		C_0 = 1\\
		C_n = \sum\limits_{i =0}^{n - 1}C_iC_{n - 1 - i}
	\end{align*}
	Например, это количество правильных скобочных последовательностей длины $2n$.
}

Более подробно на \href{https://ru.wikipedia.org/wiki/%D0%A7%D0%B8%D1%81%D0%BB%D0%B0_%D0%9A%D0%B0%D1%82%D0%B0%D0%BB%D0%B0%D0%BD%D0%B0}{Вики}.

\T{
	Количество неизоморфных упорядоченных бинарных деревьев с корнем и $n + 1$ листьями = $n$-ому числу Каталана. 
	
	Здесь \textit{упорядоченные} означает, что ребра, выходящие из каждой вершины, упорядочены.
}

\D{
	Помеченное дерево c $n$ вершинами - дерево c $n$ вершинами, вершинам которого взаимно однозначно соответствуют числа от 1 до $n$.
}

\T[Кэли]{
	Число помеченных деревьев с $n$ вершинами равняется $n^{n - 2}$.
}
\begin{proof}
	Доказательство и еще много всего интересного смотри \href{https://neerc.ifmo.ru/wiki/index.php?title=%D0%9F%D0%BE%D0%B4%D1%81%D1%87%D0%B5%D1%82_%D0%B4%D0%B5%D1%80%D0%B5%D0%B2%D1%8C%D0%B5%D0%B2}{тут}. 
\end{proof}

\subsubsection{Метод производящих функций}
Вот \href{https://neerc.ifmo.ru/wiki/index.php?title=%D0%9F%D1%80%D0%BE%D0%B8%D0%B7%D0%B2%D0%BE%D0%B4%D1%8F%D1%89%D0%B0%D1%8F_%D1%84%D1%83%D0%BD%D0%BA%D1%86%D0%B8%D1%8F}{отсюда} можно прочитать определение и пример для решения рекуррент


\subsection{Параллельное программирование. Консенсусное число. Стек Трайбера. Очередь Майкла-Скотта.}

\href{https://www.babichev.org/tpmtp/Lecture09.pdf}{Про консенсус}.
\D{
	Задача консенсуса: есть $N$ потоков, нужно чтоб они все пришли к согласию по поводу какого-то одного значения.
}

\D{
	Любой последовательный объект можно реализовать без ожидания (wait-free) для N потоков используя консенсусный протокол для N потоков.
	
	Такое построение называется универсальная конструкция
}	

\href{https://neerc.ifmo.ru/wiki/index.php?title=%D0%A1%D1%82%D0%B5%D0%BA_%D0%A2%D1%80%D0%B0%D0%B9%D0%B1%D0%B5%D1%80%D0%B0}{Стек Трайбера}

\href{https://neerc.ifmo.ru/wiki/index.php?title=%D0%9E%D1%87%D0%B5%D1%80%D0%B5%D0%B4%D1%8C_%D0%9C%D0%B0%D0%B9%D0%BA%D0%BB%D0%B0_%D0%B8_%D0%A1%D0%BA%D0%BE%D1%82%D1%82%D0%B0}{Очередь Майкла-Скотта}.

\subsection{Компьютерные сети. Модель OSI. Устройства коммутации и маршрутизации. Протоколы Ethernet, IP, TCP, UDP.}

\href{https://www.dropbox.com/sh/4st5b16mvdf8gkj/AAAI9sbKs_C3TFgRbqWGrAeca/Programming/16%20%D0%9A%D0%BE%D0%BC%D0%BF%D1%8C%D1%8E%D1%82%D0%B5%D1%80%D0%BD%D1%8B%D0%B5%20%D1%81%D0%B5%D1%82%D0%B8.pdf?dl=0}{dropbox}

\subsection{Контекстно-свободные грамматики. Эффективные методы разбора: LL(k)-, LR(k)- и LALR-грамматики.}

\D{
	Формальная грамматика (англ. Formal grammar) — способ описания формального языка, представляющий собой четверку
	
	$\Gamma = \langle \Sigma, N, S \in N, P\rangle$
	
	, где:
	\begin{itemize}
		\item $\Sigma$ — алфавит, элементы которого называют терминалами (англ. terminals);
		\item $N$ — множество, элементы которого называют нетерминалами (англ. nonterminals);
		\item $S$ — начальный символ грамматики (англ. start symbol);
		\item $P$ — набор правил вывода (англ. production rules или productions) $\alpha \rightarrow \beta$.
	\end{itemize}
}

\D{
	Левосторонним выводом слова (англ. leftmost derivation) $\alpha$ называется такой вывод слова $\alpha$, в котором каждая последующая строка получена из предыдущей путем замены по одному из правил самого левого встречающегося в строке нетерминала.
}

Контекстно-свободная грамматика: \href{https://neerc.ifmo.ru/wiki/index.php?title=%D0%9A%D0%BE%D0%BD%D1%82%D0%B5%D0%BA%D1%81%D1%82%D0%BD%D0%BE-%D1%81%D0%B2%D0%BE%D0%B1%D0%BE%D0%B4%D0%BD%D1%8B%D0%B5_%D0%B3%D1%80%D0%B0%D0%BC%D0%BC%D0%B0%D1%82%D0%B8%D0%BA%D0%B8,_%D0%B2%D1%8B%D0%B2%D0%BE%D0%B4,_%D0%BB%D0%B5%D0%B2%D0%BE-_%D0%B8_%D0%BF%D1%80%D0%B0%D0%B2%D0%BE%D1%81%D1%82%D0%BE%D1%80%D0%BE%D0%BD%D0%BD%D0%B8%D0%B9_%D0%B2%D1%8B%D0%B2%D0%BE%D0%B4,_%D0%B4%D0%B5%D1%80%D0%B5%D0%B2%D0%BE_%D1%80%D0%B0%D0%B7%D0%B1%D0%BE%D1%80%D0%B0}{neerc}.

Наша задача - понять принадлежит ли входное слово языку. Для этого нам нужно предъявить вывод из стартового нетерминала. 
\begin{itemize}
	\item $LR(k)$-анализатор просматривает символы слева направо, складывает их в стек. Если в стеке накопилась правая часть какого-то правила, нужно принимать решение: заменять ли содержание стека на левую часть этого правила или ждать еще, чтоб накопилась правая часть еще какого-то правила (являющаяся надстрокой для текущей гипотезы). Чтобы принять это решение, анализатор смотрит еще $k$ символов.
	
	\href{https://neerc.ifmo.ru/wiki/index.php?title=LR(k)-%D0%B3%D1%80%D0%B0%D0%BC%D0%BC%D0%B0%D1%82%D0%B8%D0%BA%D0%B8}{neerc}
	
	\item $LL(k)$-анализатор
	
	\href{https://neerc.ifmo.ru/wiki/index.php?title=LL(k)-%D0%B3%D1%80%D0%B0%D0%BC%D0%BC%D0%B0%D1%82%D0%B8%D0%BA%D0%B8,_%D0%BC%D0%BD%D0%BE%D0%B6%D0%B5%D1%81%D1%82%D0%B2%D0%B0_FIRST_%D0%B8_FOLLOW}{neerc}
	
	\item \href{https://ru.wikipedia.org/wiki/SLR(1)}{$SLR(1)$ грамматика} - это грамматика, в которой усовершенствованно построение таблицы $LR(0)$ грамматики при помощи применения FIRST и FOLLOW (см. $LL$-анализатор).
	
	\href{https://ru.wikipedia.org/wiki/LALR(1)}{$LALR(1)$} - усовершенствование $SLR(1)$
\end{itemize}


\subsection{Базы данных. Типы баз данных. Реляционные БД. Нормальные формы РБД. Язык SQL}

\href{https://www.dropbox.com/sh/4st5b16mvdf8gkj/AACJWcRNv0lFmPzhlkMSxN7Va/Programming/12%20%D0%91%D0%B0%D0%B7%D1%8B%20%D0%B4%D0%B0%D0%BD%D0%BD%D1%8B%D1%85.pdf?dl=0}{dropbox}

\subsection{Марковские цепи, Эргодические цепи, Регулярные цепи. Алгоритм Витерби.}

\href{https://neerc.ifmo.ru/wiki/index.php?title=%D0%9C%D0%B0%D1%80%D0%BA%D0%BE%D0%B2%D1%81%D0%BA%D0%B0%D1%8F_%D1%86%D0%B5%D0%BF%D1%8C}{Марковская цепь}

На состояниях Марковской цепи можно построить ориентированный граф.
\D{
	Эргодическая цепь Маркова - цепь, имеющая сильно связный граф. 
}
\href{https://neerc.ifmo.ru/wiki/index.php?title=%D0%AD%D1%80%D0%B3%D0%BE%D0%B4%D0%B8%D1%87%D0%B5%D1%81%D0%BA%D0%B0%D1%8F_%D0%BC%D0%B0%D1%80%D0%BA%D0%BE%D0%B2%D1%81%D0%BA%D0%B0%D1%8F_%D1%86%D0%B5%D0%BF%D1%8C}{Эргодическая цепь}

\D{
	Цепь регулярна тогда и только тогда, когда существует такое $n$, что в матрице $P^n$ все элементы ненулевые, то есть из любого состояния можно перейти в любое за $n$ переходов.
}
\href{https://neerc.ifmo.ru/wiki/index.php?title=%D0%A0%D0%B5%D0%B3%D1%83%D0%BB%D1%8F%D1%80%D0%BD%D0%B0%D1%8F_%D0%BC%D0%B0%D1%80%D0%BA%D0%BE%D0%B2%D1%81%D0%BA%D0%B0%D1%8F_%D1%86%D0%B5%D0%BF%D1%8C}{Регулярная цепь}.

\subsubsection{Алгоритм Витерби}
\href{https://neerc.ifmo.ru/wiki/index.php?title=%D0%A1%D0%BA%D1%80%D1%8B%D1%82%D1%8B%D0%B5_%D0%9C%D0%B0%D1%80%D0%BA%D0%BE%D0%B2%D1%81%D0%BA%D0%B8%D0%B5_%D0%BC%D0%BE%D0%B4%D0%B5%D0%BB%D0%B8}{Скрытая Марковская цепь}

\href{https://neerc.ifmo.ru/wiki/index.php?title=%D0%90%D0%BB%D0%B3%D0%BE%D1%80%D0%B8%D1%82%D0%BC_%D0%92%D0%B8%D1%82%D0%B5%D1%80%D0%B1%D0%B8}{Алгоритм Витерби}

\subsection{Линейные структуры данных. Амортизационный анализ. Поисковые структуры данных. Запросы на отрезках. Персистентные структуры данных.}

Линейные структуры данных:
\begin{itemize}
    \item стек
    \item очередь
    \item дек
    \item массив
\end{itemize}

\href{https://neerc.ifmo.ru/wiki/index.php?title=%D0%90%D0%BC%D0%BE%D1%80%D1%82%D0%B8%D0%B7%D0%B0%D1%86%D0%B8%D0%BE%D0%BD%D0%BD%D1%8B%D0%B9_%D0%B0%D0%BD%D0%B0%D0%BB%D0%B8%D0%B7}{Амортизационный анализ}

\href{https://neerc.ifmo.ru/wiki/index.php?title=%D0%9F%D0%BE%D0%B8%D1%81%D0%BA%D0%BE%D0%B2%D1%8B%D0%B5_%D1%81%D1%82%D1%80%D1%83%D0%BA%D1%82%D1%83%D1%80%D1%8B_%D0%B4%D0%B0%D0%BD%D0%BD%D1%8B%D1%85}{Поисковые структуры данных}

\href{https://e-maxx.ru/algo/segment_tree}{Запросы на отрезках}

\href{https://neerc.ifmo.ru/wiki/index.php?title=%D0%9F%D0%B5%D1%80%D1%81%D0%B8%D1%81%D1%82%D0%B5%D0%BD%D1%82%D0%BD%D1%8B%D0%B5_%D1%81%D1%82%D1%80%D1%83%D0%BA%D1%82%D1%83%D1%80%D1%8B_%D0%B4%D0%B0%D0%BD%D0%BD%D1%8B%D1%85}{Персистентные структуры данных}

\subsection{Теория кодирования. Блоковые коды и их параметры. Критерии декодирования и метрики. Границы Хемминга и Варшамова-Гилберта.}

\href{https://docs.google.com/document/d/1Q9Q3T_WhroC04ByS0BtFDbvEZ6DrQbuFPd7fAl2XSWM/edit#heading=h.y6j1v1w9p3kd}{Билеты от ИС}


\subsection{Теория кодирования. Линейные коды. Границы Синглтона, Варшамова-Гилберта и Грайсмера. Вероятность ошибки декодирования и необнаружения ошибки.}

\href{https://ru.wikipedia.org/wiki/%D0%9B%D0%B8%D0%BD%D0%B5%D0%B9%D0%BD%D1%8B%D0%B9_%D0%BA%D0%BE%D0%B4}{Линейный код}

\href{https://ru.wikipedia.org/wiki/%D0%93%D1%80%D0%B0%D0%BD%D0%B8%D1%86%D0%B0_%D0%A1%D0%B8%D0%BD%D0%B3%D0%BB%D1%82%D0%BE%D0%BD%D0%B0}{Граница Синглтона}

\href{https://ru.wikipedia.org/wiki/%D0%93%D1%80%D0%B0%D0%BD%D0%B8%D1%86%D0%B0_%D0%92%D0%B0%D1%80%D1%88%D0%B0%D0%BC%D0%BE%D0%B2%D0%B0_%E2%80%94_%D0%93%D0%B8%D0%BB%D0%B1%D0%B5%D1%80%D1%82%D0%B0}{Граница Варшамова-Гилберта}

\href{https://wikicsu.ru/wiki/Griesmer_bound}{Граница Грайсмера}

\href{http://jre.cplire.ru/jre/apr14/4/text.html}{ОЦЕНКА ВЕРОЯТНОСТИ ОШИБКИ НА БИТ ПО РЕЗУЛЬТАТАМ ДЕКОДИРОВАНИЯ КОДОВЫХ СЛОВ}

\subsection{Машинное обучение. Понятие машинного обучения в искусственном интеллекте. Классификация задач машинного обучения, их примеры и особенности.}

\href{https://ru.wikipedia.org/wiki/%D0%9C%D0%B0%D1%88%D0%B8%D0%BD%D0%BD%D0%BE%D0%B5_%D0%BE%D0%B1%D1%83%D1%87%D0%B5%D0%BD%D0%B8%D0%B5#:~:text=%D0%9C%D0%B0%D1%88%D0%B8%D0%BD%D0%BD%D0%BE%D0%B5%20%D0%BE%D0%B1%D1%83%D1%87%D0%B5%D0%BD%D0%B8%D0%B5%20(%D0%B0%D0%BD%D0%B3%D0%BB.,%D0%BF%D1%80%D0%B8%D0%BC%D0%B5%D0%BD%D0%B5%D0%BD%D0%B8%D1%8F%20%D1%80%D0%B5%D1%88%D0%B5%D0%BD%D0%B8%D0%B9%20%D0%BC%D0%BD%D0%BE%D0%B6%D0%B5%D1%81%D1%82%D0%B2%D0%B0%20%D1%81%D1%85%D0%BE%D0%B4%D0%BD%D1%8B%D1%85%20%D0%B7%D0%B0%D0%B4%D0%B0%D1%87.}{Википедия}

\href{https://docs.google.com/document/d/1Q9Q3T_WhroC04ByS0BtFDbvEZ6DrQbuFPd7fAl2XSWM/edit#}{Билеты от ИС}

\subsection{Машинное обучение. Многослойная нейронная сеть и алгоритм обратного распространения ошибок. Методы инициализации и методы оптимизации в нейронных сетях. Архитектуры сетей, их применение}

\href{https://ru.wikipedia.org/wiki/%D0%9C%D0%B5%D1%82%D0%BE%D0%B4_%D0%BE%D0%B1%D1%80%D0%B0%D1%82%D0%BD%D0%BE%D0%B3%D0%BE_%D1%80%D0%B0%D1%81%D0%BF%D1%80%D0%BE%D1%81%D1%82%D1%80%D0%B0%D0%BD%D0%B5%D0%BD%D0%B8%D1%8F_%D0%BE%D1%88%D0%B8%D0%B1%D0%BA%D0%B8}{Метод обратного распространения ошибки}

\href{https://machinelearningmastery.ru/initialization-techniques-for-neural-networks-f4ce8e64effc/}{Методы инициализации для нейронных сетей}

\href{https://habr.com/ru/company/skillfactory/blog/552394/}{Принцип действия оптимизаторов для нейронных сетей}

\href{https://habr.com/ru/post/318970/}{Еще про оптимизацию}.

Как я понял, оптимизация для нейронных сетей - это всякие навесы на градиентный спуск.

\href{https://habr.com/ru/company/oleg-bunin/blog/340184/}{Введение в архитектуры нейронных сетей}


\end{document}
