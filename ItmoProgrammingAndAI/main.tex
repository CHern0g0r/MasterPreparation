%----------------------------------------------------------------------------------------
%	PACKAGES AND OTHER DOCUMENT CONFIGURATIONS
%----------------------------------------------------------------------------------------

\documentclass{article}
% \documentclass[14pt]{extarticle}
\usepackage{pdfpages}

%%%%%%%%%%%%%%%%%%%%%%%%%%%%%%%%%%%%%%%%%
% Lachaise Assignment
% Structure Specification File
% Version 1.0 (26/6/2018)
%
% This template originates from:
% http://www.LaTeXTemplates.com
%
% Authors:
% Marion Lachaise & François Févotte
% Vel (vel@LaTeXTemplates.com)
%
% License:
% CC BY-NC-SA 3.0 (http://creativecommons.org/licenses/by-nc-sa/3.0/)
% 
%%%%%%%%%%%%%%%%%%%%%%%%%%%%%%%%%%%%%%%%%

%----------------------------------------------------------------------------------------
%	PACKAGES AND OTHER DOCUMENT CONFIGURATIONS
%----------------------------------------------------------------------------------------

\usepackage{amsmath,amsfonts,stmaryrd,amssymb,amsthm} % Math packages

\usepackage{mathtext}

\usepackage{enumerate} % Custom item numbers for enumerations

\usepackage[ruled]{algorithm2e} % Algorithms

\usepackage[framemethod=tikz]{mdframed} % Allows defining custom boxed/framed environments

\usepackage{listings} % File listings, with syntax highlighting
\lstset{
	basicstyle=\ttfamily, % Typeset listings in monospace font
}

\usepackage[unicode]{hyperref}
%----------------------------------------------------------------------------------------
%	DOCUMENT MARGINS
%----------------------------------------------------------------------------------------

\usepackage{geometry} % Required for adjusting page dimensions and margins

\geometry{
	paper=a4paper, % Paper size, change to letterpaper for US letter size
	top=2.5cm, % Top margin
	bottom=3cm, % Bottom margin
	left=2.5cm, % Left margin
	right=2.5cm, % Right margin
	headheight=14pt, % Header height
	footskip=1.5cm, % Space from the bottom margin to the baseline of the footer
	headsep=1.2cm, % Space from the top margin to the baseline of the header
	%showframe, % Uncomment to show how the type block is set on the page
}

%----------------------------------------------------------------------------------------
%	FONTS
%----------------------------------------------------------------------------------------

\usepackage[utf8]{inputenc} % Required for inputting international characters
\usepackage[T1, T2A]{fontenc} % Output font encoding for international characters

\usepackage[english,russian]{babel}

\usepackage{XCharter} % Use the XCharter fonts

%----------------------------------------------------------------------------------------
%	COMMAND LINE ENVIRONMENT
%----------------------------------------------------------------------------------------

% Usage:
% \begin{commandline}
%	\begin{verbatim}
%		$ ls
%		
%		Applications	Desktop	...
%	\end{verbatim}
% \end{commandline}

\mdfdefinestyle{commandline}{
	leftmargin=10pt,
	rightmargin=10pt,
	innerleftmargin=15pt,
	middlelinecolor=black!50!white,
	middlelinewidth=2pt,
	frametitlerule=false,
	backgroundcolor=black!5!white,
	frametitle={Command Line},
	frametitlefont={\normalfont\sffamily\color{white}\hspace{-1em}},
	frametitlebackgroundcolor=black!50!white,
	nobreak,
}

% Define a custom environment for command-line snapshots
\newenvironment{commandline}{
	\medskip
	\begin{mdframed}[style=commandline]
}{
	\end{mdframed}
	\medskip
}

%----------------------------------------------------------------------------------------
%	FILE CONTENTS ENVIRONMENT
%----------------------------------------------------------------------------------------

% Usage:
% \begin{file}[optional filename, defaults to "File"]
%	File contents, for example, with a listings environment
% \end{file}

\mdfdefinestyle{file}{
	innertopmargin=1.6\baselineskip,
	innerbottommargin=0.8\baselineskip,
	topline=false, bottomline=false,
	leftline=false, rightline=false,
	leftmargin=2cm,
	rightmargin=2cm,
	singleextra={%
		\draw[fill=black!10!white](P)++(0,-1.2em)rectangle(P-|O);
		\node[anchor=north west]
		at(P-|O){\ttfamily\mdfilename};
		%
		\def\l{3em}
		\draw(O-|P)++(-\l,0)--++(\l,\l)--(P)--(P-|O)--(O)--cycle;
		\draw(O-|P)++(-\l,0)--++(0,\l)--++(\l,0);
	},
	nobreak,
}

% Define a custom environment for file contents
\newenvironment{file}[1][File]{ % Set the default filename to "File"
	\medskip
	\newcommand{\mdfilename}{#1}
	\begin{mdframed}[style=file]
}{
	\end{mdframed}
	\medskip
}

%----------------------------------------------------------------------------------------
%	NUMBERED QUESTIONS ENVIRONMENT
%----------------------------------------------------------------------------------------

% Usage:
% \begin{question}[optional title]
%	Question contents
% \end{question}

\mdfdefinestyle{question}{
	innertopmargin=1.2\baselineskip,
	innerbottommargin=0.8\baselineskip,
	roundcorner=5pt,
	nobreak,
	singleextra={%
		\draw(P-|O)node[xshift=1em,anchor=west,fill=white,draw,rounded corners=5pt]{%
		Question \theQuestion\questionTitle};
	},
}

\newcounter{Question} % Stores the current question number that gets iterated with each new question

% Define a custom environment for numbered questions
\newenvironment{question}[1][\unskip]{
	\bigskip
	\stepcounter{Question}
	\newcommand{\questionTitle}{~#1}
	\begin{mdframed}[style=question]
}{
	\end{mdframed}
	\medskip
}

%----------------------------------------------------------------------------------------
%	WARNING TEXT ENVIRONMENT
%----------------------------------------------------------------------------------------

% Usage:
% \begin{warn}[optional title, defaults to "Warning:"]
%	Contents
% \end{warn}

\mdfdefinestyle{warning}{
	topline=false, bottomline=false,
	leftline=false, rightline=false,
	nobreak,
	singleextra={%
		\draw(P-|O)++(-0.5em,0)node(tmp1){};
		\draw(P-|O)++(0.5em,0)node(tmp2){};
		\fill[black,rotate around={45:(P-|O)}](tmp1)rectangle(tmp2);
		\node at(P-|O){\color{white}\scriptsize\bf !};
		\draw[very thick](P-|O)++(0,-1em)--(O);%--(O-|P);
	}
}

% Define a custom environment for warning text
\newenvironment{warn}[1][Warning:]{ % Set the default warning to "Warning:"
	\medskip
	\begin{mdframed}[style=warning]
		\noindent{\textbf{#1}}
}{
	\end{mdframed}
}

%----------------------------------------------------------------------------------------
%	INFORMATION ENVIRONMENT
%----------------------------------------------------------------------------------------

% Usage:
% \begin{info}[optional title, defaults to "Info:"]
% 	contents
% 	\end{info}

\mdfdefinestyle{info}{%
	topline=false, bottomline=false,
	leftline=false, rightline=false,
	nobreak,
	singleextra={%
		\fill[black](P-|O)circle[radius=0.4em];
		\node at(P-|O){\color{white}\scriptsize\bf i};
		\draw[very thick](P-|O)++(0,-0.8em)--(O);%--(O-|P);
	}
}

% Define a custom environment for information
\newenvironment{info}[1][Info:]{ % Set the default title to "Info:"
	\medskip
	\begin{mdframed}[style=info]
		\noindent{\textbf{#1}}
}{
	\end{mdframed}
}

\newcommand{\D}[1]{\begin{warn}[Def:] #1 \end{warn}}
\newcommand{\pred}[1]{\textbf{Предложение:} #1\\}
\newcommand{\T}[2][]{\begin{warn}[Theorem: #1] \\#2 \end{warn}}
\newcommand{\m}[1]{\mathfrak{#1}}
\newcommand{\inc}[1]{\includepdf[pages=-]{#1}}

\newcommand\invisiblesection[1]{%
  \refstepcounter{section}%
  \addcontentsline{toc}{section}{\protect\numberline{\thesection}#1}%
  \sectionmark{#1}}
 % Include the file specifying the document structure and custom commands

%----------------------------------------------------------------------------------------
%	ASSIGNMENT INFORMATION
%----------------------------------------------------------------------------------------

\title{Programming and AI} % Title of the assignment

%----------------------------------------------------------------------------------------

\DeclareMathOperator*\uplim{\overline{lim}}

\usepackage[parfill]{parskip}
\begin{document}
	
\maketitle % Print the title
\section{Математика и Теоретическая информатика}

\subsection{Числовые ряды. Абсолютная и условная сходимость. Признаки сходимости числовых рядов.}

\subsubsection{Числовые ряды}

$\sum\limits_{k=1}^{\infty} a_{k} = a_{1} + a_{2} + a_{3} + \cdots$ -- числовой ряд

Сходимость ряда означает существование конечной суммы, т.е. $\sum\limits_{k=1}^{\infty} a_{k} = S$ где $S$ -- конечное число, иначе ряд считается расходящимся.

\subsubsection{Абсолютная и условная сходимость}

Ряд $\sum\limits_{k=1}^{\infty} a_{k}$ называется {\bf абсолютно} сходящимся, если сходится ряд из модулей $\sum\limits_{k=1}^{\infty} |a_{k}|$, иначе ряд называется {\bf условно} сходящимся

\subsubsection{Признаки сходимости числовых рядов} 

{\bf Знакоположительные ряды} (ряды с положительными членами):

Критерий сходимости знакоположительных рядов-- знакоположительный ряд $\sum\limits_{k=1}^{\infty} a_{k}$ сходится тогда и только тогда, когда последовательность его частичных сумм $S(n) = \sum\limits_{k=1}^{k=n}a_{k}$ ограничена сверху

{\bf Док-во:}

=>: ряд сходится, значит последовательность частичных сумм $\S(n) =\sum\limits_{k=1}^{n} a_{k}$ имеет предел равный $\sum\limits_{k=1}^{\infty} a_{k} = S$

<=: Пусть дан положительный ряд и последовательность частичных сумм ограничена сверху, заметим что последовательность частичных сумм неубывающая:
$$S_{n + 1} - S_{n} = a_{n + 1} \ge 0$$. Используя свойство из теоремы о монотонной последовательности получаем, что т.к. последовательность частичных сумм монотонно не убывает и ограничена сверху, значит она сходится и потому ряд сходится по определению.

{\bf Признак сравнения с мажорантой}

Пусть даны два положительных ряда $\sum\limits_{k=1}^{\infty} a_{k}$ и $\sum\limits_{k=1}^{\infty} b_{k}$. Если начиная с некоторого номера $n > N$ выполняется неравенство $0 \le a_n \le b_n$, то:

\begin{itemize}
	\item из сходимости рядя $\sum\limits_{k=1}^{\infty} b_{k}$ следует сходимость ряда $\sum\limits_{k=1}^{\infty} a_{k}$
	\item из расходимости ряда $\sum\limits_{k=1}^{\infty} a_{k}$ следует расходимость $\sum\limits_{k=1}^{\infty} b_{k}$
\end{itemize}

{\bf Док-во:}

Из неравенств на члены следует неравенство на частичные суммы $0 \le S_n \le \sigma_n$, дальше очев.


{\bf Признак Раабе}

Если для ряда $\sum\limits_{k=1}^{\infty} a_{k}$ существует предел $$R = \lim\limits_{n \rightarrow \infty} n (\frac{a_n}{a_{n+1}} - 1)$$, то при $R > 1$ ряд сходится, а при $R < 1$ -- расходится. Если $R = 1$, то жанный признак не говорит ничего.

{\bf Признак Гаусса}

Пусть для знакоположительного ряда $\sum\limits_{n=1}^{\infty} a_{n}$ отношение $\frac{a_n}{a_{n + 1}}$ может быть представлено в виде $$\frac{a_n}{a_{n + 1}} = \lambda + \frac{\mu}{n} + \frac{\theta_n}{n^2}$$, где $\lambda, \mu$ -- постоянные, а последовательность $\theta_n$ ограничена. Тогда 
\begin{itemize}
	\item ряд расходится если либо $\lambda > 1$, либо $\lambda = 1, \mu > 1$
	\item ряд расходится, если либо $\lambda < 1$, либо $\lambda = 1, \mu \le 1$
\end{itemize}


{\bf Знакопеременные ряды}

\D{Знакопеременными называются ряды, члены которых могут (стоять) быть как положительными, так и отрицательными.}


{\bf Признак Даламбера}

Слабее признака Коши, но зато проще

Если существует $\lim\limits_{n \rightarrow \infty}|\frac{a_{n + 1}}{a_n}| = r$, то 

\begin{itemize}
	\item если $r < 1$, то ряд абсолютно сходится
	\item если $r > 1$, то ряд расходится
	\item если $r = 1$, то данный признак ничего не говорит (сука)
\end{itemize}

{\bf Док-во:}

1. Пусть начиная с некоторого номера N верно неравенство $|\frac{a_{n+1}}{a_n}| \le q, 0 < q < 1$. Тогда перемножив члены начиная с N будем иметь что $\frac{a_{N+n}}{a_N} \le q^n$ откуда $|a_{N+n}| \le |a_{N}q^n|$, значит ряд $|a_{N+1}| + |a_{N+2}| + ...$ меньше бесконечной суммы убывающей геометрической прогрессии, поэтому он сходится

2. $|\frac{a_{n + 1}}{a_n}| \ge 1$ (с некоторого N), тогда можно записать $|a_{n+1}| \ge |a_n|$ значит модуль членов $a$ не стремится к 0 на бесконечности, значит последовательность не стремится к 0 а значит ряд не сходится.

3. Если просто меньше 1 до там хуйня какая-то мне впадлу
\\

{\bf Радикальный признак Коши} (ебаная оппозиция)

Если существует $\lim\lim\limits_{n \rightarrow \infty} \sqrt[n]{|a_n|} = r$, то

\begin{itemize}
	\item если $r < 1$ то ряд сходится абсолютно
	\item если $r > 1$ то ряд расходится
	\item если $r = 1$ то хз (опять??)
\end{itemize}

{\bf Док-во:} \href{https://ru.wikipedia.org/wiki/%D0%A0%D0%B0%D0%B4%D0%B8%D0%BA%D0%B0%D0%BB%D1%8C%D0%BD%D1%8B%D0%B9_%D0%BF%D1%80%D0%B8%D0%B7%D0%BD%D0%B0%D0%BA_%D0%9A%D0%BE%D1%88%D0%B8}{тут}
\\

{\bf Признак Лейбница}

Пусть для знакочередующегося ряда $$S = \sum\limits_{n=1}^{\infty}(-1)^{n-1}a_n, a_n \ge 0$$
выполняются следующие условия

\begin{itemize}
	\item С некоторого $N$ последовательность $a$ монотонно убывает, т.е. $a_{n+1} \le a_n$
	\item $\lim\limits_{n \rightarrow \infty}a_n = 0$
\end{itemize}

Тогда такой ряд сходится

{\bf Док-во:} \href{https://ru.wikipedia.org/wiki/%D0%A2%D0%B5%D0%BE%D1%80%D0%B5%D0%BC%D0%B0_%D0%9B%D0%B5%D0%B9%D0%B1%D0%BD%D0%B8%D1%86%D0%B0_%D0%BE_%D1%81%D1%85%D0%BE%D0%B4%D0%B8%D0%BC%D0%BE%D1%81%D1%82%D0%B8_%D0%B7%D0%BD%D0%B0%D0%BA%D0%BE%D1%87%D0%B5%D1%80%D0%B5%D0%B4%D1%83%D1%8E%D1%89%D0%B8%D1%85%D1%81%D1%8F_%D1%80%D1%8F%D0%B4%D0%BE%D0%B2}{здесь}\\

{\bf Признак Абеля}

\T {Числовой ряд $\sum\limits_{n=1}^{\infty}a_nb_n$ сходится, если выполнены следующие условия
	
	\begin{itemize}
		\item Последовательность \{$a_n$\} монотонна и ограничена
		\item Ряд $\sum\limits_{n=1}^{\infty}b_n$ сходится
	\end{itemize}
}
{\bf Proof:} \href{https://ib.mazurok.com/2015/06/16/%D0%BF%D1%80%D0%B8%D0%B7%D0%BD%D0%B0%D0%BA%D0%B8-%D0%B0%D0%B1%D0%B5%D0%BB%D1%8F-%D0%B8-%D0%B4%D0%B8%D1%80%D0%B8%D1%85%D0%BB%D0%B5/}{вот}\\

{\bf Признак Дирихле}

\T{Пусть выполнены условия:
	\begin{itemize}
		\item последовательность частичных сумм $B_n = \sum\limits_{k=1}^{n}$ ограничена
		\item последовательность $a_n$, начиная с некоторого номера, монотонно убывает $a_n \ge a_{n+1}$
		\item $\lim\limits_{n\rightarrow\infty}a_n = 0$
	\end{itemize}
	Тогда ряд $\sum\limits_{n=1}^{\infty}a_nb_b$ сходится
}

{\bf Proof:} \href{https://ib.mazurok.com/2015/06/16/%D0%BF%D1%80%D0%B8%D0%B7%D0%BD%D0%B0%D0%BA%D0%B8-%D0%B0%D0%B1%D0%B5%D0%BB%D1%8F-%D0%B8-%D0%B4%D0%B8%D1%80%D0%B8%D1%85%D0%BB%D0%B5/}{вот}\\

\subsection{Кратные, поверхностные и криволинейные интегралы. Формулы Грина, Стокса и Остроградского}


\subsubsection{Интеграль4ики}
\D{Пусть дана $f(x)$ -- функция действительной переменной. {\bf Неопределенным интегралом} функции $f(x)$, или ее первообразной, называется такая функция $F(x)$, производная которой равна $f(x)$, т.е. $F^{'}(x) = f(x)$. Обозначается $F(x) = \int f(x)dx$ }


\D{Кратным интегралом называют множество интегралов, взятых от $d > 1$, например  $$\underbrace{\int...\int f(x_1,...,x_d)dx_{1}...dx_{d}}_{d}$$}

Замечание -- кратный интеграл -- определенный интеграл, при его вычислении всегда получается число

\D{Криволинейный интеграл -- интеграл вычисляемый вдоль какой-либо прямой.
	
	Пусть $l$ -- пгладкая, без особых точек и пересечений кривая (может быть замкнутой), заданая параметрически $l: r(t)$, где $r$ -- радиус вектор, конец которого описывает кривую, а параметр $t$ направлен от начального значения $a$ к конечному значению $b$. Для интеграла второго рода направление, в котором движется параметр, определяет направление кривой $l$.
	
	Также есть скалярная или векторная функция , которая рассматривается вдоль кривой $l: f(r)$
	
	Еще есть разбиение отрезка параметризации, и разбиение кривой. Они соответствуют друг-другу (параметризация от параметра по факту сопостовляет точке из отрезка параметризации $[a, b]$ точку на прямой, и по разбиению параметризации разбивается кривая по соответствующим точкам, подробнее можно почитать на вики ссылку вставить не получилось:( )
	
	Интегральная сумма для интеграла {\bf первого рода} -- сумма вида $\sum\limits_{k=1}^{n} f(r(\xi_i))\cdot|l_k|$ где $|l_k|$ -- длина соответствующего отрезка, $\xi_i$ -- точка на соответствующем отрезке
	
	Интегральная сумма для интеграла {\bf первого рода} -- сумма вида $\sum\limits_{k=1}^{n} f(r(\xi_i))\cdot(r(t_k) = r(t_{k - 1})))$
	
	Собственно, криволинейный интеграл это интегральная сумма с $n$ устремленным в бесконечность
}

Похоже на обычный определенный интеграл, только тут мы вместо оси выравниваемся на кривую какую-то, и по факту считаем площидь криволинейного цилиндра между кривой в пространстве и ее проекцией (вроде бы, но это не точно)


\D{Пусть $\Phi$ -- гладкая, ограниченная полная поверхность. Пусть далее на $\Phi$ задана функция $f(M) = f(x, y, z)$. Рассмотрим разбиение $T$ этой поверхности на часть $\Phi_i (i=1, ..., n)$ кусочно-гладкими кривыми и на каждой такой части выберем произвольную точку $M_i(x_i, y_i, z_i)$. Вычислив значение функции в этой точке $f(M_i) = f(x_i, y_i, z_i)$ и, приняв за $\sigma_i$ площадь поверхность $\Phi_i$, рассмотрим сумму $$I\{\Phi_i, M_i\} = \sum_i f(M_i)\sigma_i$$. Тогда число I называется пределом сумм $i\{\Phi_i, M_i\}$ если $$\forall \epsilon > 0 \exists \delta > 0 \forall T: d(T) < \delta \forall \{M_i\} |I\{\Phi_i, M_i\} - I| < \epsilon$$
	Предел $I$ сумм $I\{\Phi_i, M_i\}$ при $d(T) \rightarrow 0$
	называется {\bf поверхностным интегралом первого рода} от функции $f(M)$ по поверхности $\Phi$ и обозначается $$I = \iint\limits_{\Phi} f(M)d\sigma$$} 

По сути -- берем поверхность в пространстве, а дальше как в криволинейном -- вместо отрезков оже куски пространства и т.д. получается магия какая-то.

\subsubsection{Формула Грина}

\D{Пусть $C$ -- положительно ориентированная кусочно-гладкая замкнутая кривая на плоскости, а $D$ -- область, ограниченная кривой $C$. Если фунеции $P = P(x, y)$, $Q = Q(x, y)$ определены в области $D$ и имеют неприрывные частные производные $\frac{\partial P}{\partial y}, \frac{\partial Q}{\partial x}$, то $$\oint Pdx + Qdy = \iint\limits_{D}(\frac{\partial Q}{\partial x} - \frac{\partial P}{\partial y})dxdy$$}
{\bf Док-во и еще:} \href{https://ru.wikipedia.org/wiki/%D0%A2%D0%B5%D0%BE%D1%80%D0%B5%D0%BC%D0%B0_%D0%93%D1%80%D0%B8%D0%BD%D0%B0}{here}

\subsubsection{Формула Стокса}

\D{Пусть на ориентируемом многообразии $M$ размерности $n$ заданы положительно ориентированное ограниченное $p-$мерное подмногообразие $\sigma (1 \le p \le n)$ и дифференциальная форма $\omega$ степени $p - 1$ класса $C^1$. Тогда если граница подмногообразия $\partial \sigma$ положительно ориентированаб то $$\int\limits_{\sigma}d\omega = \int\limits_{\partial \sigma \omega}$$}

Грубо говоря взяли поверхность, и с помощью дифференциалов перешли к интегралу по границе поверхности, как-то так, но надо глубже разбираться потому что очень много определений которые надо помнить

\subsubsection{Формула Остроградского (Гаусс сосать)}

\D{Пусть теперь $\partial V$ -- кусочно-гладкая гипперповерхность $(p = n - 1)$, ограничивающая некоторую область $V$ в $n-$мерном пространстве. Тогда интеграл дивергенции (это оператор который отображает векторное поле на скалярное -- $div F = \lim\limits_{V \rightarrow 0} \frac{\Phi_F}{V}$, где $\Phi_F$ -- поток векторного поля $F$ через сферическую поверхность площадью $S$ ограничивающую объем $V$, хуита какая-то хочу объяснение на пальцах) поля по области равен потоку поля через границу области $\partial V$: $$\int\limits_{V} div F dV = \int\limits_{\partial V} F d \Sigma$$.
	
	В трехмерном пространстве $(n = 3)$ с координатами $\{x, y, z\}$ эквивалентнно $$\int\limits_{\partial V} F d \Sigma = \int\limits_{V}(\frac{\partial P}{\partial x} + \frac{\partial Q}{\partial y} + \frac{\partial R}{\partial z}) dV$$, или $$\iiint\limits_{\partial V} Pdydz + Qdzdx + Rdxdy = \iint\limits_{V} (\frac{\partial P}{\partial x} + \frac{\partial Q}{\partial y} + \frac{\partial R}{\partial z})dxdydz$$ }


Понятно что тут взяли и применили стокса на какой-то случай, но чет пиздец ребята)))


\subsection{Функциональные ряды, свойства равномерно сходящихся функциональных рядов. Степенные ряды. Ряд Тейлора.}

\subsubsection{Функциональные ряды}

\D {Функциональный ряд -- ряд, каждым членом которого является функция $u_k(x)$
	
	Обозначается $\sum\limits_{k=1}^{\infty} u_k(x)$}

Функциональная последовательность $u_k(x)$ сходится {\bf поточечно} к функции $u(x)$, если $\forall x \in E \exists \lim\limits_{k \rightarrow \infty} u_k(x) = u(x)$

{\bf Равномерная сходимость} -- существует функция $u(x): E \rightarrow \mathbb{C}$ такая, что 

$sup |u_k(x) - u(x)| \xrightarrow {k \rightarrow \infty} 0, x \in E$

Функциональный ряд называется сходящимся {\bf поточечно}, если последовательность $S_n(x) = \sum\limits_{k=1}^{n} u_k(n)$ сходится поточечно. Аналогично для равномерной сходимости.

{\bf Необходимое условие равноменой сходимости ряда}

$u_k(x) \rightrightarrows 0$ при $k \rightarrow \infty$

Или, что эквивалентно $\forall \epsilon > 0 \exists n_0(\epsilon) \in \mathbb{N} : \forall x \in X, \forall n > n_0 |u_n(x)| < \epsilon$, где $X$ -- область сходимости

{\bf Свойства}

\begin{enumerate}
	\item {\bf Теоремы о непрерывности}
	
	Последовательность непрерывных в точке функций сходится к функции, непрерывной в этой точке.
	
	Последовательность $u_k(x) \rightrightarrows u(x)$
	
	$\forall k:$ функция $u_k(x)$ непрерывна в точке $x_0$
	
	Тогда и $u(x)$ непрерывна в $x_0$
	
	Ряд непрерывных в точке функций сходится к функции, непрерывной в этой точке.
	
	Ряд $\sum\limits_{k=0}^{\infty}u_k(x) \rightrightarrows S(x)$
	
	$\forall k$: функция непрерывна в точкке $x_0$
	
	Тогда $S(x)$ непрерывна в  $x_0$
	
	\item {\bf Теоремы об интегрировании}
	
	Рассматриваются действительнозначные функции на отрезке действительной оси
	
	{\it Теорема о переходе к пределу под знаком интеграла}
	
	$\forall k:$ функция $u_k(x)$ непрерывна на отрезке $[a, b]$
	
	$u_k(x) \rightrightarrows u(x)$ на $[a, b]$
	
	Тогда числовая последовательность $\{\int\limits_{a}^{b} u_k(x) dx\}$ сходится к конечному пределу $\int\limits_a^b u(x) dx$
	
	{\it Теорема о почленном интегрировании}
	
	$\forall k:$ функция $u_k(x)$ непрерывна на отрезке $[a, b]$
	
	$\sum\limits_{k=1}^{\infty}u_k(x) \rightrightarrows S(x)$ на $[a, b]$
	
	Тогда числовой ряд $\sum\limits_{k=1}^{\infty}\int\limits_{a}^{b} u_k(x) dx$ сходится и равен $\int\limits_a^b S(x) dx$
	
	\item {\bf Теоремы о дифференцировании}
	
	Рассматриваются действительнозначные функции на отрезке действительной оси
	
	{\it Теорема о дифференцировании под пределом}
	
	$\forall k:$ функция $u_k(x)$ дифференцируема (имеет непрерывную производную) на отрезке $[a, b]$
	
	$\exists c \in [a, b]: u_k(c)$ сходится к конечному пределу
	
	$u_k^{\prime}(x)  \rightrightarrows \omega(x)$ на отрезке $[a, b]$
	
	Тогда $\exists u(x): u_k(x) \rightrightarrows u(x),\ u(x)$ -- дифференцируема на $[a, b],\ u^{\prime}(x) = \omega(x)$ на $[a, b]$
	
	{\it Теорема о почленном дифференцировании}
	
	$\forall k:$ функция $u_k(x)$ -- дифференцируема на отрезке $[a, b]$
	
	$\exists c \in [a, b]: \sum\limits_{k=1}^{\infty} u_k(c)$ сходится
	
	$\sum\limits_{k=1}^{\infty}u_k^{\prime}(x)$ равномерно сходится на отрезке $[a, b]$
	
	Тогда $\exists S(x): \sum\limits_{k=1}^{\infty}u_k(x) \rightrightarrows S(x),\ S(x)$ -- дифференцируем на $[a, b], S^{\prime}(x) = \sum\limits_{k=1}^{\infty}u_k^{\prime}(x)$ на $[a, b]$
	
\end{enumerate}

\subsubsection{Степенные ряды}

\D {{\bf Степенной ряд с одной переменной} -- это формальное алгебраическое вырадение вида $$F(x) = \sum\limits_{n=0}^{\infty}a_nX^n$$ в котором коэффициенты $a_n$ берутся из некоторого кольца $R$, обычно вещественные или комплексные числа}

Для степенных рядов есть несколько теорем об их сходимости

\begin{itemize}
	\item Певая теорема Абеля
	
	Пусть ряд $\sum a_n x^n$ сходится в точке $x_0$. Тогда этот ряд сходится абсолютно в круге $|x| < |x_0|$ и равномерно по $x$ на любом компактном подмножестве этого круга.
	
	Отсюда можно сделать вывод что если ряд расходится при $x = x_0$, то он расходится при всех $|x| > |x_0|$
	
	Появляется понятие радиуса сходимости $R$, при котором при $|x| < R$ ряд сходится абсолютно, про $|x| > R$ расходится
	
	\item Формула Коши-Адамара (Коши-Амидамару)
	
	Значение радиуса сходимости степенного ряда может быть вычислено по формуле $\frac{1}{R} = \uplim\limits_{n \rightarrow +\infty}|a_n|^{1/n}$
	
	\item Признак Даламбера
	
	Если при $n > N$ и $\alpha > 1$ выполнено неравенство $|\frac{a_n}{a_{n+1}}| \ge R(1 + \frac{\alpha}{n})$ тогда степенной ряд $\sum a_n x^n$ сходится во всех точках окружности $|x| = R$ абсолютно и равномерно по $x$
	
	\item Признак Дирихле
	
	Если все коэффициенты степенного ряда $\sum a_n x^n$ положительны и последовательность $a_n$ монотонно сходится к 0б тогда этот ряд сходится во всех точках окружности $|x| = 1$, кроме, может быть, точки $x = 1$
\end{itemize}


\subsubsection{Ряд Тейлора}

\D {Ряд Тейлора -- разложение функции в бесконечную сумму степенных функций
	
	Многочленом Тейлора функции $f(x)$ вещественной переменной $x$, дифференцируемой $k$ раз в точке $a$ называется конечная сумма 
	$$f(x) = \sum\limits_{n=0}{k}\frac{f^{(n)}(a)}{n!} (x - a)^n = f(a) + f^{\prime}(a)(x - a) + \frac{f^{(2)}(a)}{2!}(x - a)^2 + ... + \frac{f^{(k)}(a)}{k!}(x - k)^k$$
	
	Рядом Тейлора в точке $a$ функции $f(x)$ , бесконечно диффиренцируемой в окрестности точки $a$, называется формальный степенной ряд
	$$f(x) = \sum\limits_{n=0}^{+\infty}\frac{f^{(n)}(a)}{n!}(x - a)^n$$
	
	Другими словами, рядом Тейлора функции $f(x)$ в точке $a$ называется ряд разложения функции по положительным степеням двучлена $(x - a)$
	
}

Еще есть формула Тейлора, это просто частичная сумма ряда вроде как.

В случае $a = 0$ это все безобразие -- {\bf ряд Маклорена}


\subsection{Определители и их свойства. Системы линейных алгебраических уравнений и их исследование. Методы решения систем линейных алгебраических уравнений.}

\subsubsection{Определитель}

\D {Определитель -- скалярная величина, которая характеризует ориентированное "растяжение"\ или "сжатие" \  многомерного евклидова пространства после преобразования матрицей. Имеет смысл только для квадратных матриц. Стандартные обозначения -- $\det(A), |A|, \Delta (A)$ }

{\bf Определение через перестановки}

Для квадратной матрицы $A = (a_{ij})$ размера $n \times n$ ее определитель вычисляется по формуле $$det A = \sum\limits_{\alpha_1, \alpha_2, ..., \alpha_n}(-1)^{N(\alpha_1, \alpha_2, ..., \alpha_n)} \cdot a_{1 \alpha_1} a_{2 \alpha_2} ... a_{n \alpha_n}$$

Где суммирование проводится по всем перестановкам $\alpha_1, \alpha_2, ..., \alpha_n$ чисел $1, 2, ..., n$, а $N(\alpha_1, \alpha_2, ..., \alpha_n)$ обозначает число инверсий в перестановке $\alpha_1, ..., \alpha_n$

Таким образом в определитель входит $n!$ слагаемых.


{\bf Аксиоматическое построение}

Понятие определителя может быть введено на основе его свойств. А именно, определителем вещественной матрицы называется функция $det: \mathbb{R}^{n \times n} \rightarrow \mathbb{R}$, обладающая следующими тремя свойствами

\begin{itemize}
	\item $\det(A)$ -- кососимметрическая функция строк(столбцов) матрицы $A$, т.е. не меняется при четных перестановках аргументов
	\item $\det(A)$ -- полилинейная функция строк (столбцов) матрицы $A$
	\item $\det(E) = 1$, где $E$ -- единичная $n \times n$ матрица.
\end{itemize}

Еще свойства

\begin{enumerate}
	\item $\det E = 1$
	\item $\det cA = c^n \det A$
	\item $\det A^T = \det A$
	\item $\det(AB) = \det A \cdot \det B$
	\item $\det A^{-1} = (\det A)^{-1}$, причем матрица обратима тогда и только тогда, когда обратим ее определитель
	\item Существует ненулевое решение уравнения $AX = 0$ тогда и только тогда, когда $\det A = 0$ 
\end{enumerate}

\subsubsection{Системы линейных уравнений}

В классическом варианте коэффициенты при переменных, свободные члены и неизвестные считаются вещественными числами

Общий вид системы линейных алгебраических уравнений:

$$
{\begin{cases}a_{11}x_{1}+a_{12}x_{2}+\dots +a_{1n}x_{n}=b_{1}\\a_{21}x_{1}+a_{22}x_{2}+\dots +a_{2n}x_{n}=b_{2}\\\dots \\a_{m1}x_{1}+a_{m2}x_{2}+\dots +a_{mn}x_{n}=b_{m}\\\end{cases}}$$

где $m$ -- количество уравнений, а $n$ -- количество переменных.

Система называется {\bf однородной}, если все ее свободные члены ($b_i$) равны нулю, иначе -- {\bf неоднородной}

Система называется {\bf совместной}, если она имеет хотя бы одно решение, иначе несовместной. Решения считаются различными, если хотя бы одно из значений переменных не совпадает. Если решение одно, то система {\bf определенная}

Также есть запись в матричной форме

$
\begin{pmatrix}
	a_{11} & a_{12} & \cdots & a_{1n} \\
	a_{21} & a_{22} & \cdots & a_{2n} \\
	\vdots & \vdots & \ddots & \vdots \\
	a_{m1} & a_{m2} & \cdots & a_{mn} 
\end{pmatrix}
\begin{pmatrix}
	x_1 \\
	x_2 \\
	\vdots \\
	x_n
\end{pmatrix} 
=
\begin{pmatrix}
	b_1 \\
	b_2 \\
	\vdots \\
	b_m
\end{pmatrix}$

или $Ax=b$. Если к матрицу $A$ приписать справа столбец свободных членов, матрица будет называться расширенной.

Системы называются {\bf эквивалентными}, если множество их решений совпадает, т.е. если решение одной системы является решением другой.

Можно менять уравнения домножением на константу кроме 0, на сумму с другим уравнением, на линейную комбинацию с учетом этой. Будут получаться эквивалентные системы.

\subsubsection{Методы решения систем уравнений}

\begin{itemize}
	\item {\it Метод Гаусса}
	
	Приводим матрицу к ступенчатому виду, остались какие-то переменные. Назовем главными те, которые на диагонали, остальные -- свободные. Теперь переносим свободные через =, и присваивая им все возможные значения легко получить решения для главных, а значит для всей системы.
	
	Подробнее \href{https://ru.wikipedia.org/wiki/%D0%9C%D0%B5%D1%82%D0%BE%D0%B4_%D0%93%D0%B0%D1%83%D1%81%D1%81%D0%B0}{здесь}
	
	\item {\it Метод Гаусса-Жордана}
	
	\begin{enumerate}
		\item Выбирают первый слева столбец матрицы, в котором есть хоть одно отличное от нуля значение.
		\item Если самое верхнее число в этом столбце ноль, то меняют всю первую строку матрицы с другой строкой матрицы, где в этой колонке нет нуля.
		\item Все элементы первой строки делят на верхний элемент выбранного столбца.
		\item Из оставшихся строк вычитают первую строку, умноженную на первый элемент соответствующей строки, с целью получить первым элементом каждой строки (кроме первой) ноль.
		\item Далее проводят такую же процедуру с матрицей, получающейся из исходной матрицы после вычёркивания первой строки и первого столбца.
		\item После повторения этой процедуры $(n-1)$ раз получают верхнюю треугольную матрицу
		\item Вычитают из предпоследней строки последнюю строку, умноженную на соответствующий коэффициент, с тем, чтобы в предпоследней строке осталась только 1 на главной диагонали.
		\item Повторяют предыдущий шаг для последующих строк. В итоге получают единичную матрицу и решение на месте свободного вектора (с ним необходимо проводить все те же преобразования).
		
	\end{enumerate}
	
	Короче приводим к единичной, что осталось у свободных членов и есть решение
	
	\item {\it Метод Крамера}
	
	Для системы $n$ линейных уравнений с $n$ неизвестными (над произвольным полем)
	
	$\begin{cases}a_{11}x_{1}+a_{12}x_{2}+\ldots +a_{1n}x_{n}=b_{1}\\a_{21}x_{1}+a_{22}x_{2}+\ldots +a_{2n}x_{n}=b_{2}\\\cdots \cdots \cdots \cdots \cdots \cdots \cdots \cdots \cdots \cdots \\a_{n1}x_{1}+a_{n2}x_{2}+\ldots +a_{nn}x_{n}=b_{n}\\
	\end{cases}$
	
	с определителем матрицы системы $\Delta$ , отличным от нуля, решение записывается в виде
	
	$ x_{i}={\frac {1}{\Delta }}{\begin{vmatrix}a_{11}&\ldots &a_{1,i-1}&b_{1}&a_{1,i+1}&\ldots &a_{1n}\\a_{21}&\ldots &a_{2,i-1}&b_{2}&a_{2,i+1}&\ldots &a_{2n}\\\ldots &\ldots &\ldots &\ldots &\ldots &\ldots &\ldots \\a_{n-1,1}&\ldots &a_{n-1,i-1}&b_{n-1}&a_{n-1,i+1}&\ldots &a_{n-1,n}\\a_{n1}&\ldots &a_{n,i-1}&b_{n}&a_{n,i+1}&\ldots &a_{nn}\\\end{vmatrix}}$
	(i-ый столбец матрицы системы заменяется столбцом свободных членов).
	
	Подставляем вместо соответствующего столбца свободные члены, считаем определитель, делим на определитель всей матрицы, и получаем $x_i$ соответствующий данному столбцу.
	
	\item {\it Матричный метод}
	
	Есть система вида $AX=B$, тогда решением будет $X=A^{-1}B$
	
	Чтобы работало, нужно чтобы матрица $A$ была невырождена, т.е. чтобы определитель был не равен 0
	
\end{itemize}

Еще есть какие-то итерационные и другие методы, но они звучат и выглядят не очень полезными, но можно посмотреть \href{https://ru.wikipedia.org/wiki/%D0%A1%D0%B8%D1%81%D1%82%D0%B5%D0%BC%D0%B0_%D0%BB%D0%B8%D0%BD%D0%B5%D0%B9%D0%BD%D1%8B%D1%85_%D0%B0%D0%BB%D0%B3%D0%B5%D0%B1%D1%80%D0%B0%D0%B8%D1%87%D0%B5%D1%81%D0%BA%D0%B8%D1%85_%D1%83%D1%80%D0%B0%D0%B2%D0%BD%D0%B5%D0%BD%D0%B8%D0%B9#:~:text=%D0%A1%D0%B8%D1%81%D1%82%D0%B5%D0%BC%D0%B0%20%D0%BB%D0%B8%D0%BD%D0%B5%D0%B9%D0%BD%D1%8B%D1%85%20%D0%B0%D0%BB%D0%B3%D0%B5%D0%B1%D1%80%D0%B0%D0%B8%D1%87%D0%B5%D1%81%D0%BA%D0%B8%D1%85%20%D1%83%D1%80%D0%B0%D0%B2%D0%BD%D0%B5%D0%BD%D0%B8%D0%B9%20(%D0%BB%D0%B8%D0%BD%D0%B5%D0%B9%D0%BD%D0%B0%D1%8F,%D0%BB%D0%B8%D0%BD%D0%B5%D0%B9%D0%BD%D1%8B%D0%BC%20%E2%80%94%20%D0%B0%D0%BB%D0%B3%D0%B5%D0%B1%D1%80%D0%B0%D0%B8%D1%87%D0%B5%D1%81%D0%BA%D0%B8%D0%BC%20%D1%83%D1%80%D0%B0%D0%B2%D0%BD%D0%B5%D0%BD%D0%B8%D0%B5%D0%BC%20%D0%BF%D0%B5%D1%80%D0%B2%D0%BE%D0%B9%20%D1%81%D1%82%D0%B5%D0%BF%D0%B5%D0%BD%D0%B8}{тут}

\subsection{Линейные операторы в конечномерном пространстве и их матричноепредставление. Характеристический многочлен, собственные числа и собственные вектора линейного оператора. Сопряженные и самосопряженные операторы.}

\subsubsection{ Линейные операторы}

\D {Пусть $X$ и $Y$ -- линейные пространства над полем $F$. Отображение $\mathcal{A}: X \rightarrow Y$ называется линейным оператором, если $\forall x_1, x_2 \in X, \forall \lambda \in F$
	\begin{itemize}
		\item $\mathcal{A}(x_1 + x_2) = \mathcal{A}(x_1) + \mathcal{A}(x_2)$
		\item $\mathcal{A}(\lambda \cdot x_1) = \lambda \cdot \mathcal{A}(x_1)$
	\end{itemize}
	
	Линейный оператор $\mathcal{A}: X \rightarrow X$ называется автоморфизмом (или гомоморфизмом)
	
}

Операторы равны, если переводят элементы первого пространства в одинаковые элементы второго пространства.

\D{Пусть $\mathcal{A}: X \rightarrow Y$
	
	Пусть п.п. $X \leftrightarrow \{e_k\}_{k=1}^{n},\ \dim X = n$
	
	Пусть п.п. $Y \leftrightarrow \{h_k\}_{k=1}^{n},\ \dim Y = m$
	
	$\underset{{1\le k \le n}}{\mathcal{A}e_k} = \sum\limits_{i = 1}^{m} \alpha_k^i \cdot h_i \Rightarrow A = ||\alpha_k^i||,$ где $1 \le i \le m, 1 \le k \le m$
	
	$A = \begin{pmatrix}
		\alpha_1^1 & ... & \alpha_n^1\\
		\alpha_1^2 & ... & \alpha_n^2\\
		... & ... & ...\\
		\alpha_1^n & ... & \alpha_n^n\\
		
	\end{pmatrix}$
	
}

\subsubsection{Характеристический многочлен}

\D {Для данной матрицы $A$, $\chi(\lambda) = \det (A - \lambda E)$, где $E$ -- единичная матрица, является многочленом от $\lambda$, который называется {\bf характеристическим многочленом} матрицы $A$ (видимо можно отождествить матрицу с линейным оператором, тогда будет многочлен для оператора)}

Ценность характеристического многочлена в том, то собственные значения матрицы являются его корнями. Действительно, если уравнение $Av = \lambda v$ имеет ненулевое решение, то $(A - \lambda E)v = 0$, значит матрица $A - \lambda E$ вырождена и ее определитель $\det (A - \lambda E) = \chi(\lambda)$ равен 0

{\bf Свойства}

\begin{itemize}
	\item Для матрицы $n \times n$ характеристический многочлен имеет степень $n$
	\item Все корни характеристического многочлена матрицы являются ее собственными значениями
	\item Теорема Гамильтона-Кэли -- если $\chi(\lambda)$ -- характеристический многочлен матрицы $A$, то $\chi(A) = 0$
	\item Характеристические многочлены подобных матриц совпадают
	\item Характеристический многочлен обратной матрицы $\chi_{A^{-1}}(\lambda) = \frac{(-\lambda)^n}{\det A}\chi_A(1/\lambda)$
	\item Если $A$ и $B$ две матрицы $n \times n$, то $\chi_{AB} = \chi_{BA}$. В частности $tr(AB) = tr(BA), \det(AB) = \det (BA)$
	\item В более общем виде, если $A$ --матрица $m \times n$, а $B$ -- матрица $n \times m$, причем $m < n$, так что $AB$ и $BA$ --квадратные матрицы размеров $m$ и $n$ соответственно, то $\chi_{BA}(\lambda) = \lambda ^{n - m}\chi_{AB}(\lambda)$ 
\end{itemize}


\D{Пусть $L$ -- линейное пространство над полем $K$,
	
	$\mathcal{A}:L \rightarrow L$ -- линейный оператор
	
	{\bf Собственным вектором} линейного оператора $\mathcal{A}$ называется такой ненулевой вектор $x \in L$, что для некоторого $\lambda \in K: \mathcal{A}x = \lambda x$
	
	При этом $\lambda$ называют {\bf собственным числом} оператора $\mathcal{A}$
	
}

{\bf Свойства}

\begin{itemize}
	\item Собственные векторы, отвечающие различным собственным значениям, образуют ЛНЗ набор
	\item Еще какие-то леммы есть, подробнее см на \href{https://neerc.ifmo.ru/wiki/index.php?title=%D0%A1%D0%BE%D0%B1%D1%81%D1%82%D0%B2%D0%B5%D0%BD%D0%BD%D1%8B%D0%B5_%D0%B2%D0%B5%D0%BA%D1%82%D0%BE%D1%80%D1%8B_%D0%B8_%D1%81%D0%BE%D0%B1%D1%81%D1%82%D0%B2%D0%B5%D0%BD%D0%BD%D1%8B%D0%B5_%D0%B7%D0%BD%D0%B0%D1%87%D0%B5%D0%BD%D0%B8%D1%8F}{говне}
\end{itemize}

\subsubsection{Сопряженные и самосопряженные операторы}

\D{Пусть $E, L$ -- линейные пространства, а $E^*, L^*$ -- сопряженные линейные пространства (пространства линейных функционалов, определенных на $E$ и $L$). Тогда для любого линейного оператора $\mathcal{A}: E \rightarrow L$ и любого линейного функционала $g \in L^*$ определен линейный функционал $F \in E^*$ -- суперпозиция $g$ и $A: f(x) = g(A(x))$. Отображение $g \rightarrow f$ называется сопряженным линейным оператором и обозначается $\mathcal{A^*}: L^* \rightarrow E^*$. Если кратко, то $(\mathcal{A^*}g, x) = (g, \mathcal{A}x)$
	
	Если же $\mathcal{A^*} = \mathcal{A}$, то такой оператор называется самосопряженным, для него $(\mathcal{A}x, y) = (x, \mathcal{A}y)$
	
}

\subsection{Задача Коши для системы обыкновенных дифференциальных уравнений. Существование и единственность решения. Устойчивость.}

\subsection{Линейные обыкновенные дифференциальные уравнения и системы. Фундаментальная система решений. Метод вариации постоянных для решения неоднородных уравнений.}

\subsection{Дискретные случайные величины. Математическое ожидание и дисперсия. Стандартные дискретные распределения (Бернулли, биномиальное, геометрическое, Пуассона).}

\subsection{Непрерывные случайные величины и их функции распределения. Математическое ожидание и дисперсия. Стандартные непрерывные распределения (равномерное, показательное, нормальное).}

\subsection{Вероятностные неравенства Йенсена, Маркова и Чебышёва. Правило трёх сигм. Закон больших чисел.}

\subsection{Множества и операции над ними. Булевы функции, КНФ, ДНФ. Базисы, теорема Поста.}

\subsection{Комбинаторные объекты. Коды Грея. Формула включения-исключения. Лемма Бернсайда и Теорема Пойа. Числа Стирлинга. Подсчёт деревьев. Метод производящих функций.}

\subsection{Детерминированные и недетерминированные конечные автоматы, их эквивалентность. Минимизация ДКА.}

\subsection{Математическая логика. Понятие доказательства. Правила вывода. Теоремы Гёделя.}

\subsection{Контекстно-свободные грамматики. Эффективные методы разбора: LL(k)-, LR(k)- и LALR-грамматики.}

\subsection{Комбинаторная теория сложности. Временная и емкостная сложность. Сложностные классы P, NP, PS. Сведение, NP-полные задачи.}

\subsection{Марковские цепи, Эргодические цепи, Регулярные цепи. Алгоритм Витерби.}

\subsection{Линейные структуры данных. Амортизационный анализ. Поисковые структуры данных. Запросы на отрезках. Персистентные структуры данных.}

\subsection{Графы. Обход графов. Поиск кратчайших путей. Задача о паросочетании, максимальном потоке и максимальном потоке минимальной стоимости.}

\subsection{Строки. Поиск строки в подстроке. Бор, алгоритм Ахо-Корасика. Суффиксные массивы и деревья.}

\subsection{Постановка задачи линейного программирования. Двойственность задачи ЛП.}

\subsection{Градиентные методы. Метод сопряжения градиентов. Минимизация квадратичных функций. Метод Ньютона.}

\newpage

\section{Программирование и вычислительная техника}

\subsection{Архитектура ЭВМ. Архитектура фон Неймана и гарвардская архитектура. Основные принципы и их альтернативы. Архитектура набора команд (ISA), CISC и RISC архитектуры.}

\subsubsection{Архитектура ЭВМ}
\textbf{Архитектура ЭВМ} - 
это модель, устанавливающая принципы организации вычислительной системы, состав, 
порядок и взаимодействие основных частей ЭВМ, функциональные возможности, 
удобство эксплуатации, стоимость, надежность.

\subsubsection{Архитектура фон Неймана и гарвардская архитектура. Основные принципы и их альтернативы.}


\href{https://www.currentschoolnews.com/ru/%D0%BD%D0%BE%D0%B2%D0%BE%D1%81%D1%82%D0%B8-%D0%BE%D0%B1%D1%80%D0%B0%D0%B7%D0%BE%D0%B2%D0%B0%D0%BD%D0%B8%D1%8F/%D0%A0%D0%B0%D0%B7%D0%BD%D0%B8%D1%86%D0%B0-%D0%BC%D0%B5%D0%B6%D0%B4%D1%83-%D1%84%D0%BE%D0%BD-%D0%9D%D0%B5%D0%B9%D0%BC%D0%B0%D0%BD%D0%B0-%D0%B8-%D0%93%D0%B0%D1%80%D0%B2%D0%B0%D1%80%D0%B4%D1%81%D0%BA%D0%BE%D0%B9-%D0%B0%D1%80%D1%85%D0%B8%D1%82%D0%B5%D0%BA%D1%82%D1%83%D1%80%D1%8B/}{Подробнее тут}
\\

\textbf{Особенности архитектуры фон Неймана:}
\begin{enumerate}
	\item Архитектура фон Неймана - это теоретический проект, основанный на концепции компьютера с хранимой программой.
	\item Архитектура фон Неймана имеет только одну шину, которая используется как для извлечения инструкций, так и для передачи данных. Что еще более важно, операции должны быть запланированы, потому что они не могут быть выполнены одновременно.
	\item В архитектуре фон Неймана процессору потребовалось бы два тактовых цикла для выполнения инструкции.
	\item Архитектура фон Неймана обычно используется буквально на всех машинах, от настольных компьютеров, ноутбуков, высокопроизводительных компьютеров до рабочих станций.
\end{enumerate}

\textbf{Особенности Гарвардской aрхитектуры:}
\begin{enumerate}
	\item Гарвардская архитектура - это современная компьютерная архитектура, основанная на компьютерной модели ретранслятора Harvard Mark I.
	\item Гарвардская архитектура имеет отдельное пространство памяти для инструкций и данных, которое физически разделяет сигналы и код хранения и память данных, что, в свою очередь, позволяет получить доступ к каждой из систем памяти одновременно.
	\item В гарвардской архитектуре процессор может выполнить инструкцию за один цикл, если были установлены соответствующие планы конвейерной обработки.
	\item Гарвардская архитектура - это новая концепция, используемая специально в микроконтроллерах и цифровой обработке сигналов (DSP).
	\item Гарвардская архитектура - сложный вид архитектуры, поскольку в ней используются две шины для команд и данных, что делает разработку блока управления сравнительно более дорогой.
\end{enumerate}

\subsubsection{Архитектура набора команд (ISA), CISC и RISC архитектуры.}

\textbf{ISA} - архитектура набора команд, которая включает в себя систему и режимы адресации, спецификацию команд процессора, ригистры и типы данных, систему прерываний (для обработки ошибок во время вычисления)

Процессоры можно разделить по сложности набора команд:
\emph{CISC} (Complex Inst. Set Computer) vs \emph{RISC} (Reduced Inst. Set Computer).

\textbf{RISC} - Главная идея в поддержании небольшого набора простых и быстрых команд, под которые соптимизирован процессор. Предполагалось, что  сложные вызываются редко. Для этой архитектуры характерен фиксированная длина кодов команд, так как их проще декодировать и она не большая.

\textbf{CISC} - Тут у нас много команд для всяких сложных операций которые реализованы непосредственно на плате, что позволяет их ускорить, но это усложняет архитектуру и может мешать эффективной реализации простых команд, поэтому по количество операций в секунду такие процессоры проигрывают RICS. Так как количество команды большое, то характерна переменная длина кода (с Хаффман кодированием).

В современном мире по факту используется компромисс между этими подходами. Процессоры - CISC по спецификации, но внутри скорее RICS и конвертируют сложные в более простые + 0-level cache.

\subsection{Архитектура ЭВМ. Кэш-память. Многоуровневая организация кэш-памяти. Протоколы когерентности кэш-памяти}

\end{document}
