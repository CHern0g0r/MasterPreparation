%----------------------------------------------------------------------------------------
%	PACKAGES AND OTHER DOCUMENT CONFIGURATIONS
%----------------------------------------------------------------------------------------

\documentclass{article}
% \documentclass[14pt]{extarticle}
\usepackage{pdfpages}

%%%%%%%%%%%%%%%%%%%%%%%%%%%%%%%%%%%%%%%%%
% Lachaise Assignment
% Structure Specification File
% Version 1.0 (26/6/2018)
%
% This template originates from:
% http://www.LaTeXTemplates.com
%
% Authors:
% Marion Lachaise & François Févotte
% Vel (vel@LaTeXTemplates.com)
%
% License:
% CC BY-NC-SA 3.0 (http://creativecommons.org/licenses/by-nc-sa/3.0/)
% 
%%%%%%%%%%%%%%%%%%%%%%%%%%%%%%%%%%%%%%%%%

%----------------------------------------------------------------------------------------
%	PACKAGES AND OTHER DOCUMENT CONFIGURATIONS
%----------------------------------------------------------------------------------------

\usepackage{amsmath,amsfonts,stmaryrd,amssymb,amsthm} % Math packages

\usepackage{mathtext}

\usepackage{enumerate} % Custom item numbers for enumerations

\usepackage[ruled]{algorithm2e} % Algorithms

\usepackage[framemethod=tikz]{mdframed} % Allows defining custom boxed/framed environments

\usepackage{listings} % File listings, with syntax highlighting
\lstset{
	basicstyle=\ttfamily, % Typeset listings in monospace font
}

\usepackage[unicode]{hyperref}
%----------------------------------------------------------------------------------------
%	DOCUMENT MARGINS
%----------------------------------------------------------------------------------------

\usepackage{geometry} % Required for adjusting page dimensions and margins

\geometry{
	paper=a4paper, % Paper size, change to letterpaper for US letter size
	top=2.5cm, % Top margin
	bottom=3cm, % Bottom margin
	left=2.5cm, % Left margin
	right=2.5cm, % Right margin
	headheight=14pt, % Header height
	footskip=1.5cm, % Space from the bottom margin to the baseline of the footer
	headsep=1.2cm, % Space from the top margin to the baseline of the header
	%showframe, % Uncomment to show how the type block is set on the page
}

%----------------------------------------------------------------------------------------
%	FONTS
%----------------------------------------------------------------------------------------

\usepackage[utf8]{inputenc} % Required for inputting international characters
\usepackage[T1, T2A]{fontenc} % Output font encoding for international characters

\usepackage[english,russian]{babel}

\usepackage{XCharter} % Use the XCharter fonts

%----------------------------------------------------------------------------------------
%	COMMAND LINE ENVIRONMENT
%----------------------------------------------------------------------------------------

% Usage:
% \begin{commandline}
%	\begin{verbatim}
%		$ ls
%		
%		Applications	Desktop	...
%	\end{verbatim}
% \end{commandline}

\mdfdefinestyle{commandline}{
	leftmargin=10pt,
	rightmargin=10pt,
	innerleftmargin=15pt,
	middlelinecolor=black!50!white,
	middlelinewidth=2pt,
	frametitlerule=false,
	backgroundcolor=black!5!white,
	frametitle={Command Line},
	frametitlefont={\normalfont\sffamily\color{white}\hspace{-1em}},
	frametitlebackgroundcolor=black!50!white,
	nobreak,
}

% Define a custom environment for command-line snapshots
\newenvironment{commandline}{
	\medskip
	\begin{mdframed}[style=commandline]
}{
	\end{mdframed}
	\medskip
}

%----------------------------------------------------------------------------------------
%	FILE CONTENTS ENVIRONMENT
%----------------------------------------------------------------------------------------

% Usage:
% \begin{file}[optional filename, defaults to "File"]
%	File contents, for example, with a listings environment
% \end{file}

\mdfdefinestyle{file}{
	innertopmargin=1.6\baselineskip,
	innerbottommargin=0.8\baselineskip,
	topline=false, bottomline=false,
	leftline=false, rightline=false,
	leftmargin=2cm,
	rightmargin=2cm,
	singleextra={%
		\draw[fill=black!10!white](P)++(0,-1.2em)rectangle(P-|O);
		\node[anchor=north west]
		at(P-|O){\ttfamily\mdfilename};
		%
		\def\l{3em}
		\draw(O-|P)++(-\l,0)--++(\l,\l)--(P)--(P-|O)--(O)--cycle;
		\draw(O-|P)++(-\l,0)--++(0,\l)--++(\l,0);
	},
	nobreak,
}

% Define a custom environment for file contents
\newenvironment{file}[1][File]{ % Set the default filename to "File"
	\medskip
	\newcommand{\mdfilename}{#1}
	\begin{mdframed}[style=file]
}{
	\end{mdframed}
	\medskip
}

%----------------------------------------------------------------------------------------
%	NUMBERED QUESTIONS ENVIRONMENT
%----------------------------------------------------------------------------------------

% Usage:
% \begin{question}[optional title]
%	Question contents
% \end{question}

\mdfdefinestyle{question}{
	innertopmargin=1.2\baselineskip,
	innerbottommargin=0.8\baselineskip,
	roundcorner=5pt,
	nobreak,
	singleextra={%
		\draw(P-|O)node[xshift=1em,anchor=west,fill=white,draw,rounded corners=5pt]{%
		Question \theQuestion\questionTitle};
	},
}

\newcounter{Question} % Stores the current question number that gets iterated with each new question

% Define a custom environment for numbered questions
\newenvironment{question}[1][\unskip]{
	\bigskip
	\stepcounter{Question}
	\newcommand{\questionTitle}{~#1}
	\begin{mdframed}[style=question]
}{
	\end{mdframed}
	\medskip
}

%----------------------------------------------------------------------------------------
%	WARNING TEXT ENVIRONMENT
%----------------------------------------------------------------------------------------

% Usage:
% \begin{warn}[optional title, defaults to "Warning:"]
%	Contents
% \end{warn}

\mdfdefinestyle{warning}{
	topline=false, bottomline=false,
	leftline=false, rightline=false,
	nobreak,
	singleextra={%
		\draw(P-|O)++(-0.5em,0)node(tmp1){};
		\draw(P-|O)++(0.5em,0)node(tmp2){};
		\fill[black,rotate around={45:(P-|O)}](tmp1)rectangle(tmp2);
		\node at(P-|O){\color{white}\scriptsize\bf !};
		\draw[very thick](P-|O)++(0,-1em)--(O);%--(O-|P);
	}
}

% Define a custom environment for warning text
\newenvironment{warn}[1][Warning:]{ % Set the default warning to "Warning:"
	\medskip
	\begin{mdframed}[style=warning]
		\noindent{\textbf{#1}}
}{
	\end{mdframed}
}

%----------------------------------------------------------------------------------------
%	INFORMATION ENVIRONMENT
%----------------------------------------------------------------------------------------

% Usage:
% \begin{info}[optional title, defaults to "Info:"]
% 	contents
% 	\end{info}

\mdfdefinestyle{info}{%
	topline=false, bottomline=false,
	leftline=false, rightline=false,
	nobreak,
	singleextra={%
		\fill[black](P-|O)circle[radius=0.4em];
		\node at(P-|O){\color{white}\scriptsize\bf i};
		\draw[very thick](P-|O)++(0,-0.8em)--(O);%--(O-|P);
	}
}

% Define a custom environment for information
\newenvironment{info}[1][Info:]{ % Set the default title to "Info:"
	\medskip
	\begin{mdframed}[style=info]
		\noindent{\textbf{#1}}
}{
	\end{mdframed}
}

\newcommand{\D}[1]{\begin{warn}[Def:] #1 \end{warn}}
\newcommand{\pred}[1]{\textbf{Предложение:} #1\\}
\newcommand{\T}[2][]{\begin{warn}[Theorem: #1] \\#2 \end{warn}}
\newcommand{\m}[1]{\mathfrak{#1}}
\newcommand{\inc}[1]{\includepdf[pages=-]{#1}}

\newcommand\invisiblesection[1]{%
  \refstepcounter{section}%
  \addcontentsline{toc}{section}{\protect\numberline{\thesection}#1}%
  \sectionmark{#1}}
 % Include the file specifying the document structure and custom commands

%----------------------------------------------------------------------------------------
%	ASSIGNMENT INFORMATION
%----------------------------------------------------------------------------------------

\title{Generative AI} % Title of the assignment

%----------------------------------------------------------------------------------------

\DeclareMathOperator*\uplim{\overline{lim}}


\begin{document}

\maketitle % Print the title
\section{Математика и анализ данных}

\subsection{Числовые ряды. Абсолютная и условная сходимость. Признаки сходимости числовых рядов.}

\subsubsection{Числовые ряды}

$\sum\limits_{k=1}^{\infty} a_{k} = a_{1} + a_{2} + a_{3} + \cdots$ -- числовой ряд

Сходимость ряда означает существование конечной суммы, т.е. $\sum\limits_{k=1}^{\infty} a_{k} = S$ где $S$ -- конечное число, иначе ряд считается расходящимся.

\subsubsection{Абсолютная и условная сходимость}

Ряд $\sum\limits_{k=1}^{\infty} a_{k}$ называется {\bf абсолютно} сходящимся, если сходится ряд из модулей $\sum\limits_{k=1}^{\infty} |a_{k}|$, иначе ряд называется {\bf условно} сходящимся

\subsubsection{Признаки сходимости числовых рядов} 

{\bf Знакоположительные ряды} (ряды с положительными членами):

Критерий сходимости знакоположительных рядов-- знакоположительный ряд $\sum\limits_{k=1}^{\infty} a_{k}$ сходится тогда и только тогда, когда последовательность его частичных сумм $S(n) = \sum\limits_{k=1}^{k=n}a_{k}$ ограничена сверху

{\bf Док-во:}

=>: ряд сходится, значит последовательность частичных сумм $\S(n) =\sum\limits_{k=1}^{n} a_{k}$ имеет предел равный $\sum\limits_{k=1}^{\infty} a_{k} = S$

<=: Пусть дан положительный ряд и последовательность частичных сумм ограничена сверху, заметим что последовательность частичных сумм неубывающая:
$$S_{n + 1} - S_{n} = a_{n + 1} \ge 0$$. Используя свойство из теоремы о монотонной последовательности получаем, что т.к. последовательность частичных сумм монотонно не убывает и ограничена сверху, значит она сходится и потому ряд сходится по определению.

{\bf Признак сравнения с мажорантой}

Пусть даны два положительных ряда $\sum\limits_{k=1}^{\infty} a_{k}$ и $\sum\limits_{k=1}^{\infty} b_{k}$. Если начиная с некоторого номера $n > N$ выполняется неравенство $0 \le a_n \le b_n$, то:

\begin{itemize}
	\item из сходимости рядя $\sum\limits_{k=1}^{\infty} b_{k}$ следует сходимость ряда $\sum\limits_{k=1}^{\infty} a_{k}$
	\item из расходимости ряда $\sum\limits_{k=1}^{\infty} a_{k}$ следует расходимость $\sum\limits_{k=1}^{\infty} b_{k}$
\end{itemize}

{\bf Док-во:}

Из неравенств на члены следует неравенство на частичные суммы $0 \le S_n \le \sigma_n$, дальше очев.


{\bf Признак Раабе}

Если для ряда $\sum\limits_{k=1}^{\infty} a_{k}$ существует предел $$R = \lim\limits_{n \rightarrow \infty} n (\frac{a_n}{a_{n+1}} - 1)$$, то при $R > 1$ ряд сходится, а при $R < 1$ -- расходится. Если $R = 1$, то жанный признак не говорит ничего.

{\bf Признак Гаусса}

Пусть для знакоположительного ряда $\sum\limits_{n=1}^{\infty} a_{n}$ отношение $\frac{a_n}{a_{n + 1}}$ может быть представлено в виде $$\frac{a_n}{a_{n + 1}} = \lambda + \frac{\mu}{n} + \frac{\theta_n}{n^2}$$, где $\lambda, \mu$ -- постоянные, а последовательность $\theta_n$ ограничена. Тогда 
\begin{itemize}
	\item ряд расходится если либо $\lambda > 1$, либо $\lambda = 1, \mu > 1$
	\item ряд расходится, если либо $\lambda < 1$, либо $\lambda = 1, \mu \le 1$
\end{itemize}


{\bf Знакопеременные ряды}

\D{Знакопеременными называются ряды, члены которых могут (стоять) быть как положительными, так и отрицательными.}


{\bf Признак Даламбера}

Слабее признака Коши, но зато проще

Если существует $\lim\limits_{n \rightarrow \infty}|\frac{a_{n + 1}}{a_n}| = r$, то 

\begin{itemize}
	\item если $r < 1$, то ряд абсолютно сходится
	\item если $r > 1$, то ряд расходится
	\item если $r = 1$, то данный признак ничего не говорит (сука)
\end{itemize}

{\bf Док-во:}

1. Пусть начиная с некоторого номера N верно неравенство $|\frac{a_{n+1}}{a_n}| \le q, 0 < q < 1$. Тогда перемножив члены начиная с N будем иметь что $\frac{a_{N+n}}{a_N} \le q^n$ откуда $|a_{N+n}| \le |a_{N}q^n|$, значит ряд $|a_{N+1}| + |a_{N+2}| + ...$ меньше бесконечной суммы убывающей геометрической прогрессии, поэтому он сходится

2. $|\frac{a_{n + 1}}{a_n}| \ge 1$ (с некоторого N), тогда можно записать $|a_{n+1}| \ge |a_n|$ значит модуль членов $a$ не стремится к 0 на бесконечности, значит последовательность не стремится к 0 а значит ряд не сходится.

3. Если просто меньше 1 до там хуйня какая-то мне впадлу
\\

{\bf Радикальный признак Коши} (ебаная оппозиция)

Если существует $\lim\lim\limits_{n \rightarrow \infty} \sqrt[n]{|a_n|} = r$, то

\begin{itemize}
	\item если $r < 1$ то ряд сходится абсолютно
	\item если $r > 1$ то ряд расходится
	\item если $r = 1$ то хз (опять??)
\end{itemize}

{\bf Док-во:} \href{https://ru.wikipedia.org/wiki/%D0%A0%D0%B0%D0%B4%D0%B8%D0%BA%D0%B0%D0%BB%D1%8C%D0%BD%D1%8B%D0%B9_%D0%BF%D1%80%D0%B8%D0%B7%D0%BD%D0%B0%D0%BA_%D0%9A%D0%BE%D1%88%D0%B8}{тут}
\\

{\bf Признак Лейбница}

Пусть для знакочередующегося ряда $$S = \sum\limits_{n=1}^{\infty}(-1)^{n-1}a_n, a_n \ge 0$$
выполняются следующие условия

\begin{itemize}
	\item С некоторого $N$ последовательность $a$ монотонно убывает, т.е. $a_{n+1} \le a_n$
	\item $\lim\limits_{n \rightarrow \infty}a_n = 0$
\end{itemize}

Тогда такой ряд сходится

{\bf Док-во:} \href{https://ru.wikipedia.org/wiki/%D0%A2%D0%B5%D0%BE%D1%80%D0%B5%D0%BC%D0%B0_%D0%9B%D0%B5%D0%B9%D0%B1%D0%BD%D0%B8%D1%86%D0%B0_%D0%BE_%D1%81%D1%85%D0%BE%D0%B4%D0%B8%D0%BC%D0%BE%D1%81%D1%82%D0%B8_%D0%B7%D0%BD%D0%B0%D0%BA%D0%BE%D1%87%D0%B5%D1%80%D0%B5%D0%B4%D1%83%D1%8E%D1%89%D0%B8%D1%85%D1%81%D1%8F_%D1%80%D1%8F%D0%B4%D0%BE%D0%B2}{здесь}\\
	
{\bf Признак Абеля}

\T {Числовой ряд $\sum\limits_{n=1}^{\infty}a_nb_n$ сходится, если выполнены следующие условия

\begin{itemize}
	\item Последовательность \{$a_n$\} монотонна и ограничена
	\item Ряд $\sum\limits_{n=1}^{\infty}b_n$ сходится
\end{itemize}
}
{\bf Proof:} \href{https://ib.mazurok.com/2015/06/16/%D0%BF%D1%80%D0%B8%D0%B7%D0%BD%D0%B0%D0%BA%D0%B8-%D0%B0%D0%B1%D0%B5%D0%BB%D1%8F-%D0%B8-%D0%B4%D0%B8%D1%80%D0%B8%D1%85%D0%BB%D0%B5/}{вот}\\
	
{\bf Признак Дирихле}

\T{Пусть выполнены условия:
\begin{itemize}
	\item последовательность частичных сумм $B_n = \sum\limits_{k=1}^{n}$ ограничена
	\item последовательность $a_n$, начиная с некоторого номера, монотонно убывает $a_n \ge a_{n+1}$
	\item $\lim\limits_{n\rightarrow\infty}a_n = 0$
\end{itemize}
Тогда ряд $\sum\limits_{n=1}^{\infty}a_nb_b$ сходится
}
	
{\bf Proof:} \href{https://ib.mazurok.com/2015/06/16/%D0%BF%D1%80%D0%B8%D0%B7%D0%BD%D0%B0%D0%BA%D0%B8-%D0%B0%D0%B1%D0%B5%D0%BB%D1%8F-%D0%B8-%D0%B4%D0%B8%D1%80%D0%B8%D1%85%D0%BB%D0%B5/}{вот}\\

\subsection{Кратные, поверхностные и криволинейные интегралы. Формулы Грина, Стокса и Остроградского}


\subsubsection{Интеграль4ики}
\D{Пусть дана $f(x)$ -- функция действительной переменной. {\bf Неопределенным интегралом} функции $f(x)$, или ее первообразной, называется такая функция $F(x)$, производная которой равна $f(x)$, т.е. $F^{'}(x) = f(x)$. Обозначается $F(x) = \int f(x)dx$ }


\D{Кратным интегралом называют множество интегралов, взятых от $d > 1$, например  $$\underbrace{\int...\int f(x_1,...,x_d)dx_{1}...dx_{d}}_{d}$$}

Замечание -- кратный интеграл -- определенный интеграл, при его вычислении всегда получается число

\D{Криволинейный интеграл -- интеграл вычисляемый вдоль какой-либо прямой.

Пусть $l$ -- пгладкая, без особых точек и пересечений кривая (может быть замкнутой), заданая параметрически $l: r(t)$, где $r$ -- радиус вектор, конец которого описывает кривую, а параметр $t$ направлен от начального значения $a$ к конечному значению $b$. Для интеграла второго рода направление, в котором движется параметр, определяет направление кривой $l$.

Также есть скалярная или векторная функция , которая рассматривается вдоль кривой $l: f(r)$

Еще есть разбиение отрезка параметризации, и разбиение кривой. Они соответствуют друг-другу (параметризация от параметра по факту сопостовляет точке из отрезка параметризации $[a, b]$ точку на прямой, и по разбиению параметризации разбивается кривая по соответствующим точкам, подробнее можно почитать на вики ссылку вставить не получилось:( )

Интегральная сумма для интеграла {\bf первого рода} -- сумма вида $\sum\limits_{k=1}^{n} f(r(\xi_i))\cdot|l_k|$ где $|l_k|$ -- длина соответствующего отрезка, $\xi_i$ -- точка на соответствующем отрезке

Интегральная сумма для интеграла {\bf первого рода} -- сумма вида $\sum\limits_{k=1}^{n} f(r(\xi_i))\cdot(r(t_k) = r(t_{k - 1})))$

Собственно, криволинейный интеграл это интегральная сумма с $n$ устремленным в бесконечность
}

Похоже на обычный определенный интеграл, только тут мы вместо оси выравниваемся на кривую какую-то, и по факту считаем площидь криволинейного цилиндра между кривой в пространстве и ее проекцией (вроде бы, но это не точно)


\D{Пусть $\Phi$ -- гладкая, ограниченная полная поверхность. Пусть далее на $\Phi$ задана функция $f(M) = f(x, y, z)$. Рассмотрим разбиение $T$ этой поверхности на часть $\Phi_i (i=1, ..., n)$ кусочно-гладкими кривыми и на каждой такой части выберем произвольную точку $M_i(x_i, y_i, z_i)$. Вычислив значение функции в этой точке $f(M_i) = f(x_i, y_i, z_i)$ и, приняв за $\sigma_i$ площадь поверхность $\Phi_i$, рассмотрим сумму $$I\{\Phi_i, M_i\} = \sum_i f(M_i)\sigma_i$$. Тогда число I называется пределом сумм $i\{\Phi_i, M_i\}$ если $$\forall \epsilon > 0 \exists \delta > 0 \forall T: d(T) < \delta \forall \{M_i\} |I\{\Phi_i, M_i\} - I| < \epsilon$$
Предел $I$ сумм $I\{\Phi_i, M_i\}$ при $d(T) \rightarrow 0$
 называется {\bf поверхностным интегралом первого рода} от функции $f(M)$ по поверхности $\Phi$ и обозначается $$I = \iint\limits_{\Phi} f(M)d\sigma$$} 

По сути -- берем поверхность в пространстве, а дальше как в криволинейном -- вместо отрезков оже куски пространства и т.д. получается магия какая-то.

\subsubsection{Формула Грина}

\D{Пусть $C$ -- положительно ориентированная кусочно-гладкая замкнутая кривая на плоскости, а $D$ -- область, ограниченная кривой $C$. Если фунеции $P = P(x, y)$, $Q = Q(x, y)$ определены в области $D$ и имеют неприрывные частные производные $\frac{\partial P}{\partial y}, \frac{\partial Q}{\partial x}$, то $$\oint Pdx + Qdy = \iint\limits_{D}(\frac{\partial Q}{\partial x} - \frac{\partial P}{\partial y})dxdy$$}
{\bf Док-во и еще:} \href{https://ru.wikipedia.org/wiki/%D0%A2%D0%B5%D0%BE%D1%80%D0%B5%D0%BC%D0%B0_%D0%93%D1%80%D0%B8%D0%BD%D0%B0}{here}
	
\subsubsection{Формула Стокса}

\D{Пусть на ориентируемом многообразии $M$ размерности $n$ заданы положительно ориентированное ограниченное $p-$мерное подмногообразие $\sigma (1 \le p \le n)$ и дифференциальная форма $\omega$ степени $p - 1$ класса $C^1$. Тогда если граница подмногообразия $\partial \sigma$ положительно ориентированаб то $$\int\limits_{\sigma}d\omega = \int\limits_{\partial \sigma \omega}$$}

Грубо говоря взяли поверхность, и с помощью дифференциалов перешли к интегралу по границе поверхности, как-то так, но надо глубже разбираться потому что очень много определений которые надо помнить

\subsubsection{Формула Остроградского (Гаусс сосать)}

\D{Пусть теперь $\partial V$ -- кусочно-гладкая гипперповерхность $(p = n - 1)$, ограничивающая некоторую область $V$ в $n-$мерном пространстве. Тогда интеграл дивергенции (это оператор который отображает векторное поле на скалярное -- $div F = \lim\limits_{V \rightarrow 0} \frac{\Phi_F}{V}$, где $\Phi_F$ -- поток векторного поля $F$ через сферическую поверхность площадью $S$ ограничивающую объем $V$, хуита какая-то хочу объяснение на пальцах) поля по области равен потоку поля через границу области $\partial V$: $$\int\limits_{V} div F dV = \int\limits_{\partial V} F d \Sigma$$.
	
В трехмерном пространстве $(n = 3)$ с координатами $\{x, y, z\}$ эквивалентнно $$\int\limits_{\partial V} F d \Sigma = \int\limits_{V}(\frac{\partial P}{\partial x} + \frac{\partial Q}{\partial y} + \frac{\partial R}{\partial z}) dV$$, или $$\iiint\limits_{\partial V} Pdydz + Qdzdx + Rdxdy = \iint\limits_{V} (\frac{\partial P}{\partial x} + \frac{\partial Q}{\partial y} + \frac{\partial R}{\partial z})dxdydz$$ }


Понятно что тут взяли и применили стокса на какой-то случай, но чет пиздец ребята)))


\subsection{Функциональные ряды, свойства равномерно сходящихся функциональных рядов. Степенные ряды. Ряд Тейлора.}

\subsubsection{Функциональные ряды}

\D {Функциональный ряд -- ряд, каждым членом которого является функция $u_k(x)$

Обозначается $\sum\limits_{k=1}^{\infty} u_k(x)$}

Функциональная последовательность $u_k(x)$ сходится {\bf поточечно} к функции $u(x)$, если $\forall x \in E \exists \lim\limits_{k \rightarrow \infty} u_k(x) = u(x)$

{\bf Равномерная сходимость} -- существует функция $u(x): E \rightarrow \mathbb{C}$ такая, что 

$sup |u_k(x) - u(x)| \xrightarrow {k \rightarrow \infty} 0, x \in E$

Функциональный ряд называется сходящимся {\bf поточечно}, если последовательность $S_n(x) = \sum\limits_{k=1}^{n} u_k(n)$ сходится поточечно. Аналогично для равномерной сходимости.

{\bf Необходимое условие равноменой сходимости ряда}

$u_k(x) \rightrightarrows 0$ при $k \rightarrow \infty$

Или, что эквивалентно $\forall \epsilon > 0 \exists n_0(\epsilon) \in \mathbb{N} : \forall x \in X, \forall n > n_0 |u_n(x)| < \epsilon$, где $X$ -- область сходимости

{\bf Свойства}

\begin{enumerate}
	\item {\bf Теоремы о непрерывности}
	
	Последовательность непрерывных в точке функций сходится к функции, непрерывной в этой точке.
	
	Последовательность $u_k(x) \rightrightarrows u(x)$
	
	$\forall k:$ функция $u_k(x)$ непрерывна в точке $x_0$
	
	Тогда и $u(x)$ непрерывна в $x_0$
	
	Ряд непрерывных в точке функций сходится к функции, непрерывной в этой точке.
	
	Ряд $\sum\limits_{k=0}^{\infty}u_k(x) \rightrightarrows S(x)$
	
	$\forall k$: функция непрерывна в точкке $x_0$
	
	Тогда $S(x)$ непрерывна в  $x_0$
	
	\item {\bf Теоремы об интегрировании}
	
	Рассматриваются действительнозначные функции на отрезке действительной оси
	
	{\it Теорема о переходе к пределу под знаком интеграла}
	
	$\forall k:$ функция $u_k(x)$ непрерывна на отрезке $[a, b]$
	
	$u_k(x) \rightrightarrows u(x)$ на $[a, b]$
	
	Тогда числовая последовательность $\{\int\limits_{a}^{b} u_k(x) dx\}$ сходится к конечному пределу $\int\limits_a^b u(x) dx$
	
	{\it Теорема о почленном интегрировании}
	
	$\forall k:$ функция $u_k(x)$ непрерывна на отрезке $[a, b]$
	
	$\sum\limits_{k=1}^{\infty}u_k(x) \rightrightarrows S(x)$ на $[a, b]$
	
	Тогда числовой ряд $\sum\limits_{k=1}^{\infty}\int\limits_{a}^{b} u_k(x) dx$ сходится и равен $\int\limits_a^b S(x) dx$
	
	\item {\bf Теоремы о дифференцировании}
	
	Рассматриваются действительнозначные функции на отрезке действительной оси
	
	{\it Теорема о дифференцировании под пределом}
	
	$\forall k:$ функция $u_k(x)$ дифференцируема (имеет непрерывную производную) на отрезке $[a, b]$
	
	$\exists c \in [a, b]: u_k(c)$ сходится к конечному пределу
	
	$u_k^{\prime}(x)  \rightrightarrows \omega(x)$ на отрезке $[a, b]$
	
	Тогда $\exists u(x): u_k(x) \rightrightarrows u(x),\ u(x)$ -- дифференцируема на $[a, b],\ u^{\prime}(x) = \omega(x)$ на $[a, b]$
	
	{\it Теорема о почленном дифференцировании}

	$\forall k:$ функция $u_k(x)$ -- дифференцируема на отрезке $[a, b]$
	
	$\exists c \in [a, b]: \sum\limits_{k=1}^{\infty} u_k(c)$ сходится
	
	$\sum\limits_{k=1}^{\infty}u_k^{\prime}(x)$ равномерно сходится на отрезке $[a, b]$
	
	Тогда $\exists S(x): \sum\limits_{k=1}^{\infty}u_k(x) \rightrightarrows S(x),\ S(x)$ -- дифференцируем на $[a, b], S^{\prime}(x) = \sum\limits_{k=1}^{\infty}u_k^{\prime}(x)$ на $[a, b]$
	
\end{enumerate}

\subsubsection{Степенные ряды}

\D {{\bf Степенной ряд с одной переменной} -- это формальное алгебраическое вырадение вида $$F(x) = \sum\limits_{n=0}^{\infty}a_nX^n$$ в котором коэффициенты $a_n$ берутся из некоторого кольца $R$, обычно вещественные или комплексные числа}

Для степенных рядов есть несколько теорем об их сходимости

\begin{itemize}
	\item Певая теорема Абеля
	
	Пусть ряд $\sum a_n x^n$ сходится в точке $x_0$. Тогда этот ряд сходится абсолютно в круге $|x| < |x_0|$ и равномерно по $x$ на любом компактном подмножестве этого круга.
	
	Отсюда можно сделать вывод что если ряд расходится при $x = x_0$, то он расходится при всех $|x| > |x_0|$
	
	Появляется понятие радиуса сходимости $R$, при котором при $|x| < R$ ряд сходится абсолютно, про $|x| > R$ расходится
	
	\item Формула Коши-Адамара (Коши-Амидамару)
	
	 Значение радиуса сходимости степенного ряда может быть вычислено по формуле $\frac{1}{R} = \uplim\limits_{n \rightarrow +\infty}|a_n|^{1/n}$
	  
	 \item Признак Даламбера
	 
	 Если при $n > N$ и $\alpha > 1$ выполнено неравенство $|\frac{a_n}{a_{n+1}}| \ge R(1 + \frac{\alpha}{n})$ тогда степенной ряд $\sum a_n x^n$ сходится во всех точках окружности $|x| = R$ абсолютно и равномерно по $x$
	 
	 \item Признак Дирихле
	 
	 Если все коэффициенты степенного ряда $\sum a_n x^n$ положительны и последовательность $a_n$ монотонно сходится к 0б тогда этот ряд сходится во всех точках окружности $|x| = 1$, кроме, может быть, точки $x = 1$
\end{itemize}


\subsubsection{Ряд Тейлора}

\D {Ряд Тейлора -- разложение функции в бесконечную сумму степенных функций

Многочленом Тейлора функции $f(x)$ вещественной переменной $x$, дифференцируемой $k$ раз в точке $a$ называется конечная сумма 
$$f(x) = \sum\limits_{n=0}{k}\frac{f^{(n)}(a)}{n!} (x - a)^n = f(a) + f^{\prime}(a)(x - a) + \frac{f^{(2)}(a)}{2!}(x - a)^2 + ... + \frac{f^{(k)}(a)}{k!}(x - k)^k$$

Рядом Тейлора в точке $a$ функции $f(x)$ , бесконечно диффиренцируемой в окрестности точки $a$, называется формальный степенной ряд
$$f(x) = \sum\limits_{n=0}^{+\infty}\frac{f^{(n)}(a)}{n!}(x - a)^n$$

Другими словами, рядом Тейлора функции $f(x)$ в точке $a$ называется ряд разложения функции по положительным степеням двучлена $(x - a)$

}

Еще есть формула Тейлора, это просто частичная сумма ряда вроде как.

В случае $a = 0$ это все безобразие -- {\bf ряд Маклорена}

\end{document}
