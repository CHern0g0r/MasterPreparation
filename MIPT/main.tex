%----------------------------------------------------------------------------------------
%	PACKAGES AND OTHER DOCUMENT CONFIGURATIONS
%----------------------------------------------------------------------------------------

\documentclass{article}
% \documentclass[14pt]{extarticle}
\usepackage{pdfpages}

%%%%%%%%%%%%%%%%%%%%%%%%%%%%%%%%%%%%%%%%%
% Lachaise Assignment
% Structure Specification File
% Version 1.0 (26/6/2018)
%
% This template originates from:
% http://www.LaTeXTemplates.com
%
% Authors:
% Marion Lachaise & François Févotte
% Vel (vel@LaTeXTemplates.com)
%
% License:
% CC BY-NC-SA 3.0 (http://creativecommons.org/licenses/by-nc-sa/3.0/)
% 
%%%%%%%%%%%%%%%%%%%%%%%%%%%%%%%%%%%%%%%%%

%----------------------------------------------------------------------------------------
%	PACKAGES AND OTHER DOCUMENT CONFIGURATIONS
%----------------------------------------------------------------------------------------

\usepackage{amsmath,amsfonts,stmaryrd,amssymb,amsthm} % Math packages

\usepackage{mathtext}

\usepackage{enumerate} % Custom item numbers for enumerations

\usepackage[ruled]{algorithm2e} % Algorithms

\usepackage[framemethod=tikz]{mdframed} % Allows defining custom boxed/framed environments

\usepackage{listings} % File listings, with syntax highlighting
\lstset{
	basicstyle=\ttfamily, % Typeset listings in monospace font
}

\usepackage[unicode]{hyperref}
%----------------------------------------------------------------------------------------
%	DOCUMENT MARGINS
%----------------------------------------------------------------------------------------

\usepackage{geometry} % Required for adjusting page dimensions and margins

\geometry{
	paper=a4paper, % Paper size, change to letterpaper for US letter size
	top=2.5cm, % Top margin
	bottom=3cm, % Bottom margin
	left=2.5cm, % Left margin
	right=2.5cm, % Right margin
	headheight=14pt, % Header height
	footskip=1.5cm, % Space from the bottom margin to the baseline of the footer
	headsep=1.2cm, % Space from the top margin to the baseline of the header
	%showframe, % Uncomment to show how the type block is set on the page
}

%----------------------------------------------------------------------------------------
%	FONTS
%----------------------------------------------------------------------------------------

\usepackage[utf8]{inputenc} % Required for inputting international characters
\usepackage[T1, T2A]{fontenc} % Output font encoding for international characters

\usepackage[english,russian]{babel}

\usepackage{XCharter} % Use the XCharter fonts

%----------------------------------------------------------------------------------------
%	COMMAND LINE ENVIRONMENT
%----------------------------------------------------------------------------------------

% Usage:
% \begin{commandline}
%	\begin{verbatim}
%		$ ls
%		
%		Applications	Desktop	...
%	\end{verbatim}
% \end{commandline}

\mdfdefinestyle{commandline}{
	leftmargin=10pt,
	rightmargin=10pt,
	innerleftmargin=15pt,
	middlelinecolor=black!50!white,
	middlelinewidth=2pt,
	frametitlerule=false,
	backgroundcolor=black!5!white,
	frametitle={Command Line},
	frametitlefont={\normalfont\sffamily\color{white}\hspace{-1em}},
	frametitlebackgroundcolor=black!50!white,
	nobreak,
}

% Define a custom environment for command-line snapshots
\newenvironment{commandline}{
	\medskip
	\begin{mdframed}[style=commandline]
}{
	\end{mdframed}
	\medskip
}

%----------------------------------------------------------------------------------------
%	FILE CONTENTS ENVIRONMENT
%----------------------------------------------------------------------------------------

% Usage:
% \begin{file}[optional filename, defaults to "File"]
%	File contents, for example, with a listings environment
% \end{file}

\mdfdefinestyle{file}{
	innertopmargin=1.6\baselineskip,
	innerbottommargin=0.8\baselineskip,
	topline=false, bottomline=false,
	leftline=false, rightline=false,
	leftmargin=2cm,
	rightmargin=2cm,
	singleextra={%
		\draw[fill=black!10!white](P)++(0,-1.2em)rectangle(P-|O);
		\node[anchor=north west]
		at(P-|O){\ttfamily\mdfilename};
		%
		\def\l{3em}
		\draw(O-|P)++(-\l,0)--++(\l,\l)--(P)--(P-|O)--(O)--cycle;
		\draw(O-|P)++(-\l,0)--++(0,\l)--++(\l,0);
	},
	nobreak,
}

% Define a custom environment for file contents
\newenvironment{file}[1][File]{ % Set the default filename to "File"
	\medskip
	\newcommand{\mdfilename}{#1}
	\begin{mdframed}[style=file]
}{
	\end{mdframed}
	\medskip
}

%----------------------------------------------------------------------------------------
%	NUMBERED QUESTIONS ENVIRONMENT
%----------------------------------------------------------------------------------------

% Usage:
% \begin{question}[optional title]
%	Question contents
% \end{question}

\mdfdefinestyle{question}{
	innertopmargin=1.2\baselineskip,
	innerbottommargin=0.8\baselineskip,
	roundcorner=5pt,
	nobreak,
	singleextra={%
		\draw(P-|O)node[xshift=1em,anchor=west,fill=white,draw,rounded corners=5pt]{%
		Question \theQuestion\questionTitle};
	},
}

\newcounter{Question} % Stores the current question number that gets iterated with each new question

% Define a custom environment for numbered questions
\newenvironment{question}[1][\unskip]{
	\bigskip
	\stepcounter{Question}
	\newcommand{\questionTitle}{~#1}
	\begin{mdframed}[style=question]
}{
	\end{mdframed}
	\medskip
}

%----------------------------------------------------------------------------------------
%	WARNING TEXT ENVIRONMENT
%----------------------------------------------------------------------------------------

% Usage:
% \begin{warn}[optional title, defaults to "Warning:"]
%	Contents
% \end{warn}

\mdfdefinestyle{warning}{
	topline=false, bottomline=false,
	leftline=false, rightline=false,
	nobreak,
	singleextra={%
		\draw(P-|O)++(-0.5em,0)node(tmp1){};
		\draw(P-|O)++(0.5em,0)node(tmp2){};
		\fill[black,rotate around={45:(P-|O)}](tmp1)rectangle(tmp2);
		\node at(P-|O){\color{white}\scriptsize\bf !};
		\draw[very thick](P-|O)++(0,-1em)--(O);%--(O-|P);
	}
}

% Define a custom environment for warning text
\newenvironment{warn}[1][Warning:]{ % Set the default warning to "Warning:"
	\medskip
	\begin{mdframed}[style=warning]
		\noindent{\textbf{#1}}
}{
	\end{mdframed}
}

%----------------------------------------------------------------------------------------
%	INFORMATION ENVIRONMENT
%----------------------------------------------------------------------------------------

% Usage:
% \begin{info}[optional title, defaults to "Info:"]
% 	contents
% 	\end{info}

\mdfdefinestyle{info}{%
	topline=false, bottomline=false,
	leftline=false, rightline=false,
	nobreak,
	singleextra={%
		\fill[black](P-|O)circle[radius=0.4em];
		\node at(P-|O){\color{white}\scriptsize\bf i};
		\draw[very thick](P-|O)++(0,-0.8em)--(O);%--(O-|P);
	}
}

% Define a custom environment for information
\newenvironment{info}[1][Info:]{ % Set the default title to "Info:"
	\medskip
	\begin{mdframed}[style=info]
		\noindent{\textbf{#1}}
}{
	\end{mdframed}
}

\newcommand{\D}[1]{\begin{warn}[Def:] #1 \end{warn}}
\newcommand{\pred}[1]{\textbf{Предложение:} #1\\}
\newcommand{\T}[2][]{\begin{warn}[Theorem: #1] \\#2 \end{warn}}
\newcommand{\m}[1]{\mathfrak{#1}}
\newcommand{\inc}[1]{\includepdf[pages=-]{#1}}

\newcommand\invisiblesection[1]{%
  \refstepcounter{section}%
  \addcontentsline{toc}{section}{\protect\numberline{\thesection}#1}%
  \sectionmark{#1}}
 % Include the file specifying the document structure and custom commands

%----------------------------------------------------------------------------------------
%	ASSIGNMENT INFORMATION
%----------------------------------------------------------------------------------------

\title{Gamedev} % Title of the assignment

%----------------------------------------------------------------------------------------

\begin{document}

\maketitle % Print the title

\section{Базовая часть}

\subsection{Алгебра и геометрия, линейная алгебра}

\subsubsection{Группы, кольца, поля. Определения и примеры. Циклические группы. Теорема о
гомоморфизме.}

\D{
    Группа = $\langle M, \cdot\rangle$ - непустое множество с операцией

    Верны следующие свойства
    \begin{itemize}
        \item Ассоциативность: $(a \cdot b) \cdot c = a \cdot (b\cdot c)$
        \item Существование нейтрального: $\exists 1 \forall a : a \cdot 1 = 1 \cdot a = a$
        \item Существование обратного: $\forall a \exists a^{-1}: a \cdot a^{-1} = 1$
    \end{itemize}
}

Примеры $\mathbb{Z}, \mathbb{Z}/n\mathbb{Z}, \langle\mathbb{R}, \cdot\rangle$

\D{
    Абелева группа = группа + операция коммутативна.
}


\D{
    Циклическая группа = группа, порожденная степенями одного элемента.

    $G' = \langle g \rangle, g \in G$, $G$ - группа.
}

\D{
    Кольцо = $\langle M, +, \cdot \rangle$
    \begin{itemize}
        \item Абелева группа относительно $+$
        \item $\forall a, b, c \in M: a(b + c) = ab + ac \wedge (a+b)c = ac + bc$
    \end{itemize}

    Кольцо называется коммутативным, если умножение в нем коммутативно.

    Ассоциативным, если ассоциативно.

    Кольцо с единицей, если есть единица.
}

Следствия:
\begin{itemize}
    \item $a 0 = 0 a = 0$
    \item $a(-b) = (-a) b = -ab$
    \item $a(b-c) = ab - ac$
\end{itemize}


\D {
    Поле = ассоциативное коммутативное кольцо с единицей, в котором каждый
    элемент обратим.

    Абелева группа по умножению.
}

\D{
    Гомоморфизм групп = отображение $f: G \to H$, т.ч.
    $\forall a, b \in G: f(ab) = f(a) f(b)$

    \begin{itemize}
        \item $Im f = \{f(a): a \in G\} < H$
        \item $Ker f = \{a\in G : f(a) = e\} \lhd  G$ (нормальная)
    \end{itemize}
}

\T[о гомоморфизме групп]{
    $f: G \to H$ - гомоморфизм. Тогда
    $Im f \simeq G / Ker f$
}



\subsubsection{Подстановки. Определение подстановки, четность подстановок. Произведение
подстановок, разложение подстановок в произведение транспозиций и независимых
циклов.}

\D{
    Подстановка = биективное преобразование множества.

    $\sigma = \left({i_1 \atop j_1} ... {i_n \atop j_n}\right)$

    $\sigma(i_i) = j_i$
}

Пример на умножение подстановок:

$\left({1 \atop 3} {2 \atop 4} {3 \atop 1} {4\atop 2}\right)
\left({1 \atop 4} {2 \atop 3} {3 \atop 2} {4\atop 1}\right) =
\left({4\atop 2} {3 \atop 1} {2 \atop 4} {1 \atop 3}\right)
\left({1 \atop 4} {2 \atop 3} {3 \atop 2} {4\atop 1}\right) =
\left({1 \atop 2} {2 \atop 1} {3 \atop 4} {4\atop 3}\right)$

\D{
    $S_n$ - группа подстановок (симметрическая) степени $n$ = совокупность
    всех биективных преобразований множества размера $n$.
}

\D {
    Транспозиция = подстановка, меняющая местами 2 элемента.
}

\T{
    Группа $S_n$ порождается транспозициями. $\iff$ Любая перестановка
    представляется как произведение транспозиций.

    \begin{proof}
        $(i j) \cdot \left({1\atop k_1} ... {n \atop k_n}\right)$
        поменяет местами $i$ и $j$ в нижней строке.

        Такими преобразованиями любую перестановку можно привести к
        тождественной: сначала меняем местами $k_1$ и $1$, затем ставим
        двойку на второе место и т.д.

        $\Rightarrow \exists \tau_1, ..., \tau_s$ - транспозиции, т.ч.
        $\tau_s ... \tau_2 \tau_1 \sigma = id$

        $\Rightarrow \sigma = \tau_1 ...\tau_s$, т.к. транспозиция обратна сама себе.
    \end{proof}
}

Знак перестановки $\varepsilon_\pi = (-1)^t$, где $t$ = количество
транспозиций в каком-то разложении $\pi$. При $\varepsilon_\pi = 1$ подстановка
является четной.



\subsubsection{Комплексные числа. Геометрическое изображение, алгебраическая и
тригонометрическая форма записи, извлечение корней, корни из единицы.}

\D{
    $\mathbb{C}$ - поле комплексных чисел = минимальное поле, содержащее
    $\mathbb{R}$ в качестве подполя и элемент $i$, т.ч. $i^2 = -1$.

    Моножество пар вида $(a, b) \forall a, b \in \mathbb{R}$, где сложение
    и умножение определено следующими формулами:
    \begin{itemize}
        \item $(a, b) + (c, d) = (a + c, b + d)$
        \item $(a, b) \cdot (c, d) = (ac - bd, ad + bc)$
    \end{itemize}
}

$a + bi$ - алгебраическая форма записи комплексного числа.

Комплексное число можно рассматривать как вектор или точку с координатами $(a, b)$.

Иногда вместо декартовых координат удобнее рассматривать полярные.

\textbf{Модулем} комплексного числа называется длина вектора, изображающего это число.

$|a + b i| = \sqrt{a^2 + b^2}$

\textbf{Аргументом} комплексного числа называется угол между вектором и положительным
направлением оси X.

$r, \phi$ - модуль и аргумент числа $c \Rightarrow a = r \cos \phi, \;\; b = r \sin \phi$

$\Rightarrow c = r (\cos \phi + i \sin\phi)$ - тригонометрическая форма записи.

$r_1(\cos \phi_1 + i \sin\phi_1) = r_2(\cos \phi_2 + i \sin \phi_2) \iff
(r_1 = r_2) \wedge (\phi_1 = \phi_2 + 2 \pi k, k \in \mathbb{Z})$

Умножение комплексных чисел в тригонометрической форме:

$r_1(\cos \phi_1 + i \sin\phi_1) \cdot r_2 (\cos \phi_2 + i \sin\phi_2) =
r_1 r_2 (\cos(\phi_1 + \phi_2) + i \sin (\phi_1 + \phi_2))$

Формула Муавра:

$[r(\cos\phi + i \sin\phi)]^n = r^n (\cos n\phi + i \sin n\phi)$

Извлечение корней:

$z^n = [s(\cos\psi + i \sin\psi)]^n = r (\cos \phi + i\sin\phi) \Rightarrow
s = \sqrt[n]{r} \wedge \psi = \frac{\phi + 2 \pi k}{n}$

$z = \sqrt[n]{r}(\cos\frac{\phi + 2 \pi k}{n} + i \sin\frac{\phi + 2 \pi k}{n})$

Одинаковые значения $z$ получаются, когда в качестве $k$ берутся значения,
сравнимые по модулю $n$.



\subsubsection{Системы линейных уравнений. Прямоугольные матрицы. Приведение матриц и систем
линейных уравнений к ступенчатому виду. Метод Гаусса.}

\D{
    Линейным уравнением с неизвестными $x_1, ..., x_n$ над полем $K$
    называется уравнение вида $\sum\limits_{i=1}^n a_i x_i = b$, 
    где $a_i, b \in K$
}

Матрица коэффициентов системы:
$\begin{pmatrix}
    a_{11} & a_{12} & ... & a_{1n} \\
    \vdots & \ddots &  & \vdots\\
    a_{m1} & a_{m2} & ... & a_{mn}
\end{pmatrix}$

Расширенная матрица системы:
$\begin{pmatrix}
    a_{11} & a_{12} & ... & a_{1n} & b_1 \\
    \vdots & \ddots &  & \vdots & \vdots \\
    a_{m1} & a_{m2} & ... & a_{mn} & b_m
\end{pmatrix}$

Система уравнений называется совместной, если имеет хотя бы одно решение.

Две системы уравнений называются совместными, если их множества решений совпадают.

\textbf{Метод Гаусса:}

С помощью элементарных преобразований:
\begin{itemize}
    \item прибавления к одной строке другой, умноженной на коэффициент
    \item перестановка двух строк
    \item умножение строки на ненулевой коэффициент
\end{itemize}
любую матрицу можно привести к ступенчатому виду (номера ведущих элементов
ее ненулевых строк образуют строго возрастающую последовательность, а
нулевые строки стоят в конце).

Пусть $A$ - ступенчатая матрица системы, а $A'$ - ступенчатая расширенная матрица.

Тогда имеют место следующие случаи:
\begin{itemize}
    \item $rk A < rk A' \Rightarrow$ 0 решений (система несовместна)
    \item $rk A = rk A' = n \Rightarrow$ 1 решение (определенная система)
    \item $rk A = rk A' < n \Rightarrow \infty$ решений (неопределенная система)
\end{itemize}

\T{
    \begin{itemize}
        \item Совокупность решений однородной системы является подпространством
        в $K^n$
        \item Совокупность решений произвольной совместной системы есть сумма
        одного из решений и подпространства решений однородной системы с той же матрицей.
    \end{itemize}
}



\subsubsection{Линейная зависимость и ранг. Линейная зависимость строк (столбцов). Основная лемма
о линейной зависимости, базис и ранг системы строк (столбцов). Ранг матрицы.
Критерий совместности и определенности системы линейных уравнений в терминах
рангов матриц. Фундаментальная система решений однородной системы линейных
уравнений.}

\D{
    Векторы $a_1, ..., a_n$ называются линейно зависимыми, если существует их
    нетривиальная линейная комбинация, равная нулю.
}

\textbf{Лемма} Векторы $a_1, ..., a_n$ линейно зависимы $\iff$ один из
векторов линейно выражается через остальные.

\T[Основная лемма о линейной зависимости]{
    Если векторы $b_1, ..., b_m$ линейно выражаются через векторы
    $a_1, ..., a_n$, причем $m > n$, то векторы $b_1, ..., b_m$ линейно зависимы.
    \begin{proof}
        Пусть $b_i = \sum\limits_j \mu_{ij} a_j$

        $\forall \lambda_1, ..., \lambda_m \in K: \sum\limits_{i=1}^m \lambda_i b_i =
        \sum\limits_{j=1}^n (\sum\limits_{k=1}^m \lambda_k \mu_{kj})a_j$

        Рассмотрим систему $n$ однородных линейных уравнений с $m$ неизвестными.
        \begin{equation*}
            \begin{cases}
                \mu_{11} x_1 + ... + \mu_{m1} x_m = 0 \\
                ... \\
                \mu_{1n} x_1 + ... + \mu_{mn} x_m = 0
            \end{cases}
        \end{equation*}

        Если $(\lambda_1, ..., \lambda_m)$ - произвольное решение системы,
        то $\sum \lambda_i b_i = 0$

        С другой стороны, количество уравнений меньше, чем количество неизвестных,
        значит существует нетривиальное решение данной системы.
    \end{proof}
}

\D{
    Базис векторного пространства = всякая ЛНС, порождающая пространство.
}

Векторное пространство называется конечномерным, если порождается конечным
числом векторов.

\T{
    Всякое конечномерное пространство $V$ обладает базисом.

    Из всякого конечного порождающего множества $S \subset V$ можно выбрать
    базис пространства $V$.
    \begin{proof}
        Если множество $S$ линейно зависимо, то в нем найдется вектор, линейно
        выражающийся через остальные. Такой вектор можно выкидывать, пока не получим
        ЛНС.
    \end{proof}
}

\T{
    Все базисы конечномерного векторного пространства содержат одинаковое
    число векторов. (Размерность пространства)
    \begin{proof}
        Из основной леммы о линейной зависимости.
    \end{proof}
}

\T{
    Всякое максимальное по включению линейно независимое подмножество
    $\{e_1, ..., e_k\}$ множества $S$ является базисом $\langle S \rangle$
}

\T{
    Конечномерные векторные пространства над одним полем изоморфны $\iff$
    они имеют одинаковую размерность.
}

\D{
    Ранг системы векторов = размерность ее линейной оболочки.

    Ранг матрицы = ранг системы ее строк.
}

Ранг матрицы равен числу ненулевых строк любой ступенчатой матрицы к которой 
она приводится элементарными преобразованиями строк.

\T[Кронекера-Капелли]{
    Система линейных уравнений совместна $\iff$ ранг ее матрицы равен
    рангу расширенной матрицы.
}

\T{
    Совместная система линейных уравнений определена (имеет единственное решение) $\iff$ ранг матрицы
    коэффициентов равен числу неизвестных.
}

\T{
    Размерность пространства решений однородной СЛУ с $n$ неизвестными равна
    $n - rk A$.
    \begin{proof}
        Рассмотрим СЛУ и приведем ее матрицу к ступенчатому виду. Число
        ненулевых уравнений в ступенчатой матрице равно $r = rk A$.
        Общее решение будет содержать $r$ главных неизвестных и иметь вид:
        \begin{equation}
            \begin{cases}
                x_1 = \sum\limits_{i=1}^{n-r} c_{1i} x_{r + i}\\
                \vdots \\
                x_r = \sum\limits_{i=1}^{n-r} c_{ri} x_{r + i}
            \end{cases}
        \end{equation}

        Придавая одной из свободных неизвестных значение 1, а остальным 0
        получим следующие $n-r$ решений системы:

        $u_i = (c_{1i}, c_{2_i}, ..., c_{r_i}, 0, ..., 1, ..., 0)$

        Докажем, что они составляют базис пространства решений.

        $\forall \lambda_i \in K:\; u = \sum \lambda_i u_i$ - является решением
        системы.

        Любое решение системы является линейной комбинацией $u_i$, т.к. значения
        главных неизвестных однозначно определяются по значениям свободных.

        $u_i$ являются ЛНС.
    \end{proof}
}

Всякий базис пространства решений системы однородных уравнений называется
\textbf{фундаментальной системой решений}.



\subsubsection{Определители. Определитель квадратной матрицы, его основные свойства. Критерий
равенства определителя нулю. Формула разложения определителя матрицы по строке
(столбцу).}

\D{
    Определителем квадратной матрицы $A = (a_{ij})$ порядка $n$
    называется число
    
    $\det A = \sum\limits_{(k_1, ..., k_n)} sgn(k_1, ..., k_n) a_{1k_1}... a_{n k_n}$

    Где $(k_1, ..., k_n)$ - перестановка.
}

\T{
    \begin{itemize}
        \item Определитель является кососимметрической полилинейной
        функцией строк матрицы.
        \item Всякая $f$ на множестве квадратных матриц порядка $n$,
        являющаяся кососимметрической полилинейной функцией строк матрицы,
        имеет вид $f(A) = f(E) \det A$
    \end{itemize}
}

Определитель матрицы не изменяется при сложении строк с коэффициентом.

\T {
    Квадратная матрица невырожденна $\iff\; \det A \neq 0$
    \begin{proof}
        Приведем матрицу к ступенчатому виду, сохраняется свойство
        неравенства определителя нулю. Матрица $A$ невырожденна титт,
        когда полученная ступенчатая матрица является строго треугольной.
        Равносильно тому, что определитель отличен от нуля.
    \end{proof}
}

\begin{itemize}
    \item $\det A = \det A^T$
    \item $\det AB = \det A \cdot \det B$
\end{itemize}

\T{
    $\det A = \sum\limits_j a_{ij} A_{ij} = \sum\limits_i a_{ij} A_{ij}$

    Где алгебраическое дополнение $A_{ij} = (-1)^{i+j} M_{ij}$, $M_{ij}$ - минор матрицы (определитель
    подматрицы без i строки и j столбца).
}



\subsubsection{Операции над матрицами. Операции над матрицами и их свойства. Теорема о ранге
произведения двух матриц. Определитель произведения квадратных матриц. Обратная
матрица, ее явный вид (формула), способ выражения с помощью элементарных
преобразований строк.}

Матрицы образуют модуль над соответствующим кольцом.

Квадратные матрицы одного размера образуют ассоциативное кольцо с единицей.

\T{
    Ранг произведения матриц не превосходит ранга каждого из множителей.
    \begin{proof}
        $AB = C \Rightarrow c_{ik} = \sum\limits_j a_{ij}b_{jk}$.

        Рассматривая эти равенства при фиксированном $i$, видно, что строка
        $i$ матрицы $C$ - линейная комбинация строк матрицы $B$ с коэффициентами
        из $i$ строки матрицы $A$. Следовательно, линейная оболочка строк
        матрицы $C$ содержится в линейной оболочке строк матрицы $B \Rightarrow rk C \leq rk B$.

        Фиксируя $k$ видно, что линейная оболочка столбцов матрицы $C$ содержится
        в линейной оболочке столбцов матрицы $A \Rightarrow rk C \leq rk A$
    \end{proof}
}

\T{
    $\forall A, B$ - квадратных матриц.

    $\det AB = \det A \cdot \det B$

    \begin{proof}
        $c_1, ..., c_n$ - строки матрицы $AB$.
        
        $c_i = a_i B$. Отсюда следует, что при фиксированной матрице $B$
        $\det AB$ - кососимметрическая полилинейная функция строк матрицы $A$.
        (нужна проверка)

        Значит она представляется в виде $f(E) \cdot \det A\Rightarrow$

        $\det AB = \det EB \cdot \det A = \det A \cdot \det B$
    \end{proof}
}

\D{
    Обратная матрица к $A$ это такая матрица $A^{-1}$, что $A A^{-1} = E$
}

Чтобы найти обратную матрицу нужно выписать рядом нужную матрицу и единичную,
затем, применяя одинаковые преобразования строк, привести нужную матрицу к
единичному виду.

$A^{-1} = \frac{adj A}{|A|}$,

где $adj A$ - взаимная матрица (матрица, составленная из алгебраических дополнений
и транспонированная).



\subsubsection{Векторные пространства; базис. Векторное пространство, его базис и размерность.
Преобразования координат в векторном пространстве. Подпространства как множества
решений систем однородных линейных уравнений. Связь между размерностями суммы
и пересечения двух подпространств. Линейная независимость подпространств. Базис и
размерность прямой суммы подпространств.}

\D{
    Линейное пространство над полем $K$ называется множество $V$ с операциями
    сложения и умножения на элементы $K$ со следующими свойствами:
    \begin{itemize}
        \item $V$ - абелева группа по сложению
        \item $\lambda(a+b) = \lambda a + \lambda b$
        \item $(\lambda + \mu) a = \lambda a + \mu a$
        \item $(\lambda \mu) a = \lambda (\mu a)$
        \item $1 a = a$
    \end{itemize}
}

\D{
    Базис векторного пространства = система векторов $\{e_1, ..., e_n\} \subset V$,
    т.ч. $\forall a \in V \exists! (a_1, ..., a_n): a = \sum a_i e_i$
}

\textbf{Пр.} Базис = максимальное ЛНС. Минимальное порождающее семейство.

\T{
    Всякое векторное пространство над $K$ изоморфно $K^n$
    \begin{proof}
        $\phi: V \to K^n$, которое ставит вектору в соответствие его координаты
        в выбранном базисе.
    \end{proof}
}

\T{
    \begin{itemize}
        \item Совокупность решений однородной системы является подпространством
        в $K^n$
        \item Совокупность решений произвольной совместной системы есть сумма
        одного из решений и подпространства решений однородной системы с той же матрицей.
    \end{itemize}
}


\end{document}
