%----------------------------------------------------------------------------------------
%	PACKAGES AND OTHER DOCUMENT CONFIGURATIONS
%----------------------------------------------------------------------------------------

\documentclass{article}
% \documentclass[14pt]{extarticle}
\usepackage{pdfpages}

\input{structure.tex} % Include the file specifying the document structure and custom commands

%----------------------------------------------------------------------------------------
%	ASSIGNMENT INFORMATION
%----------------------------------------------------------------------------------------

\title{Gamedev} % Title of the assignment

%----------------------------------------------------------------------------------------

\begin{document}

\maketitle % Print the title

\section{Базовая часть}

\subsection{Алгебра и геометрия, линейная алгебра}

\subsubsection{Группы, кольца, поля. Определения и примеры. Циклические группы. Теорема о
гомоморфизме.}

\D{
    Группа = $\langle M, \cdot\rangle$ - непустое множество с операцией

    Верны следующие свойства
    \begin{itemize}
        \item Ассоциативность: $(a \cdot b) \cdot c = a \cdot (b\cdot c)$
        \item Существование нейтрального: $\exists 1 \forall a : a \cdot 1 = 1 \cdot a = a$
        \item Существование обратного: $\forall a \exists a^{-1}: a \cdot a^{-1} = 1$
    \end{itemize}
}

Примеры $\mathbb{Z}, \mathbb{Z}/n\mathbb{Z}, \langle\mathbb{R}, \cdot\rangle$

\D{
    Абелева группа = группа + операция коммутативна.
}


\D{
    Циклическая группа = группа, порожденная степенями одного элемента.

    $G' = \langle g \rangle, g \in G$, $G$ - группа.
}

\D{
    Кольцо = $\langle M, +, \cdot \rangle$
    \begin{itemize}
        \item Абелева группа относительно $+$
        \item $\forall a, b, c \in M: a(b + c) = ab + ac \wedge (a+b)c = ac + bc$
    \end{itemize}

    Кольцо называется коммутативным, если умножение в нем коммутативно.

    Ассоциативным, если ассоциативно.

    Кольцо с единицей, если есть единица.
}

Следствия:
\begin{itemize}
    \item $a 0 = 0 a = 0$
    \item $a(-b) = (-a) b = -ab$
    \item $a(b-c) = ab - ac$
\end{itemize}


\D {
    Поле = ассоциативное коммутативное кольцо с единицей, в котором каждый
    элемент обратим.

    Абелева группа по умножению.
}

\D{
    Гомоморфизм групп = отображение $f: G \to H$, т.ч.
    $\forall a, b \in G: f(ab) = f(a) f(b)$

    \begin{itemize}
        \item $Im f = \{f(a): a \in G\} < H$
        \item $Ker f = \{a\in G : f(a) = e\} \lhd  G$ (нормальная)
    \end{itemize}
}

\T[о гомоморфизме групп]{
    $f: G \to H$ - гомоморфизм. Тогда
    $Im f \simeq G / Ker f$
}


\end{document}
