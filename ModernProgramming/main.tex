%----------------------------------------------------------------------------------------
%	PACKAGES AND OTHER DOCUMENT CONFIGURATIONS
%----------------------------------------------------------------------------------------

\documentclass{article}
% \documentclass[14pt]{extarticle}
\usepackage{pdfpages}
\usepackage{float}

\input{structure.tex} % Include the file specifying the document structure and custom commands

%----------------------------------------------------------------------------------------
%	ASSIGNMENT INFORMATION
%----------------------------------------------------------------------------------------

\title{Programming and AI} % Title of the assignment

%----------------------------------------------------------------------------------------

\DeclareMathOperator*\uplim{\overline{lim}}

\usepackage[parfill]{parskip}
\usepackage{listings}

\begin{document}
	
\maketitle % Print the title
%\tableofcontents
\section{Алгоритмы и структуры данных}

\subsection{Сложность алгоритмов по времени и методы ее оценки}

\href{https://ru.wikipedia.org/wiki/%D0%92%D1%8B%D1%87%D0%B8%D1%81%D0%BB%D0%B8%D1%82%D0%B5%D0%BB%D1%8C%D0%BD%D0%B0%D1%8F_%D1%81%D0%BB%D0%BE%D0%B6%D0%BD%D0%BE%D1%81%D1%82%D1%8C}{Асимптотическая сложность}


\subsection{Структуры данных}

\subsubsection{расширяющийся массив}

\href{https://ru.algorithmica.org/cs/basic-structures/vector/}{Динамический массив}

\subsubsection{список}

\href{https://neerc.ifmo.ru/wiki/index.php?title=%D0%A1%D0%BF%D0%B8%D1%81%D0%BE%D0%BA#:~:text=List)%20%E2%80%94%20%D1%81%D1%82%D1%80%D1%83%D0%BA%D1%82%D1%83%D1%80%D0%B0%20%D0%B4%D0%B0%D0%BD%D0%BD%D1%8B%D1%85%2C%20%D1%81%D0%BE%D1%81%D1%82%D0%BE%D1%8F%D1%89%D0%B0%D1%8F,%D0%B4%D0%B0%D0%BD%D0%BD%D1%8B%D1%85%20%D0%BA%D0%B0%D0%BA%20%D1%81%D1%82%D0%B5%D0%BA%20%D0%B8%20%D0%BE%D1%87%D0%B5%D1%80%D0%B5%D0%B4%D1%8C}{Список}
\subsubsection{двоичная куча}

\href{https://ru.algorithmica.org/cs/basic-structures/vector/}{Куча}

\subsection{Абстрактные типы данных}

\subsubsection{стек}

\href{https://neerc.ifmo.ru/wiki/index.php?title=%D0%A1%D1%82%D0%B5%D0%BA}{Стек}

\subsubsection{очередь}

\href{https://neerc.ifmo.ru/wiki/index.php?title=%D0%9E%D1%87%D0%B5%D1%80%D0%B5%D0%B4%D1%8C}{Очередь}

\subsubsection{система непересекающихся множеств}

\href{https://e-maxx.ru/algo/dsu}{С доказательствами времени работы}

\href{https://ru.algorithmica.org/cs/set-structures/dsu/#:~:text=%D0%A1%D0%B8%D1%81%D1%82%D0%B5%D0%BC%D0%B0%20%D0%BD%D0%B5%D0%BF%D0%B5%D1%80%D0%B5%D1%81%D0%B5%D0%BA%D0%B0%D1%8E%D1%89%D0%B8%D1%85%D1%81%D1%8F%20%D0%BC%D0%BD%D0%BE%D0%B6%D0%B5%D1%81%D1%82%D0%B2%20(%D0%B0%D0%BD%D0%B3%D0%BB.,%D1%87%D0%B5%D0%BC%D1%83%20%D1%80%D0%B0%D0%B2%D0%B5%D0%BD%20%D1%80%D0%B0%D0%B7%D0%BC%D0%B5%D1%80%20%D0%B4%D0%B0%D0%BD%D0%BD%D0%BE%D0%B3%D0%BE%20%D0%BC%D0%BD%D0%BE%D0%B6%D0%B5%D1%81%D1%82%D0%B2%D0%B0%C2%BB}{Без доказательств}

\subsubsection{Алгоритмы сортировки}

\href{https://neerc.ifmo.ru/wiki/index.php?title=%D0%A1%D0%BE%D1%80%D1%82%D0%B8%D1%80%D0%BE%D0%B2%D0%BA%D0%B8}{Сортировки}

\subsubsection{Сортировка вставкой}

\href{https://neerc.ifmo.ru/wiki/index.php?title=%D0%A1%D0%BE%D1%80%D1%82%D0%B8%D1%80%D0%BE%D0%B2%D0%BA%D0%B0_%D0%B2%D1%81%D1%82%D0%B0%D0%B2%D0%BA%D0%B0%D0%BC%D0%B8}{Вставкой}

\subsubsection{Сортировка слиянием}

\href{https://neerc.ifmo.ru/wiki/index.php?title=%D0%A1%D0%BE%D1%80%D1%82%D0%B8%D1%80%D0%BE%D0%B2%D0%BA%D0%B0_%D1%81%D0%BB%D0%B8%D1%8F%D0%BD%D0%B8%D0%B5%D0%BC}{Слиянием}

\subsubsection{Быстрая сортировка}

\href{https://neerc.ifmo.ru/wiki/index.php?title=%D0%91%D1%8B%D1%81%D1%82%D1%80%D0%B0%D1%8F_%D1%81%D0%BE%D1%80%D1%82%D0%B8%D1%80%D0%BE%D0%B2%D0%BA%D0%B0}{Быстрая}

\subsubsection{Сортировка кучей}

\href{https://neerc.ifmo.ru/wiki/index.php?title=%D0%A1%D0%BE%D1%80%D1%82%D0%B8%D1%80%D0%BE%D0%B2%D0%BA%D0%B0_%D0%BA%D1%83%D1%87%D0%B5%D0%B9}{Кучей}

\subsection{Методы построения алгоритмов}

\subsubsection{разделяй и властвуй}

\href{https://codechick.io/tutorials/dsa/dsa-divide-and-conquer}{Разделяй и властвуй}

\subsubsection{динамическое программирование}

\href{https://neerc.ifmo.ru/wiki/index.php?title=%D0%94%D0%B8%D0%BD%D0%B0%D0%BC%D0%B8%D1%87%D0%B5%D1%81%D0%BA%D0%BE%D0%B5_%D0%BF%D1%80%D0%BE%D0%B3%D1%80%D0%B0%D0%BC%D0%BC%D0%B8%D1%80%D0%BE%D0%B2%D0%B0%D0%BD%D0%B8%D0%B5}{neerc}

\href{https://ru.wikipedia.org/wiki/%D0%94%D0%B8%D0%BD%D0%B0%D0%BC%D0%B8%D1%87%D0%B5%D1%81%D0%BA%D0%BE%D0%B5_%D0%BF%D1%80%D0%BE%D0%B3%D1%80%D0%B0%D0%BC%D0%BC%D0%B8%D1%80%D0%BE%D0%B2%D0%B0%D0%BD%D0%B8%D0%B5}{Википедия, есть примеры задач}

\subsubsection{жадные алгоритмы}

\href{https://ru.wikipedia.org/wiki/%D0%96%D0%B0%D0%B4%D0%BD%D1%8B%D0%B9_%D0%B0%D0%BB%D0%B3%D0%BE%D1%80%D0%B8%D1%82%D0%BC}{Жадный алгоритм}

\subsection{ Основная теорема о времени работы рекурсивных алгоритмов}

\href{https://neerc.ifmo.ru/wiki/index.php?title=%D0%9C%D0%B0%D1%81%D1%82%D0%B5%D1%80-%D1%82%D0%B5%D0%BE%D1%80%D0%B5%D0%BC%D0%B0}{Мастер теорема}

\subsection{Алгоритмы поиска в графе }

\subsubsection{поиск в ширину}

\href{https://neerc.ifmo.ru/wiki/index.php?title=%D0%9E%D0%B1%D1%85%D0%BE%D0%B4_%D0%B2_%D1%88%D0%B8%D1%80%D0%B8%D0%BD%D1%83}{bfs}
\subsubsection{поиск в глубину}

\href{http://neerc.ifmo.ru/wiki/index.php?title=%D0%9E%D0%B1%D1%85%D0%BE%D0%B4_%D0%B2_%D0%B3%D0%BB%D1%83%D0%B1%D0%B8%D0%BD%D1%83,_%D1%86%D0%B2%D0%B5%D1%82%D0%B0_%D0%B2%D0%B5%D1%80%D1%88%D0%B8%D0%BD}{dfs}

\subsubsection{алгоритм Дейкстры}

\href{https://neerc.ifmo.ru/wiki/index.php?title=%D0%90%D0%BB%D0%B3%D0%BE%D1%80%D0%B8%D1%82%D0%BC_%D0%94%D0%B5%D0%B9%D0%BA%D1%81%D1%82%D1%80%D1%8B}{Дейкстра}

\subsubsection{алгоритм Беллмана — Форда}

\href{https://neerc.ifmo.ru/wiki/index.php?title=%D0%90%D0%BB%D0%B3%D0%BE%D1%80%D0%B8%D1%82%D0%BC_%D0%A4%D0%BE%D1%80%D0%B4%D0%B0-%D0%91%D0%B5%D0%BB%D0%BB%D0%BC%D0%B0%D0%BD%D0%B0}{Форд-Беллман}

\subsubsection{алгоритм Флойда — Уоршала}

\href{https://neerc.ifmo.ru/wiki/index.php?title=%D0%90%D0%BB%D0%B3%D0%BE%D1%80%D0%B8%D1%82%D0%BC_%D0%A4%D0%BB%D0%BE%D0%B9%D0%B4%D0%B0#:~:text=%D0%90%D0%BB%D0%B3%D0%BE%D1%80%D0%B8%D1%82%D0%BC%20%D0%A4%D0%BB%D0%BE%D0%B9%D0%B4%D0%B0%20(%D0%B0%D0%BB%D0%B3%D0%BE%D1%80%D0%B8%D1%82%D0%BC%20%D0%A4%D0%BB%D0%BE%D0%B9%D0%B4%D0%B0%E2%80%93%D0%A3%D0%BE%D1%80%D1%88%D0%B5%D0%BB%D0%BB%D0%B0,%D1%85%D0%BE%D1%82%D1%8F%20%D0%B1%D1%8B%20%D0%BE%D0%B4%D0%B8%D0%BD%20%D1%82%D0%B0%D0%BA%D0%BE%D0%B9%20%D1%86%D0%B8%D0%BA%D0%BB}{Флойд}

\subsection{Двоичные деревья поиска}

\subsubsection{(АВЛ-дерево}

\href{https://neerc.ifmo.ru/wiki/index.php?title=%D0%90%D0%92%D0%9B-%D0%B4%D0%B5%D1%80%D0%B5%D0%B2%D0%BE}{АВЛ}

\href{https://ru.wikipedia.org/wiki/%D0%90%D0%92%D0%9B-%D0%B4%D0%B5%D1%80%D0%B5%D0%B2%D0%BE}{Все вращения}

\subsubsection{Красно-черное дерево}

\href{https://neerc.ifmo.ru/wiki/index.php?title=%D0%9A%D1%80%D0%B0%D1%81%D0%BD%D0%BE-%D1%87%D0%B5%D1%80%D0%BD%D0%BE%D0%B5_%D0%B4%D0%B5%D1%80%D0%B5%D0%B2%D0%BE}{Красно-черное}

\subsection{Хэш-таблицы и операции над ними}

\href{https://neerc.ifmo.ru/wiki/index.php?title=%D0%A5%D0%B5%D1%88-%D1%82%D0%B0%D0%B1%D0%BB%D0%B8%D1%86%D0%B0}{Хеш-таблицы}

\subsection{Алгоритмы поиска подстроки}

\subsubsection{Алгоритм Кнута-Морриса-Пратта}

\href{https://neerc.ifmo.ru/wiki/index.php?title=%D0%90%D0%BB%D0%B3%D0%BE%D1%80%D0%B8%D1%82%D0%BC_%D0%9A%D0%BD%D1%83%D1%82%D0%B0-%D0%9C%D0%BE%D1%80%D1%80%D0%B8%D1%81%D0%B0-%D0%9F%D1%80%D0%B0%D1%82%D1%82%D0%B0}{Кнута-Морриса-Пратта}

\subsubsection{Алгоритм Рабина-Карпа}

\href{https://neerc.ifmo.ru/wiki/index.php?title=%D0%9F%D0%BE%D0%B8%D1%81%D0%BA_%D0%BF%D0%BE%D0%B4%D1%81%D1%82%D1%80%D0%BE%D0%BA%D0%B8_%D0%B2_%D1%81%D1%82%D1%80%D0%BE%D0%BA%D0%B5_%D1%81_%D0%B8%D1%81%D0%BF%D0%BE%D0%BB%D1%8C%D0%B7%D0%BE%D0%B2%D0%B0%D0%BD%D0%B8%D0%B5%D0%BC_%D1%85%D0%B5%D1%88%D0%B8%D1%80%D0%BE%D0%B2%D0%B0%D0%BD%D0%B8%D1%8F._%D0%90%D0%BB%D0%B3%D0%BE%D1%80%D0%B8%D1%82%D0%BC_%D0%A0%D0%B0%D0%B1%D0%B8%D0%BD%D0%B0-%D0%9A%D0%B0%D1%80%D0%BF%D0%B0}{Рабина-Карпа}

\subsubsection{Алгоритм Ахо-Корасик}

\href{https://neerc.ifmo.ru/wiki/index.php?title=%D0%90%D0%BB%D0%B3%D0%BE%D1%80%D0%B8%D1%82%D0%BC_%D0%90%D1%85%D0%BE-%D0%9A%D0%BE%D1%80%D0%B0%D1%81%D0%B8%D0%BA}{Ахо-Корасик}

\href{https://ru.wikipedia.org/wiki/%D0%90%D0%BB%D0%B3%D0%BE%D1%80%D0%B8%D1%82%D0%BC_%D0%90%D1%85%D0%BE_%E2%80%94_%D0%9A%D0%BE%D1%80%D0%B0%D1%81%D0%B8%D0%BA}{Асимптотика алгоритма}

\subsection{Сложность вычислений}

\subsubsection{ классы сложности P и NP}

\href{https://neerc.ifmo.ru/wiki/index.php?title=%D0%9A%D0%BB%D0%B0%D1%81%D1%81_P}{Класс $P$}

\href{https://neerc.ifmo.ru/wiki/index.php?title=%D0%9D%D0%B5%D0%B4%D0%B5%D1%82%D0%B5%D1%80%D0%BC%D0%B8%D0%BD%D0%B8%D1%80%D0%BE%D0%B2%D0%B0%D0%BD%D0%BD%D1%8B%D0%B5_%D0%B2%D1%8B%D1%87%D0%B8%D1%81%D0%BB%D0%B5%D0%BD%D0%B8%D1%8F}{Недетерминированные вычисления}

\subsubsection{примеры NP-полных задач}

\href{https://neerc.ifmo.ru/wiki/index.php?title=%D0%A2%D0%B5%D0%BE%D1%80%D0%B5%D0%BC%D0%B0_%D0%9A%D1%83%D0%BA%D0%B0}{Теорема Кука-Левина}

\href{https://cgi.csc.liv.ac.uk/~igor/COMP309/3CP.pdf}{$3$-Coloring}

\href{https://neerc.ifmo.ru/wiki/index.php?title=NP-%D0%BF%D0%BE%D0%BB%D0%BD%D0%BE%D1%82%D0%B0_%D0%B7%D0%B0%D0%B4%D0%B0%D1%87%D0%B8_%D0%BE_%D0%BD%D0%B5%D0%B7%D0%B0%D0%B2%D0%B8%D1%81%D0%B8%D0%BC%D0%BE%D0%BC_%D0%BC%D0%BD%D0%BE%D0%B6%D0%B5%D1%81%D1%82%D0%B2%D0%B5}{Независимое множество}

\href{https://neerc.ifmo.ru/wiki/index.php?title=%D0%A2%D0%B5%D0%BE%D1%80%D0%B8%D1%8F_%D1%81%D0%BB%D0%BE%D0%B6%D0%BD%D0%BE%D1%81%D1%82%D0%B8}{Другие задачи}


\section{Статистика и машинное обучение}

\subsection{Нормальное распределение}

\href{https://ru.wikipedia.org/wiki/%D0%9D%D0%BE%D1%80%D0%BC%D0%B0%D0%BB%D1%8C%D0%BD%D0%BE%D0%B5_%D1%80%D0%B0%D1%81%D0%BF%D1%80%D0%B5%D0%B4%D0%B5%D0%BB%D0%B5%D0%BD%D0%B8%D0%B5}{Нормальное расперделение}

\href{https://ru.wikipedia.org/wiki/%D0%A6%D0%B5%D0%BD%D1%82%D1%80%D0%B0%D0%BB%D1%8C%D0%BD%D0%B0%D1%8F_%D0%BF%D1%80%D0%B5%D0%B4%D0%B5%D0%BB%D1%8C%D0%BD%D0%B0%D1%8F_%D1%82%D0%B5%D0%BE%D1%80%D0%B5%D0%BC%D0%B0}{ЦПТ}


\subsection{Вероятностная модель}

\href{https://teach-in.ru/file/synopsis/pdf/probabilistic-model-M.pdf}{Есть какие-то термины}

\subsection{Эмпирическая функция распределения}

\href{https://ru.wikipedia.org/wiki/%D0%92%D1%8B%D0%B1%D0%BE%D1%80%D0%BE%D1%87%D0%BD%D0%B0%D1%8F_%D1%84%D1%83%D0%BD%D0%BA%D1%86%D0%B8%D1%8F_%D1%80%D0%B0%D1%81%D0%BF%D1%80%D0%B5%D0%B4%D0%B5%D0%BB%D0%B5%D0%BD%D0%B8%D1%8F}{Эмпирическая фукнция распределения}

\href{https://tvims.nsu.ru/chernova/ms/lec/node4.html}{Тоже что - то есть}

\subsection{Гистограмма}

\href{https://ru.wikipedia.org/wiki/%D0%93%D0%B8%D1%81%D1%82%D0%BE%D0%B3%D1%80%D0%B0%D0%BC%D0%BC%D0%B0_(%D1%81%D1%82%D0%B0%D1%82%D0%B8%D1%81%D1%82%D0%B8%D0%BA%D0%B0)}{Гистограмма}

\subsection{Выборочные моменты, мода, медиана}

\href{https://ru.wikipedia.org/wiki/%D0%92%D1%8B%D0%B1%D0%BE%D1%80%D0%BE%D1%87%D0%BD%D1%8B%D0%B5_%D0%BC%D0%BE%D0%BC%D0%B5%D0%BD%D1%82%D1%8B}{Выборочные моменты}

\href{https://ru.wikipedia.org/wiki/%D0%9C%D0%B5%D0%B4%D0%B8%D0%B0%D0%BD%D0%B0_(%D1%81%D1%82%D0%B0%D1%82%D0%B8%D1%81%D1%82%D0%B8%D0%BA%D0%B0)}{Медиана}

\href{https://ru.wikipedia.org/wiki/%D0%9C%D0%BE%D0%B4%D0%B0_(%D1%81%D1%82%D0%B0%D1%82%D0%B8%D1%81%D1%82%D0%B8%D0%BA%D0%B0)}{Мода}

\href{https://neerc.ifmo.ru/wiki/index.php?title=%D0%9D%D0%B5%D1%80%D0%B0%D0%B2%D0%B5%D0%BD%D1%81%D1%82%D0%B2%D0%BE_%D0%9C%D0%B0%D1%80%D0%BA%D0%BE%D0%B2%D0%B0#:~:text=%D0%9D%D0%B5%D1%80%D0%B0%D0%B2%D0%B5%D0%BD%D1%81%D1%82%D0%B2%D0%BE%20%D0%A7%D0%B5%D0%B1%D1%8B%D1%88%D0%B5%D0%B2%D0%B0%20(%D0%B0%D0%BD%D0%B3%D0%BB.,%D0%B1%D0%BB%D0%B8%D0%B7%D0%BA%D0%B8%D0%B5%20%D0%BA%20%D0%B7%D0%BD%D0%B0%D1%87%D0%B5%D0%BD%D0%B8%D1%8E%20%D0%BC%D0%B0%D1%82%D0%B5%D0%BC%D0%B0%D1%82%D0%B8%D1%87%D0%B5%D1%81%D0%BA%D0%BE%D0%B3%D0%BE%20%D0%BE%D0%B6%D0%B8%D0%B4%D0%B0%D0%BD%D0%B8%D1%8F}{Неравенства Маркова и Чебышева}

\subsection{Метод максимального правдоподобия}

\href{https://ru.wikipedia.org/wiki/%D0%A4%D1%83%D0%BD%D0%BA%D1%86%D0%B8%D1%8F_%D0%BF%D1%80%D0%B0%D0%B2%D0%B4%D0%BE%D0%BF%D0%BE%D0%B4%D0%BE%D0%B1%D0%B8%D1%8F}{Функция правдоподобия}

\href{https://ru.wikipedia.org/wiki/%D0%9C%D0%B5%D1%82%D0%BE%D0%B4_%D0%BC%D0%B0%D0%BA%D1%81%D0%B8%D0%BC%D0%B0%D0%BB%D1%8C%D0%BD%D0%BE%D0%B3%D0%BE_%D0%BF%D1%80%D0%B0%D0%B2%D0%B4%D0%BE%D0%BF%D0%BE%D0%B4%D0%BE%D0%B1%D0%B8%D1%8F}{Метод максимального правдоподобия}

\subsection{Методология машинного обучения и примеры прикладных задач}

\href{https://www.osp.ru/cio/2018/05/13054535}{Машинное обучение: методы и способы}

\href{http://www.machinelearning.ru/wiki/index.php?title=%D0%9A%D0%BB%D0%B0%D1%81%D1%81%D0%B8%D1%84%D0%B8%D0%BA%D0%B0%D1%86%D0%B8%D1%8F}{Приеладные задачи классификации}

\subsection{Бинарные, категориальные, ординальные (порядковые) и вещественные признаки}

\href{http://www.machinelearning.ru/wiki/index.php?title=%D0%9A%D0%BB%D0%B0%D1%81%D1%81%D0%B8%D1%84%D0%B8%D0%BA%D0%B0%D1%86%D0%B8%D1%8F}{Признаковое пространство}

\subsection{Постановка задач классификации, кластеризации и ранжирования}

\subsubsection{Классификация}

\href{http://www.machinelearning.ru/wiki/index.php?title=%D0%9A%D0%BB%D0%B0%D1%81%D1%81%D0%B8%D1%84%D0%B8%D0%BA%D0%B0%D1%86%D0%B8%D1%8F#:~:text=%D0%9A%D0%BB%D0%B0%D1%81%D1%81%D0%B8%D1%84%D0%B8%D0%BA%D0%B0%D1%86%D0%B8%D1%8F%20%E2%80%94%20%D0%BE%D0%B4%D0%B8%D0%BD%20%D0%B8%D0%B7%20%D1%80%D0%B0%D0%B7%D0%B4%D0%B5%D0%BB%D0%BE%D0%B2%20%D0%BC%D0%B0%D1%88%D0%B8%D0%BD%D0%BD%D0%BE%D0%B3%D0%BE,%D0%AD%D1%82%D0%BE%20%D0%BC%D0%BD%D0%BE%D0%B6%D0%B5%D1%81%D1%82%D0%B2%D0%BE%20%D0%BD%D0%B0%D0%B7%D1%8B%D0%B2%D0%B0%D0%B5%D1%82%D1%81%D1%8F%20%D0%BE%D0%B1%D1%83%D1%87%D0%B0%D1%8E%D1%89%D0%B5%D0%B9%20%D0%B2%D1%8B%D0%B1%D0%BE%D1%80%D0%BA%D0%BE%D0%B9}{Классификация}
\subsubsection{Кластеризация}

\href{https://datascience.eu/ru/%D0%BC%D0%B0%D1%88%D0%B8%D0%BD%D0%BD%D0%BE%D0%B5-%D0%BE%D0%B1%D1%83%D1%87%D0%B5%D0%BD%D0%B8%D0%B5/%D0%BA%D0%BB%D0%B0%D1%81%D1%82%D0%B5%D1%80%D0%BD%D1%8B%D0%B5-%D0%B0%D0%BB%D0%B3%D0%BE%D1%80%D0%B8%D1%82%D0%BC%D1%8B-%D0%B8-%D0%B8%D1%85-%D0%B7%D0%BD%D0%B0%D1%87%D0%B5%D0%BD%D0%B8%D0%B5-%D0%B2-%D0%BC/}{Кластеризация}

\subsubsection{Ранжирование}

\href{https://neerc.ifmo.ru/wiki/index.php?title=%D0%A0%D0%B0%D0%BD%D0%B6%D0%B8%D1%80%D0%BE%D0%B2%D0%B0%D0%BD%D0%B8%D0%B5}{Ранжирование}


\subsection{Функции потерь для задач классификации, регрессии и ранжирования}

\href{https://id-lab.ru/posts/developers/funkcii/}{MSE и кросс-энтропия}

\href{https://russianblogs.com/article/6840440533/}{5 функций потери регрессии}

\href{https://neerc.ifmo.ru/wiki/index.php?title=%D0%A0%D0%B0%D0%BD%D0%B6%D0%B8%D1%80%D0%BE%D0%B2%D0%B0%D0%BD%D0%B8%D0%B5}{Метрики качества ранжирования}


\subsection{Методы отбора признаков}

\href{https://habr.com/ru/company/jetinfosystems/blog/470622/}{Обзор методов отбора признаков}

\subsection{Наивный байесовский классификатор}

\href{https://ru.wikipedia.org/wiki/%D0%9D%D0%B0%D0%B8%D0%B2%D0%BD%D1%8B%D0%B9_%D0%B1%D0%B0%D0%B9%D0%B5%D1%81%D0%BE%D0%B2%D1%81%D0%BA%D0%B8%D0%B9_%D0%BA%D0%BB%D0%B0%D1%81%D1%81%D0%B8%D1%84%D0%B8%D0%BA%D0%B0%D1%82%D0%BE%D1%80#:~:text=%D0%9D%D0%B0%D0%B8%CC%81%D0%B2%D0%BD%D1%8B%D0%B9%20%D0%B1%D0%B0%CC%81%D0%B9%D0%B5%D1%81%D0%BE%D0%B2%D1%81%D0%BA%D0%B8%D0%B9%20%D0%BA%D0%BB%D0%B0%D1%81%D1%81%D0%B8%D1%84%D0%B8%D0%BA%D0%B0%CC%81%D1%82%D0%BE%D1%80%20%E2%80%94%20%D0%BF%D1%80%D0%BE%D1%81%D1%82%D0%BE%D0%B9%20%D0%B2%D0%B5%D1%80%D0%BE%D1%8F%D1%82%D0%BD%D0%BE%D1%81%D1%82%D0%BD%D1%8B%D0%B9,%D0%BA%D0%BB%D0%B0%D1%81%D1%81%D0%B8%D1%84%D0%B8%D0%BA%D0%B0%D1%82%D0%BE%D1%80%D1%8B%20%D0%BC%D0%BE%D0%B3%D1%83%D1%82%20%D0%BE%D0%B1%D1%83%D1%87%D0%B0%D1%82%D1%8C%D1%81%D1%8F%20%D0%BE%D1%87%D0%B5%D0%BD%D1%8C%20%D1%8D%D1%84%D1%84%D0%B5%D0%BA%D1%82%D0%B8%D0%B2%D0%BD%D0%BE.}{Наивный Байесовский классификатор}

\section{ Программирование и архитектура ПО}

\subsection{Основные принципы ООП}

\href{https://tproger.ru/translations/oop-principles-cheatsheet/#:~:text=%D0%91%D0%B0%D0%B7%D0%BE%D0%B2%D1%8B%D0%B5%20%D0%BF%D1%80%D0%B8%D0%BD%D1%86%D0%B8%D0%BF%D1%8B%20%D0%9E%D0%9E%D0%9F,%D0%BC%D0%B5%D1%85%D0%B0%D0%BD%D0%B8%D0%B7%D0%BC%20%D0%B4%D0%BB%D1%8F%20%D0%BF%D0%BE%D0%B2%D1%82%D0%BE%D1%80%D0%BD%D0%BE%D0%B3%D0%BE%20%D0%B8%D1%81%D0%BF%D0%BE%D0%BB%D1%8C%D0%B7%D0%BE%D0%B2%D0%B0%D0%BD%D0%B8%D1%8F%20%D0%BA%D0%BE%D0%B4%D0%B0.}{Кратко}

\href{https://training.epam.ua/News/Items/275?lang=ru}{Подробно}

\subsection{Паттерны проектирования}

\href{https://ru.wikipedia.org/wiki/%D0%A8%D0%B0%D0%B1%D0%BB%D0%BE%D0%BD_%D0%BF%D1%80%D0%BE%D0%B5%D0%BA%D1%82%D0%B8%D1%80%D0%BE%D0%B2%D0%B0%D0%BD%D0%B8%D1%8F}{Шаблон проектирования}

\subsection{SOLID принципы}

\href{https://ru.wikipedia.org/wiki/SOLID_(%D0%BE%D0%B1%D1%8A%D0%B5%D0%BA%D1%82%D0%BD%D0%BE-%D0%BE%D1%80%D0%B8%D0%B5%D0%BD%D1%82%D0%B8%D1%80%D0%BE%D0%B2%D0%B0%D0%BD%D0%BD%D0%BE%D0%B5_%D0%BF%D1%80%D0%BE%D0%B3%D1%80%D0%B0%D0%BC%D0%BC%D0%B8%D1%80%D0%BE%D0%B2%D0%B0%D0%BD%D0%B8%D0%B5)#:~:text=%D0%9F%D1%80%D0%B8%D0%BD%D1%86%D0%B8%D0%BF%D1%8B%20SOLID%20%E2%80%94%20%D1%8D%D1%82%D0%BE%20%D1%80%D1%83%D0%BA%D0%BE%D0%B2%D0%BE%D0%B4%D1%81%D1%82%D0%B2%D0%B0%2C%20%D0%BA%D0%BE%D1%82%D0%BE%D1%80%D1%8B%D0%B5,%D0%BA%D0%BE%D0%B4%D0%B0%20%D1%81%20%D1%81%D0%BE%D0%B1%D0%BB%D1%8E%D0%B4%D0%B5%D0%BD%D0%B8%D0%B5%D0%BC%20%D0%BF%D1%80%D0%B8%D0%BD%D1%86%D0%B8%D0%BF%D0%BE%D0%B2%20SOLID.}{SOLID}

\subsection{Современные языки ООП}

\href{https://ru.wikipedia.org/wiki/%D0%9E%D0%B1%D1%8A%D0%B5%D0%BA%D1%82%D0%BD%D0%BE-%D0%BE%D1%80%D0%B8%D0%B5%D0%BD%D1%82%D0%B8%D1%80%D0%BE%D0%B2%D0%B0%D0%BD%D0%BD%D1%8B%D0%B9_%D1%8F%D0%B7%D1%8B%D0%BA_%D0%BF%D1%80%D0%BE%D0%B3%D1%80%D0%B0%D0%BC%D0%BC%D0%B8%D1%80%D0%BE%D0%B2%D0%B0%D0%BD%D0%B8%D1%8F}{Список}

\href{https://tproger.ru/translations/sovremennye-jazyki-programmirovanija-kotorye-zastavjat-vas-stradat-chast-1-oop/}{Особенности}

\subsection{ Виртуальные машины (основы работы с типами, работа с памятью, загрузка сборок, сериализация)}

\href{https://ru.wikipedia.org/wiki/%D0%92%D0%B8%D1%80%D1%82%D1%83%D0%B0%D0%BB%D1%8C%D0%BD%D0%B0%D1%8F_%D0%BC%D0%B0%D1%88%D0%B8%D0%BD%D0%B0}{Википедия}

\href{https://azure.microsoft.com/ru-ru/resources/cloud-computing-dictionary/what-is-a-virtual-machine/#overview}{Что-то еще}


\subsection{Многопоточность, основные примитивы синхронизации}

\href{https://habr.com/ru/company/otus/blog/549814/}{Multithreading}

\href{https://sites.google.com/site/interviewknowages/primitivy-sinhronizacii}{Примитивы синхронизации}

\subsection{Монолит и микросервисная архитектура}

\href{https://www.atlassian.com/ru/microservices/microservices-architecture/microservices-vs-monolith#:~:text=%D0%9C%D0%BE%D0%BD%D0%BE%D0%BB%D0%B8%D1%82%D0%BD%D0%BE%D0%B5%20%D0%BF%D1%80%D0%B8%D0%BB%D0%BE%D0%B6%D0%B5%D0%BD%D0%B8%D0%B5%20%E2%80%94%20%D1%8D%D1%82%D0%BE%20%D0%B5%D0%B4%D0%B8%D0%BD%D1%8B%D0%B9%20%D0%BE%D0%B1%D1%89%D0%B8%D0%B9,%D0%BD%D0%B0%D0%B1%D0%BE%D1%80%20%D0%BD%D0%B5%D0%B1%D0%BE%D0%BB%D1%8C%D1%88%D0%B8%D1%85%20%D0%BD%D0%B5%D0%B7%D0%B0%D0%B2%D0%B8%D1%81%D0%B8%D0%BC%D0%BE%20%D1%80%D0%B0%D0%B7%D0%B2%D0%B5%D1%80%D1%82%D1%8B%D0%B2%D0%B0%D0%B5%D0%BC%D1%8B%D1%85%20%D1%81%D0%BB%D1%83%D0%B6%D0%B1.}{Сравнение микросервисной и монолитной архитектур}


\section{Математическая логика и теория алгоритмов}

\subsection{Логика высказываний, схемы из функциональных элементов}

\href{https://ru.wikipedia.org/wiki/%D0%9B%D0%BE%D0%B3%D0%B8%D0%BA%D0%B0_%D0%B2%D1%8B%D1%81%D0%BA%D0%B0%D0%B7%D1%8B%D0%B2%D0%B0%D0%BD%D0%B8%D0%B9}{Логика высказываний}

\href{http://mathhelpplanet.com/static.php?p=skhemy-iz-funktsionalnykh-elementov}{Схемы из функциональных элементов}

\subsection{Исчисление высказываний, теорема о полноте.}

\href{https://lms2.sseu.ru/courses/eresmat/metod/met6/parmet6_8.htm#:~:text=%D0%98%D1%81%D1%87%D0%B8%D1%81%D0%BB%D0%B5%D0%BD%D0%B8%D0%B5%20%D0%B2%D1%8B%D1%81%D0%BA%D0%B0%D0%B7%D1%8B%D0%B2%D0%B0%D0%BD%D0%B8%D0%B9%20%E2%80%93%20%D1%8D%D1%82%D0%BE%20%D0%B0%D0%BA%D1%81%D0%B8%D0%BE%D0%BC%D0%B0%D1%82%D0%B8%D1%87%D0%B5%D1%81%D0%BA%D0%B0%D1%8F%20%D0%BB%D0%BE%D0%B3%D0%B8%D1%87%D0%B5%D1%81%D0%BA%D0%B0%D1%8F,%D1%81%D0%B8%D0%BC%D0%B2%D0%BE%D0%BB%D0%BE%D0%B2%20%D0%B8%20%D0%BE%D0%BF%D1%80%D0%B5%D0%B4%D0%B5%D0%BB%D0%B5%D0%BD%D0%B8%D0%B5%20%D0%B2%D1%8B%D0%B2%D0%BE%D0%B4%D0%B8%D0%BC%D1%8B%D1%85%20%D1%84%D0%BE%D1%80%D0%BC%D1%83%D0%BB.}{Исчисление высказываний}

\href{https://neerc.ifmo.ru/wiki/index.php?title=%D0%98%D1%81%D1%87%D0%B8%D1%81%D0%BB%D0%B5%D0%BD%D0%B8%D0%B5_%D0%B2%D1%8B%D1%81%D0%BA%D0%B0%D0%B7%D1%8B%D0%B2%D0%B0%D0%BD%D0%B8%D0%B9}{neerc}

\href{https://mipt.lectoriy.ru/file/synopsis/pdf/Maths-MathemLogic-M05-Musatov-141001.04.pdf}{Хороший конспект по исчислению высказываний}

\href{https://mipt.lectoriy.ru/file/synopsis/pdf/Maths-MathemLogic-M06-Musatov-141008.03.pdf}{теорема о полноте}


\subsection{Исчисление секвенций}

\href{https://ru.wikipedia.org/wiki/%D0%98%D1%81%D1%87%D0%B8%D1%81%D0%BB%D0%B5%D0%BD%D0%B8%D0%B5_%D1%81%D0%B5%D0%BA%D0%B2%D0%B5%D0%BD%D1%86%D0%B8%D0%B9}{Исчисление секвенций}


\subsection{ Интуиционистская пропозициональная логика}

\href{https://ru.wikipedia.org/wiki/%D0%98%D0%BD%D1%82%D1%83%D0%B8%D1%86%D0%B8%D0%BE%D0%BD%D0%B8%D1%81%D1%82%D1%81%D0%BA%D0%B0%D1%8F_%D0%BB%D0%BE%D0%B3%D0%B8%D0%BA%D0%B0}{ Интуиционистская пропозициональная логика}

\href{https://ru.frwiki.wiki/wiki/Logique_intuitionniste}{Тут подробней, но странно}

\subsection{Предикаты, истинность и выразимость}

\href{https://mipt.lectoriy.ru/file/synopsis/pdf/Maths-MathemLogic-M07-Musatov-141015.03.pdf}{Предикаты}

\href{http://ru.discrete-mathematics.org/fall2017/1/logic_pmi/lecture-10-11-expressibility.pdf}{Выразимость предикатов}


\subsection{Элиминация кванторов}

\href{https://ru.wikipedia.org/wiki/%D0%AD%D0%BB%D0%B8%D0%BC%D0%B8%D0%BD%D0%B0%D1%86%D0%B8%D1%8F_%D0%BA%D0%B2%D0%B0%D0%BD%D1%82%D0%BE%D1%80%D0%BE%D0%B2#:~:text=%D0%AD%D0%BB%D0%B8%D0%BC%D0%B8%D0%BD%D0%B0%D1%86%D0%B8%D1%8F%20%D0%BA%D0%B2%D0%B0%D0%BD%D1%82%D0%BE%D1%80%D0%BE%D0%B2%20%E2%80%94%20%D0%BF%D0%BE%D0%BB%D1%83%D1%87%D0%B5%D0%BD%D0%B8%D0%B5%20%D0%BF%D0%BE%20%D0%B7%D0%B0%D0%B4%D0%B0%D0%BD%D0%BD%D0%BE%D0%B9,%D1%81%D0%BE%D0%B4%D0%B5%D1%80%D0%B6%D0%B0%D1%82%D0%B5%D0%BB%D1%8C%D0%BD%D1%8B%D1%85%20%D1%80%D0%B5%D0%B7%D1%83%D0%BB%D1%8C%D1%82%D0%B0%D1%82%D0%BE%D0%B2%20%D0%BE%D0%B1%20%D1%8D%D1%82%D0%BE%D0%B9%20%D1%82%D0%B5%D0%BE%D1%80%D0%B8%D0%B8.}{Элиминация кванторов}

\href{https://studfile.net/preview/6844674/page:4/}{Пример}

\subsection{Арифметика Пресбургера}

\href{https://ru.wikipedia.org/wiki/%D0%90%D1%80%D0%B8%D1%84%D0%BC%D0%B5%D1%82%D0%B8%D0%BA%D0%B0_%D0%9F%D1%80%D0%B5%D1%81%D0%B1%D1%83%D1%80%D0%B3%D0%B5%D1%80%D0%B0}{Арифметика Пресбургера}

\subsection{Исчисление предикатов, выводы и полнота}

\href{https://matica.org.ua/metodichki-i-knigi-po-matematike/matematicheskaia-logika-i-teoriia-algoritmov/6-ischislenie-predikatov}{Исчисление предикатов}

\href{https://neerc.ifmo.ru/wiki/index.php?title=%D0%98%D1%81%D1%87%D0%B8%D1%81%D0%BB%D0%B5%D0%BD%D0%B8%D0%B5_%D0%BF%D1%80%D0%B5%D0%B4%D0%B8%D0%BA%D0%B0%D1%82%D0%BE%D0%B2}{neerc}

\href{https://ru.wikipedia.org/wiki/%D0%A2%D0%B5%D0%BE%D1%80%D0%B5%D0%BC%D0%B0_%D0%93%D1%91%D0%B4%D0%B5%D0%BB%D1%8F_%D0%BE_%D0%BF%D0%BE%D0%BB%D0%BD%D0%BE%D1%82%D0%B5#:~:text=%D0%A2%D0%B5%D0%BE%D1%80%D0%B5%CC%81%D0%BC%D0%B0%20%D0%93%D1%91%D0%B4%D0%B5%D0%BB%D1%8F%20%D0%BE%20%D0%BF%D0%BE%D0%BB%D0%BD%D0%BE%D1%82%D0%B5%CC%81%20%D0%B8%D1%81%D1%87%D0%B8%D1%81%D0%BB%D0%B5%CC%81%D0%BD%D0%B8%D1%8F,%D0%B4%D0%BE%D0%BA%D0%B0%D0%B7%D0%B0%D0%BD%D0%B0%20%D0%9A%D1%83%D1%80%D1%82%D0%BE%D0%BC%20%D0%93%D1%91%D0%B4%D0%B5%D0%BB%D0%B5%D0%BC%20%D0%B2%201929.}{Теорема Геделя о полноте}

\subsection{Метод резолюции}

\href{https://denchick.github.io/logic4humans/chapter3.html}{Метод резолюций}

\subsection{Предваренная нормальная форма}

\href{http://www.alex111007.ru/book/dm/zan/z24/z24v03.htm#:~:text=%D0%93%D0%BE%D0%B2%D0%BE%D1%80%D1%8F%D1%82%2C%20%D1%87%D1%82%D0%BE%20%D1%84%D0%BE%D1%80%D0%BC%D1%83%D0%BB%D0%B0%20%D0%BB%D0%BE%D0%B3%D0%B8%D0%BA%D0%B8%20%D0%BF%D1%80%D0%B5%D0%B4%D0%B8%D0%BA%D0%B0%D1%82%D0%BE%D0%B2,%D0%BE%D1%82%D1%80%D0%B8%D1%86%D0%B0%D0%BD%D0%B8%D1%8F%20%D0%BE%D1%82%D0%BD%D0%B5%D1%81%D0%B5%D0%BD%D0%B0%20%D0%BA%20%D1%8D%D0%BB%D0%B5%D0%BC%D0%B5%D0%BD%D1%82%D0%B0%D1%80%D0%BD%D1%8B%D0%BC%20%D1%84%D0%BE%D1%80%D0%BC%D1%83%D0%BB%D0%B0%D0%BC.&text=xi%20%D0%B8%D0%BB%D0%B8%20xi%20%2C%20%D0%B0%20%D1%84%D0%BE%D1%80%D0%BC%D1%83%D0%BB%D0%B0%20%D0%90%20%D0%BA%D0%B2%D0%B0%D0%BD%D1%82%D0%BE%D1%80%D0%BE%D0%B2%20%D0%BD%D0%B5%20%D1%81%D0%BE%D0%B4%D0%B5%D1%80%D0%B6%D0%B8%D1%82.&text=%D0%A2%D0%B5%D0%BE%D1%80%D0%B5%D0%BC%D0%B0.}{Предваренная нормальная форма}

\subsection{Теорема Эрбрана}

\href{https://intuit.ru/studies/professional_skill_improvements/2162/courses/133/lecture/3745?page=2}{Теорема Эрбрана}

\subsection{Сколемизация}

\href{https://helpiks.org/6-22101.html}{Сколемизация}

\subsection{Нормальный алгоритм}

\href{https://ru.wikipedia.org/wiki/%D0%9D%D0%BE%D1%80%D0%BC%D0%B0%D0%BB%D1%8C%D0%BD%D1%8B%D0%B9_%D0%B0%D0%BB%D0%B3%D0%BE%D1%80%D0%B8%D1%82%D0%BC}{Номральный алгоритм}

\subsection{Машины Тьюринга}

\href{https://neerc.ifmo.ru/wiki/index.php?title=%D0%9C%D0%B0%D1%88%D0%B8%D0%BD%D0%B0_%D0%A2%D1%8C%D1%8E%D1%80%D0%B8%D0%BD%D0%B3%D0%B0}{Машина Тьюринга}


\subsection{Рекурсивная функция}

\href{https://ru.wikipedia.org/wiki/%D0%A0%D0%B5%D0%BA%D1%83%D1%80%D1%81%D0%B8%D0%B2%D0%BD%D0%B0%D1%8F_%D1%84%D1%83%D0%BD%D0%BA%D1%86%D0%B8%D1%8F_(%D1%82%D0%B5%D0%BE%D1%80%D0%B8%D1%8F_%D0%B2%D1%8B%D1%87%D0%B8%D1%81%D0%BB%D0%B8%D0%BC%D0%BE%D1%81%D1%82%D0%B8)}{Рекурсивная функция}

\href{https://ru.wikipedia.org/wiki/%D0%A4%D1%83%D0%BD%D0%BA%D1%86%D0%B8%D1%8F_%D0%90%D0%BA%D0%BA%D0%B5%D1%80%D0%BC%D0%B0%D0%BD%D0%B0}{Функция Акермана}

\subsection{Неперечислимость, язык диагонализации}

\href{http://mathhelpplanet.com/static.php?p=razreshimost-i-perechislimost-mnozhestv}{Разрешимость и перечислимость множеств}

\href{https://mipt.lectoriy.ru/file/synopsis/pdf/Maths-CompComplexity-M04-Musatov-140924.04.pdf}{метод диагонализации}

\subsection{Разрешимость и неразрешимость, теорема Райса}

\href{https://ru.wikipedia.org/wiki/%D0%90%D0%BB%D0%B3%D0%BE%D1%80%D0%B8%D1%82%D0%BC%D0%B8%D1%87%D0%B5%D1%81%D0%BA%D0%B0%D1%8F_%D1%80%D0%B0%D0%B7%D1%80%D0%B5%D1%88%D0%B8%D0%BC%D0%BE%D1%81%D1%82%D1%8C}{Алгоритмическая разрешимость}

\href{https://ru.wikipedia.org/wiki/%D0%A0%D0%B0%D0%B7%D1%80%D0%B5%D1%88%D0%B8%D0%BC%D0%BE%D0%B5_%D0%BC%D0%BD%D0%BE%D0%B6%D0%B5%D1%81%D1%82%D0%B2%D0%BE}{Разрешимое множество}

\href{https://neerc.ifmo.ru/wiki/index.php?title=%D0%A1%D0%B2%D0%BE%D0%B9%D1%81%D1%82%D0%B2%D0%B0_%D0%BF%D0%B5%D1%80%D0%B5%D1%87%D0%B8%D1%81%D0%BB%D0%B8%D0%BC%D1%8B%D1%85_%D1%8F%D0%B7%D1%8B%D0%BA%D0%BE%D0%B2._%D0%A2%D0%B5%D0%BE%D1%80%D0%B5%D0%BC%D0%B0_%D0%A3%D1%81%D0%BF%D0%B5%D0%BD%D1%81%D0%BA%D0%BE%D0%B3%D0%BE-%D0%A0%D0%B0%D0%B9%D1%81%D0%B0}{Теорема Успенского-Райса}

\subsection{Проблема соответствий. Поста и ее неразрешимость}

\href{https://neerc.ifmo.ru/wiki/index.php?title=%D0%9F%D1%80%D0%B8%D0%BC%D0%B5%D1%80%D1%8B_%D0%BD%D0%B5%D1%80%D0%B0%D0%B7%D1%80%D0%B5%D1%88%D0%B8%D0%BC%D1%8B%D1%85_%D0%B7%D0%B0%D0%B4%D0%B0%D1%87:_%D0%BF%D1%80%D0%BE%D0%B1%D0%BB%D0%B5%D0%BC%D0%B0_%D1%81%D0%BE%D0%BE%D1%82%D0%B2%D0%B5%D1%82%D1%81%D1%82%D0%B2%D0%B8%D0%B9_%D0%9F%D0%BE%D1%81%D1%82%D0%B0}{Проблема соответствий Поста}
\end{document}
