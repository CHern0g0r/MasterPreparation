%----------------------------------------------------------------------------------------
%	PACKAGES AND OTHER DOCUMENT CONFIGURATIONS
%----------------------------------------------------------------------------------------

\documentclass{article}
% \documentclass[14pt]{extarticle}
\usepackage{pdfpages}

%%%%%%%%%%%%%%%%%%%%%%%%%%%%%%%%%%%%%%%%%
% Lachaise Assignment
% Structure Specification File
% Version 1.0 (26/6/2018)
%
% This template originates from:
% http://www.LaTeXTemplates.com
%
% Authors:
% Marion Lachaise & François Févotte
% Vel (vel@LaTeXTemplates.com)
%
% License:
% CC BY-NC-SA 3.0 (http://creativecommons.org/licenses/by-nc-sa/3.0/)
% 
%%%%%%%%%%%%%%%%%%%%%%%%%%%%%%%%%%%%%%%%%

%----------------------------------------------------------------------------------------
%	PACKAGES AND OTHER DOCUMENT CONFIGURATIONS
%----------------------------------------------------------------------------------------

\usepackage{amsmath,amsfonts,stmaryrd,amssymb,amsthm} % Math packages

\usepackage{mathtext}

\usepackage{enumerate} % Custom item numbers for enumerations

\usepackage[ruled]{algorithm2e} % Algorithms

\usepackage[framemethod=tikz]{mdframed} % Allows defining custom boxed/framed environments

\usepackage{listings} % File listings, with syntax highlighting
\lstset{
	basicstyle=\ttfamily, % Typeset listings in monospace font
}

\usepackage[unicode]{hyperref}
%----------------------------------------------------------------------------------------
%	DOCUMENT MARGINS
%----------------------------------------------------------------------------------------

\usepackage{geometry} % Required for adjusting page dimensions and margins

\geometry{
	paper=a4paper, % Paper size, change to letterpaper for US letter size
	top=2.5cm, % Top margin
	bottom=3cm, % Bottom margin
	left=2.5cm, % Left margin
	right=2.5cm, % Right margin
	headheight=14pt, % Header height
	footskip=1.5cm, % Space from the bottom margin to the baseline of the footer
	headsep=1.2cm, % Space from the top margin to the baseline of the header
	%showframe, % Uncomment to show how the type block is set on the page
}

%----------------------------------------------------------------------------------------
%	FONTS
%----------------------------------------------------------------------------------------

\usepackage[utf8]{inputenc} % Required for inputting international characters
\usepackage[T1, T2A]{fontenc} % Output font encoding for international characters

\usepackage[english,russian]{babel}

\usepackage{XCharter} % Use the XCharter fonts

%----------------------------------------------------------------------------------------
%	COMMAND LINE ENVIRONMENT
%----------------------------------------------------------------------------------------

% Usage:
% \begin{commandline}
%	\begin{verbatim}
%		$ ls
%		
%		Applications	Desktop	...
%	\end{verbatim}
% \end{commandline}

\mdfdefinestyle{commandline}{
	leftmargin=10pt,
	rightmargin=10pt,
	innerleftmargin=15pt,
	middlelinecolor=black!50!white,
	middlelinewidth=2pt,
	frametitlerule=false,
	backgroundcolor=black!5!white,
	frametitle={Command Line},
	frametitlefont={\normalfont\sffamily\color{white}\hspace{-1em}},
	frametitlebackgroundcolor=black!50!white,
	nobreak,
}

% Define a custom environment for command-line snapshots
\newenvironment{commandline}{
	\medskip
	\begin{mdframed}[style=commandline]
}{
	\end{mdframed}
	\medskip
}

%----------------------------------------------------------------------------------------
%	FILE CONTENTS ENVIRONMENT
%----------------------------------------------------------------------------------------

% Usage:
% \begin{file}[optional filename, defaults to "File"]
%	File contents, for example, with a listings environment
% \end{file}

\mdfdefinestyle{file}{
	innertopmargin=1.6\baselineskip,
	innerbottommargin=0.8\baselineskip,
	topline=false, bottomline=false,
	leftline=false, rightline=false,
	leftmargin=2cm,
	rightmargin=2cm,
	singleextra={%
		\draw[fill=black!10!white](P)++(0,-1.2em)rectangle(P-|O);
		\node[anchor=north west]
		at(P-|O){\ttfamily\mdfilename};
		%
		\def\l{3em}
		\draw(O-|P)++(-\l,0)--++(\l,\l)--(P)--(P-|O)--(O)--cycle;
		\draw(O-|P)++(-\l,0)--++(0,\l)--++(\l,0);
	},
	nobreak,
}

% Define a custom environment for file contents
\newenvironment{file}[1][File]{ % Set the default filename to "File"
	\medskip
	\newcommand{\mdfilename}{#1}
	\begin{mdframed}[style=file]
}{
	\end{mdframed}
	\medskip
}

%----------------------------------------------------------------------------------------
%	NUMBERED QUESTIONS ENVIRONMENT
%----------------------------------------------------------------------------------------

% Usage:
% \begin{question}[optional title]
%	Question contents
% \end{question}

\mdfdefinestyle{question}{
	innertopmargin=1.2\baselineskip,
	innerbottommargin=0.8\baselineskip,
	roundcorner=5pt,
	nobreak,
	singleextra={%
		\draw(P-|O)node[xshift=1em,anchor=west,fill=white,draw,rounded corners=5pt]{%
		Question \theQuestion\questionTitle};
	},
}

\newcounter{Question} % Stores the current question number that gets iterated with each new question

% Define a custom environment for numbered questions
\newenvironment{question}[1][\unskip]{
	\bigskip
	\stepcounter{Question}
	\newcommand{\questionTitle}{~#1}
	\begin{mdframed}[style=question]
}{
	\end{mdframed}
	\medskip
}

%----------------------------------------------------------------------------------------
%	WARNING TEXT ENVIRONMENT
%----------------------------------------------------------------------------------------

% Usage:
% \begin{warn}[optional title, defaults to "Warning:"]
%	Contents
% \end{warn}

\mdfdefinestyle{warning}{
	topline=false, bottomline=false,
	leftline=false, rightline=false,
	nobreak,
	singleextra={%
		\draw(P-|O)++(-0.5em,0)node(tmp1){};
		\draw(P-|O)++(0.5em,0)node(tmp2){};
		\fill[black,rotate around={45:(P-|O)}](tmp1)rectangle(tmp2);
		\node at(P-|O){\color{white}\scriptsize\bf !};
		\draw[very thick](P-|O)++(0,-1em)--(O);%--(O-|P);
	}
}

% Define a custom environment for warning text
\newenvironment{warn}[1][Warning:]{ % Set the default warning to "Warning:"
	\medskip
	\begin{mdframed}[style=warning]
		\noindent{\textbf{#1}}
}{
	\end{mdframed}
}

%----------------------------------------------------------------------------------------
%	INFORMATION ENVIRONMENT
%----------------------------------------------------------------------------------------

% Usage:
% \begin{info}[optional title, defaults to "Info:"]
% 	contents
% 	\end{info}

\mdfdefinestyle{info}{%
	topline=false, bottomline=false,
	leftline=false, rightline=false,
	nobreak,
	singleextra={%
		\fill[black](P-|O)circle[radius=0.4em];
		\node at(P-|O){\color{white}\scriptsize\bf i};
		\draw[very thick](P-|O)++(0,-0.8em)--(O);%--(O-|P);
	}
}

% Define a custom environment for information
\newenvironment{info}[1][Info:]{ % Set the default title to "Info:"
	\medskip
	\begin{mdframed}[style=info]
		\noindent{\textbf{#1}}
}{
	\end{mdframed}
}

\newcommand{\D}[1]{\begin{warn}[Def:] #1 \end{warn}}
\newcommand{\pred}[1]{\textbf{Предложение:} #1\\}
\newcommand{\T}[2][]{\begin{warn}[Theorem: #1] \\#2 \end{warn}}
\newcommand{\m}[1]{\mathfrak{#1}}
\newcommand{\inc}[1]{\includepdf[pages=-]{#1}}

\newcommand\invisiblesection[1]{%
  \refstepcounter{section}%
  \addcontentsline{toc}{section}{\protect\numberline{\thesection}#1}%
  \sectionmark{#1}}
 % Include the file specifying the document structure and custom commands

%----------------------------------------------------------------------------------------
%	ASSIGNMENT INFORMATION
%----------------------------------------------------------------------------------------

\title{Software Engineering} % Title of the assignment

%----------------------------------------------------------------------------------------

\begin{document}

\maketitle % Print the title

\section{Математика и алгоритмы}

\subsection{Системы линейных алгебраических уравнений. Теорема Кронекера-
Капелли. Общее решение системы алгебраических уравнений.}

\D{
    СЛАУ = система уровнений, каждое из которых линейно. Коэффициенты
    в уравнениях, а также переменные принимают значения в каком-либо поле.
}

СЛАУ можно представить в матричном виде: $Ax = b,\ A \in F^{m\times n}, b \in F^m$

Две системы эквивалентны, если множество их решений совпадает.

Эквивалентные преобразования:
\begin{itemize}
    \item Умножение на коэффициент
    \item Сложение с другим уравнением
\end{itemize}

$C$ - невырожденная матрица (полный ранг, det != 0)
$\Rightarrow Ax = b \sim CA x = C b$

$\Rightarrow \exists A^{-1} \rightarrow x = A^{-1} b$

\T[Кронекера-Капелли]{
    Система совместна $\iff rk A = rk (A | b)$
}

Методы решения:
\begin{itemize}
    \item Гаусса (Приведение к треугольному виду)
    \item Кронекера (Заменяем по очереди каждый столбец на $b$,
    считаем определитель $\Delta_i$, $x_i = \frac{\Delta_i}{\Delta}$)
    \item Обратной матрицей
    \item 
\end{itemize}


\subsection{Матрицы. Ранг матрицы, ранг произведения матриц, ранг 
транспонированной матрицы. Определитель матрицы. Определитель
произведения.}

\href{https://ru.wikipedia.org/wiki/%D0%9C%D0%B0%D1%82%D1%80%D0%B8%D1%86%D0%B0_(%D0%BC%D0%B0%D1%82%D0%B5%D0%BC%D0%B0%D1%82%D0%B8%D0%BA%D0%B0)}{Матрицы}

Ранг матрицы = количество линейно независимых столбцов (строк).

$rk A = rk A^T$

$rk AB \leq min(rk A, rk B)$ (Столбец $AB$ представляется как линейная
комбинация столбцов $A (B)$)

\href{https://ru.wikipedia.org/wiki/%D0%9E%D0%BF%D1%80%D0%B5%D0%B4%D0%B5%D0%BB%D0%B8%D1%82%D0%B5%D0%BB%D1%8C}{Определитель матрицы}

$det(AB) = det A \cdot det B$


\subsection{Основная теорема арифметики. Малая теорема Ферма, функция
Эйлера. Мультипликативность функции Эйлера. Теорема Эйлера.}

\href{https://ru.wikipedia.org/wiki/%D0%9E%D1%81%D0%BD%D0%BE%D0%B2%D0%BD%D0%B0%D1%8F_%D1%82%D0%B5%D0%BE%D1%80%D0%B5%D0%BC%D0%B0_%D0%B0%D1%80%D0%B8%D1%84%D0%BC%D0%B5%D1%82%D0%B8%D0%BA%D0%B8}{Основная теорема арифметики}

\T[Малая теорема ферма]{
    $p$ - простое, $a$ - целое $a \nmid p \Rightarrow a^{p-1}\equiv 1 (mod\ p)$
}

\href{https://ru.wikipedia.org/wiki/%D0%9C%D0%B0%D0%BB%D0%B0%D1%8F_%D1%82%D0%B5%D0%BE%D1%80%D0%B5%D0%BC%D0%B0_%D0%A4%D0%B5%D1%80%D0%BC%D0%B0}{мтф}

\D{
    Функция Эйлера = $\phi(n) = |\{x\in \mathbb{N}: x < n \wedge gcd(x, n) = 1\}|$

    $\forall n, m :\ gcd(n, m) = 1 \rightarrow \phi(nm) = \phi(n) \cdot \phi(m)$
}
\href{https://ru.wikipedia.org/wiki/%D0%A4%D1%83%D0%BD%D0%BA%D1%86%D0%B8%D1%8F_%D0%AD%D0%B9%D0%BB%D0%B5%D1%80%D0%B0}{ссыль}

\T[Теорема Эйлера]{
    $gcd(a, m) = 1 \Rightarrow a^{\phi(m)} \equiv 1 (mod\ m)$
}

\href{https://ru.wikipedia.org/wiki/%D0%A2%D0%B5%D0%BE%D1%80%D0%B5%D0%BC%D0%B0_%D0%AD%D0%B9%D0%BB%D0%B5%D1%80%D0%B0_(%D1%82%D0%B5%D0%BE%D1%80%D0%B8%D1%8F_%D1%87%D0%B8%D1%81%D0%B5%D0%BB)}{ТЭ}


\subsection{Вероятностное пространство. Независимые события. Теорема
сложения. Условная вероятность. Полная система событий. Формула
полной вероятности. Формула Байеса.}

\D{
    Вероятностное пространство = $(\Omega, \m{A}, \mathbb{P})$
    \begin{itemize}
        \item $\Omega$ = произвольное непустое множество элементарных событий
        \item $\m{A}$ = $\sigma$-алгебра (замкнуто по дополнениям + объединение счетного) подмножеств $\Omega$, называемых случайными событиями
        \item $\mathbb{P}$ = вероятностная мера ($\sigma$-аддитивная мера), т.ч. $\mathbb{P}(\Omega) = 1$
    \end{itemize}
}

\href{https://ru.wikipedia.org/wiki/%D0%92%D0%B5%D1%80%D0%BE%D1%8F%D1%82%D0%BD%D0%BE%D1%81%D1%82%D0%BD%D0%BE%D0%B5_%D0%BF%D1%80%D0%BE%D1%81%D1%82%D1%80%D0%B0%D0%BD%D1%81%D1%82%D0%B2%D0%BE}{Вероятностное пространство}

\D{
    События $A, B$ независимы $\iff P(AB) = P(A) \cdot P(B)$
}
\href{https://ru.wikipedia.org/wiki/%D0%9D%D0%B5%D0%B7%D0%B0%D0%B2%D0%B8%D1%81%D0%B8%D0%BC%D0%BE%D1%81%D1%82%D1%8C_(%D1%82%D0%B5%D0%BE%D1%80%D0%B8%D1%8F_%D0%B2%D0%B5%D1%80%D0%BE%D1%8F%D1%82%D0%BD%D0%BE%D1%81%D1%82%D0%B5%D0%B9)}{Независимые события}

\D{
    Условная вероятность: $P(B) > 0 \Rightarrow P(A|B) = \frac{P(AB)}{P(B)}$
}

\D{
    Полная система событий = система случайных событий такая, что в результате произведённого случайного эксперимента непременно произойдет одно и только одно из них.
}

\D{
    Формула полной вероятности:

    $P(B_i) > 0 \wedge B_i \cap B_j = \emptyset \wedge \bigcup B_i = \Omega \Rightarrow
    P(A) = \sum P(A | B_i) P(B_i)$
}

\D{
    Формула Байеса:

    $P(A | B) = \frac{P(B | A) P(A)}{P(B)}$
}


\subsection{Случайная величина и её функция распределения. Совместное
распределение случайных величин. Распределение суммы независимых случайных величин.}

\href{https://ru.wikipedia.org/wiki/%D0%A1%D0%BB%D1%83%D1%87%D0%B0%D0%B9%D0%BD%D0%B0%D1%8F_%D0%B2%D0%B5%D0%BB%D0%B8%D1%87%D0%B8%D0%BD%D0%B0}{Случайная величина}

\D{
    Функция распределения случайной величины $\xi$:

    $F_\xi(x) = \mathbb{P}(\xi \leq x)$
}

\href{https://mipt.ru/education/chair/mathematics/study/methods/%D0%A1%D0%A0%D0%A1%D0%92_%D0%A1%D0%B0%D0%BC%D0%BE%D1%80%D0%BE%D0%B2%D0%B0(2).pdf}{Совместное распределение, сумма независимых}


\subsection{Математическое ожидание и дисперсия случайной величины, их
свойства.}

$E[X] = \int\limits_{\Omega} \xi(\omega) d \mathbb{P}(d \omega) =
\int\limits_{-\infty}^\infty x d F_\xi(x)$

\href{https://ru.wikipedia.org/wiki/%D0%9C%D0%B0%D1%82%D0%B5%D0%BC%D0%B0%D1%82%D0%B8%D1%87%D0%B5%D1%81%D0%BA%D0%BE%D0%B5_%D0%BE%D0%B6%D0%B8%D0%B4%D0%B0%D0%BD%D0%B8%D0%B5}{Матожидание}

Свойства:
\begin{itemize}
    \item $E[a] = a$
    \item $E[a X + b Y] = a E[X] + b E[Y]$
    \item $0 \leq X \leq Y$ п.в. $\wedge E[Y] < \infty \Rightarrow 0 \leq E[X] \leq E[Y]$
    \item $X = Y$ п.в. $\Rightarrow E[X] = E[Y]$
    \item $X, Y$ - нез $\Rightarrow E[XY] = E[X] E[Y]$
\end{itemize}

$D[X] = E(X - E(X))^2$

\href{https://ru.wikipedia.org/wiki/%D0%94%D0%B8%D1%81%D0%BF%D0%B5%D1%80%D1%81%D0%B8%D1%8F_%D1%81%D0%BB%D1%83%D1%87%D0%B0%D0%B9%D0%BD%D0%BE%D0%B9_%D0%B2%D0%B5%D0%BB%D0%B8%D1%87%D0%B8%D0%BD%D1%8B}{Дисперсия}

Свойства:
\begin{itemize}
    \item $D[X] \geq 0$
    \item $D[X] < \infty \Rightarrow E[X] < \infty$
    \item $D[X] = 0 \iff X = E[X]$ п.в.
    \item $D[X + Y] = D[X] + D[Y] + cov(X, Y)$
    \item $D[\sum c_i X_i] = \sum c_i^2 D[X_i] + 2 \sum c_i c_j cov(X_i, X_j)$
    \item $D[a X] = a^2 D[X]$
    \item $D[-X] = D[X]$
    \item $D[X + b] = D[X]$
    \item 
\end{itemize}


\subsection{Теорема Больцано-Вейерштрасса и критерий Коши для числовой
последовательности.}

\T[Больцано-Вейерштрасса]{
    $\{x_i\} \subset R^n$ - последовательность, $\forall k ||x_k|| \leq C \in R_+$

    $\Rightarrow$

    $\exists \{k_i\}: \{x_{k_i}\}$ сходится к некоторой точке пространства $R^n$

}

\href{https://ru.wikipedia.org/wiki/%D0%A2%D0%B5%D0%BE%D1%80%D0%B5%D0%BC%D0%B0_%D0%91%D0%BE%D0%BB%D1%8C%D1%86%D0%B0%D0%BD%D0%BE_%E2%80%94_%D0%92%D0%B5%D0%B9%D0%B5%D1%80%D1%88%D1%82%D1%80%D0%B0%D1%81%D1%81%D0%B0}{Т. Больцано-Вейерштрасса}

\T[Критерий Коши]{
    $\{a_n\} \subset R^n$

    $\exists lim a_n \iff \forall \varepsilon > 0 \exists N: \forall m, n > N: |a_m - a_n| < \varepsilon$
}

\href{https://ru.wikipedia.org/wiki/%D0%9A%D1%80%D0%B8%D1%82%D0%B5%D1%80%D0%B8%D0%B9_%D0%9A%D0%BE%D1%88%D0%B8}{Критерий Коши}


\subsection{Два определения предела функции одной и нескольких переменных:
с помощью окрестностей и через пределы последовательностей.}

По Гейне:
$A = \lim\limits_{x \to x_0} f(x) \iff
\forall \{x_n\}: (\lim x_n = x_0 \wedge x_0 \notin \{x_n\})
\rightarrow (\lim f(x_n) = A)$

По Коши:
$\lim\limits_{x \to x_0} f(x) = A \iff
\forall \varepsilon > 0 \exists \delta = \delta(\varepsilon) > 0:
\forall x : (0 < |x - x_0| < \delta) \rightarrow (|f(x) - A| < \varepsilon)$



\section{Программирование}


\end{document}
