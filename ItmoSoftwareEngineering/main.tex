%----------------------------------------------------------------------------------------
%	PACKAGES AND OTHER DOCUMENT CONFIGURATIONS
%----------------------------------------------------------------------------------------

\documentclass{article}
% \documentclass[14pt]{extarticle}
\usepackage{pdfpages}

\input{structure.tex} % Include the file specifying the document structure and custom commands

%----------------------------------------------------------------------------------------
%	ASSIGNMENT INFORMATION
%----------------------------------------------------------------------------------------

\title{Software Engineering} % Title of the assignment

%----------------------------------------------------------------------------------------

\begin{document}

\maketitle % Print the title

\section{Математика и алгоритмы}

\subsection{Системы линейных алгебраических уравнений. Теорема Кронекера-
Капелли. Общее решение системы алгебраических уравнений.}

\D{
    СЛАУ = система уровнений, каждое из которых линейно. Коэффициенты
    в уравнениях, а также переменные принимают значения в каком-либо поле.
}

СЛАУ можно представить в матричном виде: $Ax = b,\ A \in F^{m\times n}, b \in F^m$

Две системы эквивалентны, если множество их решений совпадает.

Эквивалентные преобразования:
\begin{itemize}
    \item Умножение на коэффициент
    \item Сложение с другим уравнением
\end{itemize}

$C$ - невырожденная матрица (полный ранг, det != 0)
$\Rightarrow Ax = b \sim CA x = C b$

$\Rightarrow \exists A^{-1} \rightarrow x = A^{-1} b$

\T[Кронекера-Капелли]{
    Система совместна $\iff rk A = rk (A | b)$
}

Методы решения:
\begin{itemize}
    \item Гаусса (Приведение к треугольному виду)
    \item Кронекера (Заменяем по очереди каждый столбец на $b$,
    считаем определитель $\Delta_i$, $x_i = \frac{\Delta_i}{\Delta}$)
    \item Обратной матрицей
    \item 
\end{itemize}


\subsection{Матрицы. Ранг матрицы, ранг произведения матриц, ранг транс-
понированной матрицы. Определитель матрицы. Определитель про-
изведения.}




\section{Программирование}


\end{document}
